\section{Groupes, anneaux}

% \begin{equation}
% 
% %%
% \label{}
% \end{equation}

% \begin{sol}
% \begin{enumerate}
% \item %%
% %%
% \item %%
% \end{enumerate}
% \end{sol}

\begin{exer}
Soit $E$ un ensemble.

Montrer que $E$ est infini si et seulement si, toute bijection de $E$ sur lui-même stabilise au moins une partie stricte de $E$.
\end{exer}

\begin{exer}[Cas particulier du théorème de Cauchy]
Soient $p$ un nombre premier impair, et $G$ un groupe d'ordre $2p$.\\
Montrer que $G$ admet un élément d'ordre $p$.
%Indications : (i) Quels sont les ordres possibles des éléments de $G$ ?\\
%(ii) Un groupe dont tous les éléments sont d'ordre $2$ est abélien, et muni par l'exponentiation d'une structure d'espace vectoriel sur $\mathbb{Z} / 2 \mathbb{Z}$.\\
%Conclure à l'aide d'un argument de dimension.
\end{exer}

\begin{exer}
Soit $(G,.)$ un groupe abélien fini, noté multiplicativement. Pour tout élément $x$ de $G$, on note $O(x)$ l'ordre de $x$.
\begin{enumerate}
\item Soient $x$ et $y$ deux éléments de $G$ tels que $O(x)$ et $O(y)$ soient premiers entre eux. Déterminer l'ordre de $xy$.
\item On suppose ici $x$ et $y$ quelconques. Montrer qu'il existe un élément $z$ de $G$ tel que : $O(z) = O(x) \vee O(y)$.
\item\label{ques:ord_ppcm} En déduire l'existence d'un élément de $G$ dont l'ordre $m$ est le ppcm des ordres des éléments de $G$. %
$m$ est appelé exposant de $G$.
\item Supposons maintenant :\[\forall d \in \mathbb{N}^{\ast} , \lvert \{ x \in G / x^d = 1 \} \rvert \leqslant d\]
Montrer que $G$ est cyclique.
\item Soit $\mathbb{K}$ un corps commutatif. Montrer que %
tout sous-groupe fini du groupe multiplicatif de $\mathbb{K}$ est cyclique.
\end{enumerate}
\end{exer}

\begin{sol}
\begin{enumerate}
\item Supposons :

\begin{equation}
(xy)^k=1
%%
\label{}
\end{equation}
%%
pour un certain entier \(k\). %
La commutativit\'e de l'anneau \(A\) permet d'\'ecrire :

\begin{equation}
(xy)^k = x^k\,y^k
%%
\label{}
\end{equation}

Ainsi, \(x^k\) est une puissance de \(y\), ce qui entra\^ine :

\begin{equation}
(x^k)^{O(y)} = 1
%%
\label{eq:xk_oy}
\end{equation}

Mais \(x^k\) est aussi une puissance de \(x\), donc :

\begin{equation}
(x^k)^{O(x)} = 1
%%
\label{eq:xk_ox}
\end{equation}

Or \(O(x)\) et \(O(y)\) sont premiers entre eux : il existe deux entiers \(u\)et \(v\) tels que

\begin{equation}
O(x) u + O(y) v = 1
%%
\label{eq:bezout_oxy}
\end{equation}

Les relations~\rfeq{eq:bezout_oxy}, \rfeq{eq:xk_ox} et~\rfeq{eq:xk_oy} entra\^inent alors :

\begin{equation}
(x^k)^{O(x) u + O(y) v} = x^k \quad \text{donc} \quad x^k = 1
%%
\label{}
\end{equation}

On en d\'eduit que \(y^k=1\). %
Il s'ensuit que \(k\) est divisible par \(O(x)\) et par \(O(y)\), donc par leur produit car ces entiers sont premiers entre eux \textbf{raccourci possible ?}. %
R\'eciproquement, il est facile de voir que \((xy)^{O(x)\,O(y)}=1\) : %
l'ordre de \(xy\) est donc exactement \(O(x)\,O(y)\).

%%
\item Soient \(x\) et \(y\) deux \'el\'ements de \(G\), d'ordres respectifs \(m\) et \(n\). %
On peut \'ecrire :

\begin{equation}
m = dm' \quad\text{et}\quad n=dn'\, ,
%%
\label{}
\end{equation}
%%
o\g{u} \(d=m\wedge n\), %





%%
\item\label{seq_Aa_oppcm} On se va construire une suite \((A_k, a_k)_k\), o\g{u} \((A_k)_k\) est une suite strictement croissante de parties de \(G\), %
\(a_k\in A_k\) pour tout \(k\) et \(O(a_k)=\vee_{x\in A_k} \, O(x)\). %
Le dernier terme de cette suite founrira la solution.

\begin{itemize}
%%
\item %
On pose : \(A_1=\{1\}\) et \(a_1=1\), %
ce couple \((A_1, a_1)\) v\'erifie les propri\'et\'es voulues pour initialiser la suite.

\item Supposons la suite construite jusqu'\g{a} un rang \(k\). %
Alors deux possibilit\'es existent :

\begin{description}
%%
\item [\(A_k=G\) :] la construction d'arr\^ete ;
%%
\item [\(A_k \nsubseteq G\) :] on choisit un \'el\'ement \(b\) dans l'ensemble fini \(G \setminus A_k\). %
% Si 
Alors \(a_k \, b\) est d'ordre \(\left(\vee_{x\in A_k} \, O(x)\right) \vee O(b)\) d'apr\g{e}s la question pr\'ec\'edente. %\(\)
Comme le ppcm est associatif - il correspond \g{a} une intersectin d'id\'eaux de \(\mathbb{Z}\), %
on peut encore \'ecrire :

\begin{equation}
O(a_k\, b) = \vee_{x\in A_k \cup \{b\}} \, O(x).
%%
\label{}
\end{equation}
%%
On peut encore \'ecrire, pour la m\^eme raison et par idempotence du ppcm :

\begin{equation}
O(a_k\, b) = \vee_{x\in A_k \cup \{b, a_k\,b\}} \, O(x).
%%
\label{}
\end{equation}

Ainsi, en posant \(A_{k+1} = A_k \cup \{b, a_k\,b\}\) et \(a_{k+1}=a_k\, b\), %
on obtient le terme de rang \(k+1\) de la suite.
%%
\end{description}

Cet algorithme s'arr\^ete car \((A_k)_k\) est une suite strictement croissante de parties de l'ensemble fini \(G\), %
on peut aussi invoquer la stricte croissance et la bornitude de \(\sharp (A_k)_k\). %
%%
Le couple de rang maximal \((A_{k_{\max}}, a_{k_{\max}})\) nous donne la r\'eponse \g{a} la question.
%%
\end{itemize}

\item Soit \(m\) le ppcm des ordres des \'el\'ements de \(G\). %\(\mathbb{K}^{\times}\)
% et \(a\) un \'el\'ement de \(G\) d'ordre \(m\).
On note que tout \'e\'ement \(x\) de \(G\) est racine du polyn\^ome

\begin{equation}
X^m - 1
%%
\label{}
\end{equation}
%%
de \(\mathbb{K}[X]\). %
Comme ce polyn\^ome admet au plus \(m\) racines distinctes, l'ordre de \(G\) est inf\'erieur ou \'egal \g{a} \(m\). %
%%
Puisque \(m\) est l'ordre d'un \'el\'ement \(a\) de \(G\) d'apr\g{e}s la question pr\'ec\'edente, %
le th\'eor\g{e}me de Lagrange prouve que \(m\) divise \(\sharp G\) ; %
\(G\) est donc d'ordre \(m\). %
%%
L'ordre de \(a\) \'etant aussi celui de \(G\), ce groupe est cyclique et \(a\) en est un g\'en\'erateur.
%%
\end{enumerate}
\end{sol}

\begin{rema}
Nous verrons au chapitre~\ref{} un r\'esulat analogue \g{a} celui d\'emontr\'e \g{a} la question~\ref{ques:ord_ppcm} de cet exercice : %
le th\'eor\g{e}me de Caley-Hamilton.
\end{rema}

\begin{rema}
%%
En appliquant le r\'esultat pr\'ec\'edent au corps fini \(\mathbb{Z} / p\mathbb{Z}\), on \'etablit que le groupe multiplicatif \(\mathbb{Z} / p\mathbb{Z}^{\times}\) est cyclique, %
autrement dit qu'il existe une racine primitive modulo \(p\), pour tout nombre premier \(p\).
%%
\end{rema}

\begin{rema}
%%
La question~\ref{seq_Aa_oppcm} de l'exercice fournit une construction de racine primitive modulo \(p\). %
Toutefois, celle-ci n'est a priori pas plus efficace qu'une recherche exhaustive, on sait seulement qu'elle ach\g{e}ve.
%%
\end{rema}

\begin{exer}[Tout anneau intègre et fini est un corps]%Questions intermédiaires : pas d'obligation.
%Je pose tout d'abord le problème directement à l'élève, et je le laisse réfléchir quelques minutes sans indication de solution. Je continue l'interrogation en fonction de ses réactions, en posant éventuellement les questions intermédiaires qui suivent.\\
Soit en effet $(A,+,\times)$ un anneau, non nécessairement supposé unitaire, intègre et fini. %
Pour tout élément $a$ de $A \setminus \{0\}$, on note $m_a$ l'endomorphisme de multiplication à gauche %
$x \mapsto a \times x$ du groupe $(A,+)$ -pourquoi est-ce un endomorphisme ?
\begin{enumerate}
\item Montrer que les termes de la famille $(m_a)$ sont des automorphismes de $(A,+)$.
\item Montrer que $a \mapsto m_a$ induit un morphisme injectif de la structure algébrique associative %
$(A \setminus \{0\}, \times)$ dans le groupe des automorphismes de $(A,+)$.
\item Démontrer le lemme suivant :\textit{Soient $(G,\ast)$ un groupe et $F$ %
une partie finie de $G$ stable par $\ast$. Alors $(F,\ast)$ est un groupe.}
\item Conclure.
\end{enumerate}
\end{exer}