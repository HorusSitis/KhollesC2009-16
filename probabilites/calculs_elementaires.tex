\section{calculs \'el\'ementaires}

\begin{exer}[Une distraction du savant cosinus]
%\begin{center}
%\begin{minipage}{0.8\linewidth}
\textit{N voyageurs (au moins deux) s'apprêtent à monter dans un avion contenant N places num\'erot\'ees. %
Le premier d'entre eux s'av\`ere être le savant cosinus qui, distrait comme il l'est toujours, ne regarde pas le num\'ero de sa r\'eservation. %
Les passagers suivants, quand ils montent, en ordre, dans l'avion, s'asseyent alors \`a leur place r\'eserv\'ee si elle est encore libre, %
et sinon choisissent une place au hasard parmi celles qui restent.}
%\end{minipage}
%\end{center}
\ligneinter
\begin{enumerate}
\item Calculer, par r\'ecurrence sur $N$, la probabilit\'e $p_N$ que le dernier passager soit assis \`a sa place.
\item Retrouver le r\'esultat pr\'ec\'edent par un argument direct.
\end{enumerate}
\end{exer}

\begin{exer}
On jette plusieurs fois de suite et ind\'ependamment une pi\`ece de monnaie non \'equilibr\'ee, la probabilit\'e de tomber sur "Pile" est un r\'eel $p$ compris entre $0$ et $1$ strictement.\\
Calculer :
\begin{enumerate}
\item La probabilit\'e de ne pas avoir de "Face" au cours des $n$ premiers jets pour tout $n$ sup\'erieur \`a $1$ ;
\item La prpobabilit\'e d'obtenir "Face" pour la premi\`ere fios au $n-$i\`eme jet ;
\item l'esp\'erance du nombre de jets jusqu'\`a la premi\`ere apparition de "Face".
\end{enumerate}
\end{exer}

\begin{exer}
Pour tout entier $n$ sup\'erieur ou \'egal à $1$, on consid\`ere $n$ boules num\'erot\'ees de $1$ \`a $n$, que l'on place dans $n$ urnes, num\'erot\'ees de la m\^eme mani\`ere.
\begin{enumerate}
\item Calculer, pour tout $n$, la probabilit\'e $p_n$ de l'\'ev\`enement : chaque urne contient exactement une boule \`a la fin de l'op\'eration.
\item Montrer que la suite $(p_n)$ est d\'ecroissante et tend vers $0$ en $+\infty$.
\end{enumerate}
\end{exer}