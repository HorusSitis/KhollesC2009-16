\section{Arithm\'etique}

% \begin{equation}
% 
% %%
% \label{}
% \end{equation}

% \begin{sol}
% \begin{enumerate}
% \item %%
% %%
% \item %%
% \end{enumerate}
% \end{sol}

\begin{exer}
Montrer que le $\mathbb{C}$-espace vectoriel $\mathbb{C} (X)$ est de dimension non d\'enombrable.

On consid\'erera la famille $(\frac{1}{X-\lambda})_{\lambda \in \mathbb{C}}$.
\end{exer}

Remarquons que $\mathbb{C} (X)$, qui admet $\mathbb{C} [X]$ comme sous-espace vectoriel, %
n'est pas de dimension finie. %
%%
Deux solutions sont possible pour cet exercice, %
selon le sens accord\'e \g{a} l'expression \og{} dimension non d\'enombrable \fg{} . %
%%
Elles reposent toutes les deux sur l'unicit\'e de la d\'ecomposition en \'el\'ements simples %
d'une fraction rationnelle.

\begin{sol}[Existence d'une famille libre non d\'enombrable]
Pour r\'epondre \g{a} la question, il suffit de montrer que la famille \((\frac{1}{X-\lambda})_{\lambda \in \mathbb{C}}\), %
qui est non d\'enombrable, est libre. %

\par
Une combinaison lin\'eaire de termes de cette famille s'\'ecrit :

\begin{equation}
\sum\limits_{i=1}^n\, \frac{\alpha_i}{X-\lambda}
%%
\label{eq:def_cl_frac}
\end{equation}

\((\alpha_i)_{i=1}^n\) \'etant une famille finie de nombres complexes. %
Supposons que \(n>1\) et que tous les coefficients \(\alpha_i\) soient non nuls, %
et que la fraction rationnelle d\'efinie par~\rfeq{eq:def_cl_frac} soit nulle. %
%%
L'\'egalit\'e

\[\sum\limits_{i=1}^n\, \frac{\alpha_i}{X-\lambda}=0,\]
%%
est la d\'ecomposition en \'el\'ements simples de la fraction nulle, ce qui est absurde car celle-ci n'a pas de p\^ole. %
%%
Conclusion : il n'existe pas de combinaison lin\'eaire, \g{a} coefficients non nuls, des termes de \((\frac{1}{X-\lambda})_{\lambda \in \mathbb{C}}\), %
dont la valeur soit \(0\). %
Cette famille est donc libre.
\end{sol}

\begin{sol}[Non existence d'une famille d\'enombrable g\'en\'eratrice]
Le cas d'une famille g\'en\'eratrice finie est d\'ej\g{a} trait\'e, car un espace vectoriel de dimension finie ne peut avoir %
$\mathbb{C} [X]$ comme sous-espace vectoriel.

\par
Soit \((F_i(X))_{i\in\mathbb{N}}\) une suite de frations rationnelles. %
Pour tout \(i\), \(F_i(X)\) admet une suite finie de p\^oles \((\lambda_j)_{j=1}^{n_i}\). %
%%
L'ensemble

\begin{equation}
\bigcup\limits_{i\in\mathbb{N}}\,\text{Im}((\lambda_j)_{j=1}^{n_i})
%%
\label{eq:def_poles_ij}
\end{equation}
%%
de tous les p\^oles des termes de \((F_i(X))\) est une union finie d'ensembles d\'enombrables, %
il est donc d\'enombrable - on peut r\'eindexer les p\^oles qui apparaissent dans l'expression~\rfeq{eq:def_poles_ij} %
pour obtenir une suite index\'ee dans \(\mathbb{N}\). %
%%
Puisque \(\mathbb{C}\), comme \(\mathbb{R}\)\footnote{Voir la section concernant le espaces complets, dans ancien programme.}%
est non d\'enombrable, il existe une nombre complexe \(\lambda_{\infty}\) qui n'appartient pas \g{a} l'ensemble des p\^oles des fractions \(F_i(X)\). %
%%
La fraction

\[\frac{1}{X-\lambda_{\infty}},\]
%%
n'est pas une combinaison lin\'eaire de la famille \((F_i(X))_{i\in\mathbb{N}}\), %
car de telles combinaisons ont des p\^oles distincts de \(\lambda_{\infty}\). %
%%
Ainsi \((F_i(X))\) n'est-elle pas g\'en\'eratrice de \(\mathbb{C}(X)\).
\end{sol}

% Th\'eor\g{e}me de Mason, th\'eor\g{e}me de Liouville.

\begin{exer}[Densit\'e naturelle et diviseurs communs]
Soit $A$ une partie de $\mathbb{N}$.

\par
On dit que $A$ admet une densit\'e naturelle, si et seulement si la suite :
\[\left(\frac{\sharp A \cap [\![1,n]\!]}{n}\right)_{n \in \mathbb{N}^{\ast}}\]
%%
converge. %
On appelle densit\'e de $A$, et note $d(A)$, cette limite lorsqu'elle existe.
%%
\begin{enumerate}
\item Soit $\alpha$ un entier strictement positif. Montrer que $\alpha \mathbb{N}$ admet une densit\'e naturelle, %
calculer cette densit\'e.
%%
\item Soit \(F\) une partie finie de \(\mathbb{N}\). Que dire de la densit\'e de \(A\cup F\), en fonction de celle de \(A\) - existence, valeur ?
%%
\item Montrer que si $A$ est une partie de $\mathbb{N}$ qui admet une densit\'e naturelle \'egale à $1$, %
alors $A$ contient une infinit\'e d'entiers premiers entre eux deux-à-deux.
%%
\item D\'eduire de la question pr\'ec\'edente un r\'esultat classique en arithm\'etique.%Il s'agit du th\'eor\g{e}me d'infinitude de l'ensemble des nombres premiers.
\end{enumerate}
\end{exer}

\begin{sol}
%%
\begin{enumerate}
\item Remarquons que pour tout entier \(n\) strictement positif :

\begin{equation}
n=\lfloor \frac{n}{\alpha} \rfloor \, \alpha + r_n ,
%%
\label{eq:div_alpha_n}
\end{equation}
%%
o\g{u} \(r_n\) est un entier compris entre \(0\) et \(\alpha-1\). %
On peut aussi \'ecrire :

\begin{equation}
\sharp A \cap [\![1,n]\!] = \lfloor \frac{n}{\alpha} \rfloor.
%%
\label{eq:card_n_sur_alpha}
\end{equation}
%%
En combinant~\rfeq{eq:div_alpha_n} et~\rfeq{eq:card_n_sur_alpha} on obtient :

\begin{equation}
\dfrac{\sharp A \cap [\![1,n]\!]}{n} = \dfrac{1}{\alpha}\left(1-\frac{r_n}{n}\right) .
%%
\label{eq:dens_n_rn}
\end{equation}
%%
Puisque la suite \((r_n)_n\) est born\'ee, le terme de droite de~\rfeq{eq:dens_n_rn} tend vers \(\frac{1}{\alpha}\) lorsque \(n\) tend vers \(+\infty\). %
Cette limite est la densit\'e naturelle de \(\alpha\mathbb{N}^{\ast}\), qui en particulier existe.
%%
\item \dots
%%
\item Raisonnons par contrapos\'ee et soit \(A\) une partie de \(\mathbb{N}\) %
qui ne contient pas une infinit\'e d'entiers premiers entre eux deux \g{a} deux.

\begin{itemize}
\item Il existe une suite finie \((u_n)_{n\in \![1,N_{\text{max}}]\!]}\) d'\'el\'ements de \(A\) premiers entre eux deux \g{a} deux, %
qui est maximale pour cette propri\'et\'e. %
%%
En effet, dans le cas contraire, on construit par r\'ecurrence une suite \((u_n)_{n\in\mathbb{N}}\) %
d'\'el\'ements de \(A\), premiers entre eux deux \g{a} deux ce qui est impossible par hypoth\g{e}se.

\item Soit maintenant \(a\in A \setminus \mathrm{Im}\,(u_n)_{n\in \![1,N_{\text{max}}]\!]}\). %
%%
Il existe un terme \(u_{n_a}\) de \((u_n)_{n\in \![1,N_{\text{max}}]\!]}\) qui n'est pas premier avec \(a\) : %
en particulier, \(a\) et \(u_{n_a}\) ont un facteur premier \(p_a\) en commun. %
La relation :

\begin{equation}
a \in p_a\,\mathbb{N}
%%
\label{eq:fact_p_a}
\end{equation}
%%
se g\'en\'eralise ainsi :

\begin{equation}
\left(A \setminus \mathrm{Im}\,(u_n) \right)\subset \bigcup\limits_{p\in\,P} p\,\mathbb{N},
%%
\label{eq:fact_P_A_minus_u}
\end{equation}
%%
o\g{u} \(P\) est l'ensemble des nombres premiers divisant au moins l'un des termes de \((u_n)_{n\in \![1,N_{\text{max}}]\!]}\). %
De m\^eme, pour tout \(n\), \(u_n\) vaut \(1\) ou appartient \g{a} l'union pr\'ec\'ecente. %
Ainsi :

\begin{equation}
A \subset \{1\} \cup \bigcup\limits_{p\in\,P} p\,\mathbb{N},
%%
\label{eq:fact_P_A}
\end{equation}

\item On note que, comme chacun des ensembles \(p\,\mathbb{N}\), \(\bigcup\limits_{p\in\,P} p\,\mathbb{N}\) est p\'eriodique, %
et que l'une de ses p\'eriodes est

\begin{equation}
\prod\limits_{p\in\,P}p.
%%
\label{eq:period_P}
\end{equation}

On montre, de la m\^eme mani\g{e}re que dans la question pr\'ec\'edente, que l'ensemble \(\bigcup\limits_{p\in\,P} p\,\mathbb{N}\) %
admet une densit\'e naturelle, qui est \'egale \g{a}

\begin{equation}
\dfrac{\sharp \left(\bigcup\limits_{p\in\,P} p\,\mathbb{N} \cap [\![1,n]\!]\right)}{\prod\limits_{p\in\,P}p}.
%%
\label{eq:period_P}
\end{equation}
%%
Puisque \(1\) n'appartient pas \g{a} \(\bigcup\limits_{p\in\,P} p\,\mathbb{N}\), %
ce quotient est strictement inf\'erieur \g{a} \(1\).

\par
Remarquons enfin que la valeur ou l'existence de la densit\'e naturelle d'une partie de \(\mathbb{N}\) est inchang\'ee %
par r\'eunion avec un ensemble fini : \(\{1\} \cup \bigcup\limits_{p\in\,P} p\,\mathbb{N}\) admet une densit\'e strictement inf\'erieure \g{a} \(1\).

\item D'apr\g{e}s la relation~\rfeq{eq:fact_P_A}, \(A\) ne peut donc pas admettre une densit\'e \'egale \g{a} \(1\). %
En particulier, il admet une densit\'e strictement inf\'erieure \g{a} \(1\) ou n'admet pas de densit\'e.
%%
\end{itemize}
%%
\item Remarquons que \(\mathbb{N}^{\ast}\) admet une densit\'e naturelle, qui est \'egale \g{a} \(1\). %
Il existe donc une infinit\'e d'entiers naturels premiers entre eux deux \g{a} deux. %
%%
Comme les facteurs premiers de deux quelconques de ces entiers sont distincts, il existe donc une infinit\'e de nombres premiers.
\end{enumerate}
%%
\end{sol}

\begin{exer}[Le th\'eor\g{e}me d'infinitude de l'ensemble des nombres premiers revisit\'e]
%%
On d\'efinit un ensemble $\tau$ parties de $\mathbb{Z}$ par :

\begin{equation}
% \[
\forall P \in \mathbb{N} , P \in \tau \Leftrightarrow ( \forall m \in P , \exists a \in \mathbb{Z} | m + a \mathbb{Z} \subseteq P)
% \]
%%
\label{eq:def_topo_Zab}
\end{equation}

\begin{enumerate}
\item Montrer que $\tau$ est une topologie sur $\mathbb{Z}$.
\item Montrer que, si $a$ est un entier et $b$ un entier naturel non nul, alors : %
$a \mathbb{Z} + b$ est ferm\'e dans $(\mathbb{Z} , \tau)$.
\item Montrer que l'ensemble des nombres premiers est infini.
\end{enumerate}
\end{exer}

\begin{sol}
%%
\begin{enumerate}
\item Montrons que $\tau$ est une topologie sur $\mathbb{Z}$ \dots
%%
\item Montrons que, si $a$ est un entier et $b$ un entier naturel non nul, alors : %
$a \mathbb{Z} + b$ est ferm\'e dans $(\mathbb{Z} , \tau)$ \dots
%%
\item Supposons que l'ensemble des nombres premiers soit infini. %
%%
L'union \(\bigcup\limits_{p\,\text{premier}}\,p\,\mathbb{Z}\), %
qui est une r\'eunion de ferm\'es d'apr\g{e}s la question pr\'ec\'edente, %
est donc finie : %
puisque \(\tau\) satisfait les axiomes d'une toppologie, il s'ensuit que

\begin{equation}
\bigcup\limits_{p\,\text{premier}}\,p\,\mathbb{Z}
%%
\label{}
\end{equation}
%%
est un ferm\'e de cette topologie. %
Puique tout entier non inversible admet au moins un diviseur premier, %
on peut aussi \'ecrire :

\begin{equation}
\bigcup\limits_{p\,\text{premier}}\,p\,\mathbb{Z} = \mathbb{Z} \setminus \{-1, 1\}.
%%
\label{}
\end{equation}

\par
L'ensemble \(\{-1, 1\}\) est donc un ouvert non vide de \(\mathbb{Z}\), %
ce qui est absurde car par d\'efinition de \(\tau\), tout ouvert non vide est infini. %
% 
% \par
Conclusion : l'ensemble des nombres premiers est infini.
%%
\end{enumerate}
%%
\end{sol}

\begin{exer}
Soitn $n$ un entier naturel non nul tel que la suite $(a_k)_k$ des entiers strictement positifs, %
inf\'erieurs à $n$ et premiers à $n$ soit arithm\'etique.\\
Montrer que $n$ est une puissance de deux, ou un nombre premier impair.
\end{exer}

\begin{exer}
D\'emontrer qu'une fonction rationnelle complexe non constante omet au plus une valeur dans $\mathbb{C}$.
%Indication : On utilisera le th\'eor\g{e}me de D'Alembert.
\end{exer}

\begin{exer}
Soit $k$ un entier naturel sup\'erieur ou \'egal à $2$.\\
Montrer que le produit de trois entiers naturels non nuls cons\'ecutifs n'est jamais une puissance $k-$i\g{e}me.
\end{exer}

\begin{exer}
\begin{enumerate}
\item Montrer qu'il existe une infinit\'e de nombres premiers congrus à $3$ modulo $4$.
\item Montrer qu'il existe une infinit\'e de nombres premiers congrus à $5$ modulo $6$.
\end{enumerate}
\end{exer}

\begin{exer}
Soit $A$ un anneau commutatif.\\
On dit que $A$ est Noetherien si et seulement si tout id\'eal de $A$ est engendr\'e map un nombre fini d'\'el\'ements de $A$.
\begin{enumerate}
\item Montrer que toute suite croissante d'id\'eaux d'un anneau Noetherien est stationnaire.
\item Montrer que, si $A$ est un anneau int\g{e}gre et Noetherien, alors tout \'el\'ement de $A$ est d\'ecomposable en produit de facteurs irr\'eductibles.
\end{enumerate}
%Indications : (i) Raisonner par l'absurde.\\
%(ii) Soit $a$ un \'el\'ement de $A$ non d\'ecomposable. Montrer que $a$ est le produit de deux \'el\'ements de $A$ qui ne sont pas des unit\'es, et que l'un d'entre eux est non d\'ecomposable.\\
%(iii) Montrer que si un anneau est principal, alors toute suite croissante de ses id\'eaux est stationnaire.\\
%(iv) Consid\'erer la suite des id\'eaux respectivement engendr\'es par les termes de la suite d'\'el\'ements ind\'ecomposables de $A$ d\'efinie aux questions (i) et (ii). Conclure.
\end{exer}

\begin{sol}
%%
\begin{enumerate}
\item Soit \((I_n)_{n\in\mathbb{N}}\) unes suite croissante d'id\'eaux de \(A\). %
%%
Montrons que \(\bigcup\limits_{i\in\mathbb{N}}\,I_n\), autrement not\'e \(I\), est un id\'eal de \(A\). %
Soient en effet \(x\) et \(y\) deux \'el\'ements de cete union. %
En particulier, il existe deux enters \(n_x\) et \(n_y\) tels que \(x\) et \(y\) appartiennent respectivement \g{a} %
\(I_{n_x}\) et \(I_{n_y}\). %
%%
Mais la famille \((I_n)_{n\in\mathbb{N}}\) est croissante : \(x\) et \(y\), et donc leur somme, appartiennent \g{a} l'id\'eal \(I_{\max{n_x, n_y}}\) de \(A\). %
%%
Enfin, si \(x\) appartient \g{a} \(I_n\) pour un certain \(n\) et si \(a\) est un \'el\'ement de \(A\), %
alors \(ax\) appartient \g{a} \(I_n\), donc \g{a} \(\bigcup\limits_{i\in\mathbb{N}}\,I_n\).

\par
On peut maintenant d\'emontrer le r\'esultat principal de cette section. %
En effet, \(A\) \'etant n\"otherien, \(I\) est engendr\'e par une famille finie \((x_i)_{i=1}^{n_I}\) de ses \'el\'ements. %
En particulier, chaque \(x_i\) appartient \g{a} au moins un terme \(I_{n_i}\) de la suite \(I_n\) %
- on peut par exemple prendre le premier terme de \((I_n)_n\) o\g{u} l'on reencontre \(x_i\). %
%%
Alors, tous les termes de la famille g\'en\'eratrice \((x_i)_i\) de \(I\) appartiennnent \g{a} \(I_{\max\limits_i{n_i}}\). %
Cet id\'eal est donc confondu avec la limite \(I\) de la suite, qui est donc stationnaire.

%%
\item Raisonnons par l'absurde et supposons qu'il existe un \'el\'ement \(a\) de \(A\) qui n'est pas d\'ecomposable en produit de facteurs irr\'eductibles. %
%%
Plus particuli\g{e}rement, \(a\) ne peut \^etre ni irr\'eductible, ni inversible. %
On peut donc \'ecrire

\begin{equation}
a = bc
%%
\label{eq:hyp_a_nondec}
\end{equation}
%%
o\g{u} \(b\) et \(c\) sont non inversibles. %
%%
Les id\'eaux principaux \(b\,A\) et \(c\,A\) incluent \(a\,A\) d'apr\g{e}s ces relations de divisibilit\'e et la commutativit\'e de \(A\). %
Supposons par exemple que l'inclusion r\'eciproque \(b\,A\subset a\,A\) soit v\'erifi\'ee. %
On peut en particulier \'ecrire :

\begin{equation}
b=au\quad \text{donc} \quad b=bcu
%%
\label{eq:}
\end{equation}
%%
Puisque \(A\) est int\g{e}gre, ceci implique que \(cu=1\), donc que \(c\) est inversible, absurde. %
Par ailleurs, l'un au moins des facteurs \(b\) et \(c\) est ind\'ecomposable en produit de facteurs irr\'eductibles de \(A\), %
car dans le cas contraire~\rfeq{eq:hyp_a_nondec} fournirait une d\'ecomposition de \(a\). %
%%
On a ainsi construit un \'el\'ement \(a_1\) de \(A\), \(b\) ou \(c\), %
ind\'ecomposable comme \(a\), et tel que

\begin{equation}
a\,A \subsetneq a_1\, A
%%
\label{eq:st_inc}
\end{equation}

En it\'erant cette construction, on obtient donc une suite strictement croissante d'id\'eaux de \(A\), %
ce qui est impossible d'apr\g{e}s la question pr\'ec\'edente.
%%
\end{enumerate}
%%
\end{sol}