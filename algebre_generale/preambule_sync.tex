\usepackage[top=2cm,bottom=2cm,left=2cm,right=2cm]{geometry}% On peut changer en cours de route avec la commande \newgeometry

\usepackage[utf8]{inputenc}
\usepackage[french]{babel}
\usepackage{amsmath,amsfonts,amssymb,graphicx}
\usepackage[dvipsnames]{pstricks}
\usepackage{pstricks-add,pst-plot,pst-node}
%\usepackage{fp}
%\usepackage{multido}
%\usepackage{pdftricks}

%\usepackage{color} %(black, white, red, green, blue, yellow, magenta et cyan)

%\usepackage{amsthm}
\usepackage{framed}
\usepackage[amsthm,thmmarks,framed]{ntheorem}

\usepackage{multicol}
%\begin{multicols}[titre]{nb colonnes}
%\setlength{\columnseprule}{0.25pt}
\usepackage{array}

\usepackage{titletoc}
%\usepackage{hyperref} Probl\`emes avec la mise en page des exercices.

%\dottedcontents{section}[left]{above}{labelwidth}{leaderwidth = espacement entre les pointillés par exemple}
\dottedcontents{section}[0em]{\vspace{0.35cm}\bfseries}{3em}{0.75em}
\dottedcontents{subsection}[1.5em]{\vspace{0.15cm}}{4em}{0.75em}
\dottedcontents{subsubsection}[3em]{}{5em}{0.75em}

\setlength{\parindent}{0cm}

\newcommand{\g}[1]{\`#1}

\newcommand*{\etoile}
{
\begin{center}
\hspace{1pt}\par
*\hspace{5pt}*\hspace{5pt}*
\end{center}
}


\newcommand*{\ligneinter}
{
\begin{center}
\vspace{-2pt}
\hfill\rule{0.5\linewidth}{0.4pt}\hfill\null
\end{center}
\vspace{7pt}
}

% Je veux cr\'eer un environnement pour les énoncés préliminaires des exercices, afin d'éviter l'utilisation manuelle de \linebreak et, 
%pourquoi pas, éviter le bug de centrage systématique de l'en-tête.
  
{%
\theoremstyle{break}
\theoremprework{\vspace{0.5cm}\begin{minipage}{\textwidth}}
\theorempostwork{\etoile\end{minipage}}
\theoremheaderfont{\scshape}
\theorembodyfont{\normalfont}
\theoremseparator{ :\newline\vspace{0.2cm}}
\newtheorem{exer}{Exercice}[section]
}
  
{
\theoremstyle{break}
%\theoremprework{\vspace{0.2cm}}
\theorempostwork{\ligneinter}%\hfill\rule{0.6\linewidth}{0.4pt}\hfill\null}

\theoremheaderfont{\scshape}
\theorembodyfont{\normalfont}
\theoremseparator{ :\newline\vspace{0.2cm}}
\newtheorem{ques}{Question}[section] %Autre pr\'esentation : encadrer les \'enonc\'es ?
} 
  
%\setcounter{section}{-1}
\renewcommand{\thesection}{\Roman{section}}
