\section{Programme de sup}

\begin{exer}
Soit $n$ un entier naturel non nul.

On d\'efinit, sur $\mathbb{R}_n[X]$, l'application $\varphi$ par : $\varphi (P) = Q$ o\`u
\[\forall x \in\mathbb{R} , \overset{\sim}{Q}(x)=\int\limits_x^{x+1}P(x)dx\]
D\'emontrer que $\varphi$ r\'ealise un endomorphisme de $\mathbb{R}_n[X]$ dont on calculera le d\'eterminant.
\end{exer}

\begin{exer}
Soient $A$ et $B$ deux matrices $n\times n$ \`a coefficients dans $\mathbb{Z}$ telles que : $det A \wedge det B = 1$.

Montrer qu'il existe deux matrices $U$ et $V$, \`a coefficients entiers, telles que :\[UA+VB=AU+BV ; AU+BV=I_n\]
\end{exer}

\begin{exer}
Soient $\mathbb{K}$ un corps commutatif, $E$ et $F$ deux $\mathbb{K}-$espaces vectoriels de dimension finie. %
Soit de plus $G$ un sous-espace vectoriel de $E$, on note $A$ l'ensemble :\[\{u\in\mathcal{L}(E,F) |\ker u\supseteq G\}\]
\begin{enumerate}
\item Montrer que $A$ est un sous-espace vectoriel de $\mathcal{L}(E,F)$ et pr\'eciser sa dimension.
\item Que retrouve-t-on dans le cas o\`u $F=\mathbb{K}$ ?
\end{enumerate}
\end{exer}

\begin{exer}
Soient $n$, $k$ et $p$ trois entiers naturels non nuls, on suppose que : $n\leq p$ %
Soit $E$ un espace vectoriel de dimension finie sur un corps $\mathbb{K}$. On suppose que le cardinal de $\mathbb{K}$ est strictement sup\'erieur \`a $k(n-1)$.

Montrer que toute famille $(F_i)_{i\in [\![1,k]\!]}$ de sous-espaces vectoriels de m\^me dimension $p$, on peut construire un suppl\'ementaire commun \`a chacun des $F_i$.\\
\textit{On traite le cas o\`u $\mathbb{K}$ est infini, on affine ensuite \'eventuellement.}
\end{exer}

\begin{exer}
Montrer que la famille $(x\mapsto\arg\sinh \lambda x)_{\lambda\in\mathbb{R}_+{\ast}}$ est libre %
dans $\mathcal{C}^{\infty}(\mathbb{R},\mathbb{R})$.
\end{exer}