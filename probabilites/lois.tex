\section{Lois de probabilit\'es}

\begin{exer}[Recensement d'une population d'\'ecureuils]
%\begin{center}
%\begin{minipage}{\0.8\linewidth}
\textit{On veut estimer le nombre $N$ d'\'ecureuils dans une for\^et. %
Pour cela on en capture $k$, on leur met une petite marque sur la patte et on les rel\^ache. %
Une semaine apr\`es (on suppose qu'aucun \'ecureuil est mort ou n\'e dans cet intervalle), %
on en capture $l$ et on compte ceux d'entre eux qui portent la marque.}
%\end{minipage}
%\end{center}
\ligneinter
\begin{enumerate}
\item Calculer la probabilit\'e d'observer une valeur $m$ \'ecureuils marqu\'es  en fonction de $N$, $k$ et $l$.
\item Calculer, quand $N$ est grand, la valeur $N_{max}$ pour laquelle la probabilit\'e ci-dessus est la plus grande. On appelle cette valeur \textit{estimateur du maximum de vraisemblance} pour le nombre d'\'ecureuils.
\end{enumerate}
\end{exer}

\begin{exer}
Soit $(X_n)$ une suite de variables al\'eatoires de lois respectives $\mathcal{B}(n,p_n)$ o\`u $np_n$ est constant, de valeur unique $\lambda$.\\
On note, pour tout $n$ : $A_n = (X_n \geq 1)$.\\
Soit, de plus, $Y$ une variable de Poisson de param\`etre $\lambda$.\\
Montrer, pour $j$ fix\'e et sup\'erieur \`a $1$, que :\[\mathbb{P}(X_n=j|A_n)\underset{n\rightarrow +\infty}{\longrightarrow}\mathbb{P}(Y=j|y\geq 1)\]
\end{exer}

\begin{exer}
Soit $X$ une variable al\'eatoire de loi $\mathcal{B}(n,p)$.\\
Quel entier $j$ compris entre $0$ et $n$, maximise $\mathbb{P}(X=j)$ ?
\end{exer}

\begin{exer}[Optimisations avec la loi de Poisson]
\begin{enumerate}
\item Soit $\lambda$ un r\'eel strictement positif. Quel entier $j$ maximise $\mathbb{P}(X=j)$, pour une variable de Poisson de param\`etre $\lambda$ ?
\item Soit $j$ un entier positif. Quel est le r\'eel $\lambda$ pour lequel $\mathbb{P}(X=j)$ et maximal, o\`u $X$ est choisie avec une loi $\mathcal{P}(\lambda)$ ?
\end{enumerate}
\end{exer}

\begin{exer}
Soit $X$ une variable al\'eatoire de loi $\mathcal{G}(p)$.\\
Montrer que :\[\mathbb{E}\left(\frac{1}{1+X}\right)=\ln((1-p)^{\frac{p}{p-1}}))\]
\end{exer}

%\newpage

\begin{exer}
Soit $p$ un réel compris, strictement entre $0$ et $1$, et soient $X$ et $Y$ deux variables al\'eatoires ind\'ependantes, de m\^eme loi $\mathcal{B}(p)$.\\
Donner les lois de $max(X,Y)$ et $min(X,Y)$.
\end{exer}

\begin{exer}
Soient $X$ et $Y$ deux variables al\'eatoires d\'efinies sur un m\^eme espace $\Omega$, ind\'ependantes, et telles que :
\begin{itemize}
\item $X$ suit une loi de Poisson de param\`etre $\lambda$ ;
\item $Y$ vaut $1$ ou $2$, avec la m\^eme probabilit\'e $\frac{1}{2}$.
\end{itemize}
\begin{enumerate}
\item Donner la loi de la variable $Z$ d\'efinie par : $Z=XY$.
\item Calculer la probabilit\'e que $Z$ prenne une valeur paire.
\end{enumerate}
\end{exer}

\begin{exer}
Soit $X$ une variable al\'eatoire de Poisson, de param\`etre $\lambda$. On note $Y$ la variable $(-1)^X$.
\begin{enumerate}
\item Donner l'ensemble des valeurs prises par $Y$, puis calculer son esp\'erance.
\item D\'eterminer la loi de $Y$.
\end{enumerate}
\end{exer}