\section{Programmme actuel}

\begin{exer}
Montrer que le plan Euclidien peut être pavé par des polygones réguliers, %
si et seulement si leurs nombres de côtés est $3$, $4$ ou $6$.
%Indication : on considèrera les angles des sommets de ces polygones.
\end{exer}

\begin{exer}
On se donne $1000$ points du plan affine Euclidien.

Montrer qu'il existe une droite affine qui partitionne la plan en deux demi-plans ouverts, %
contenant chacun exactement $500$ points de la famille précédente.
\end{exer}

\begin{exer}[Théorème de Sylvester]
On concidère $n$ points du plan affine Euclidien.

Montrer que, si toute droite menée par deux de ces points en contient un troisième, alors tous ces points sont alignés.
\end{exer}

\begin{exer}
Montrer que le plan affine Euclidien ne peut être partitionné en cercles Euclidiens de rayons non nuls.
\end{exer}