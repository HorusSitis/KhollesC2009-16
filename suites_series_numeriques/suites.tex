\section{Suites numériques}

\begin{exer}
Donner une condition nécessaire et suffisante sur une suite réelle $(u_n)_n$ pour qu'il existe %
une permutation $\sigma$ de $\mathbb{N}$ telle que $(u_{\sigma (n)})_{n \in \mathbb{N}}$ soit ultimement monotone.
\end{exer}

\begin{exer}
Déterminer le nombre de chemins permettant de joindre deux sommets opposés d'un cube %
-une étape est définie par le franchissement d'une arète.
%\textit{Je demande d'étudier le problème afin de trouver une formalisation, je m'attends à une question de récurrence linéaire. Eu égard à la complexité des matrices étudiées, je ne demande pas de calcul explicite mais, éventuellement, une version manuscrite d'un code Maple adéquat.}
\end{exer}

\begin{exer}
Soient $(u_n)$ et $(v_n)$ deux suites réelles telles que $0$ soit une valeur d'adhérence de $(u_n v_n)$.

Montrer que $0$ est une valeur d'adhérence de $(u_n)$ ou de $(v_n)$.
\end{exer}

\begin{exer}
Soit $(x_n)$ une suite d'un espace vectoriel normé -le résultat est vrai dans un espace topologique quelconque-, %
telle que$(x_{2n})$, $(x_{2n+1})$ et $(x_{3n})$ convergent.

Montrer que $(x_n)$ converge.
\end{exer}
