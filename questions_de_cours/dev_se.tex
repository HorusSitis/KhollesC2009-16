\section{D\'eveloppements en s\'erie enti\g{e}re}

\begin{ques}
Donner le développement en s\'erie enti\g{e}re; au voisinage de $0$, de la fonction $\arg \tanh$.
\end{ques}

\begin{ques}
Donner le d\'eveloppement en s\'erie enti\g{e}re, au voisinage de $0$, de la fonction $\arg \sinh$.
\end{ques}

\begin{ques}
Donner le d\'eveloppement en s\'erie enti\g{e}re, au voisinage de $0$, de la fonction $\arg \cosh$.
\end{ques}

\begin{ques}
Soit $\alpha$ un r\'eel quelconque. Donner le d\'eveloppement en s\'erie enti\`ere de la fonction $x\mapsto (1+x)^{\alpha}$.
\end{ques}

\begin{ques}
Donner le développement en série entière; au voisinage de $0$, de la fonction $\arctan$.
\end{ques}

\begin{ques}
Donner le d\'eveloppement en s\'erie enti\g{e}re, au voisinage de $0$, de la fonction $\arcsin$.
\end{ques}

\begin{ques}
Donner le d\'eveloppement en s\'erie enti\g{e}re, au voisinage de $0$, de la fonction $\arccos$.
\end{ques}

\begin{ques}
Donner le d\'evelppoement en s\'erie enti\g{e}re au voisinage de $0$ de $x\mapsto\ln (1+x)$. %
Peut-on prolonger ce d\'eveloppement au bord de l'intervalle de convergence ?
\end{ques}

\begin{ques}
Donner un d\'eveloppement en s\'erie enti\`ere de la fonction $x\mapsto\frac{\sin x}{x}$ au voisinage de $0$.
\end{ques}