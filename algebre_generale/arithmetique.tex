\section{Arithmétique}

% \begin{equation}
% 
% %%
% \label{}
% \end{equation}

\begin{exer}
Montrer que le $\mathbb{C}$-espace vectoriel $\mathbb{C} (X)$ est de dimension non dénombrable.

On considérera la famille $(\frac{1}{X-\lambda})_{\lambda \in \mathbb{C}}$.
\end{exer}

Remarquons que $\mathbb{C} (X)$, qui admet $\mathbb{C} [X]$ comme sous-espace vectoriel, %
n'est pas de dimension finie. %
%%
Deux solutions sont possible pour cet exercice, %
selon le sens accord\'e \g{a} l'expression \og{} dimension non dénombrable \fg{} . %
%%
Elles reposent toutes les deux sur l'unicit\'e de la d\'ecomposition en \'el\'ements simples %
d'une fraction rationnelle.

\begin{sol}[Existence d'une famille libre non d\'enombrable]
Pour r\'epondre \g{a} la question, il suffit de montrer que la famille \((\frac{1}{X-\lambda})_{\lambda \in \mathbb{C}}\), %
qui est non d\'enombrable, est libre. %

\par
Une combinaison lin\'eaire de termes de cette famille s'\'ecrit :

\begin{equation}
\sum\limits_{i=1}^n\, \frac{\alpha_i}{X-\lambda}
%%
\label{eq:def_cl_frac}
\end{equation}

\((\alpha_i)_{i=1}^n\) \'etant une famille finie de nombres complexes. %
Supposons que \(n>1\) et que tous les coefficients \(\alpha_i\) soient non nuls, %
et que la fraction rationnelle d\'efinie par~\rfeq{eq:def_cl_frac} soit nulle. %
%%
L'\'egalit\'e

\[\sum\limits_{i=1}^n\, \frac{\alpha_i}{X-\lambda}=0,\]
%%
est la d\'ecomposition en \'el\'ements simples de la fraction nulle, ce qui est absurde car celle-ci n'a pas de p\^ole. %
%%
Conclusion : il n'existe pas de combinaison lin\'eaire, \g{a} coefficients non nuls, des termes de \((\frac{1}{X-\lambda})_{\lambda \in \mathbb{C}}\), %
dont la valeur soit \(0\). %
Cette famille est donc libre.
\end{sol}

\begin{sol}[Non existence d'une famille d\'enombrable g\'en\'eratrice]
Le cas d'une famille g\'en\'eratrice finie est d\'ej\g{a} trait\'e, car un espace vectoriel de dimension finie ne peut avoir %
$\mathbb{C} [X]$ comme sous-espace vectoriel.

\par
Soit \((F_i(X))_{i\in\mathbb{N}}\) une suite de frations rationnelles. %
Pour tout \(i\), \(F_i(X)\) admet une suite finie de p\^oles \((\lambda_j)_{j=1}^{n_i}\). %
%%
L'ensemble

\begin{equation}
\bigcup\limits_{i\in\mathbb{N}}\,\text{Im}((\lambda_j)_{j=1}^{n_i})
%%
\label{eq:def_poles_ij}
\end{equation}
%%
de tous les p\^oles des termes de \((F_i(X))\) est une union finie d'ensembles d\'enombrables, %
il est donc d\'enombrable - on peut r\'eindexer les p\^oles qui apparaissent dans l'expression~\rfeq{eq:def_poles_ij} %
pour obtenir une suite index\'ee dans \(\mathbb{N}\). %
%%
Puisque \(\mathbb{C}\), comme \(\mathbb{R}\)\footnote{Voir la section concernant le espaces complets, dans ancien programme.}%
est non d\'enombrable, il existe une nombre complexe \(\lambda_{\infty}\) qui n'appartient pas \g{a} l'ensemble des p\^oles des fractions \(F_i(X)\). %
%%
La fraction

\[\frac{1}{X-\lambda_{\infty}},\]
%%
n'est pas une combinaison lin\'eaire de la famille \((F_i(X))_{i\in\mathbb{N}}\), %
car de telles combinaisons ont des p\^oles distincts de \(\lambda_{\infty}\). %
%%
Ainsi \((F_i(X))\) n'est-elle pas g\'en\'eratrice de \(\mathbb{C}(X)\).
\end{sol}

% Théorème de Mason, théorème de Liouville.

\begin{exer}[Densité naturelle et diviseurs communs]
Soit $A$ une partie de $\mathbb{N}$.\\
On dit que $A$ admet une densité naturelle, si et seulement si la suite :
\[\left(\frac{\sharp A \cap [\![1,n]\!]}{n}\right)_{n \in \mathbb{N}^{\ast}}\]
converge. On appelle densité de $A$, et note $d(A)$, cette limite lorsqu'elle existe.
\begin{enumerate}
\item Soit $\alpha$ un entier strictement positif. Montrer que $\alpha \mathbb{N}$ admet une densité naturelle, %
calculer cette densité.
\item Montrer que si $A$ est une partie de $\mathbb{N}$ qui admet une densité naturelle égale à $1$, %
alors $A$ contient une infinité d'entiers premiers entre eux deux-à-deux.
%Raisonner par l'absurde.
%Lemme important : il existe au moins une famille (x_i) d'entiers de $A$, premiers entre eux deux à deux, maximale parmi les familles d'entiers de A qui ont cette propriété.
\item Déduire de la question précédente un résultat classique en arithmétique.%Il s'agit du théorème d'infinitude de l'ensemble des nombres premiers.
\end{enumerate}
\end{exer}

\begin{sol}
%%
\begin{enumerate}
\item Remarquons que pour tout entier \(n\) strictement positif :

\begin{equation}
n=\lfloor \frac{n}{\alpha} \rfloor \, \alpha + r_n ,
%%
\label{eq:div_alpha_n}
\end{equation}
%%
o\g{u} \(r_n\) est un entier compris entre \(0\) et \(\alpha-1\). %
On peut aussi \'ecrire :

\begin{equation}
\sharp A \cap [\![1,n]\!] = \lfloor \frac{n}{\alpha} \rfloor.
%%
\label{eq:card_n_sur_alpha}
\end{equation}
%%
En combinant~\rfeq{eq:div_alpha_n} et~\rfeq{eq:card_n_sur_alpha} on obtient :

\begin{equation}
\dfrac{\sharp A \cap [\![1,n]\!]}{n} = \dfrac{1}{\alpha}\left(1-\frac{r_n}{n}\right) .
%%
\label{eq:dens_n_rn}
\end{equation}
%%
Puisque la suite \((r_n)_n\) est born\'ee, le terme de droite de~\rfeq{eq:dens_n_rn} tend vers \(\frac{1}{\alpha}\) lorsque \(n\) tend vers \(+\infty\). %
Cette limite est la densit\'e naturelle de \(\alpha\mathbb{N}^{\ast}\), qui en particulier existe.
%%
\item \dots
%%
\item Remarquons que \(\mathbb{N}^{\ast}\) admet ue densit\'e naturelle, qui est \'egale \g{a} \(1\). %
Il existe donc une infinit\'e d'entiers naturels premiers entre eux deux \g{a} deux. %
%%
Comme les facteurs premiers de deux quelconques de ces entiers sont distincts, il existe donc une infinit\'e de nombres premiers.
\end{enumerate}
%%
\end{sol}

\begin{exer}[Le théorème d'infinitude de l'ensemble des nombres premiers revisité]
On définit un ensemble $\tau$ parties de $\mathbb{Z}$ par :
\[\forall P \in \mathbb{N} , P \in \tau \Leftrightarrow ( \forall m \in P , \exists a \in \mathbb{Z} | m + a \mathbb{Z} \subseteq P)\]
\begin{enumerate}
\item Montrer que $\tau$ est une topologie sur $\mathbb{Z}$.
\item Montrer que, si $a$ est un entier et $b$ un entier naturel non nul, alors : %
$a \mathbb{Z} + b$ est fermé dans $(\mathbb{Z} , \tau)$.
\item Montrer que l'ensemble des nombres premiers est infini.
\end{enumerate}
\end{exer}

\begin{exer}
Soitn $n$ un entier naturel non nul tel que la suite $(a_k)_k$ des entiers strictement positifs, %
inférieurs à $n$ et premiers à $n$ soit arithmétique.\\
Montrer que $n$ est une puissance de deux, ou un nombre premier impair.
\end{exer}

\begin{exer}
Démontrer qu'une fonction rationnelle complexe non constante omet au plus une valeur dans $\mathbb{C}$.
%Indication : On utilisera le théorème de D'Alembert.
\end{exer}

\begin{exer}
Soit $k$ un entier naturel supérieur ou égal à $2$.\\
Montrer que le produit de trois entiers naturels non nuls consécutifs n'est jamais une puissance $k-$ième.
\end{exer}

\begin{exer}
\begin{enumerate}
\item Montrer qu'il existe une infinité de nombres premiers congrus à $3$ modulo $4$.
\item Montrer qu'il existe une infinité de nombres premiers congrus à $5$ modulo $6$.
\end{enumerate}
\end{exer}

\begin{exer}
Soit $A$ un anneau commutatif.\\
On dit que $A$ est Noetherien si et seulement si tout idéal de $A$ est engendré map un nombre fini d'éléments de $A$.
\begin{enumerate}
\item Montrer que toute suite croissante d'idéaux d'un anneau Noetherien est stationnaire.
\item Montrer que, si $A$ est un anneau intègre et Noetherien, alors tout élément de $A$ est décomposable en produit de facteurs irréductibles.
\end{enumerate}
%Indications : (i) Raisonner par l'absurde.\\
%(ii) Soit $a$ un élément de $A$ non décomposable. Montrer que $a$ est le produit de deux éléments de $A$ qui ne sont pas des unités, et que l'un d'entre eux est non décomposable.\\
%(iii) Montrer que si un anneau est principal, alors toute suite croissante de ses idéaux est stationnaire.\\
%(iv) Considérer la suite des idéaux respectivement engendrés par les termes de la suite d'éléments indécomposables de $A$ définie aux questions (i) et (ii). Conclure.
\end{exer}

\begin{sol}
%%
\begin{enumerate}
\item Soit \((I_n)_{n\in\mathbb{N}}\) unes suite croissante d'id\'eaux de \(A\). %
%%
Montrons que \(\bigcup\limits_{i\in\mathbb{N}}\,I_n\), autrement not\'e \(I\), est un id\'eal de \(A\). %
Soient en effet \(x\) et \(y\) deux \'el\'ements de cete union. %
En particulier, il existe deux enters \(n_x\) et \(n_y\) tels que \(x\) et \(y\) appartiennent respectivement \g{a} %
\(I_{n_x}\) et \(I_{n_y}\). %
%%
Mais la famille \((I_n)_{n\in\mathbb{N}}\) est croissante : \(x\) et \(y\), et donc leur somme, appartiennent \g{a} l'id\'eal \(I_{\max{n_x, n_y}}\) de \(A\). %
%%
Enfin, si \(x\) appartient \g{a} \(I_n\) pour un certain \(n\) et si \(a\) est un \'el\'ement de \(A\), %
alors \(ax\) appartient \g{a} \(I_n\), donc \g{a} \(\bigcup\limits_{i\in\mathbb{N}}\,I_n\).

\par
On peut maintenant d\'emontrer le r\'esultat principal de cette section. %
En effet, \(A\) \'etant n\"otherien, \(I\) est engendr\'e par une famille finie \((x_i)_{i=1}^{n_I}\) de ses \'el\'ements. %
En particulier, chaque \(x_i\) appartient \g{a} au moins un terme \(I_{n_i}\) de la suite \(I_n\) %
- on peut par exemple prendre le premier terme de \((I_n)_n\) o\g{u} l'on reencontre \(x_i\). %
%%
Alors, tous les termes de la famille g\'en\'eratrice \((x_i)_i\) de \(I\) appartiennnent \g{a} \(I_{\max\limits_i{n_i}}\). %
Cet id\'eal est donc confondu avec la limite \(I\) de la suite, qui est donc stationnaire.

%%
\item Raisonnons par l'absurde et supposons qu'il existe un \'el\'ement \(a\) de \(A\) qui n'est pas d\'ecomposable en produit de facteurs irr\'eductibles. %
%%
Plus particuli\g{e}rement, \(a\) ne peut \^etre ni irr\'eductible, ni inversible. %
On peut donc \'ecrire

\begin{equation}
a = bc
%%
\label{eq:hyp_a_nondec}
\end{equation}
%%
o\g{u} \(b\) et \(c\) sont non inversibles. %
%%
Les id\'eaux principaux \(b\,A\) et \(c\,A\) incluent \(a\,A\) d'apr\g{e}s ces relations de divisibilit\'e et la commutativit\'e de \(A\). %
Supposons par exemple que l'inclusion r\'eciproque \(b\,A\subset a\,A\) soit v\'erifi\'ee. %
On peut en particulier \'ecrire :

\begin{equation}
b=au\quad \text{donc} \quad b=bcu
%%
\label{eq:}
\end{equation}
%%
Puisque \(A\) est int\g{e}gre, ceci implique que \(cu=1\), donc que \(c\) est inversible, absurde. %
Par ailleurs, l'un au moins des facteurs \(b\) et \(c\) est ind\'ecomposable en produit de facteurs irr\'eductibles de \(A\), %
car dans le cas contraire~\rfeq{eq:hyp_a_nondec} fournirait une d\'ecomposition de \(a\). %
%%
On a ainsi construit un \'el\'ement \(a_1\) de \(A\), \(b\) ou \(c\), %
ind\'ecomposable comme \(a\), et tel que

\begin{equation}
a\,A \subsetneq a_1\, A
%%
\label{eq:st_inc}
\end{equation}

En it\'erant cette construction, on obtient donc une suite strictement croissante d'id\'eaux de \(A\), %
ce qui est impossible d'apr\g{e}s la question pr\'ec\'edente.
%%
\end{enumerate}
%%
\end{sol}