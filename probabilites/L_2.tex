\section{$L^2$}

\begin{exer}
Soit $X$ une variable al\'eatoire $L^2$. Quel r\'eel minimise $c\mapsto\mathbb{E}((X-c)^2)$ ? Quel est le minimum atteint ?
\end{exer}

\begin{exer}
Soit $X$ une variable al\'eatoire $L^2$.\\
On suppose que : $V(X)=0$.\\
Montrer que $X$ prend la valeur $\mathbb{E}(X)$ avec une probabilit\'e $1$.
\end{exer}

\begin{exer}[L'in\'egalit\'e de Tchebycheff est-elle optimale ?]
\begin{enumerate}
\item (Oui) Montrer que, si $a$ est un r\'eel strictement positif, alors il existe une variable al\'eatoire $X$, $L^2$, telle que :\[\mathbb{P}(|X-\mathbb{E}(X)|\geq a)=\frac{V(X)}{a^2}\]
\item (Non) Soit $X$ une variable al\'eatoire $L^2$. Montrer que : \[a^2\mathbb{P}(|X-\mathbb{E}(X)|\geq a)\underset{a\rightarrow\infty}{\rightarrow}0\]
\end{enumerate}
\end{exer}