\documentclass[a4paper,10pt]{report}
\usepackage[utf8]{inputenc}

\usepackage[top=2cm,bottom=2cm,left=2cm,right=2cm]{geometry}% On peut changer en cours de route avec la commande \newgeometry

\usepackage[utf8]{inputenc}
\usepackage[french]{babel}
\usepackage{amsmath,amsfonts,amssymb,graphicx}
\usepackage[dvipsnames]{pstricks}
\usepackage{pstricks-add,pst-plot,pst-node}
%\usepackage{fp}
%\usepackage{multido}
%\usepackage{pdftricks}

%\usepackage{color} %(black, white, red, green, blue, yellow, magenta et cyan)

%\usepackage{amsthm}
\usepackage{framed}
\usepackage[amsthm,thmmarks,framed]{ntheorem}

\usepackage{multicol}
%\begin{multicols}[titre]{nb colonnes}
%\setlength{\columnseprule}{0.25pt}
\usepackage{array}

\usepackage{titletoc}
%\usepackage{hyperref} Probl\`emes avec la mise en page des exercices.

%\dottedcontents{section}[left]{above}{labelwidth}{leaderwidth = espacement entre les pointillés par exemple}
\dottedcontents{section}[0em]{\vspace{0.35cm}\bfseries}{3em}{0.75em}
\dottedcontents{subsection}[1.5em]{\vspace{0.15cm}}{4em}{0.75em}
\dottedcontents{subsubsection}[3em]{}{5em}{0.75em}

\setlength{\parindent}{0cm}

\newcommand{\g}[1]{\`#1}

\newcommand*{\etoile}
{
\begin{center}
\hspace{1pt}\par
*\hspace{5pt}*\hspace{5pt}*
\end{center}
}


\newcommand*{\ligneinter}
{
\begin{center}
\vspace{-2pt}
\hfill\rule{0.5\linewidth}{0.4pt}\hfill\null
\end{center}
\vspace{7pt}
}

% Je veux cr\'eer un environnement pour les énoncés préliminaires des exercices, afin d'éviter l'utilisation manuelle de \linebreak et, 
%pourquoi pas, éviter le bug de centrage systématique de l'en-tête.
  
{%
\theoremstyle{break}
\theoremprework{\vspace{0.5cm}\begin{minipage}{\textwidth}}
\theorempostwork{\etoile\end{minipage}}
\theoremheaderfont{\scshape}
\theorembodyfont{\normalfont}
\theoremseparator{ :\newline\vspace{0.2cm}}
\newtheorem{exer}{Exercice}[section]
}
  
{
\theoremstyle{break}
%\theoremprework{\vspace{0.2cm}}
\theorempostwork{\ligneinter}%\hfill\rule{0.6\linewidth}{0.4pt}\hfill\null}

\theoremheaderfont{\scshape}
\theorembodyfont{\normalfont}
\theoremseparator{ :\newline\vspace{0.2cm}}
\newtheorem{ques}{Question}[section] %Autre pr\'esentation : encadrer les \'enonc\'es ?
} 
  
%\setcounter{section}{-1}
\renewcommand{\thesection}{\Roman{section}}


%%% ------------------------------------------ %%%

\includeonly{%
questions_de_cours/questions_c, %
%%
algebre_generale/algebre_generale, %
alg_l_re/alg_l_re, %
alg_bi_fq/alg_bi_fq, %
geo_euclidienne/geo_euclidienne, %
topologie/topologie, %
suites_series_numeriques/suites_series_numeriques, %
analyse_fonctions/analy., %
edo/edo, %
calcul_differentiel/calcul_differentiel, %
% combinatoire/combinatoire, %
probabilites/probabilites, %
%%
retravailler/exxa_c, %
anc_programme/anc_programme, %
}

%%% ------------------------------------------ %%%

% Title Page
\title{Kh\^olles en MP-MP${}^{\ast}$ : sujets pos\'es aux Lazaristes et leurs corrig\'es, 2009-2016}
\author{Antoine Moreau}


\begin{document}

\maketitle

\begin{abstract}
Ce document rassemble les \'enonc\'es des exercices et questions de cours que j'ai r\'ealis\'es pour des kh\^olles aux Lazaristes, %
entre l'automne 2009 et le printemps 2016, %
ainsi que leurs solutions et r\'eponses. %
%Tous les \'enonc\'es, dont quelques-uns n'ont jamais \'et\'e pos\'es en kh\^olle, ont \'et\'e \'elabor\'es entre les automnes 2009 et 2015, et sont d\'ej\g{a} regroup\'es dans \cite{}.
Sauf mention du contraire, les solutions sont personnelles, par ailleurs la provenance de chaque \'enonc\'e est indiqu\'ee.
\end{abstract}

%%% ------------------------------------------ %%%

\tableofcontents

%%% ------------------------------------------ %%%

% \chapter{Questions de cours}
\chapter{Questions de cours}

\section{Enonc\'es seuls}

\begin{ques}
Enoncer le th\'eor\`eme de convergence domin\'ee de Lebesgue.
\end{ques}

\begin{ques}
Enoncer le th\'eor\`eme d'int\'egration terme \`a terme d'une s\'erie de fonctions.
\end{ques}

\begin{ques}
Enoncer et nommer le th\'eor\`eme de d\'erivation sous le signe $\int$.
\end{ques}

\begin{ques}
\begin{itemize}
\item Enoncer le th\'eor\g{e}me de comparaison des sommes partielles de deux s\'eries \g{a} termes positifs.
\item Enoncer le th\'eor\g{e}me de comparaison des restes de deux s\'eries \g{a} termes positifs.
\end{itemize}
\end{ques}

\begin{ques}
Enoncer et d\'emontrer le th\'eor\`eme de repr\'esentation des formes lin\'eaires d'un espace Euclidien.
\end{ques}

\begin{ques}
Que dire d'une suite $(u_n)_n$ telle que : $u_n = \underset{n \rightarrow + \infty}{o}(0)$ ?
\end{ques}
\section{Propositions \`a d\'emontrer}

\begin{ques}
Enoncer et d\'emontrer le th\'eor\g{e}me de d\'rivabilit\'e, et la formule de d\'erivation, d'une fonction r\'eciproque.
\end{ques}

\begin{ques}[Ancien programme]
Exprimer $\arg\cosh$ et $\arg\sinh$ \`a l'aide de la fonction logarithme et de fonctions alg\'ebriques.
\end{ques}

\begin{ques}
Enoncer et d\'emontrer le crit\`ere de convergence des int\'egrales de Bertrand.
\end{ques}

\begin{ques}
Enoncer et d\'emontrer le th\'eor\`eme de convergence des s\'eries de Bertrand.
\end{ques}

\begin{ques}
D\'ecomposer sur $\mathbb{C}$, $\frac{P'}{P}$ en \'el\'ements simples, o\`u $P$ est un polyn\^ome complexe non constant.
\end{ques}

\begin{ques}
Définir les polynômes de Tchebycheff, et établir une relation de récurrence entre les termes de la suite de ces polynômes. Calculer le polynôme de Tchebycheff de degré $5$.
\end{ques}

\begin{ques}
Montrer que, pour tout espace vectoriel de dimension finie $E$, %
toute famille $(u_i)$ d'endomorphismes de $E$ qui commutent deux \`a deux admet une base de diagonalisation commune dans $E$.
\end{ques}

\begin{ques}
Démontrer qu'une matrice r\'eelle sym\'etrique est diagonalisable \textit{id est} \'enoncer et d\'emontrer le th\'eor\g{e}me spectral.
\end{ques}

\begin{ques}
Calculer $\int\limits_{\mathbb{R}} e^{-\frac{x^2}{2}} dx$.
\end{ques}

\begin{ques}
Soient $r_1$ et $r_2$ deux rotations vectorielles de $\mathbb{R}^3$ qui ne sont pas des retournements. Montrer que ces rotations commutent si et seulement si elles ont le même axe.
\end{ques}

\begin{ques}[Ancien programme]
Donner une classification complète des quadriques vectorielles en dimension $3$.
\end{ques}

\begin{ques}
Classifier totalement les automorphismes de l'espace Euclidien $\mathbb{R}^3$.
\end{ques}
\input{questions_de_cours/primitives.tex}
\input{questions_de_cours/trigonometrie.tex}
\input{questions_de_cours/dev_se.tex}




%%% ------------------------------------------ %%%

\newcounter{stock}

% \chapter{Alg\g{e}bre g\'en\'erale}
\chapter{Alg\g{e}bre g\'en\'erale}

\section{Groupes, anneaux}

% \begin{equation}
% 
% %%
% \label{}
% \end{equation}

% \begin{sol}
% \begin{enumerate}
% \item %%
% %%
% \item %%
% \end{enumerate}
% \end{sol}

\begin{exer}
Soit $E$ un ensemble.

Montrer que $E$ est infini si et seulement si, toute bijection de $E$ sur lui-même stabilise au moins une partie stricte de $E$.
\end{exer}

\begin{exer}[Cas particulier du théorème de Cauchy]
Soient $p$ un nombre premier impair, et $G$ un groupe d'ordre $2p$.\\
Montrer que $G$ admet un élément d'ordre $p$.
%Indications : (i) Quels sont les ordres possibles des éléments de $G$ ?\\
%(ii) Un groupe dont tous les éléments sont d'ordre $2$ est abélien, et muni par l'exponentiation d'une structure d'espace vectoriel sur $\mathbb{Z} / 2 \mathbb{Z}$.\\
%Conclure à l'aide d'un argument de dimension.
\end{exer}

\begin{exer}
Soit $(G,.)$ un groupe abélien fini, noté multiplicativement. Pour tout élément $x$ de $G$, on note $O(x)$ l'ordre de $x$.
\begin{enumerate}
\item Soient $x$ et $y$ deux éléments de $G$ tels que $O(x)$ et $O(y)$ soient premiers entre eux. Déterminer l'ordre de $xy$.
\item On suppose ici $x$ et $y$ quelconques. Montrer qu'il existe un élément $z$ de $G$ tel que : $O(z) = O(x) \vee O(y)$.
\item\label{ques:ord_ppcm} En déduire l'existence d'un élément de $G$ dont l'ordre $m$ est le ppcm des ordres des éléments de $G$. %
$m$ est appelé exposant de $G$.
\item Supposons maintenant :\[\forall d \in \mathbb{N}^{\ast} , \lvert \{ x \in G / x^d = 1 \} \rvert \leqslant d\]
Montrer que $G$ est cyclique.
\item Soit $\mathbb{K}$ un corps commutatif. Montrer que %
tout sous-groupe fini du groupe multiplicatif de $\mathbb{K}$ est cyclique.
\end{enumerate}
\end{exer}

\begin{sol}
\begin{enumerate}
\item Supposons :

\begin{equation}
(xy)^k=1
%%
\label{}
\end{equation}
%%
pour un certain entier \(k\). %
La commutativit\'e de l'anneau \(A\) permet d'\'ecrire :

\begin{equation}
(xy)^k = x^k\,y^k
%%
\label{}
\end{equation}

Ainsi, \(x^k\) est une puissance de \(y\), ce qui entra\^ine :

\begin{equation}
(x^k)^{O(y)} = 1
%%
\label{eq:xk_oy}
\end{equation}

Mais \(x^k\) est aussi une puissance de \(x\), donc :

\begin{equation}
(x^k)^{O(x)} = 1
%%
\label{eq:xk_ox}
\end{equation}

Or \(O(x)\) et \(O(y)\) sont premiers entre eux : il existe deux entiers \(u\)et \(v\) tels que

\begin{equation}
O(x) u + O(y) v = 1
%%
\label{eq:bezout_oxy}
\end{equation}

Les relations~\rfeq{eq:bezout_oxy}, \rfeq{eq:xk_ox} et~\rfeq{eq:xk_oy} entra\^inent alors :

\begin{equation}
(x^k)^{O(x) u + O(y) v} = x^k \quad \text{donc} \quad x^k = 1
%%
\label{}
\end{equation}

On en d\'eduit que \(y^k=1\). %
Il s'ensuit que \(k\) est divisible par \(O(x)\) et par \(O(y)\), donc par leur produit car ces entiers sont premiers entre eux \textbf{raccourci possible ?}. %
R\'eciproquement, il est facile de voir que \((xy)^{O(x)\,O(y)}=1\) : %
l'ordre de \(xy\) est donc exactement \(O(x)\,O(y)\).

%%
\item Soient \(x\) et \(y\) deux \'el\'ements de \(G\), d'ordres respectifs \(m\) et \(n\). %
On peut \'ecrire :

\begin{equation}
m = dm' \quad\text{et}\quad n=dn'\, ,
%%
\label{}
\end{equation}
%%
o\g{u} \(d=m\wedge n\), %





%%
\item\label{seq_Aa_oppcm} On se va construire une suite \((A_k, a_k)_k\), o\g{u} \((A_k)_k\) est une suite strictement croissante de parties de \(G\), %
\(a_k\in A_k\) pour tout \(k\) et \(O(a_k)=\vee_{x\in A_k} \, O(x)\). %
Le dernier terme de cette suite founrira la solution.

\begin{itemize}
%%
\item %
On pose : \(A_1=\{1\}\) et \(a_1=1\), %
ce couple \((A_1, a_1)\) v\'erifie les propri\'et\'es voulues pour initialiser la suite.

\item Supposons la suite construite jusqu'\g{a} un rang \(k\). %
Alors deux possibilit\'es existent :

\begin{description}
%%
\item [\(A_k=G\) :] la construction d'arr\^ete ;
%%
\item [\(A_k \nsubseteq G\) :] on choisit un \'el\'ement \(b\) dans l'ensemble fini \(G \setminus A_k\). %
% Si 
Alors \(a_k \, b\) est d'ordre \(\left(\vee_{x\in A_k} \, O(x)\right) \vee O(b)\) d'apr\g{e}s la question pr\'ec\'edente. %\(\)
Comme le ppcm est associatif - il correspond \g{a} une intersectin d'id\'eaux de \(\mathbb{Z}\), %
on peut encore \'ecrire :

\begin{equation}
O(a_k\, b) = \vee_{x\in A_k \cup \{b\}} \, O(x).
%%
\label{}
\end{equation}
%%
On peut encore \'ecrire, pour la m\^eme raison et par idempotence du ppcm :

\begin{equation}
O(a_k\, b) = \vee_{x\in A_k \cup \{b, a_k\,b\}} \, O(x).
%%
\label{}
\end{equation}

Ainsi, en posant \(A_{k+1} = A_k \cup \{b, a_k\,b\}\) et \(a_{k+1}=a_k\, b\), %
on obtient le terme de rang \(k+1\) de la suite.
%%
\end{description}

Cet algorithme s'arr\^ete car \((A_k)_k\) est une suite strictement croissante de parties de l'ensemble fini \(G\), %
on peut aussi invoquer la stricte croissance et la bornitude de \(\sharp (A_k)_k\). %
%%
Le couple de rang maximal \((A_{k_{\max}}, a_{k_{\max}})\) nous donne la r\'eponse \g{a} la question.
%%
\end{itemize}

\item Soit \(m\) le ppcm des ordres des \'el\'ements de \(G\). %\(\mathbb{K}^{\times}\)
% et \(a\) un \'el\'ement de \(G\) d'ordre \(m\).
On note que tout \'e\'ement \(x\) de \(G\) est racine du polyn\^ome

\begin{equation}
X^m - 1
%%
\label{}
\end{equation}
%%
de \(\mathbb{K}[X]\). %
Comme ce polyn\^ome admet au plus \(m\) racines distinctes, l'ordre de \(G\) est inf\'erieur ou \'egal \g{a} \(m\). %
%%
Puisque \(m\) est l'ordre d'un \'el\'ement \(a\) de \(G\) d'apr\g{e}s la question pr\'ec\'edente, %
le th\'eor\g{e}me de Lagrange prouve que \(m\) divise \(\sharp G\) ; %
\(G\) est donc d'ordre \(m\). %
%%
L'ordre de \(a\) \'etant aussi celui de \(G\), ce groupe est cyclique et \(a\) en est un g\'en\'erateur.
%%
\end{enumerate}
\end{sol}

\begin{rema}
Nous verrons au chapitre~\ref{} un r\'esulat analogue \g{a} celui d\'emontr\'e \g{a} la question~\ref{ques:ord_ppcm} de cet exercice : %
le th\'eor\g{e}me de Caley-Hamilton.
\end{rema}

\begin{rema}
%%
En appliquant le r\'esultat pr\'ec\'edent au corps fini \(\mathbb{Z} / p\mathbb{Z}\), on \'etablit que le groupe multiplicatif \(\mathbb{Z} / p\mathbb{Z}^{\times}\) est cyclique, %
autrement dit qu'il existe une racine primitive modulo \(p\), pour tout nombre premier \(p\).
%%
\end{rema}

\begin{rema}
%%
La question~\ref{seq_Aa_oppcm} de l'exercice fournit une construction de racine primitive modulo \(p\). %
Toutefois, celle-ci n'est a priori pas plus efficace qu'une recherche exhaustive, on sait seulement qu'elle ach\g{e}ve.
%%
\end{rema}

\begin{exer}[Tout anneau intègre et fini est un corps]%Questions intermédiaires : pas d'obligation.
%Je pose tout d'abord le problème directement à l'élève, et je le laisse réfléchir quelques minutes sans indication de solution. Je continue l'interrogation en fonction de ses réactions, en posant éventuellement les questions intermédiaires qui suivent.\\
Soit en effet $(A,+,\times)$ un anneau, non nécessairement supposé unitaire, intègre et fini. %
Pour tout élément $a$ de $A \setminus \{0\}$, on note $m_a$ l'endomorphisme de multiplication à gauche %
$x \mapsto a \times x$ du groupe $(A,+)$ -pourquoi est-ce un endomorphisme ?
\begin{enumerate}
\item Montrer que les termes de la famille $(m_a)$ sont des automorphismes de $(A,+)$.
\item Montrer que $a \mapsto m_a$ induit un morphisme injectif de la structure algébrique associative %
$(A \setminus \{0\}, \times)$ dans le groupe des automorphismes de $(A,+)$.
\item Démontrer le lemme suivant :\textit{Soient $(G,\ast)$ un groupe et $F$ %
une partie finie de $G$ stable par $\ast$. Alors $(F,\ast)$ est un groupe.}
\item Conclure.
\end{enumerate}
\end{exer}
\section{Arithmétique}

% \begin{equation}
% 
% %%
% \label{}
% \end{equation}

\begin{exer}
Montrer que le $\mathbb{C}$-espace vectoriel $\mathbb{C} (X)$ est de dimension non dénombrable.

On considérera la famille $(\frac{1}{X-\lambda})_{\lambda \in \mathbb{C}}$.
\end{exer}

Remarquons que $\mathbb{C} (X)$, qui admet $\mathbb{C} [X]$ comme sous-espace vectoriel, %
n'est pas de dimension finie. %
%%
Deux solutions sont possible pour cet exercice, %
selon le sens accord\'e \g{a} l'expression \og{} dimension non dénombrable \fg{} . %
%%
Elles reposent toutes les deux sur l'unicit\'e de la d\'ecomposition en \'el\'ements simples %
d'une fraction rationnelle.

\begin{sol}[Existence d'une famille libre non d\'enombrable]
Pour r\'epondre \g{a} la question, il suffit de montrer que la famille \((\frac{1}{X-\lambda})_{\lambda \in \mathbb{C}}\), %
qui est non d\'enombrable, est libre. %

\par
Une combinaison lin\'eaire de termes de cette famille s'\'ecrit :

\begin{equation}
\sum\limits_{i=1}^n\, \frac{\alpha_i}{X-\lambda}
%%
\label{eq:def_cl_frac}
\end{equation}

\((\alpha_i)_{i=1}^n\) \'etant une famille finie de nombres complexes. %
Supposons que \(n>1\) et que tous les coefficients \(\alpha_i\) soient non nuls, %
et que la fraction rationnelle d\'efinie par~\rfeq{eq:def_cl_frac} soit nulle. %
%%
L'\'egalit\'e

\[\sum\limits_{i=1}^n\, \frac{\alpha_i}{X-\lambda}=0,\]
%%
est la d\'ecomposition en \'el\'ements simples de la fraction nulle, ce qui est absurde car celle-ci n'a pas de p\^ole. %
%%
Conclusion : il n'existe pas de combinaison lin\'eaire, \g{a} coefficients non nuls, des termes de \((\frac{1}{X-\lambda})_{\lambda \in \mathbb{C}}\), %
dont la valeur soit \(0\). %
Cette famille est donc libre.
\end{sol}

\begin{sol}[Non existence d'une famille d\'enombrable g\'en\'eratrice]
Le cas d'une famille g\'en\'eratrice finie est d\'ej\g{a} trait\'e, car un espace vectoriel de dimension finie ne peut avoir %
$\mathbb{C} [X]$ comme sous-espace vectoriel.

\par
Soit \((F_i(X))_{i\in\mathbb{N}}\) une suite de frations rationnelles. %
Pour tout \(i\), \(F_i(X)\) admet une suite finie de p\^oles \((\lambda_j)_{j=1}^{n_i}\). %
%%
L'ensemble

\begin{equation}
\bigcup\limits_{i\in\mathbb{N}}\,\text{Im}((\lambda_j)_{j=1}^{n_i})
%%
\label{eq:def_poles_ij}
\end{equation}
%%
de tous les p\^oles des termes de \((F_i(X))\) est une union finie d'ensembles d\'enombrables, %
il est donc d\'enombrable - on peut r\'eindexer les p\^oles qui apparaissent dans l'expression~\rfeq{eq:def_poles_ij} %
pour obtenir une suite index\'ee dans \(\mathbb{N}\). %
%%
Puisque \(\mathbb{C}\), comme \(\mathbb{R}\)\footnote{Voir la section concernant le espaces complets, dans ancien programme.}%
est non d\'enombrable, il existe une nombre complexe \(\lambda_{\infty}\) qui n'appartient pas \g{a} l'ensemble des p\^oles des fractions \(F_i(X)\). %
%%
La fraction

\[\frac{1}{X-\lambda_{\infty}},\]
%%
n'est pas une combinaison lin\'eaire de la famille \((F_i(X))_{i\in\mathbb{N}}\), %
car de telles combinaisons ont des p\^oles distincts de \(\lambda_{\infty}\). %
%%
Ainsi \((F_i(X))\) n'est-elle pas g\'en\'eratrice de \(\mathbb{C}(X)\).
\end{sol}

% Théorème de Mason, théorème de Liouville.

\begin{exer}[Densité naturelle et diviseurs communs]
Soit $A$ une partie de $\mathbb{N}$.\\
On dit que $A$ admet une densité naturelle, si et seulement si la suite :
\[\left(\frac{\sharp A \cap [\![1,n]\!]}{n}\right)_{n \in \mathbb{N}^{\ast}}\]
converge. On appelle densité de $A$, et note $d(A)$, cette limite lorsqu'elle existe.
\begin{enumerate}
\item Soit $\alpha$ un entier strictement positif. Montrer que $\alpha \mathbb{N}$ admet une densité naturelle, %
calculer cette densité.
\item Montrer que si $A$ est une partie de $\mathbb{N}$ qui admet une densité naturelle égale à $1$, %
alors $A$ contient une infinité d'entiers premiers entre eux deux-à-deux.
%Raisonner par l'absurde.
%Lemme important : il existe au moins une famille (x_i) d'entiers de $A$, premiers entre eux deux à deux, maximale parmi les familles d'entiers de A qui ont cette propriété.
\item Déduire de la question précédente un résultat classique en arithmétique.%Il s'agit du théorème d'infinitude de l'ensemble des nombres premiers.
\end{enumerate}
\end{exer}

\begin{sol}
%%
\begin{enumerate}
\item Remarquons que pour tout entier \(n\) strictement positif :

\begin{equation}
n=\lfloor \frac{n}{\alpha} \rfloor \, \alpha + r_n ,
%%
\label{eq:div_alpha_n}
\end{equation}
%%
o\g{u} \(r_n\) est un entier compris entre \(0\) et \(\alpha-1\). %
On peut aussi \'ecrire :

\begin{equation}
\sharp A \cap [\![1,n]\!] = \lfloor \frac{n}{\alpha} \rfloor.
%%
\label{eq:card_n_sur_alpha}
\end{equation}
%%
En combinant~\rfeq{eq:div_alpha_n} et~\rfeq{eq:card_n_sur_alpha} on obtient :

\begin{equation}
\dfrac{\sharp A \cap [\![1,n]\!]}{n} = \dfrac{1}{\alpha}\left(1-\frac{r_n}{n}\right) .
%%
\label{eq:dens_n_rn}
\end{equation}
%%
Puisque la suite \((r_n)_n\) est born\'ee, le terme de droite de~\rfeq{eq:dens_n_rn} tend vers \(\frac{1}{\alpha}\) lorsque \(n\) tend vers \(+\infty\). %
Cette limite est la densit\'e naturelle de \(\alpha\mathbb{N}^{\ast}\), qui en particulier existe.
%%
\item \dots
%%
\item Remarquons que \(\mathbb{N}^{\ast}\) admet ue densit\'e naturelle, qui est \'egale \g{a} \(1\). %
Il existe donc une infinit\'e d'entiers naturels premiers entre eux deux \g{a} deux. %
%%
Comme les facteurs premiers de deux quelconques de ces entiers sont distincts, il existe donc une infinit\'e de nombres premiers.
\end{enumerate}
%%
\end{sol}

\begin{exer}[Le théorème d'infinitude de l'ensemble des nombres premiers revisité]
On définit un ensemble $\tau$ parties de $\mathbb{Z}$ par :
\[\forall P \in \mathbb{N} , P \in \tau \Leftrightarrow ( \forall m \in P , \exists a \in \mathbb{Z} | m + a \mathbb{Z} \subseteq P)\]
\begin{enumerate}
\item Montrer que $\tau$ est une topologie sur $\mathbb{Z}$.
\item Montrer que, si $a$ est un entier et $b$ un entier naturel non nul, alors : %
$a \mathbb{Z} + b$ est fermé dans $(\mathbb{Z} , \tau)$.
\item Montrer que l'ensemble des nombres premiers est infini.
\end{enumerate}
\end{exer}

\begin{exer}
Soitn $n$ un entier naturel non nul tel que la suite $(a_k)_k$ des entiers strictement positifs, %
inférieurs à $n$ et premiers à $n$ soit arithmétique.\\
Montrer que $n$ est une puissance de deux, ou un nombre premier impair.
\end{exer}

\begin{exer}
Démontrer qu'une fonction rationnelle complexe non constante omet au plus une valeur dans $\mathbb{C}$.
%Indication : On utilisera le théorème de D'Alembert.
\end{exer}

\begin{exer}
Soit $k$ un entier naturel supérieur ou égal à $2$.\\
Montrer que le produit de trois entiers naturels non nuls consécutifs n'est jamais une puissance $k-$ième.
\end{exer}

\begin{exer}
\begin{enumerate}
\item Montrer qu'il existe une infinité de nombres premiers congrus à $3$ modulo $4$.
\item Montrer qu'il existe une infinité de nombres premiers congrus à $5$ modulo $6$.
\end{enumerate}
\end{exer}

\begin{exer}
Soit $A$ un anneau commutatif.\\
On dit que $A$ est Noetherien si et seulement si tout idéal de $A$ est engendré map un nombre fini d'éléments de $A$.
\begin{enumerate}
\item Montrer que toute suite croissante d'idéaux d'un anneau Noetherien est stationnaire.
\item Montrer que, si $A$ est un anneau intègre et Noetherien, alors tout élément de $A$ est décomposable en produit de facteurs irréductibles.
\end{enumerate}
%Indications : (i) Raisonner par l'absurde.\\
%(ii) Soit $a$ un élément de $A$ non décomposable. Montrer que $a$ est le produit de deux éléments de $A$ qui ne sont pas des unités, et que l'un d'entre eux est non décomposable.\\
%(iii) Montrer que si un anneau est principal, alors toute suite croissante de ses idéaux est stationnaire.\\
%(iv) Considérer la suite des idéaux respectivement engendrés par les termes de la suite d'éléments indécomposables de $A$ définie aux questions (i) et (ii). Conclure.
\end{exer}

\begin{sol}
%%
\begin{enumerate}
\item Soit \((I_n)_{n\in\mathbb{N}}\) unes suite croissante d'id\'eaux de \(A\). %
%%
Montrons que \(\bigcup\limits_{i\in\mathbb{N}}\,I_n\), autrement not\'e \(I\), est un id\'eal de \(A\). %
Soient en effet \(x\) et \(y\) deux \'el\'ements de cete union. %
En particulier, il existe deux enters \(n_x\) et \(n_y\) tels que \(x\) et \(y\) appartiennent respectivement \g{a} %
\(I_{n_x}\) et \(I_{n_y}\). %
%%
Mais la famille \((I_n)_{n\in\mathbb{N}}\) est croissante : \(x\) et \(y\), et donc leur somme, appartiennent \g{a} l'id\'eal \(I_{\max{n_x, n_y}}\) de \(A\). %
%%
Enfin, si \(x\) appartient \g{a} \(I_n\) pour un certain \(n\) et si \(a\) est un \'el\'ement de \(A\), %
alors \(ax\) appartient \g{a} \(I_n\), donc \g{a} \(\bigcup\limits_{i\in\mathbb{N}}\,I_n\).

\par
On peut maintenant d\'emontrer le r\'esultat principal de cette section. %
En effet, \(A\) \'etant n\"otherien, \(I\) est engendr\'e par une famille finie \((x_i)_{i=1}^{n_I}\) de ses \'el\'ements. %
En particulier, chaque \(x_i\) appartient \g{a} au moins un terme \(I_{n_i}\) de la suite \(I_n\) %
- on peut par exemple prendre le premier terme de \((I_n)_n\) o\g{u} l'on reencontre \(x_i\). %
%%
Alors, tous les termes de la famille g\'en\'eratrice \((x_i)_i\) de \(I\) appartiennnent \g{a} \(I_{\max\limits_i{n_i}}\). %
Cet id\'eal est donc confondu avec la limite \(I\) de la suite, qui est donc stationnaire.

%%
\item Raisonnons par l'absurde et supposons qu'il existe un \'el\'ement \(a\) de \(A\) qui n'est pas d\'ecomposable en produit de facteurs irr\'eductibles. %
%%
Plus particuli\g{e}rement, \(a\) ne peut \^etre ni irr\'eductible, ni inversible. %
On peut donc \'ecrire

\begin{equation}
a = bc
%%
\label{eq:hyp_a_nondec}
\end{equation}
%%
o\g{u} \(b\) et \(c\) sont non inversibles. %
%%
Les id\'eaux principaux \(b\,A\) et \(c\,A\) incluent \(a\,A\) d'apr\g{e}s ces relations de divisibilit\'e et la commutativit\'e de \(A\). %
Supposons par exemple que l'inclusion r\'eciproque \(b\,A\subset a\,A\) soit v\'erifi\'ee. %
On peut en particulier \'ecrire :

\begin{equation}
b=au\quad \text{donc} \quad b=bcu
%%
\label{eq:}
\end{equation}
%%
Puisque \(A\) est int\g{e}gre, ceci implique que \(cu=1\), donc que \(c\) est inversible, absurde. %
Par ailleurs, l'un au moins des facteurs \(b\) et \(c\) est ind\'ecomposable en produit de facteurs irr\'eductibles de \(A\), %
car dans le cas contraire~\rfeq{eq:hyp_a_nondec} fournirait une d\'ecomposition de \(a\). %
%%
On a ainsi construit un \'el\'ement \(a_1\) de \(A\), \(b\) ou \(c\), %
ind\'ecomposable comme \(a\), et tel que

\begin{equation}
a\,A \subsetneq a_1\, A
%%
\label{eq:st_inc}
\end{equation}

En it\'erant cette construction, on obtient donc une suite strictement croissante d'id\'eaux de \(A\), %
ce qui est impossible d'apr\g{e}s la question pr\'ec\'edente.
%%
\end{enumerate}
%%
\end{sol}
\input{algebre_generale/series_formelles_1d.tex}


% \section{Groupes, anneaux}

% \begin{equation}
% 
% %%
% \label{}
% \end{equation}

% \begin{sol}
% \begin{enumerate}
% \item %%
% %%
% \item %%
% \end{enumerate}
% \end{sol}

\begin{exer}
Soit $E$ un ensemble.

Montrer que $E$ est infini si et seulement si, toute bijection de $E$ sur lui-même stabilise au moins une partie stricte de $E$.
\end{exer}

\begin{exer}[Cas particulier du théorème de Cauchy]
Soient $p$ un nombre premier impair, et $G$ un groupe d'ordre $2p$.\\
Montrer que $G$ admet un élément d'ordre $p$.
%Indications : (i) Quels sont les ordres possibles des éléments de $G$ ?\\
%(ii) Un groupe dont tous les éléments sont d'ordre $2$ est abélien, et muni par l'exponentiation d'une structure d'espace vectoriel sur $\mathbb{Z} / 2 \mathbb{Z}$.\\
%Conclure à l'aide d'un argument de dimension.
\end{exer}

\begin{exer}
Soit $(G,.)$ un groupe abélien fini, noté multiplicativement. Pour tout élément $x$ de $G$, on note $O(x)$ l'ordre de $x$.
\begin{enumerate}
\item Soient $x$ et $y$ deux éléments de $G$ tels que $O(x)$ et $O(y)$ soient premiers entre eux. Déterminer l'ordre de $xy$.
\item On suppose ici $x$ et $y$ quelconques. Montrer qu'il existe un élément $z$ de $G$ tel que : $O(z) = O(x) \vee O(y)$.
\item\label{ques:ord_ppcm} En déduire l'existence d'un élément de $G$ dont l'ordre $m$ est le ppcm des ordres des éléments de $G$. %
$m$ est appelé exposant de $G$.
\item Supposons maintenant :\[\forall d \in \mathbb{N}^{\ast} , \lvert \{ x \in G / x^d = 1 \} \rvert \leqslant d\]
Montrer que $G$ est cyclique.
\item Soit $\mathbb{K}$ un corps commutatif. Montrer que %
tout sous-groupe fini du groupe multiplicatif de $\mathbb{K}$ est cyclique.
\end{enumerate}
\end{exer}

\begin{sol}
\begin{enumerate}
\item Supposons :

\begin{equation}
(xy)^k=1
%%
\label{}
\end{equation}
%%
pour un certain entier \(k\). %
La commutativit\'e de l'anneau \(A\) permet d'\'ecrire :

\begin{equation}
(xy)^k = x^k\,y^k
%%
\label{}
\end{equation}

Ainsi, \(x^k\) est une puissance de \(y\), ce qui entra\^ine :

\begin{equation}
(x^k)^{O(y)} = 1
%%
\label{eq:xk_oy}
\end{equation}

Mais \(x^k\) est aussi une puissance de \(x\), donc :

\begin{equation}
(x^k)^{O(x)} = 1
%%
\label{eq:xk_ox}
\end{equation}

Or \(O(x)\) et \(O(y)\) sont premiers entre eux : il existe deux entiers \(u\)et \(v\) tels que

\begin{equation}
O(x) u + O(y) v = 1
%%
\label{eq:bezout_oxy}
\end{equation}

Les relations~\rfeq{eq:bezout_oxy}, \rfeq{eq:xk_ox} et~\rfeq{eq:xk_oy} entra\^inent alors :

\begin{equation}
(x^k)^{O(x) u + O(y) v} = x^k \quad \text{donc} \quad x^k = 1
%%
\label{}
\end{equation}

On en d\'eduit que \(y^k=1\). %
Il s'ensuit que \(k\) est divisible par \(O(x)\) et par \(O(y)\), donc par leur produit car ces entiers sont premiers entre eux \textbf{raccourci possible ?}. %
R\'eciproquement, il est facile de voir que \((xy)^{O(x)\,O(y)}=1\) : %
l'ordre de \(xy\) est donc exactement \(O(x)\,O(y)\).

%%
\item Soient \(x\) et \(y\) deux \'el\'ements de \(G\), d'ordres respectifs \(m\) et \(n\). %
On peut \'ecrire :

\begin{equation}
m = dm' \quad\text{et}\quad n=dn'\, ,
%%
\label{}
\end{equation}
%%
o\g{u} \(d=m\wedge n\), %





%%
\item\label{seq_Aa_oppcm} On se va construire une suite \((A_k, a_k)_k\), o\g{u} \((A_k)_k\) est une suite strictement croissante de parties de \(G\), %
\(a_k\in A_k\) pour tout \(k\) et \(O(a_k)=\vee_{x\in A_k} \, O(x)\). %
Le dernier terme de cette suite founrira la solution.

\begin{itemize}
%%
\item %
On pose : \(A_1=\{1\}\) et \(a_1=1\), %
ce couple \((A_1, a_1)\) v\'erifie les propri\'et\'es voulues pour initialiser la suite.

\item Supposons la suite construite jusqu'\g{a} un rang \(k\). %
Alors deux possibilit\'es existent :

\begin{description}
%%
\item [\(A_k=G\) :] la construction d'arr\^ete ;
%%
\item [\(A_k \nsubseteq G\) :] on choisit un \'el\'ement \(b\) dans l'ensemble fini \(G \setminus A_k\). %
% Si 
Alors \(a_k \, b\) est d'ordre \(\left(\vee_{x\in A_k} \, O(x)\right) \vee O(b)\) d'apr\g{e}s la question pr\'ec\'edente. %\(\)
Comme le ppcm est associatif - il correspond \g{a} une intersectin d'id\'eaux de \(\mathbb{Z}\), %
on peut encore \'ecrire :

\begin{equation}
O(a_k\, b) = \vee_{x\in A_k \cup \{b\}} \, O(x).
%%
\label{}
\end{equation}
%%
On peut encore \'ecrire, pour la m\^eme raison et par idempotence du ppcm :

\begin{equation}
O(a_k\, b) = \vee_{x\in A_k \cup \{b, a_k\,b\}} \, O(x).
%%
\label{}
\end{equation}

Ainsi, en posant \(A_{k+1} = A_k \cup \{b, a_k\,b\}\) et \(a_{k+1}=a_k\, b\), %
on obtient le terme de rang \(k+1\) de la suite.
%%
\end{description}

Cet algorithme s'arr\^ete car \((A_k)_k\) est une suite strictement croissante de parties de l'ensemble fini \(G\), %
on peut aussi invoquer la stricte croissance et la bornitude de \(\sharp (A_k)_k\). %
%%
Le couple de rang maximal \((A_{k_{\max}}, a_{k_{\max}})\) nous donne la r\'eponse \g{a} la question.
%%
\end{itemize}

\item Soit \(m\) le ppcm des ordres des \'el\'ements de \(G\). %\(\mathbb{K}^{\times}\)
% et \(a\) un \'el\'ement de \(G\) d'ordre \(m\).
On note que tout \'e\'ement \(x\) de \(G\) est racine du polyn\^ome

\begin{equation}
X^m - 1
%%
\label{}
\end{equation}
%%
de \(\mathbb{K}[X]\). %
Comme ce polyn\^ome admet au plus \(m\) racines distinctes, l'ordre de \(G\) est inf\'erieur ou \'egal \g{a} \(m\). %
%%
Puisque \(m\) est l'ordre d'un \'el\'ement \(a\) de \(G\) d'apr\g{e}s la question pr\'ec\'edente, %
le th\'eor\g{e}me de Lagrange prouve que \(m\) divise \(\sharp G\) ; %
\(G\) est donc d'ordre \(m\). %
%%
L'ordre de \(a\) \'etant aussi celui de \(G\), ce groupe est cyclique et \(a\) en est un g\'en\'erateur.
%%
\end{enumerate}
\end{sol}

\begin{rema}
Nous verrons au chapitre~\ref{} un r\'esulat analogue \g{a} celui d\'emontr\'e \g{a} la question~\ref{ques:ord_ppcm} de cet exercice : %
le th\'eor\g{e}me de Caley-Hamilton.
\end{rema}

\begin{rema}
%%
En appliquant le r\'esultat pr\'ec\'edent au corps fini \(\mathbb{Z} / p\mathbb{Z}\), on \'etablit que le groupe multiplicatif \(\mathbb{Z} / p\mathbb{Z}^{\times}\) est cyclique, %
autrement dit qu'il existe une racine primitive modulo \(p\), pour tout nombre premier \(p\).
%%
\end{rema}

\begin{rema}
%%
La question~\ref{seq_Aa_oppcm} de l'exercice fournit une construction de racine primitive modulo \(p\). %
Toutefois, celle-ci n'est a priori pas plus efficace qu'une recherche exhaustive, on sait seulement qu'elle ach\g{e}ve.
%%
\end{rema}

\begin{exer}[Tout anneau intègre et fini est un corps]%Questions intermédiaires : pas d'obligation.
%Je pose tout d'abord le problème directement à l'élève, et je le laisse réfléchir quelques minutes sans indication de solution. Je continue l'interrogation en fonction de ses réactions, en posant éventuellement les questions intermédiaires qui suivent.\\
Soit en effet $(A,+,\times)$ un anneau, non nécessairement supposé unitaire, intègre et fini. %
Pour tout élément $a$ de $A \setminus \{0\}$, on note $m_a$ l'endomorphisme de multiplication à gauche %
$x \mapsto a \times x$ du groupe $(A,+)$ -pourquoi est-ce un endomorphisme ?
\begin{enumerate}
\item Montrer que les termes de la famille $(m_a)$ sont des automorphismes de $(A,+)$.
\item Montrer que $a \mapsto m_a$ induit un morphisme injectif de la structure algébrique associative %
$(A \setminus \{0\}, \times)$ dans le groupe des automorphismes de $(A,+)$.
\item Démontrer le lemme suivant :\textit{Soient $(G,\ast)$ un groupe et $F$ %
une partie finie de $G$ stable par $\ast$. Alors $(F,\ast)$ est un groupe.}
\item Conclure.
\end{enumerate}
\end{exer}
% \section{Arithmétique}

% \begin{equation}
% 
% %%
% \label{}
% \end{equation}

\begin{exer}
Montrer que le $\mathbb{C}$-espace vectoriel $\mathbb{C} (X)$ est de dimension non dénombrable.

On considérera la famille $(\frac{1}{X-\lambda})_{\lambda \in \mathbb{C}}$.
\end{exer}

Remarquons que $\mathbb{C} (X)$, qui admet $\mathbb{C} [X]$ comme sous-espace vectoriel, %
n'est pas de dimension finie. %
%%
Deux solutions sont possible pour cet exercice, %
selon le sens accord\'e \g{a} l'expression \og{} dimension non dénombrable \fg{} . %
%%
Elles reposent toutes les deux sur l'unicit\'e de la d\'ecomposition en \'el\'ements simples %
d'une fraction rationnelle.

\begin{sol}[Existence d'une famille libre non d\'enombrable]
Pour r\'epondre \g{a} la question, il suffit de montrer que la famille \((\frac{1}{X-\lambda})_{\lambda \in \mathbb{C}}\), %
qui est non d\'enombrable, est libre. %

\par
Une combinaison lin\'eaire de termes de cette famille s'\'ecrit :

\begin{equation}
\sum\limits_{i=1}^n\, \frac{\alpha_i}{X-\lambda}
%%
\label{eq:def_cl_frac}
\end{equation}

\((\alpha_i)_{i=1}^n\) \'etant une famille finie de nombres complexes. %
Supposons que \(n>1\) et que tous les coefficients \(\alpha_i\) soient non nuls, %
et que la fraction rationnelle d\'efinie par~\rfeq{eq:def_cl_frac} soit nulle. %
%%
L'\'egalit\'e

\[\sum\limits_{i=1}^n\, \frac{\alpha_i}{X-\lambda}=0,\]
%%
est la d\'ecomposition en \'el\'ements simples de la fraction nulle, ce qui est absurde car celle-ci n'a pas de p\^ole. %
%%
Conclusion : il n'existe pas de combinaison lin\'eaire, \g{a} coefficients non nuls, des termes de \((\frac{1}{X-\lambda})_{\lambda \in \mathbb{C}}\), %
dont la valeur soit \(0\). %
Cette famille est donc libre.
\end{sol}

\begin{sol}[Non existence d'une famille d\'enombrable g\'en\'eratrice]
Le cas d'une famille g\'en\'eratrice finie est d\'ej\g{a} trait\'e, car un espace vectoriel de dimension finie ne peut avoir %
$\mathbb{C} [X]$ comme sous-espace vectoriel.

\par
Soit \((F_i(X))_{i\in\mathbb{N}}\) une suite de frations rationnelles. %
Pour tout \(i\), \(F_i(X)\) admet une suite finie de p\^oles \((\lambda_j)_{j=1}^{n_i}\). %
%%
L'ensemble

\begin{equation}
\bigcup\limits_{i\in\mathbb{N}}\,\text{Im}((\lambda_j)_{j=1}^{n_i})
%%
\label{eq:def_poles_ij}
\end{equation}
%%
de tous les p\^oles des termes de \((F_i(X))\) est une union finie d'ensembles d\'enombrables, %
il est donc d\'enombrable - on peut r\'eindexer les p\^oles qui apparaissent dans l'expression~\rfeq{eq:def_poles_ij} %
pour obtenir une suite index\'ee dans \(\mathbb{N}\). %
%%
Puisque \(\mathbb{C}\), comme \(\mathbb{R}\)\footnote{Voir la section concernant le espaces complets, dans ancien programme.}%
est non d\'enombrable, il existe une nombre complexe \(\lambda_{\infty}\) qui n'appartient pas \g{a} l'ensemble des p\^oles des fractions \(F_i(X)\). %
%%
La fraction

\[\frac{1}{X-\lambda_{\infty}},\]
%%
n'est pas une combinaison lin\'eaire de la famille \((F_i(X))_{i\in\mathbb{N}}\), %
car de telles combinaisons ont des p\^oles distincts de \(\lambda_{\infty}\). %
%%
Ainsi \((F_i(X))\) n'est-elle pas g\'en\'eratrice de \(\mathbb{C}(X)\).
\end{sol}

% Théorème de Mason, théorème de Liouville.

\begin{exer}[Densité naturelle et diviseurs communs]
Soit $A$ une partie de $\mathbb{N}$.\\
On dit que $A$ admet une densité naturelle, si et seulement si la suite :
\[\left(\frac{\sharp A \cap [\![1,n]\!]}{n}\right)_{n \in \mathbb{N}^{\ast}}\]
converge. On appelle densité de $A$, et note $d(A)$, cette limite lorsqu'elle existe.
\begin{enumerate}
\item Soit $\alpha$ un entier strictement positif. Montrer que $\alpha \mathbb{N}$ admet une densité naturelle, %
calculer cette densité.
\item Montrer que si $A$ est une partie de $\mathbb{N}$ qui admet une densité naturelle égale à $1$, %
alors $A$ contient une infinité d'entiers premiers entre eux deux-à-deux.
%Raisonner par l'absurde.
%Lemme important : il existe au moins une famille (x_i) d'entiers de $A$, premiers entre eux deux à deux, maximale parmi les familles d'entiers de A qui ont cette propriété.
\item Déduire de la question précédente un résultat classique en arithmétique.%Il s'agit du théorème d'infinitude de l'ensemble des nombres premiers.
\end{enumerate}
\end{exer}

\begin{sol}
%%
\begin{enumerate}
\item Remarquons que pour tout entier \(n\) strictement positif :

\begin{equation}
n=\lfloor \frac{n}{\alpha} \rfloor \, \alpha + r_n ,
%%
\label{eq:div_alpha_n}
\end{equation}
%%
o\g{u} \(r_n\) est un entier compris entre \(0\) et \(\alpha-1\). %
On peut aussi \'ecrire :

\begin{equation}
\sharp A \cap [\![1,n]\!] = \lfloor \frac{n}{\alpha} \rfloor.
%%
\label{eq:card_n_sur_alpha}
\end{equation}
%%
En combinant~\rfeq{eq:div_alpha_n} et~\rfeq{eq:card_n_sur_alpha} on obtient :

\begin{equation}
\dfrac{\sharp A \cap [\![1,n]\!]}{n} = \dfrac{1}{\alpha}\left(1-\frac{r_n}{n}\right) .
%%
\label{eq:dens_n_rn}
\end{equation}
%%
Puisque la suite \((r_n)_n\) est born\'ee, le terme de droite de~\rfeq{eq:dens_n_rn} tend vers \(\frac{1}{\alpha}\) lorsque \(n\) tend vers \(+\infty\). %
Cette limite est la densit\'e naturelle de \(\alpha\mathbb{N}^{\ast}\), qui en particulier existe.
%%
\item \dots
%%
\item Remarquons que \(\mathbb{N}^{\ast}\) admet ue densit\'e naturelle, qui est \'egale \g{a} \(1\). %
Il existe donc une infinit\'e d'entiers naturels premiers entre eux deux \g{a} deux. %
%%
Comme les facteurs premiers de deux quelconques de ces entiers sont distincts, il existe donc une infinit\'e de nombres premiers.
\end{enumerate}
%%
\end{sol}

\begin{exer}[Le théorème d'infinitude de l'ensemble des nombres premiers revisité]
On définit un ensemble $\tau$ parties de $\mathbb{Z}$ par :
\[\forall P \in \mathbb{N} , P \in \tau \Leftrightarrow ( \forall m \in P , \exists a \in \mathbb{Z} | m + a \mathbb{Z} \subseteq P)\]
\begin{enumerate}
\item Montrer que $\tau$ est une topologie sur $\mathbb{Z}$.
\item Montrer que, si $a$ est un entier et $b$ un entier naturel non nul, alors : %
$a \mathbb{Z} + b$ est fermé dans $(\mathbb{Z} , \tau)$.
\item Montrer que l'ensemble des nombres premiers est infini.
\end{enumerate}
\end{exer}

\begin{exer}
Soitn $n$ un entier naturel non nul tel que la suite $(a_k)_k$ des entiers strictement positifs, %
inférieurs à $n$ et premiers à $n$ soit arithmétique.\\
Montrer que $n$ est une puissance de deux, ou un nombre premier impair.
\end{exer}

\begin{exer}
Démontrer qu'une fonction rationnelle complexe non constante omet au plus une valeur dans $\mathbb{C}$.
%Indication : On utilisera le théorème de D'Alembert.
\end{exer}

\begin{exer}
Soit $k$ un entier naturel supérieur ou égal à $2$.\\
Montrer que le produit de trois entiers naturels non nuls consécutifs n'est jamais une puissance $k-$ième.
\end{exer}

\begin{exer}
\begin{enumerate}
\item Montrer qu'il existe une infinité de nombres premiers congrus à $3$ modulo $4$.
\item Montrer qu'il existe une infinité de nombres premiers congrus à $5$ modulo $6$.
\end{enumerate}
\end{exer}

\begin{exer}
Soit $A$ un anneau commutatif.\\
On dit que $A$ est Noetherien si et seulement si tout idéal de $A$ est engendré map un nombre fini d'éléments de $A$.
\begin{enumerate}
\item Montrer que toute suite croissante d'idéaux d'un anneau Noetherien est stationnaire.
\item Montrer que, si $A$ est un anneau intègre et Noetherien, alors tout élément de $A$ est décomposable en produit de facteurs irréductibles.
\end{enumerate}
%Indications : (i) Raisonner par l'absurde.\\
%(ii) Soit $a$ un élément de $A$ non décomposable. Montrer que $a$ est le produit de deux éléments de $A$ qui ne sont pas des unités, et que l'un d'entre eux est non décomposable.\\
%(iii) Montrer que si un anneau est principal, alors toute suite croissante de ses idéaux est stationnaire.\\
%(iv) Considérer la suite des idéaux respectivement engendrés par les termes de la suite d'éléments indécomposables de $A$ définie aux questions (i) et (ii). Conclure.
\end{exer}

\begin{sol}
%%
\begin{enumerate}
\item Soit \((I_n)_{n\in\mathbb{N}}\) unes suite croissante d'id\'eaux de \(A\). %
%%
Montrons que \(\bigcup\limits_{i\in\mathbb{N}}\,I_n\), autrement not\'e \(I\), est un id\'eal de \(A\). %
Soient en effet \(x\) et \(y\) deux \'el\'ements de cete union. %
En particulier, il existe deux enters \(n_x\) et \(n_y\) tels que \(x\) et \(y\) appartiennent respectivement \g{a} %
\(I_{n_x}\) et \(I_{n_y}\). %
%%
Mais la famille \((I_n)_{n\in\mathbb{N}}\) est croissante : \(x\) et \(y\), et donc leur somme, appartiennent \g{a} l'id\'eal \(I_{\max{n_x, n_y}}\) de \(A\). %
%%
Enfin, si \(x\) appartient \g{a} \(I_n\) pour un certain \(n\) et si \(a\) est un \'el\'ement de \(A\), %
alors \(ax\) appartient \g{a} \(I_n\), donc \g{a} \(\bigcup\limits_{i\in\mathbb{N}}\,I_n\).

\par
On peut maintenant d\'emontrer le r\'esultat principal de cette section. %
En effet, \(A\) \'etant n\"otherien, \(I\) est engendr\'e par une famille finie \((x_i)_{i=1}^{n_I}\) de ses \'el\'ements. %
En particulier, chaque \(x_i\) appartient \g{a} au moins un terme \(I_{n_i}\) de la suite \(I_n\) %
- on peut par exemple prendre le premier terme de \((I_n)_n\) o\g{u} l'on reencontre \(x_i\). %
%%
Alors, tous les termes de la famille g\'en\'eratrice \((x_i)_i\) de \(I\) appartiennnent \g{a} \(I_{\max\limits_i{n_i}}\). %
Cet id\'eal est donc confondu avec la limite \(I\) de la suite, qui est donc stationnaire.

%%
\item Raisonnons par l'absurde et supposons qu'il existe un \'el\'ement \(a\) de \(A\) qui n'est pas d\'ecomposable en produit de facteurs irr\'eductibles. %
%%
Plus particuli\g{e}rement, \(a\) ne peut \^etre ni irr\'eductible, ni inversible. %
On peut donc \'ecrire

\begin{equation}
a = bc
%%
\label{eq:hyp_a_nondec}
\end{equation}
%%
o\g{u} \(b\) et \(c\) sont non inversibles. %
%%
Les id\'eaux principaux \(b\,A\) et \(c\,A\) incluent \(a\,A\) d'apr\g{e}s ces relations de divisibilit\'e et la commutativit\'e de \(A\). %
Supposons par exemple que l'inclusion r\'eciproque \(b\,A\subset a\,A\) soit v\'erifi\'ee. %
On peut en particulier \'ecrire :

\begin{equation}
b=au\quad \text{donc} \quad b=bcu
%%
\label{eq:}
\end{equation}
%%
Puisque \(A\) est int\g{e}gre, ceci implique que \(cu=1\), donc que \(c\) est inversible, absurde. %
Par ailleurs, l'un au moins des facteurs \(b\) et \(c\) est ind\'ecomposable en produit de facteurs irr\'eductibles de \(A\), %
car dans le cas contraire~\rfeq{eq:hyp_a_nondec} fournirait une d\'ecomposition de \(a\). %
%%
On a ainsi construit un \'el\'ement \(a_1\) de \(A\), \(b\) ou \(c\), %
ind\'ecomposable comme \(a\), et tel que

\begin{equation}
a\,A \subsetneq a_1\, A
%%
\label{eq:st_inc}
\end{equation}

En it\'erant cette construction, on obtient donc une suite strictement croissante d'id\'eaux de \(A\), %
ce qui est impossible d'apr\g{e}s la question pr\'ec\'edente.
%%
\end{enumerate}
%%
\end{sol}
% \input{algebre_generale/series_formelles_1d.tex}

% \chapter{Alg\g{e}bre lin\'eaire}
\chapter{Alg\g{e}bre lin\'eaire}

\section{Programme de sup}

\begin{exer}
Soit $n$ un entier naturel non nul.

On d\'efinit, sur $\mathbb{R}_n[X]$, l'application $\varphi$ par : $\varphi (P) = Q$ o\`u
\[\forall x \in\mathbb{R} , \overset{\sim}{Q}(x)=\int\limits_x^{x+1}P(x)dx\]
D\'emontrer que $\varphi$ r\'ealise un endomorphisme de $\mathbb{R}_n[X]$ dont on calculera le d\'eterminant.
\end{exer}

\begin{exer}
Soient $A$ et $B$ deux matrices $n\times n$ \`a coefficients dans $\mathbb{Z}$ telles que : $det A \wedge det B = 1$.

Montrer qu'il existe deux matrices $U$ et $V$, \`a coefficients entiers, telles que :\[UA+VB=AU+BV ; AU+BV=I_n\]
\end{exer}

\begin{exer}
Soient $\mathbb{K}$ un corps commutatif, $E$ et $F$ deux $\mathbb{K}-$espaces vectoriels de dimension finie. %
Soit de plus $G$ un sous-espace vectoriel de $E$, on note $A$ l'ensemble :\[\{u\in\mathcal{L}(E,F) |\ker u\supseteq G\}\]
\begin{enumerate}
\item Montrer que $A$ est un sous-espace vectoriel de $\mathcal{L}(E,F)$ et pr\'eciser sa dimension.
\item Que retrouve-t-on dans le cas o\`u $F=\mathbb{K}$ ?
\end{enumerate}
\end{exer}

\begin{exer}
Soient $n$, $k$ et $p$ trois entiers naturels non nuls, on suppose que : $n\leq p$ %
Soit $E$ un espace vectoriel de dimension finie sur un corps $\mathbb{K}$. On suppose que le cardinal de $\mathbb{K}$ est strictement sup\'erieur \`a $k(n-1)$.

Montrer que toute famille $(F_i)_{i\in [\![1,k]\!]}$ de sous-espaces vectoriels de m\^me dimension $p$, on peut construire un suppl\'ementaire commun \`a chacun des $F_i$.\\
\textit{On traite le cas o\`u $\mathbb{K}$ est infini, on affine ensuite \'eventuellement.}
\end{exer}

\begin{exer}
Montrer que la famille $(x\mapsto\arg\sinh \lambda x)_{\lambda\in\mathbb{R}_+{\ast}}$ est libre %
dans $\mathcal{C}^{\infty}(\mathbb{R},\mathbb{R})$.
\end{exer}
\section{R\'eduction des endomorphismes}

\begin{exer}
Montrer que $\begin{pmatrix}I_n & I_n\\0&0\end{pmatrix}$ est diagonalisable.
\end{exer}

\begin{exer}
Soient $n$ un entier strictement positif, et $A$ et $B$ deux matrices de $\mathcal{M}_n(\mathbb{C})$. %
Montrer que si $AB=0$, alors $A$ et $B$ sont simultan\'ement diagonalisables.
\end{exer}

\begin{exer}
Soit $n$ un entier strictement sup\'erieur \`a $1$ et :\[f:\mathbb{C}_n[X] \rightarrow \mathbb{C}[X] : P\mapsto (X^2-X)P(1)+(X^2+X)P(-1)\]
Montrer que induit un endomorphisme de $\mathbb{C}_n[X]$ et d\'eterminer son image, son noyau et ses \'el\'ements propres.
\end{exer}

\begin{exer}
Soit $\varphi$ l'endomorphisme de $\mathbb{C}[X]$ d\'efini par :\[\forall P\in\mathbb{C}_n[X] , \varphi (P)=(3X+8)P+(X^2-5X)P'-(X^3-X^2)P''\]
D\'eterminer les \'el\'ements propres de $\varphi$.
\end{exer}

\begin{exer}
Soient $n$ un entier naturel non nul, $E$ un espace vectoriel de dimension $n$.

Montrer que si $(u_i)_{i \in [\![1,n]\!]}$ est une famille d'endomorphismes nilpotents de $E$ qui commutent deux-à-deux, %
alors le produit de cette famille d'endomorphismes est nul.
\end{exer}

\begin{exer}[Décomposition de Fitting]
Soient $E$ un espace vectoriel de dimension finie $n$ et $u$ un endomorphisme de $E$.
\begin{enumerate}
\item Montrer que les suites $(\textup{Im} u^k)_{k\in\mathbb{N}}$ et $(\ker u^k)_{k\in\mathbb{N}}$ sont strictement monotones puis stationnaires à un m\^eme rang $p$.
\item Montrer que la suite $(\ker u^k)_k$ "s'essouffle", c'est-à-dire que $(\dim \ker u^{k+1} - \dim \ker u^k)_k$ est décroissante.
\item Montrer que : $E = \textup{Im} u^p \oplus \ker u^p$.

Ces deux espaces sont respectivement appelés coeur et nilespace de $u$.
\item Montrer qu'il existe une base de $E$ dans laquelle la matrice de $u$ est diagonale par blocs, avec un bloc nilpotent et un bloc inversible.
\end{enumerate}
\end{exer}

\begin{exer}[Invariance du polynôme minimal par extension du corps de base]
Soient $M$ une matrice carrée à coefficients dans un corps $\mathbb{K}$, $\mathbb{L}$ une extension de $\mathbb{K}$.\\
On note $\mu_{M \mathbb{K}}$, respectivement $\mu_{M \mathbb{L}}$, le polynôme minimal de $M$ considérée come à coefficients dans $\mathbb{K}$, respectivement $\mathbb{L}$.

Montrer que ces polynômes sont égaux.
%Indications : i) Montrer que $\mu_{M \mathbb{L}}$ divise $\mu_{M \mathbb{K}}$. Il suffit alors de montrer que ces polynômes ont même degré.\\
%ii) Considérer le rang de $(M^q)_{q \in [0, \deg \mu_{M \mathbb{K}}]}$, et démontrer le résultat intermédiaire suivant :\\
%le rang d'une famille de vecteurs d'une puissancde cartésienne d'un corps est préservé par extenxion de ce corps.
\end{exer}

\begin{exer}
Soit : $n \in \mathbb{N}$. %
Soient $E$ un espace vectoriel de dimension $n$ -sur un corps de caractéristique nulle-, $G$ un sous-groupe fini de $GL(E)$. %
On note $E^G$ l'ensemble des éléments de $G$ invariants sous l'action de $G$.

Montrer que :\[\dim E^G = \frac{1}{|G|} \sum\limits_{g \in G} Tr g\]
On considérera le projecteur $\frac{1}{|G|} \sum\limits_{g \in G} g$. 
\end{exer}

\begin{exer}[Disques de Gershgörin]
Soit : $n \in \mathbb{N}^{\ast}$
\begin{enumerate}
\item Montrer le lemme d'Hadamard :
\begin{center}
\fbox{
\begin{minipage}{11cm}
Soit $(a_{ij}) \in M_n (\mathbb{C})$. %
Si :\[\forall i \in [\![1,n]\!] , |a_{ii}| > \sum\limits_{j \neq i} |a_{ij}|\]
alors $(a_{ij})$ est inversible.
\end{minipage}
}
\end{center}
\item En déduire une localisation des valeurs propres d'une matrice complexe.
\end{enumerate}
\end{exer}

\begin{exer}
Soit $E$ un espace vectoriel de dimension quelconque, et $u$ et $v$ deusx endomorphismes de $E$ qui commutent.

On suppose que $u$ et $v$ admettent un polyôme annulateur. Montrer qu'il en est de même pour $u+v$.
\end{exer}

\begin{exer}
Soit $A$ une matrice complexe.

Montrer que $(A^n)_n$ est bornée si et seulement si %
toutes ses valeurs propres sont de module inférieur à 1, et pour toute valeur propre $\lambda$ de $A$ de module $1$ :
\[\ker (A - \lambda I) = \ker (A - \lambda I)^2\]
\end{exer}

\begin{exer}
Soient $p$ un entier naturel non nul et $A$ une matrice complexe inversible tels que $A^p$ soit diagonalisable.
\begin{itemize}
\item Montrer que $A$ est diagonalisable.
\item Le résultat subsiste-t-il si $A$ n'est pas supposée inversible ?
\end{itemize}
\end{exer}

\begin{exer}
Soit $A$ une matrice complexe, $n\times n$, de rang $1$ -$n>0$.

Donner une condition n\'ecessaire et suffisante sur $A$ pour que $A$ soit diagonalisable.
\end{exer}

\begin{exer}
Soit : \[A=\begin{pmatrix}0&1&0\\1&0&1\\0&1&0\end{pmatrix}\]
Calculer $\exp A$.
\end{exer}

\begin{exer}
Soient $n$ un entier naturel non nul, et $G$ un sous goupe de $GL_n (\mathbb{C})$ vérifiant :
\[\forall g \in G , g^2 = I_n\]
\begin{enumerate}
\item Montrer que $G$ est abélien.
\item Montrer que les éléments de $G$ sont simultanément diagonalisables.
\item Montrer que $G$ est fini, et majorer son ordre.
\item Soit $m$ un entier naturel. %
Montrer que $GL_n (\mathbb{C})$ est isomorphe à $GL_n (\mathbb{C})$ si et seulement si $n = m$.
\end{enumerate}
\end{exer}

\begin{exer}
Soient $E$ un espace vectoriel de dimension finie, $u$ un endomorphisme de $E$. %
On note $\mu_u$ le polynôme minimal de $u$, et on appelle polynôme minimal ponctuel de $u$ en un vecteur $x$ %
le polynôme unitaire $\mu_u^x$ qui engendre l'idéal :\[\{ P \in \mathbb{K}[X] / P(u)(x) = 0\}\]
On munit $E$ d'une base $(e_i)$.
\begin{enumerate}
\item Montrer que les termes de $\mu_u^x$ divisent $\mu_u$.
\item Montrer que $\mu_u$ est le ppcm des termes de $(\mu_u^{e_i})_i$.
\item Soit $(x,y) \in E^2$. montrer que si $\mu_u^x$ et $\mu_u^y$ sont premiers entre eux, %
alors il existe un élément de $E$ dont le polynôme minimal ponctuel est $\mu_u^x\mu_u^y$.
\item Montrer qu'il existe un élément de $E$ dont le polynôme minimal ponctuel est $\mu_u$.
\end{enumerate}
\end{exer}

\subsection{Endomorphismes cycliques}

\begin{exer}
Dans tout cet exercice, $E$ est un $\mathbb{K}-$espace vectoriel de dimension finie, et $u$ un endomorphisme de $E$.
\begin{enumerate}
\item On appelle, pour un \'el\'ement $x$ quelconque de $E$, \textit{espace cyclique} de $u$ engendr\'e par $x$ le sous-espace $E_x$ de $E$ d\'efini par : $E_x=Vect (u^k(x))_{k\in\mathbb{N}}$.
\begin{enumerate}
\item Montrer que $E_x$ est le plus petit sous-espace de $E$ stable par $u$ et contenant $x$.
\item Montrer qu'il existe un polyn\^ome $\mu_u^x$, \`a coefficients dans $\mathbb{K}$, %
unique \`a une constance multiplicative pr\`es, qui divise tout polyn\^ome $P$ de $\mathbb{K}[X]$ tel que $P(u)(x)=0$.
\item Que dire du degr\'e de $\mu_u^x$ ?
\item Comparer $\mu_u^x$ et le polyn\^ome annulateur global de $u$, $\mu_u$.
\item En exprimant la matrice de la restriction $v$ de $u$ \`a $E_x$ dans une base bien choisie, %
montrer que : $\mu_u^x=\chi_v$.
\item Comparer $\mu_u^x$ et $\chi_u$, sans utiliser le th\'eo\`eme de Cayley-Hamilton.
\end{enumerate}
\item A l'aide des r\'esultats qui pr\'ec\`edent, d\'emontrer le th\'eor\`eme de Cayley-Hamilton.

\smallskip
Un endomorphisme $u$ de $E$ tel que $E_x=E$ pour un certain $x$ est dit \textit{cyclique}.
\item D\'emontrer que si $u$ est un endomorphisme cyclique, alors $\mu_u$ et $\chi_u$ sont \'egaux.
\item On suppose maintenant que le polyn\^ome $\mu_u$ est irr\'eductible.
\begin{enumerate}
%Montrer que si $x$ est un vecteur de $E$ et $F$ un sous-espace de $E$ stable par $u$, alors $F$ inclut $E_x$ ou l'intersection de ces deux espaces est triviale.
\item Montrer que les espaces cycliques de $u$ sont des sous-espaces stables minimaux, %
c'est-\`a-dire qu'ils n'incluent pas de sous-espace vectoriel strict stable par $u$.
\item Montrer que $u$ se d\'ecompose comme somme d'endomorphismes cycliques, %
c'est-\`a-dire qu'il existe une famille $(x_k)$ finie de vecteurs de $E$ telle que :
\[E=\bigoplus\limits_k E_{x_k}\]
\end{enumerate}
\end{enumerate}
\end{exer}

\begin{exer}[Lemme chinois et endomorphismes cycliques]
Montrer que la somme directe d'une famille d'endomorphismes cycliques deux à deux étrangers %
-du point de vue de leur polynôme caractéristique- est cyclique.
\end{exer}

\subsection{Semi-simplicité}

\begin{exer}
On dit que $u$ est semi-simple si et seulement si tout sous-espace de $E$ stable par $u$ admet un supplémentaire dans $E$, stable par $u$.

Montrer que $u$ est semi-simple si et seulement si son polynôme minimal n'admet que des facteurs irréductibles avec une multiplicité $1$.
\end{exer}

\begin{exer}[Endomorphismes semi-simples]
On appelle endomorphisme semi-simple d'un espace vectoriel $E$ un endomorphisme $u$ tel que, %
si $F$ est un sous-espace de $E$ stable par $u$, alors $F$ admet un supplémentaire stable par $u$. %
On définit de même les matrices semi-simples.
\begin{enumerate}
\item Que dire d'un endomorphisme semi-simple et nilpotent ?
\item Montrer que le polynôme minimal d'un endomorphisme semi-simple d'un espace vectoriel de dimension finie %
sur un corps quelconque, n'admet pas de facteur multiple.

Soit $M$ une matrice $(n,n)$ à coefficients dans un corps $\mathbb{K}$, semi-simple. %
On note $\mathbb{K}_{alg}$ la clôture algébrique de $\mathbb{K}$.
\item Montrer que si $M$ est considérée comme étant à coefficients dans $\mathbb{K}_{alg}$, alors $M$ est diagonalisable.

Soient $\mathbb{K}$ un corps commutatif, $\mathbb{L}$ un sur-corps commutatif de $\mathbb{K}$.
\item Montrer qu'une matrice $M$ -on considère les endomorphismes canoniquement associés- à coefficients dans $\mathbb{K}$, %
semi-simple lorsqu'elle est considérée à coefficients dans $\mathbb{L}$, est encore semi-simple.
\item En déduire qu'un endomorphisme d'espace vectoriel de dimension finie, annulé par un polynôme n'admettant pas de facteur multiple, est semi-simple.
\end{enumerate}
\end{exer}

\begin{exer}
Montrer que si $E$ est un espace vectoriel de dimension finie, %
et $(u_i)_i$ une famille d'endomorphismes diagonalisables de $E$ commutant deux à deux, %
alors ces endomorphismes sont diagonalisables dans une même base.
\end{exer}

\subsection{Simplicité}

\begin{exer}
On dit que $u$ est simple si et seulement si les seuls sous-espaces de $E$ stables par $u$ sont $E$ et $\{ 0\}$.
Montrer que $u$ est simple si et seulement si $\chi_u$ est irréductible.
\end{exer}


% \section{Programme de sup}

\begin{exer}
Soit $n$ un entier naturel non nul.

On d\'efinit, sur $\mathbb{R}_n[X]$, l'application $\varphi$ par : $\varphi (P) = Q$ o\`u
\[\forall x \in\mathbb{R} , \overset{\sim}{Q}(x)=\int\limits_x^{x+1}P(x)dx\]
D\'emontrer que $\varphi$ r\'ealise un endomorphisme de $\mathbb{R}_n[X]$ dont on calculera le d\'eterminant.
\end{exer}

\begin{exer}
Soient $A$ et $B$ deux matrices $n\times n$ \`a coefficients dans $\mathbb{Z}$ telles que : $det A \wedge det B = 1$.

Montrer qu'il existe deux matrices $U$ et $V$, \`a coefficients entiers, telles que :\[UA+VB=AU+BV ; AU+BV=I_n\]
\end{exer}

\begin{exer}
Soient $\mathbb{K}$ un corps commutatif, $E$ et $F$ deux $\mathbb{K}-$espaces vectoriels de dimension finie. %
Soit de plus $G$ un sous-espace vectoriel de $E$, on note $A$ l'ensemble :\[\{u\in\mathcal{L}(E,F) |\ker u\supseteq G\}\]
\begin{enumerate}
\item Montrer que $A$ est un sous-espace vectoriel de $\mathcal{L}(E,F)$ et pr\'eciser sa dimension.
\item Que retrouve-t-on dans le cas o\`u $F=\mathbb{K}$ ?
\end{enumerate}
\end{exer}

\begin{exer}
Soient $n$, $k$ et $p$ trois entiers naturels non nuls, on suppose que : $n\leq p$ %
Soit $E$ un espace vectoriel de dimension finie sur un corps $\mathbb{K}$. On suppose que le cardinal de $\mathbb{K}$ est strictement sup\'erieur \`a $k(n-1)$.

Montrer que toute famille $(F_i)_{i\in [\![1,k]\!]}$ de sous-espaces vectoriels de m\^me dimension $p$, on peut construire un suppl\'ementaire commun \`a chacun des $F_i$.\\
\textit{On traite le cas o\`u $\mathbb{K}$ est infini, on affine ensuite \'eventuellement.}
\end{exer}

\begin{exer}
Montrer que la famille $(x\mapsto\arg\sinh \lambda x)_{\lambda\in\mathbb{R}_+{\ast}}$ est libre %
dans $\mathcal{C}^{\infty}(\mathbb{R},\mathbb{R})$.
\end{exer}
% \section{R\'eduction des endomorphismes}

\begin{exer}
Montrer que $\begin{pmatrix}I_n & I_n\\0&0\end{pmatrix}$ est diagonalisable.
\end{exer}

\begin{exer}
Soient $n$ un entier strictement positif, et $A$ et $B$ deux matrices de $\mathcal{M}_n(\mathbb{C})$. %
Montrer que si $AB=0$, alors $A$ et $B$ sont simultan\'ement diagonalisables.
\end{exer}

\begin{exer}
Soit $n$ un entier strictement sup\'erieur \`a $1$ et :\[f:\mathbb{C}_n[X] \rightarrow \mathbb{C}[X] : P\mapsto (X^2-X)P(1)+(X^2+X)P(-1)\]
Montrer que induit un endomorphisme de $\mathbb{C}_n[X]$ et d\'eterminer son image, son noyau et ses \'el\'ements propres.
\end{exer}

\begin{exer}
Soit $\varphi$ l'endomorphisme de $\mathbb{C}[X]$ d\'efini par :\[\forall P\in\mathbb{C}_n[X] , \varphi (P)=(3X+8)P+(X^2-5X)P'-(X^3-X^2)P''\]
D\'eterminer les \'el\'ements propres de $\varphi$.
\end{exer}

\begin{exer}
Soient $n$ un entier naturel non nul, $E$ un espace vectoriel de dimension $n$.

Montrer que si $(u_i)_{i \in [\![1,n]\!]}$ est une famille d'endomorphismes nilpotents de $E$ qui commutent deux-à-deux, %
alors le produit de cette famille d'endomorphismes est nul.
\end{exer}

\begin{exer}[Décomposition de Fitting]
Soient $E$ un espace vectoriel de dimension finie $n$ et $u$ un endomorphisme de $E$.
\begin{enumerate}
\item Montrer que les suites $(\textup{Im} u^k)_{k\in\mathbb{N}}$ et $(\ker u^k)_{k\in\mathbb{N}}$ sont strictement monotones puis stationnaires à un m\^eme rang $p$.
\item Montrer que la suite $(\ker u^k)_k$ "s'essouffle", c'est-à-dire que $(\dim \ker u^{k+1} - \dim \ker u^k)_k$ est décroissante.
\item Montrer que : $E = \textup{Im} u^p \oplus \ker u^p$.

Ces deux espaces sont respectivement appelés coeur et nilespace de $u$.
\item Montrer qu'il existe une base de $E$ dans laquelle la matrice de $u$ est diagonale par blocs, avec un bloc nilpotent et un bloc inversible.
\end{enumerate}
\end{exer}

\begin{exer}[Invariance du polynôme minimal par extension du corps de base]
Soient $M$ une matrice carrée à coefficients dans un corps $\mathbb{K}$, $\mathbb{L}$ une extension de $\mathbb{K}$.\\
On note $\mu_{M \mathbb{K}}$, respectivement $\mu_{M \mathbb{L}}$, le polynôme minimal de $M$ considérée come à coefficients dans $\mathbb{K}$, respectivement $\mathbb{L}$.

Montrer que ces polynômes sont égaux.
%Indications : i) Montrer que $\mu_{M \mathbb{L}}$ divise $\mu_{M \mathbb{K}}$. Il suffit alors de montrer que ces polynômes ont même degré.\\
%ii) Considérer le rang de $(M^q)_{q \in [0, \deg \mu_{M \mathbb{K}}]}$, et démontrer le résultat intermédiaire suivant :\\
%le rang d'une famille de vecteurs d'une puissancde cartésienne d'un corps est préservé par extenxion de ce corps.
\end{exer}

\begin{exer}
Soit : $n \in \mathbb{N}$. %
Soient $E$ un espace vectoriel de dimension $n$ -sur un corps de caractéristique nulle-, $G$ un sous-groupe fini de $GL(E)$. %
On note $E^G$ l'ensemble des éléments de $G$ invariants sous l'action de $G$.

Montrer que :\[\dim E^G = \frac{1}{|G|} \sum\limits_{g \in G} Tr g\]
On considérera le projecteur $\frac{1}{|G|} \sum\limits_{g \in G} g$. 
\end{exer}

\begin{exer}[Disques de Gershgörin]
Soit : $n \in \mathbb{N}^{\ast}$
\begin{enumerate}
\item Montrer le lemme d'Hadamard :
\begin{center}
\fbox{
\begin{minipage}{11cm}
Soit $(a_{ij}) \in M_n (\mathbb{C})$. %
Si :\[\forall i \in [\![1,n]\!] , |a_{ii}| > \sum\limits_{j \neq i} |a_{ij}|\]
alors $(a_{ij})$ est inversible.
\end{minipage}
}
\end{center}
\item En déduire une localisation des valeurs propres d'une matrice complexe.
\end{enumerate}
\end{exer}

\begin{exer}
Soit $E$ un espace vectoriel de dimension quelconque, et $u$ et $v$ deusx endomorphismes de $E$ qui commutent.

On suppose que $u$ et $v$ admettent un polyôme annulateur. Montrer qu'il en est de même pour $u+v$.
\end{exer}

\begin{exer}
Soit $A$ une matrice complexe.

Montrer que $(A^n)_n$ est bornée si et seulement si %
toutes ses valeurs propres sont de module inférieur à 1, et pour toute valeur propre $\lambda$ de $A$ de module $1$ :
\[\ker (A - \lambda I) = \ker (A - \lambda I)^2\]
\end{exer}

\begin{exer}
Soient $p$ un entier naturel non nul et $A$ une matrice complexe inversible tels que $A^p$ soit diagonalisable.
\begin{itemize}
\item Montrer que $A$ est diagonalisable.
\item Le résultat subsiste-t-il si $A$ n'est pas supposée inversible ?
\end{itemize}
\end{exer}

\begin{exer}
Soit $A$ une matrice complexe, $n\times n$, de rang $1$ -$n>0$.

Donner une condition n\'ecessaire et suffisante sur $A$ pour que $A$ soit diagonalisable.
\end{exer}

\begin{exer}
Soit : \[A=\begin{pmatrix}0&1&0\\1&0&1\\0&1&0\end{pmatrix}\]
Calculer $\exp A$.
\end{exer}

\begin{exer}
Soient $n$ un entier naturel non nul, et $G$ un sous goupe de $GL_n (\mathbb{C})$ vérifiant :
\[\forall g \in G , g^2 = I_n\]
\begin{enumerate}
\item Montrer que $G$ est abélien.
\item Montrer que les éléments de $G$ sont simultanément diagonalisables.
\item Montrer que $G$ est fini, et majorer son ordre.
\item Soit $m$ un entier naturel. %
Montrer que $GL_n (\mathbb{C})$ est isomorphe à $GL_n (\mathbb{C})$ si et seulement si $n = m$.
\end{enumerate}
\end{exer}

\begin{exer}
Soient $E$ un espace vectoriel de dimension finie, $u$ un endomorphisme de $E$. %
On note $\mu_u$ le polynôme minimal de $u$, et on appelle polynôme minimal ponctuel de $u$ en un vecteur $x$ %
le polynôme unitaire $\mu_u^x$ qui engendre l'idéal :\[\{ P \in \mathbb{K}[X] / P(u)(x) = 0\}\]
On munit $E$ d'une base $(e_i)$.
\begin{enumerate}
\item Montrer que les termes de $\mu_u^x$ divisent $\mu_u$.
\item Montrer que $\mu_u$ est le ppcm des termes de $(\mu_u^{e_i})_i$.
\item Soit $(x,y) \in E^2$. montrer que si $\mu_u^x$ et $\mu_u^y$ sont premiers entre eux, %
alors il existe un élément de $E$ dont le polynôme minimal ponctuel est $\mu_u^x\mu_u^y$.
\item Montrer qu'il existe un élément de $E$ dont le polynôme minimal ponctuel est $\mu_u$.
\end{enumerate}
\end{exer}

\subsection{Endomorphismes cycliques}

\begin{exer}
Dans tout cet exercice, $E$ est un $\mathbb{K}-$espace vectoriel de dimension finie, et $u$ un endomorphisme de $E$.
\begin{enumerate}
\item On appelle, pour un \'el\'ement $x$ quelconque de $E$, \textit{espace cyclique} de $u$ engendr\'e par $x$ le sous-espace $E_x$ de $E$ d\'efini par : $E_x=Vect (u^k(x))_{k\in\mathbb{N}}$.
\begin{enumerate}
\item Montrer que $E_x$ est le plus petit sous-espace de $E$ stable par $u$ et contenant $x$.
\item Montrer qu'il existe un polyn\^ome $\mu_u^x$, \`a coefficients dans $\mathbb{K}$, %
unique \`a une constance multiplicative pr\`es, qui divise tout polyn\^ome $P$ de $\mathbb{K}[X]$ tel que $P(u)(x)=0$.
\item Que dire du degr\'e de $\mu_u^x$ ?
\item Comparer $\mu_u^x$ et le polyn\^ome annulateur global de $u$, $\mu_u$.
\item En exprimant la matrice de la restriction $v$ de $u$ \`a $E_x$ dans une base bien choisie, %
montrer que : $\mu_u^x=\chi_v$.
\item Comparer $\mu_u^x$ et $\chi_u$, sans utiliser le th\'eo\`eme de Cayley-Hamilton.
\end{enumerate}
\item A l'aide des r\'esultats qui pr\'ec\`edent, d\'emontrer le th\'eor\`eme de Cayley-Hamilton.

\smallskip
Un endomorphisme $u$ de $E$ tel que $E_x=E$ pour un certain $x$ est dit \textit{cyclique}.
\item D\'emontrer que si $u$ est un endomorphisme cyclique, alors $\mu_u$ et $\chi_u$ sont \'egaux.
\item On suppose maintenant que le polyn\^ome $\mu_u$ est irr\'eductible.
\begin{enumerate}
%Montrer que si $x$ est un vecteur de $E$ et $F$ un sous-espace de $E$ stable par $u$, alors $F$ inclut $E_x$ ou l'intersection de ces deux espaces est triviale.
\item Montrer que les espaces cycliques de $u$ sont des sous-espaces stables minimaux, %
c'est-\`a-dire qu'ils n'incluent pas de sous-espace vectoriel strict stable par $u$.
\item Montrer que $u$ se d\'ecompose comme somme d'endomorphismes cycliques, %
c'est-\`a-dire qu'il existe une famille $(x_k)$ finie de vecteurs de $E$ telle que :
\[E=\bigoplus\limits_k E_{x_k}\]
\end{enumerate}
\end{enumerate}
\end{exer}

\begin{exer}[Lemme chinois et endomorphismes cycliques]
Montrer que la somme directe d'une famille d'endomorphismes cycliques deux à deux étrangers %
-du point de vue de leur polynôme caractéristique- est cyclique.
\end{exer}

\subsection{Semi-simplicité}

\begin{exer}
On dit que $u$ est semi-simple si et seulement si tout sous-espace de $E$ stable par $u$ admet un supplémentaire dans $E$, stable par $u$.

Montrer que $u$ est semi-simple si et seulement si son polynôme minimal n'admet que des facteurs irréductibles avec une multiplicité $1$.
\end{exer}

\begin{exer}[Endomorphismes semi-simples]
On appelle endomorphisme semi-simple d'un espace vectoriel $E$ un endomorphisme $u$ tel que, %
si $F$ est un sous-espace de $E$ stable par $u$, alors $F$ admet un supplémentaire stable par $u$. %
On définit de même les matrices semi-simples.
\begin{enumerate}
\item Que dire d'un endomorphisme semi-simple et nilpotent ?
\item Montrer que le polynôme minimal d'un endomorphisme semi-simple d'un espace vectoriel de dimension finie %
sur un corps quelconque, n'admet pas de facteur multiple.

Soit $M$ une matrice $(n,n)$ à coefficients dans un corps $\mathbb{K}$, semi-simple. %
On note $\mathbb{K}_{alg}$ la clôture algébrique de $\mathbb{K}$.
\item Montrer que si $M$ est considérée comme étant à coefficients dans $\mathbb{K}_{alg}$, alors $M$ est diagonalisable.

Soient $\mathbb{K}$ un corps commutatif, $\mathbb{L}$ un sur-corps commutatif de $\mathbb{K}$.
\item Montrer qu'une matrice $M$ -on considère les endomorphismes canoniquement associés- à coefficients dans $\mathbb{K}$, %
semi-simple lorsqu'elle est considérée à coefficients dans $\mathbb{L}$, est encore semi-simple.
\item En déduire qu'un endomorphisme d'espace vectoriel de dimension finie, annulé par un polynôme n'admettant pas de facteur multiple, est semi-simple.
\end{enumerate}
\end{exer}

\begin{exer}
Montrer que si $E$ est un espace vectoriel de dimension finie, %
et $(u_i)_i$ une famille d'endomorphismes diagonalisables de $E$ commutant deux à deux, %
alors ces endomorphismes sont diagonalisables dans une même base.
\end{exer}

\subsection{Simplicité}

\begin{exer}
On dit que $u$ est simple si et seulement si les seuls sous-espaces de $E$ stables par $u$ sont $E$ et $\{ 0\}$.
Montrer que $u$ est simple si et seulement si $\chi_u$ est irréductible.
\end{exer}

% \chapter{Alg\g{e}bre bilin\'aire, formes quadratiques, espaces pr\'ehilbertiens}
\chapter{Alg\g{e}bre bilin\'eaire, formes quadratiques, espaces pr\'ehilbertiens}

\section{Alg\`ebre bilin\'eaire, espaces pr\'ehilbertiens}

\begin{exer}
Soit $E$ l'espace vectoriel des suites r\'eeles de carr\'e sommable, c'est-\`a-dire les suites $(u_n)_{n\in\mathbb{N}}$ telles que $\sum u_n^2$ converge. %
On pose : \[\langle(u_n)|(v_n)\rangle =\sum\limits_{n=0}^{\infty} u_n v_n\]
pour toutes les suites $(u_n)$ et $(v_n)$ de $E$.
\begin{enumerate}
\item Montrer que l'application d\'efinie pr\'ec\'edemment est un produit scalaire sur $E$. Quelle norme d\'erive de ce roduit scalaire ?

\smallskip
On note maintenant, pour tout entier naturel $k$, $\delta^{(k)}$ la suite dont le $ki$i\`eme terme vaut $1$, et dont les autres termes sont nuls.
\item D\'ecrire l'espace vectoriel engendr\'e par la suite $(\delta^{(k)})_k$.
\item Montrer que cette suite est totale dans $E$.
\item Montrer qu'il existe une partie d\'enombrable de $E$, qui est dense dans $E$. On dit que $E$ est \textit{s\'eparable}.
\end{enumerate}
On note souvent $l^2(\mathbb{R})$ l'espace $E$, muni de la structure pr\'ehilbertienne \'etud\i\'ee ici.
\end{exer}

%élève2
\begin{exer}
Calculer \[\underset{a,b\in\mathbb{R}^2}{\inf}\int\limits_0^1(x\ln x -ax^2-bx)^2 dx\]
\end{exer}

%élève3
\begin{exer}
Montrer qu'une matrice réelle inversible est le produit d'une matrice orthogonale et d'une matrice triangulaire supérieure %
-décomposition d'Iwasawa.
\end{exer}

\begin{exer}[Polyn\^omes de Tchebycheff et produit scalaire]
On d\'efinit, pour deux polyn\^omes $P$ et $Q$ de $\mathbb{R}[X]$ :
\[\langle P|Q \rangle =\int\limits_{-1}^1 \frac{\tilde{P}(t)\tilde{Q}(t)}{\sqrt{1-t^2}}dt\]
\begin{enumerate}
\item Montrer que l'application $\langle |\rangle$ d\'efinie ci-dessus est un produit scalaire sur $\mathbb{R}[X]$.
\item Soit $n$ un entier naturel. Montrer qu'il existe un unique polyn\^ome $T_n$ tel que :
\[\forall\theta\in\mathbb{R} , T_n(\cos x)=\cos nx\]
\item Etablir la relation de r\'ecurrence d'ordre $2$ entre les termes de la suite $(T_n)$.
\item Montrer que la suite $(T_n)$ est orthogonale pour $\langle |\rangle$ et calculer la norme des termes de $(T_n)$.
\end{enumerate}
\end{exer}

\begin{exer}
Soit $E$ un espace pr\'ehilbertien r\'eel. Soient de plus $F$ et $G$ deux sous-espaces vectoriels de $E$.
\begin{enumerate}
\item Comparer $(F+G)^{\perp}$ et $F^{\perp}\cap G^{\perp}$.
\item Comparer $(F\cap G)^{\perp}$ et $F^{\perp}+G^{\perp}$.
\item Que dire si $E$ est de dimension finie ?
\end{enumerate}
\end{exer}

\begin{exer}
Soit $n$ un entier naturel non nul.\\
D\'eterminer, pour toute matrice $A$ de $\mathcal{M}_n(\mathbb{R})$, la valeur :
\[\underset{M\in\mathcal{S}_n(\mathbb{R})}{\min}\sum\limits_{(i,j)\in [\![1,n]\!]^2} (A_{i,j}-M_{i,j})^2\]
\end{exer}

\begin{exer}
Soit $(E,\langle |\rangle )$ un espace euclidien de dimension sup\'erieure ou \'egale \`a $2$, $a$ et $b$ deux vecteurs unitaires et ind\'ependants de $E$.

On d\'efinit l'endomorphisme lin\'eaire $f$ de $E$ par :\[\forall x \in E, f(x)=\langle a|x\rangle a+\langle b|x \rangle b\]
\begin{enumerate}
\item Caract\'eriser $f$ lorsque $a$ et $b$ sont deux vecteurs orthogonaux.
\item Cas g\'en\'eral : d\'eterminer l'image et les \'el\'ements propres de $f$. $f$ est-il diagonalisable ?
\end{enumerate}
\end{exer}

\begin{exer}
Soit $E$ un espace Euclidien. %
On appelle centre d'un groupe $G$ l'ensemble des éléments de $G$ qui commutent avec tous les éléments de $G$.

D\'eterminer le centre de $O(E)$ et celui de $SO(E)$.
\end{exer}

\begin{exer}
Soit $E$ un espace vectoriel réel de dimension finie.
\begin{enumerate}
\item Montrer que tout endomorphisme de $E$ admet un sous-espace stable de dimension $1$ ou $2$.

\smallskip
On suppose maintenant $E$ Euclidien.
\item Montrer que si $f$ est un endomorphisme de $E$ qui stabilise l'orthogonal de tout sous-espace stable, alors $E$ se décompose en somme directe orthogonale de sous-espaces stables de $f$, de dimension inférieure ou égale à $2$.
\item En déduire, dans des bases orthogonales bien choisies, les matrices des endomorphismes symétriques ou orthogonaux.% Anien programme : endomorphjismes antisymétriques.
\end{enumerate}
\end{exer}

\begin{exer}
Soit $E$ un espace vectoriel réel de dimension finie.

Montrer que si $G$ est un sous-groupe fini de $GL(E)$, %
alors tout sous-espace vectoriel de $E$ stable par tous les éléments de $G$ admet un suppl\'ementaire stable par $G$.
\end{exer}

\begin{exer}
Soit $E$ un espace préhilbertien réel.
\begin{enumerate}
\item Soit $(e_k)$ une suite libre ordonnée, finie ou infinie, de vecteurs de $E$. Définir l'orthonormalisée de Gram-Schmidt de $(e_k)$.
On suppose maintenant $E$ Euclidien de dimension $n$.
\item Que dire de l'ensemble $B$ des bases ordonnées de $E$ par rapport à $E^n$ ?
\item Montrer que l'application de $B$ dans $B$ qui associe, à une base de $E$, son orthonormalisée, est continue.
\end{enumerate}
\end{exer}

\begin{exer}
Trouver tous les couples de sym\'etries orthogonales qui commutent.
\end{exer}% pas d'extension .tex
% \section{Ancien programme}

\subsection{Alg\g{e}bre bilin\'eaire}

\begin{exer}
Soit $A = \mathbb{R} \rightarrow M_n(\mathbb{R}) : t \mapsto A(t)$ une application continue qui prend des valeurs antisymétriques.\\
Montrer que toute solution de l'équation différentielle en $X$, fonction de $\mathbb{R}$ dans $M_n(\mathbb{R})$ :
\[\forall t \in \mathbb{R} , X'(t) = A(t)X(t)\]
telle que $X(0)$ soit une matrice orthogonale, prend ses valeurs dans l'ensemble des matrices orthogonales.
\end{exer}

\begin{exer}
%Cet exercice utilise les résultats de l'exercice ??\\
Soit $n$ un entier naturel non nul. On notera $Asym(n,\mathbb{R})$ l'espace des matrices antisymétriques d'ordre $n$ %
sur $\mathbb{R}$.\\
Montrer que l'exponentielle de matrices induit une surjection de $Asym(n,\mathbb{R})$ sur $SO(n,\mathbb{R})$.
\end{exer}

\subsection{Formes bilin\'eaires et quadratiques}

\begin{exer}
Sioent $q$ et $q'$ deux formes quadratiques de même cône isotrope.

Donner une relation simple entre $q$ et $q'$.
\end{exer}

\begin{exer}
Montrer que les formes bilinéaires $\phi$, non dégénérées, d'un espace vectoriel vérifiant :
\[\forall (x,y) \in E^2, \phi (x,y) = 0 \Rightarrow \phi (y,x) = 0\]
sont les formes bilinéaires symétriques et antisymétriques.

\medskip
On pourra \'etudier les familles de formes lin\'eaires, indexées en $x$ 
$d_x : E \rightarrow \mathbb{K} : y \mapsto \phi(y,x)$ et $g_x : E \rightarrow \mathbb{K} : y \mapsto \phi(x,y)$
\end{exer}

\newpage

\begin{center}
\fbox{
\begin{minipage}{15cm}
\textit{
Pour toute forme quadratique $q$, on appelle groupe orthogonal de $q$ et note $O(q)$ %
l'ensemble des automorphismes linéaires $u$ de $E$, encore appelés isométries, tels que :
\[\forall x \in E , q(u(x)) = q(x)\]
En particulier, une isom\'etrie pour $q$ pr\'eserve sa forme polaire $\varphi$
}
\end{minipage}
}
\end{center}

\begin{exer}[Commutant du groupe $O(q)$ dans $\mathcal{L}(E)$]
Soit $E$ un espace vectoriel de dimension finie. On considère, dans cet exercice, une forme quadratique $q$ non dégénérée sur $E$, on note $\varphi$ sa forme polaire.

Soit de plus $a$ un vecteur de $E$, non isotrope pour $q$. %
On note $A$ la droite vectorielle de $E$ engendrée par $a$, et $B$ l'espace $\{ x \in E | \varphi (a,x) = 0 \}$.
\begin{enumerate}
\item Montrer que : $E = A \oplus B$.
\item Avec cette notation, montrer que la symétrie linéaire par rapport à $A$, et parallèlement à $B$, est une isométrie pour $q$.

\medskip
On note :\[C = \{ v \in L(E) | \forall u \in O(q) , uv = vu\}\]
cet ensemble est appelé le commutant de $O(q)$ dans $L(E)$.\\
Soit $v$ un élément de $C$.
\item Soit encore $a$ un vecteur de $E$ non isotrope pour $q$. Montrer que : %
$\exists \lambda_a \in \mathbb{R} | v(a) = \lambda_a a$.
\item Que dire du scalaire $\lambda_b$, défini pour un autre quelconque vecteur non isotrope de $q$ ?
\item Etudier le cas d'un vecteur isotrope de $q$.
\item Déterminer $C$.
\end{enumerate}
\end{exer}

\begin{exer}[Condition de minimalité de $O(q)$]
%\textit{Cet exercice reprend les notations adoptées au début de l'exercice précédent, %
%et utilise le résultat concernant le vecteur isotrope de $q$.}
On reprend la notation de l'exercice pr\'ec\'edent pour le groupe orthogonal.

\medskip
Soient $q$ et $q'$ deux formes quadratiques sur $E$, on suppose que $q$ est non dégénérée. %
On note encore $b$ et $b'$ les formes polaires respectives pour $q$ et $q'$.
\begin{enumerate}
\item Montrer l'existence d'un endomorphisme $u$ de $E$ tel que :
\[\forall (x,y) \in E^2 , b'(x,y) = b(u(x),y)\]
autrement dit :
\[\forall (x,y) \in E^2 , b'(x,y) = b(x,u(y))\]
On suppose maintenant que : $O(q') \subseteq O(q)$.
\item Montrer que cette inclusion est une égalité.
\end{enumerate}
\end{exer}

\subsection{Formes sesquilin\'eaires complexes}

\begin{exer}
Montrer que la norme de $M_n (\mathbb{C})$ qui dérive du produit scalaire $(A,B) \mapsto Tr(A^{\ast} B)$ %
est une norme d'algèbre.\\
Calculer la norme d'une matrice Hermitienne.
\end{exer}

\begin{exer}
Soient $A$ et $B$ deux matrices Hermitiennes. Montrer que les valeurs propres de $AB - BA$ sont imaginaires pures.
\end{exer}

\begin{exer}
Montrer qu'une matrice réelle -respectivement complexe- inversible est le produit d'une matrice orthogonale %
par une matrice symétrique définie positive -respectivement une matrice unitaire par une matrice Hermitienne définie positive.
\end{exer}

\begin{exer}
On définit un ordre $\preceq$ sur l'ensemble des matrices hermitiennes positives par : $H \preceq K$ si et seulement si $K - H$ est positive.

Montrer que $M_n(\mathbb{C}) \rightarrow M_n(\mathbb{C}) : A \mapsto A^{\ast} A$ est convexe pour cette relation d'ordre.
\end{exer}

% \section{Alg\`ebre bilin\'eaire, espaces pr\'ehilbertiens}

\begin{exer}
Soit $E$ l'espace vectoriel des suites r\'eeles de carr\'e sommable, c'est-\`a-dire les suites $(u_n)_{n\in\mathbb{N}}$ telles que $\sum u_n^2$ converge. %
On pose : \[\langle(u_n)|(v_n)\rangle =\sum\limits_{n=0}^{\infty} u_n v_n\]
pour toutes les suites $(u_n)$ et $(v_n)$ de $E$.
\begin{enumerate}
\item Montrer que l'application d\'efinie pr\'ec\'edemment est un produit scalaire sur $E$. Quelle norme d\'erive de ce roduit scalaire ?

\smallskip
On note maintenant, pour tout entier naturel $k$, $\delta^{(k)}$ la suite dont le $ki$i\`eme terme vaut $1$, et dont les autres termes sont nuls.
\item D\'ecrire l'espace vectoriel engendr\'e par la suite $(\delta^{(k)})_k$.
\item Montrer que cette suite est totale dans $E$.
\item Montrer qu'il existe une partie d\'enombrable de $E$, qui est dense dans $E$. On dit que $E$ est \textit{s\'eparable}.
\end{enumerate}
On note souvent $l^2(\mathbb{R})$ l'espace $E$, muni de la structure pr\'ehilbertienne \'etud\i\'ee ici.
\end{exer}

%élève2
\begin{exer}
Calculer \[\underset{a,b\in\mathbb{R}^2}{\inf}\int\limits_0^1(x\ln x -ax^2-bx)^2 dx\]
\end{exer}

%élève3
\begin{exer}
Montrer qu'une matrice réelle inversible est le produit d'une matrice orthogonale et d'une matrice triangulaire supérieure %
-décomposition d'Iwasawa.
\end{exer}

\begin{exer}[Polyn\^omes de Tchebycheff et produit scalaire]
On d\'efinit, pour deux polyn\^omes $P$ et $Q$ de $\mathbb{R}[X]$ :
\[\langle P|Q \rangle =\int\limits_{-1}^1 \frac{\tilde{P}(t)\tilde{Q}(t)}{\sqrt{1-t^2}}dt\]
\begin{enumerate}
\item Montrer que l'application $\langle |\rangle$ d\'efinie ci-dessus est un produit scalaire sur $\mathbb{R}[X]$.
\item Soit $n$ un entier naturel. Montrer qu'il existe un unique polyn\^ome $T_n$ tel que :
\[\forall\theta\in\mathbb{R} , T_n(\cos x)=\cos nx\]
\item Etablir la relation de r\'ecurrence d'ordre $2$ entre les termes de la suite $(T_n)$.
\item Montrer que la suite $(T_n)$ est orthogonale pour $\langle |\rangle$ et calculer la norme des termes de $(T_n)$.
\end{enumerate}
\end{exer}

\begin{exer}
Soit $E$ un espace pr\'ehilbertien r\'eel. Soient de plus $F$ et $G$ deux sous-espaces vectoriels de $E$.
\begin{enumerate}
\item Comparer $(F+G)^{\perp}$ et $F^{\perp}\cap G^{\perp}$.
\item Comparer $(F\cap G)^{\perp}$ et $F^{\perp}+G^{\perp}$.
\item Que dire si $E$ est de dimension finie ?
\end{enumerate}
\end{exer}

\begin{exer}
Soit $n$ un entier naturel non nul.\\
D\'eterminer, pour toute matrice $A$ de $\mathcal{M}_n(\mathbb{R})$, la valeur :
\[\underset{M\in\mathcal{S}_n(\mathbb{R})}{\min}\sum\limits_{(i,j)\in [\![1,n]\!]^2} (A_{i,j}-M_{i,j})^2\]
\end{exer}

\begin{exer}
Soit $(E,\langle |\rangle )$ un espace euclidien de dimension sup\'erieure ou \'egale \`a $2$, $a$ et $b$ deux vecteurs unitaires et ind\'ependants de $E$.

On d\'efinit l'endomorphisme lin\'eaire $f$ de $E$ par :\[\forall x \in E, f(x)=\langle a|x\rangle a+\langle b|x \rangle b\]
\begin{enumerate}
\item Caract\'eriser $f$ lorsque $a$ et $b$ sont deux vecteurs orthogonaux.
\item Cas g\'en\'eral : d\'eterminer l'image et les \'el\'ements propres de $f$. $f$ est-il diagonalisable ?
\end{enumerate}
\end{exer}

\begin{exer}
Soit $E$ un espace Euclidien. %
On appelle centre d'un groupe $G$ l'ensemble des éléments de $G$ qui commutent avec tous les éléments de $G$.

D\'eterminer le centre de $O(E)$ et celui de $SO(E)$.
\end{exer}

\begin{exer}
Soit $E$ un espace vectoriel réel de dimension finie.
\begin{enumerate}
\item Montrer que tout endomorphisme de $E$ admet un sous-espace stable de dimension $1$ ou $2$.

\smallskip
On suppose maintenant $E$ Euclidien.
\item Montrer que si $f$ est un endomorphisme de $E$ qui stabilise l'orthogonal de tout sous-espace stable, alors $E$ se décompose en somme directe orthogonale de sous-espaces stables de $f$, de dimension inférieure ou égale à $2$.
\item En déduire, dans des bases orthogonales bien choisies, les matrices des endomorphismes symétriques ou orthogonaux.% Anien programme : endomorphjismes antisymétriques.
\end{enumerate}
\end{exer}

\begin{exer}
Soit $E$ un espace vectoriel réel de dimension finie.

Montrer que si $G$ est un sous-groupe fini de $GL(E)$, %
alors tout sous-espace vectoriel de $E$ stable par tous les éléments de $G$ admet un suppl\'ementaire stable par $G$.
\end{exer}

\begin{exer}
Soit $E$ un espace préhilbertien réel.
\begin{enumerate}
\item Soit $(e_k)$ une suite libre ordonnée, finie ou infinie, de vecteurs de $E$. Définir l'orthonormalisée de Gram-Schmidt de $(e_k)$.
On suppose maintenant $E$ Euclidien de dimension $n$.
\item Que dire de l'ensemble $B$ des bases ordonnées de $E$ par rapport à $E^n$ ?
\item Montrer que l'application de $B$ dans $B$ qui associe, à une base de $E$, son orthonormalisée, est continue.
\end{enumerate}
\end{exer}

\begin{exer}
Trouver tous les couples de sym\'etries orthogonales qui commutent.
\end{exer}

% \chapter{G\'eom\'etrie euclidienne}
\chapter{G\'eom\'etrie euclidienne}

\section{Programmme actuel}

\begin{exer}
Montrer que le plan Euclidien peut être pavé par des polygones réguliers, %
si et seulement si leurs nombres de côtés est $3$, $4$ ou $6$.
%Indication : on considèrera les angles des sommets de ces polygones.
\end{exer}

\begin{exer}
On se donne $1000$ points du plan affine Euclidien.

Montrer qu'il existe une droite affine qui partitionne la plan en deux demi-plans ouverts, %
contenant chacun exactement $500$ points de la famille précédente.
\end{exer}

\begin{exer}[Théorème de Sylvester]
On concidère $n$ points du plan affine Euclidien.

Montrer que, si toute droite menée par deux de ces points en contient un troisième, alors tous ces points sont alignés.
\end{exer}

\begin{exer}
Montrer que le plan affine Euclidien ne peut être partitionné en cercles Euclidiens de rayons non nuls.
\end{exer}


% \section{Programmme actuel}

\begin{exer}
Montrer que le plan Euclidien peut être pavé par des polygones réguliers, %
si et seulement si leurs nombres de côtés est $3$, $4$ ou $6$.
%Indication : on considèrera les angles des sommets de ces polygones.
\end{exer}

\begin{exer}
On se donne $1000$ points du plan affine Euclidien.

Montrer qu'il existe une droite affine qui partitionne la plan en deux demi-plans ouverts, %
contenant chacun exactement $500$ points de la famille précédente.
\end{exer}

\begin{exer}[Théorème de Sylvester]
On concidère $n$ points du plan affine Euclidien.

Montrer que, si toute droite menée par deux de ces points en contient un troisième, alors tous ces points sont alignés.
\end{exer}

\begin{exer}
Montrer que le plan affine Euclidien ne peut être partitionné en cercles Euclidiens de rayons non nuls.
\end{exer}

% \chapter{Topologie}
\chapter{Topologie}

\section{Topologie g\'en\'erale}

\begin{exer}
\begin{enumerate}
\item Rappeler la définition axiomatique des ouverts, et des fermés d'un espace topologique -ici le plus souvent, un espace vectoriel normé.

\smallskip
Soit $(E,d)$ un espace métrique, par exemple une partie d'un espace vectoriel normé munie de la distance associée à la norme.
\item Quelle est la caractérisation des ouverts de $E$ par la distance -boules ouvertes ?
\item Montrer que tout fermé de $E$ est intersection d'une suite d'ouverts.
%Indication : utiliser la caractérisation séquentielle des fermés dans un espace métrique.\\
\item Montrer que tout ouvert de $E$ est la réunion d'une suite de fermés.
\item Que penser d'une intersection infinie d'ouverts, d'une union infinie de fermés, dans un espace métrique ?
\end{enumerate}
\end{exer}

\begin{exer}
Soit $(E,d)$ un espace métrique.
\begin{enumerate}
\item L'adhérence d'une boule ouverte est-elle nécessairement la boule fermée de même centre et de même rayon ?
\item Montrer que c'est le cas si $E$ est un espace vectoriel normé.
\end{enumerate}
\end{exer}

\begin{exer}[Ouverts de $\mathbb{R}$]
\begin{enumerate}
\item Montrer qu'un ouvert de $\mathbb{R}$ est réunion disjointe d'une famille d'intervalles ouverts.
%Indication : Considérer les intervalles maximaux de l'ouvert.\\
\item Montrer que cette union est au plus dénombrable.
\end{enumerate}
\end{exer}

\begin{exer}[Le m\^eme, avec des questions interm\'ediaires]
Soit $U$ un ouvert de $\mathbb{R}$. On se propose d'\'etablir le d\'ecomposition suivante :
\[U=\bigsqcup\limits_{d\in D} I_d\]
o\`u $(I_d)$ est une famille d'intervalles ouverts, disjoints deux \`a deux, index\'ee dans un ensemble $D$ au plus d\'enombrable.

On suppose que $U$ est non vide, le cas contraire \'etant trivial.
\begin{enumerate}
\item Soit $x$ un \'el\'ement de $U$. Construire un intervalle $I_x$, incluant tous les intervalles de $\mathbb{R}$, eux-m\^eme inclus dans $U$, qui contiennent $x$.
\item Montrer qu'un intervalle de la forme $I_x$ construite pr\'ec\'edemment est \textit{maximal} dans $U$, c'est-\`a-dire qu'il n'est inclus strictement dans aucun intervalle de $\mathbb{R}$ inclus dans $U$.
\item Que peut-on dire de deux intervalles maximaux de $U$ ?
\item D\'eduire des questions pr\'ec\'edentes que $U$ est l'union disjointe de tous ses intervalles maximaux.
\item Montrer que cette union est au plus d\'enombrable. On utilisera la topologie d'une partie bien choisie de $\mathbb{R}$.
\end{enumerate}
\end{exer}

\begin{exer}
\begin{enumerate}
\item Rappeler la définition des espaces compacts par la propriété de Borel-Lebesgue.

\smallskip
Soit $K$ un espace compact.
\item Montrer que, si $(F_i)$ est une famille de fermés de $K$ dont l'intersection est vide, alors il existe une sous-famille finie de $(F_i)$ d'intersection vide.
\item Montrer qu'une suite décroissante de fermés non vides de $K$ a une intersection non vide.
\end{enumerate}
\end{exer}

\begin{exer}
Soit $K $ un espace m\'etrique compact, par exemple dans un espace vectoriel norm\'e $E$.
\begin{enumerate}
\item Montrer qu'une suite d'\'el\'ements de $K$ converge si et seulement si elle admet une unique valeur d'adh\'erence.
\end{enumerate}
% \newcounter{stock}
\setcounter{stock}{\value{enumi}}
%\smallskip
Soient maintenant $f$ une application continue de $K$ dans lui-m\^eme, et $x_0$ un \'el\'ement de $K$. %
On \'etudie la suite r\'ecurrente $(x_n)_n$, de premier terme $x_0$, et v\'erifiant : %
\[\forall n \in \mathbb{N} , x_{n+1} = f(x_n)\]
On suppose, dans toute la suite de l'exercice, que $(x_n)$ admet exactement deux valeurs d'adh\'erence $z_0$ et $z_1$.
%\smallskip
\begin{enumerate}
\setcounter{enumi}{\value{stock}}
\item Montrer que, quels que soient les voisinages $V_0$ et $V_1$ de $z_0$ et $z_1$ respectivement, %
il existe un entier naturel $N$ tel que :
\[\forall n \in \mathbb{N} , n \geq N \Rightarrow x_n \in V_0 \vee x_n \in V_1\]
\item Soit $\varphi$ une extraction telle que : $(x_{\varphi(n)})_n$ converge vers $x_0$. %
En \'etudiant la suite $(x_{\varphi(n)+1})$, montrer que :
\[f(z_0)=z_1\]
Un raisonnement semblable permet bien sûr de montrer que : $f(z_1)=z_0$.
\item En utilisant les questions pr\'ec\'edentes, montrer que les suites $(x_{2n})_n$ et $(x_{2n+1})$ convergent, %
l'une vers $z_0$, et l'autre vers $z_1$.
\end{enumerate}
\end{exer}

\begin{exer}%Exercice \`a retravailler.
Soit $(K,d)$ un espace métrique compact, par exemple un fermé borné d'un espace vectoriel normé de dimension finie.\\
On considère une isométrie $f$ de $K$ dans lui-même, c'est-à-dire une application de $K$ dans $K$ qui conserve la distance.\\
\linebreak
\begin{enumerate}
\item $f$ admet-elle nécessairement un point fixe ?
\item Montrer que $f$ est surjective.
%Indications : supposer que ce ne soit pas le cas, et considérer un élément $x_0$ de $K$ qui n'est pas dans l'image de $f$. Définir la suite $(x_n)$ des images de $x_0$ par les itérées de $f$ et montrer que $f$ induit une permutation de l'ensemble $L$ des valeurs d'adhérence de $(x_n)$. Utiliser la compacité de $K$ pour montrer que $L$ est non vide. En déduire une contradiction en considérant la suite des distances entre les termes de $(x_n)$ et $L$.\\
%Indications : Supposer que ce ne soit pas le cas, et considérer un point de $K$ dont la distance à l'image de $f$ soit maximale -pourquoi ce point existe-t-il ?-, étudier la suite des itérées de ce point par $f$ pour déduire une contradiction.
\end{enumerate}
\end{exer}

\begin{exer}[Espaces m\'etriques encha\^in\'es]
Dans tout l'exercice, on notera $(E,d)$ un espace m\'etrique.

\smallskip
Pour tout réel strictement positif, on définit la relation d'équivalence $\Re_{\epsilon}$ définie par :
\begin{center}
$x \Re_{\epsilon} y$ si et seulement si il existe une famille finie $(x_k)_{k \in [0,n]}1$ de $n+1$ points de $E$ telle que :
\end{center}
\[x_0 = x \wedge x_n = y \wedge \forall k \in [0,n] , d(x_k x_{k+1}) < \epsilon\]
On peut définir la conjonction, ou intersection de ces relations d'équivalence, que nous noterons $\Re$.\\
On dit que $(E,d)$ est \textit{bien enchaîné} -Cantor-connected- si et seulement si dux éléments de $E$ sont toujours reliés par $\Re$.
\begin{enumerate}
\item Soit : $\epsilon \in \mathbb{R}^{\ast}$. Montrer que les classes modulo $\Re_{\epsilon}$ sont ouvertes, puis qu'elles incluent les conposantes connexes de la topologie de $E$.
\item Montrer qu'un espace métrique connexe pour sa topologie usuelle, est bien enchaîné.
\item Réciproquement, un espace métrique bien enchaîné est-il nécessairement connexe ?
%\item Que peut-on dire si $(E,d)$ est complet ?
\item Montrer que, si $(E,d)$ est compact et bien enchaîné, alors il est connexe.
\end{enumerate}
\end{exer}
\section{Espaces vectoriels norm\'es}

\begin{exer}
Dans tout l'exercice, $K$ est un compact, $E$ est l'espace vectoriel des applications continues de $K$ dans $\mathbb{R}$, muni de la norme infinie.\\
\begin{enumerate}
\item Soit $\varphi$ une application uniformément continue de $\mathbb{R}$ dans $\mathbb{R}$. %
Montrer que $E \rightarrow E : f \mapsto \varphi \circ f$ est uniformément continue. Est-elle linéaire ?
\item Soit $\psi$ une application continue de $K$ dans $K$. %
Montrer que $E \rightarrow E : f \mapsto f \circ \psi$ est linéaire continue, et calculer sa norme.
\end{enumerate}
\end{exer}

\begin{exer}
Soit $\varphi$ une application lin\'eaire entre deux espaces vectoriels norm\'es $E$ et $F$.

Montrer que $\varphi$ est continue si et seulement si elle transforme toute suite qui tend vers $0$ en une suite bornée.
\end{exer}

\begin{exer}
Soit $n$ un entier naturel non nul. %
Soit de plus $G$ l'ensemble des matrices de $\mathcal{M}_n(\mathbb{R})$ triangulaires supérieures et de d\'eterminant $1$.

$G$ est-il fermé, borné, connexe par arcs ?
\end{exer}

\begin{exer}
Soit $E$ un espace vectoriel normé de dimension finie. %
On \'etudie ici, pour tout compact $A$ dans $E$, l'ensemble $L_A$ des endomorphismes de $E$ qui stabilisent $A$.
\begin{enumerate}
\item Pourquoi les \'el\'ements de $L_A$ sont-ils continus ?
\item Montrer que $L_A$ est fermé, quel que soit le compact $A$ de $E$.
\begin{center}
On veut maintenant caract\'eriser les compacts $A$ de $E$ tels que $L_A$ est lui-m\^eme compact.
\end{center}
\item D'apr\`es ce qui pr\'ec\^ede, quelle propri\'et\'e doit-on rechercher sur $A$ pour conclure ? Montrer que cette propri\'et\'e ne d\'epend pas de la norme choisie sur $\mathcal{L}(E)$.
\begin{center}
On suppose que $A$ est un compact de $E$ qui contient les vecteurs d'une base $(e_i)_i$ de $E$. On peut \'ecrire : $Vect A = E$.
\end{center}
\item Montrer que l'application qui \`a une application lin\'eaire $f$ de $E$ dans lui-m\^eme asocie le r\'eel positif : $\max\limits_i \|f(e_i)\|$ est une norme de $\mathcal{L}(E)$.
\item En d\'eduire que $L_A$ est born\'e, conclure.
\item Montrer r\'eciproquement que tout compact $A$ de $E$ tel que $L_A$ est compact v\'erifie : $Vect A = E$. On pourra raisonner par l'absurde.
\end{enumerate}
\end{exer}
\section{Ancien programme : espaces complets}

\begin{exer}
Soit $E$ un espace vectoriel normé.

Montrer que $E$ est un espace de Banach si et seulement si toute série absolument convergente de $E$ est convergente.
\end{exer}

\begin{exer}
%Je donne, si cela s'avère nécessaire, la définition d'un ensemble dénombrable, ainsi que quelques exemples de tels ensembles, sur suggestion de l'élève. Plus particulièrement, j'établis que $\mathbb{Q}$ est dénombrable.\\
Soit $(E,d)$ un espace métrique complet, par exemple un fermé d'un espace de Banach, muni de la distance associée à la norme.
\begin{enumerate}
\item Théorème de Baire : soit $(U_n)_{n \in \mathbb{N}}$ une suite d'ouverts denses de $E$.
Montrer que $\bigcap_{n \in \mathbb{N}}U_n$ est dense dans $E$.
\item Montrer que la réunion d'une suite de fermés d'intérieurs vides de $E$ est d'intérieur vide.
\item Montrer que $\mathbb{R}$ n'est pas dénombrable.
\item Montrer que $\mathbb{R} \backslash \mathbb{Q}$ est dense dans $\mathbb{R}$.

\medskip
On note maintenant $\mathbb{K}$ l'un des deux corps $\mathbb{R}$ ou $\mathbb{C}$.
\item Soit $E$ un espace de Banach sur $\mathbb{K}$. Montrer que $E$ n'admet pas de base dénombrable.
%Indication : Soit $(x_n)$ une éventuelle base dénombrable de $E$. Considérer la suite $(Vect(x_k)_{k \in 0,n})_n$.\\
\item Montrer qu'il n'existe pas de norme pour laquelle $\mathbb{K}[X]$ soit un espace de Banach.
\end{enumerate}
\end{exer}

\begin{exer}
$\mathbb{K}$ est ici le corps $\mathbb{R}$ ou $\mathbb{C}$.\\
On note $l^{\infty}(\mathbb{K})$ l'espace vectoriel des suites bornées de $\mathbb{K}$ muni de la norme $\|.\|_{\infty}$ définie par :
\[\forall (u_n) \in l^{\infty}(\mathbb{K}) , \|(u_n)\|_{\infty} = \sup_{n \in \mathbb{N}} u_n\]
Montrer que $l^{\infty}(\mathbb{K})$ est un espace de Banach.
\end{exer}

\begin{exer}
$\mathbb{K}$ est ici le corps $\mathbb{R}$ ou $\mathbb{C}$. $p \in [1,+\infty[$\\
On note, pour tout réel $p$ supérieur ou égal à $1$, $l^{p}(\mathbb{K})$ l'espace vectoriel des suites $(u_n)$ de $\mathbb{K}$ %
telles que $\sum \lvert u_n \rvert^{p}$ converge, muni de la norme $\|.\|_{p}$ définie par :
\[\forall (u_n) \in l^{p}(\mathbb{K}) , \|(u_n)\|_{p} = (\sum_{n \in \mathbb{N}} \lvert u_n \rvert^{p})^{1/p}\]
\begin{enumerate}
\item Quelles relations d'inclusion existe-t-il entre les espaces $l^{p}(\mathbb{K})$ ?
\item Montrer que ces espaces vectoriels normés sont de Banach.
\end{enumerate}
%Eventuellement, afin de faciliter les calculs, je demande de se restreindre au cas de $l^{1}(\mathbb{K})$.
\end{exer}

\begin{exer}
Soit $E$ un espace vectoriel normé.\\
Montrer que $E$ est un espace de Banach si et seulement si toute série absolument convergente de $E$ est convergente.
\end{exer}

\begin{exer}
$\mathbb{K}$ est ici le corps $\mathbb{R}$ ou $\mathbb{C}$.\\
On note $l^{\infty}(\mathbb{K})$ l'espace vectoriel des suites bornées de $\mathbb{K}$ muni de la norme %
$\|.\|_{\infty}$ définie par :
\[\forall (u_n) \in l^{\infty}(\mathbb{K}) , \|(u_n)\|_{\infty} = \sup_{n \in \mathbb{N}} u_n\]
Montrer que $l^{\infty}(\mathbb{K})$ est un espace de Banach.
\end{exer}

\begin{exer}[Dual de $l^1$]
On se place dans l'espace des suites réelles $(u_n)_n$ telles que la série $\sum_n u_n$ est absolument convergente, nous notons cet espace $l^1(\mathbb{R})$ et le munissons de la norme $\| \|_1$ définie par : $\forall (u_n)_n \in l^1 , \| (u_n) \|_1 = \sum\limits_{n=0}^{+\infty} | u_n |$. On appelle dual topologique de cet espace, et note $(l^1)'$, l'espace vectoriel des formes linéaires continues de $l^1$, que nous munirons de la norme subordonnée à $\| \|_1$à la source, $| |$ à l'arrivée.\\
Montrer qu'il existe une isométrie linaire $\phi$ de l'espace $l^{\infty}$ des suites réelles bornées muni de la norme infinie, sur $(l^1)'$, telle que :
\[\forall (a_n)_n \in l^{\infty} , \forall (u_n) \in l^1 , \phi ((a_n)_n) ((u_n)_n) = \sum\limits_{n=0}^{+\infty} a_n u_n\]
\end{exer}

\begin{exer}
\begin{enumerate}
\item Enoncer le th\'eor\`eme de convergence domin\'ee de Lebesgue.
\item Rappeler la d\'efinition de l'int\'egrabilit\'e au sens de Riemann sur un segment.
\item Que dire de l'int\'egrale d'une limite uniforme de fonctions Riemann-int\'egrables ?

\medskip
Soit maintenant $I$ un intervalle r\'eel. On appelle fonction localement int\'egrable sur $I$ une fonction int\'egrable sur tout segment de $I$. %
On peut se ramener au cas au cas d'un intervalle $I$ de la forme $[a,b[$, o\`u $a$ est un nombre r\'eel, et $b$ un e\'el\'ement de $\mathbb{R}\cup\{+\infty\}$, strictement sup\'erieur \`a $a$.
\item Montrer le th\'eor\`eme de convergence dominée pour l'intégrale de Riemann impropre :

\medskip
\fbox{
\begin{minipage}{15cm}
Soit $(f_n)$ une suite de fonctions r\'eelles d\'efinies sur $I$, localement int\'egrables et domin\'ees par une fonction $\phi$, positive et int\'egrable sur $I$. On note que les int\'egrales des termes de $(f_n)$ convergent. %
Si $(f_n)$ converge vers une limite $f$ d\'efinie sur $I$, alors $f$ est localement int\'egrable sur $I$, son int\'egrale sur $I$ converge et :
\[\lim_n \int\limits_a^b f_n(t) dt = \int\limits_a^b f(t) dt\]
\end{minipage}
}
\end{enumerate}
\end{exer}

\begin{exer}
Soit $f$ une fonction réelle définie sur $\mathbb{R}$, telle que $f$ et $f'^2$ soient intégrables.\\
Montrer que $f$ tend vers $0$ en $+ \infty$ et $- \infty$.
%Indication : penser au critère de Cauchy. On pourra utiliser le résultat de l'exercice précédent.
\end{exer}



% \section{Topologie g\'en\'erale}

\begin{exer}
\begin{enumerate}
\item Rappeler la définition axiomatique des ouverts, et des fermés d'un espace topologique -ici le plus souvent, un espace vectoriel normé.

\smallskip
Soit $(E,d)$ un espace métrique, par exemple une partie d'un espace vectoriel normé munie de la distance associée à la norme.
\item Quelle est la caractérisation des ouverts de $E$ par la distance -boules ouvertes ?
\item Montrer que tout fermé de $E$ est intersection d'une suite d'ouverts.
%Indication : utiliser la caractérisation séquentielle des fermés dans un espace métrique.\\
\item Montrer que tout ouvert de $E$ est la réunion d'une suite de fermés.
\item Que penser d'une intersection infinie d'ouverts, d'une union infinie de fermés, dans un espace métrique ?
\end{enumerate}
\end{exer}

\begin{exer}
Soit $(E,d)$ un espace métrique.
\begin{enumerate}
\item L'adhérence d'une boule ouverte est-elle nécessairement la boule fermée de même centre et de même rayon ?
\item Montrer que c'est le cas si $E$ est un espace vectoriel normé.
\end{enumerate}
\end{exer}

\begin{exer}[Ouverts de $\mathbb{R}$]
\begin{enumerate}
\item Montrer qu'un ouvert de $\mathbb{R}$ est réunion disjointe d'une famille d'intervalles ouverts.
%Indication : Considérer les intervalles maximaux de l'ouvert.\\
\item Montrer que cette union est au plus dénombrable.
\end{enumerate}
\end{exer}

\begin{exer}[Le m\^eme, avec des questions interm\'ediaires]
Soit $U$ un ouvert de $\mathbb{R}$. On se propose d'\'etablir le d\'ecomposition suivante :
\[U=\bigsqcup\limits_{d\in D} I_d\]
o\`u $(I_d)$ est une famille d'intervalles ouverts, disjoints deux \`a deux, index\'ee dans un ensemble $D$ au plus d\'enombrable.

On suppose que $U$ est non vide, le cas contraire \'etant trivial.
\begin{enumerate}
\item Soit $x$ un \'el\'ement de $U$. Construire un intervalle $I_x$, incluant tous les intervalles de $\mathbb{R}$, eux-m\^eme inclus dans $U$, qui contiennent $x$.
\item Montrer qu'un intervalle de la forme $I_x$ construite pr\'ec\'edemment est \textit{maximal} dans $U$, c'est-\`a-dire qu'il n'est inclus strictement dans aucun intervalle de $\mathbb{R}$ inclus dans $U$.
\item Que peut-on dire de deux intervalles maximaux de $U$ ?
\item D\'eduire des questions pr\'ec\'edentes que $U$ est l'union disjointe de tous ses intervalles maximaux.
\item Montrer que cette union est au plus d\'enombrable. On utilisera la topologie d'une partie bien choisie de $\mathbb{R}$.
\end{enumerate}
\end{exer}

\begin{exer}
\begin{enumerate}
\item Rappeler la définition des espaces compacts par la propriété de Borel-Lebesgue.

\smallskip
Soit $K$ un espace compact.
\item Montrer que, si $(F_i)$ est une famille de fermés de $K$ dont l'intersection est vide, alors il existe une sous-famille finie de $(F_i)$ d'intersection vide.
\item Montrer qu'une suite décroissante de fermés non vides de $K$ a une intersection non vide.
\end{enumerate}
\end{exer}

\begin{exer}
Soit $K $ un espace m\'etrique compact, par exemple dans un espace vectoriel norm\'e $E$.
\begin{enumerate}
\item Montrer qu'une suite d'\'el\'ements de $K$ converge si et seulement si elle admet une unique valeur d'adh\'erence.
\end{enumerate}
% \newcounter{stock}
\setcounter{stock}{\value{enumi}}
%\smallskip
Soient maintenant $f$ une application continue de $K$ dans lui-m\^eme, et $x_0$ un \'el\'ement de $K$. %
On \'etudie la suite r\'ecurrente $(x_n)_n$, de premier terme $x_0$, et v\'erifiant : %
\[\forall n \in \mathbb{N} , x_{n+1} = f(x_n)\]
On suppose, dans toute la suite de l'exercice, que $(x_n)$ admet exactement deux valeurs d'adh\'erence $z_0$ et $z_1$.
%\smallskip
\begin{enumerate}
\setcounter{enumi}{\value{stock}}
\item Montrer que, quels que soient les voisinages $V_0$ et $V_1$ de $z_0$ et $z_1$ respectivement, %
il existe un entier naturel $N$ tel que :
\[\forall n \in \mathbb{N} , n \geq N \Rightarrow x_n \in V_0 \vee x_n \in V_1\]
\item Soit $\varphi$ une extraction telle que : $(x_{\varphi(n)})_n$ converge vers $x_0$. %
En \'etudiant la suite $(x_{\varphi(n)+1})$, montrer que :
\[f(z_0)=z_1\]
Un raisonnement semblable permet bien sûr de montrer que : $f(z_1)=z_0$.
\item En utilisant les questions pr\'ec\'edentes, montrer que les suites $(x_{2n})_n$ et $(x_{2n+1})$ convergent, %
l'une vers $z_0$, et l'autre vers $z_1$.
\end{enumerate}
\end{exer}

\begin{exer}%Exercice \`a retravailler.
Soit $(K,d)$ un espace métrique compact, par exemple un fermé borné d'un espace vectoriel normé de dimension finie.\\
On considère une isométrie $f$ de $K$ dans lui-même, c'est-à-dire une application de $K$ dans $K$ qui conserve la distance.\\
\linebreak
\begin{enumerate}
\item $f$ admet-elle nécessairement un point fixe ?
\item Montrer que $f$ est surjective.
%Indications : supposer que ce ne soit pas le cas, et considérer un élément $x_0$ de $K$ qui n'est pas dans l'image de $f$. Définir la suite $(x_n)$ des images de $x_0$ par les itérées de $f$ et montrer que $f$ induit une permutation de l'ensemble $L$ des valeurs d'adhérence de $(x_n)$. Utiliser la compacité de $K$ pour montrer que $L$ est non vide. En déduire une contradiction en considérant la suite des distances entre les termes de $(x_n)$ et $L$.\\
%Indications : Supposer que ce ne soit pas le cas, et considérer un point de $K$ dont la distance à l'image de $f$ soit maximale -pourquoi ce point existe-t-il ?-, étudier la suite des itérées de ce point par $f$ pour déduire une contradiction.
\end{enumerate}
\end{exer}

\begin{exer}[Espaces m\'etriques encha\^in\'es]
Dans tout l'exercice, on notera $(E,d)$ un espace m\'etrique.

\smallskip
Pour tout réel strictement positif, on définit la relation d'équivalence $\Re_{\epsilon}$ définie par :
\begin{center}
$x \Re_{\epsilon} y$ si et seulement si il existe une famille finie $(x_k)_{k \in [0,n]}1$ de $n+1$ points de $E$ telle que :
\end{center}
\[x_0 = x \wedge x_n = y \wedge \forall k \in [0,n] , d(x_k x_{k+1}) < \epsilon\]
On peut définir la conjonction, ou intersection de ces relations d'équivalence, que nous noterons $\Re$.\\
On dit que $(E,d)$ est \textit{bien enchaîné} -Cantor-connected- si et seulement si dux éléments de $E$ sont toujours reliés par $\Re$.
\begin{enumerate}
\item Soit : $\epsilon \in \mathbb{R}^{\ast}$. Montrer que les classes modulo $\Re_{\epsilon}$ sont ouvertes, puis qu'elles incluent les conposantes connexes de la topologie de $E$.
\item Montrer qu'un espace métrique connexe pour sa topologie usuelle, est bien enchaîné.
\item Réciproquement, un espace métrique bien enchaîné est-il nécessairement connexe ?
%\item Que peut-on dire si $(E,d)$ est complet ?
\item Montrer que, si $(E,d)$ est compact et bien enchaîné, alors il est connexe.
\end{enumerate}
\end{exer}
% \section{Espaces vectoriels norm\'es}

\begin{exer}
Dans tout l'exercice, $K$ est un compact, $E$ est l'espace vectoriel des applications continues de $K$ dans $\mathbb{R}$, muni de la norme infinie.\\
\begin{enumerate}
\item Soit $\varphi$ une application uniformément continue de $\mathbb{R}$ dans $\mathbb{R}$. %
Montrer que $E \rightarrow E : f \mapsto \varphi \circ f$ est uniformément continue. Est-elle linéaire ?
\item Soit $\psi$ une application continue de $K$ dans $K$. %
Montrer que $E \rightarrow E : f \mapsto f \circ \psi$ est linéaire continue, et calculer sa norme.
\end{enumerate}
\end{exer}

\begin{exer}
Soit $\varphi$ une application lin\'eaire entre deux espaces vectoriels norm\'es $E$ et $F$.

Montrer que $\varphi$ est continue si et seulement si elle transforme toute suite qui tend vers $0$ en une suite bornée.
\end{exer}

\begin{exer}
Soit $n$ un entier naturel non nul. %
Soit de plus $G$ l'ensemble des matrices de $\mathcal{M}_n(\mathbb{R})$ triangulaires supérieures et de d\'eterminant $1$.

$G$ est-il fermé, borné, connexe par arcs ?
\end{exer}

\begin{exer}
Soit $E$ un espace vectoriel normé de dimension finie. %
On \'etudie ici, pour tout compact $A$ dans $E$, l'ensemble $L_A$ des endomorphismes de $E$ qui stabilisent $A$.
\begin{enumerate}
\item Pourquoi les \'el\'ements de $L_A$ sont-ils continus ?
\item Montrer que $L_A$ est fermé, quel que soit le compact $A$ de $E$.
\begin{center}
On veut maintenant caract\'eriser les compacts $A$ de $E$ tels que $L_A$ est lui-m\^eme compact.
\end{center}
\item D'apr\`es ce qui pr\'ec\^ede, quelle propri\'et\'e doit-on rechercher sur $A$ pour conclure ? Montrer que cette propri\'et\'e ne d\'epend pas de la norme choisie sur $\mathcal{L}(E)$.
\begin{center}
On suppose que $A$ est un compact de $E$ qui contient les vecteurs d'une base $(e_i)_i$ de $E$. On peut \'ecrire : $Vect A = E$.
\end{center}
\item Montrer que l'application qui \`a une application lin\'eaire $f$ de $E$ dans lui-m\^eme asocie le r\'eel positif : $\max\limits_i \|f(e_i)\|$ est une norme de $\mathcal{L}(E)$.
\item En d\'eduire que $L_A$ est born\'e, conclure.
\item Montrer r\'eciproquement que tout compact $A$ de $E$ tel que $L_A$ est compact v\'erifie : $Vect A = E$. On pourra raisonner par l'absurde.
\end{enumerate}
\end{exer}

% \chapter{Suites et s\'eries num\'eriques}
\chapter{Suites et s\'eries num\'eriques}

\section{Suites numériques}

\begin{exer}
Donner une condition nécessaire et suffisante sur une suite réelle $(u_n)_n$ pour qu'il existe %
une permutation $\sigma$ de $\mathbb{N}$ telle que $(u_{\sigma (n)})_{n \in \mathbb{N}}$ soit ultimement monotone.
\end{exer}

\begin{exer}
Déterminer le nombre de chemins permettant de joindre deux sommets opposés d'un cube %
-une étape est définie par le franchissement d'une arète.
%\textit{Je demande d'étudier le problème afin de trouver une formalisation, je m'attends à une question de récurrence linéaire. Eu égard à la complexité des matrices étudiées, je ne demande pas de calcul explicite mais, éventuellement, une version manuscrite d'un code Maple adéquat.}
\end{exer}

\begin{exer}
Soient $(u_n)$ et $(v_n)$ deux suites réelles telles que $0$ soit une valeur d'adhérence de $(u_n v_n)$.

Montrer que $0$ est une valeur d'adhérence de $(u_n)$ ou de $(v_n)$.
\end{exer}

\begin{exer}
Soit $(x_n)$ une suite d'un espace vectoriel normé -le résultat est vrai dans un espace topologique quelconque-, %
telle que$(x_{2n})$, $(x_{2n+1})$ et $(x_{3n})$ convergent.

Montrer que $(x_n)$ converge.
\end{exer}

\section{S\'eries num\'eriques}

\begin{exer}
D\'eterminer la nature des s\'eries dont le terme g\'en\'eral est donn\'e ci-dessous :
\begin{itemize}
\item $n^{\frac{1}{n}}-1$
\item $\frac{(-1)^n}{\sqrt[n]{n!}}$
\end{itemize}
\end{exer}

\begin{exer}
Montrer que les s\'eries dont le terme g\'en\'eral est donn\'e ci-dessous sont convergentes, et calculer leurs sommes.
\begin{enumerate}
\item $\ln(1-\frac{1}{n^2})$
\item $\frac{1}{n}$ lorsque $n$ est un carr\'e; $\frac{1}{n^2}$ sinon.
\end{enumerate}
\end{exer}

\begin{exer}
Soit $(a_n)_{n \in \mathbb{N}}$ une suite de r\'eels positifs tels que : $\sum a_n$ converge.
\begin{enumerate}
\item Soit $\alpha$ un r\'eel strictement sup\'erieur \`a $\frac{1}{2}$. %
Montrer que la s\'erie $\sum \frac{\sqrt{a_n}}{n^{\alpha}}$ converge.
\item Que dire du cas o\`u : $\alpha = \frac{1}{2}$ ?
\end{enumerate}
\end{exer}

\begin{exer}
Soit $(u_n)$ une suite décroissante de réels positifs qui tend vers $0$.
\begin{enumerate}
\item Montrer que les séries $\sum u_n$ et $\sum n(u_n - u_{n+1})$ sont de même nature.
\item Montrer de plus que lorsque ces séries convergent, elles ont même somme.
\end{enumerate}
\end{exer}

\begin{exer}
Donner la nature, lorsque cela est possible, de la série de terme général $n! \prod_{k=1}^n \sin(\frac{x}{k})$, %
où $x$ est un réel quelconque.
\end{exer}

\begin{exer}
\begin{enumerate}
\item Montrer que si une série $\sum x_n$ d'éléments d'un espace de Banach est absolument convergente, %
alors elle est commutativement convergente, c'est-à-dire :

%\begin{center}
Pour toute permutation $\sigma$ de $\mathbb{N}$, la série $\sum x_{\sigma(n)}$ est convergente, %
de somme $\sum\limits_{n=0}^{\infty} x_n$.
%\end{center}

\medskip
On se propose maintenant de démontrer la réciproque de ce résultat dans le cas des séries réelles. %
Soit $(u_n)$ une suite réelle telle que la série de terme général $u_n$ soit semi-convergente.
\item Démontrer le théorème de réarrangement de Riemann : Pour tout réel $\alpha$, %
il existe une permutation $\sigma$ de $\mathbb{N}$ telle que la série %
de terme général $u_{\sigma(n)}$ soit convergente de somme $\alpha$.
%Indication : Soient $E_+$ et $E_-$ les ensembles définis par : $E_+ = \{ n \in \mathbb{N} \arrowvert u_n \geq 0 \}$, $E_- = \mathbb{N} \backslash E_+$. Il existe deux bijections croissantes $\varphi$ et $\psi$ de $\mathbb{N}$, respectivement sur $E_+$ et $E_-$. Montrer que les suites $(\Sigma_{k=0}^n u_{\varphi (k)})_{n \in \mathbb{N}}$ et $(\Sigma_{k=0}^n u_{\psi (k)})_{n \in \mathbb{N}}$ tendent respectivement vers $+ \infty$ et $- \infty$.\\
\item Montrer qu'il existe une permutation $\sigma$ de $\mathbb{N}$ telle que %
la suite des sommes partielles de la série de terme général $u_{\sigma(n)}$ n'admette pas de limite %
dans $\mathbb{R} \bigcup \{ +\infty ,-\infty \}$.
%Indication : Ici encore, on utilise les suites $(u_{\varphi(n)})$ et $(u_{\psi(n)})$.
\end{enumerate}
\end{exer}

\begin{exer}
\begin{enumerate}
\item Soit $(q_n)$ une suite croissante d'entiers strictement supérieurs à $1$. %
Montrer que la série de terme général $\frac{1}{\Pi_{j=0}^n q_j}$ converge vers un élément de $]0,1]$, %
que nous noterons $\varphi ((q_n)_{n \in \mathbb{N}})$.
\item A titre d'exemple, soit $k$ un entier naturel non nul. On constate que $1/k$ est élément de $]0,1]$. %
Décomposer ce nombre en somme d'une telle série.
\item Montrer que l'application $\varphi$ de l'ensemble $S$ des suites croissantes de %
$\mathbb{N} \backslash \{0,1\}$ dans $]0,1]$ définie à la question 1. est bijective.

L'antécédent par $\varphi$ d'un élément $x$ de $]0,1]$ est appelé \textit{développement de $x$ en série de Engel}.
\item\label{rateng} A quelle condition sur $(q_n)$ le réel $\varphi ((q_n))$ est-il rationel ?

On appelle nombre Egyptien l'inverse d'un entier naturel non nul.
\item Montrer que tout rationel de $]0,1]$ s'écrit d'une unique manière comme somme d'une suite finie de nombres Egyptiens distincts.
\item Déduire de la question \ref{rateng} l'irrationalité d'un réel bien connu en analyse.
%Indication : Il s'agit de $e$.
\end{enumerate}
\end{exer}

\begin{exer}[Convergence et densit\'e pour une suite d'entiers naturels]
Soit $A$ un ensemble d'entiers naturels. On dit que $A$ admet une densité naturelle sus $\mathbb{N}$ si et seulement si :
\[\left(\frac{\sharp A \cap [\![1,n]\!]}{n}\right)_n\text{ admet une limite en $+ \infty$.}\]
Lorsque cette limite existe, nous l'appelons densité de $A$, et la notons $d(A)$.

Soit maintenant $(a_n)_n$ une suite strictement croissante d'entiers natuels non nuls. %
Montrer que si la série $\sum \frac{1}{a_k}$ converge, %
alors l'ensemble des termes de $(a_k)$ admet, dans $\mathbb{N}$, une densit\'e naturelle égale à $0$.
\end{exer}

\begin{exer}[Crit\g{e}re de condensation de Cauchy]
\begin{enumerate}
\item Soit $(u_n)_n$ une suite de réels décroisssante qui tend vers $0$ en $+ \infty$. %
Montrer que, pour tout entier $p$ strictement supérieur à $1$, %
la série $\sum u_n$ converge si et seulement si la série $\sum p^n u_{p^n}$ converge.
\item Soit $(u_n)_n$ une suite de réels positifs telle que $\sum u_n$ diverge. %
Montrer que $\sum min(u_n,1/n)$ diverge.
\end{enumerate}
\end{exer}

\begin{exer}
Déterminer la nature de la série des entiers naturels qui s'écrivent, en base $10$, sans le chiffre $9$.
\end{exer}

\begin{exer}
Soient $(a_n)$ une suite d'éléments d'un espace de Banach telle que %
$\left(\sum\limits_{k=0}^n a_k\right)_{n \in \mathbb{N}}$ soit bornée, %
et $(\epsilon_n)$ une suite réelle décroissante qui converge vers $0$.
\begin{enumerate}
\item Montrer que la série de terme général $a_n \epsilon_n$ converge. Cette règle est appelée critère de convergence d'Abel.
\item Démontrer, en utilisant la question précédente, le critère spécial des séries alternées.
\end{enumerate}
\end{exer}


% \section{Suites numériques}

\begin{exer}
Donner une condition nécessaire et suffisante sur une suite réelle $(u_n)_n$ pour qu'il existe %
une permutation $\sigma$ de $\mathbb{N}$ telle que $(u_{\sigma (n)})_{n \in \mathbb{N}}$ soit ultimement monotone.
\end{exer}

\begin{exer}
Déterminer le nombre de chemins permettant de joindre deux sommets opposés d'un cube %
-une étape est définie par le franchissement d'une arète.
%\textit{Je demande d'étudier le problème afin de trouver une formalisation, je m'attends à une question de récurrence linéaire. Eu égard à la complexité des matrices étudiées, je ne demande pas de calcul explicite mais, éventuellement, une version manuscrite d'un code Maple adéquat.}
\end{exer}

\begin{exer}
Soient $(u_n)$ et $(v_n)$ deux suites réelles telles que $0$ soit une valeur d'adhérence de $(u_n v_n)$.

Montrer que $0$ est une valeur d'adhérence de $(u_n)$ ou de $(v_n)$.
\end{exer}

\begin{exer}
Soit $(x_n)$ une suite d'un espace vectoriel normé -le résultat est vrai dans un espace topologique quelconque-, %
telle que$(x_{2n})$, $(x_{2n+1})$ et $(x_{3n})$ convergent.

Montrer que $(x_n)$ converge.
\end{exer}

% \section{S\'eries num\'eriques}

\begin{exer}
D\'eterminer la nature des s\'eries dont le terme g\'en\'eral est donn\'e ci-dessous :
\begin{itemize}
\item $n^{\frac{1}{n}}-1$
\item $\frac{(-1)^n}{\sqrt[n]{n!}}$
\end{itemize}
\end{exer}

\begin{exer}
Montrer que les s\'eries dont le terme g\'en\'eral est donn\'e ci-dessous sont convergentes, et calculer leurs sommes.
\begin{enumerate}
\item $\ln(1-\frac{1}{n^2})$
\item $\frac{1}{n}$ lorsque $n$ est un carr\'e; $\frac{1}{n^2}$ sinon.
\end{enumerate}
\end{exer}

\begin{exer}
Soit $(a_n)_{n \in \mathbb{N}}$ une suite de r\'eels positifs tels que : $\sum a_n$ converge.
\begin{enumerate}
\item Soit $\alpha$ un r\'eel strictement sup\'erieur \`a $\frac{1}{2}$. %
Montrer que la s\'erie $\sum \frac{\sqrt{a_n}}{n^{\alpha}}$ converge.
\item Que dire du cas o\`u : $\alpha = \frac{1}{2}$ ?
\end{enumerate}
\end{exer}

\begin{exer}
Soit $(u_n)$ une suite décroissante de réels positifs qui tend vers $0$.
\begin{enumerate}
\item Montrer que les séries $\sum u_n$ et $\sum n(u_n - u_{n+1})$ sont de même nature.
\item Montrer de plus que lorsque ces séries convergent, elles ont même somme.
\end{enumerate}
\end{exer}

\begin{exer}
Donner la nature, lorsque cela est possible, de la série de terme général $n! \prod_{k=1}^n \sin(\frac{x}{k})$, %
où $x$ est un réel quelconque.
\end{exer}

\begin{exer}
\begin{enumerate}
\item Montrer que si une série $\sum x_n$ d'éléments d'un espace de Banach est absolument convergente, %
alors elle est commutativement convergente, c'est-à-dire :

%\begin{center}
Pour toute permutation $\sigma$ de $\mathbb{N}$, la série $\sum x_{\sigma(n)}$ est convergente, %
de somme $\sum\limits_{n=0}^{\infty} x_n$.
%\end{center}

\medskip
On se propose maintenant de démontrer la réciproque de ce résultat dans le cas des séries réelles. %
Soit $(u_n)$ une suite réelle telle que la série de terme général $u_n$ soit semi-convergente.
\item Démontrer le théorème de réarrangement de Riemann : Pour tout réel $\alpha$, %
il existe une permutation $\sigma$ de $\mathbb{N}$ telle que la série %
de terme général $u_{\sigma(n)}$ soit convergente de somme $\alpha$.
%Indication : Soient $E_+$ et $E_-$ les ensembles définis par : $E_+ = \{ n \in \mathbb{N} \arrowvert u_n \geq 0 \}$, $E_- = \mathbb{N} \backslash E_+$. Il existe deux bijections croissantes $\varphi$ et $\psi$ de $\mathbb{N}$, respectivement sur $E_+$ et $E_-$. Montrer que les suites $(\Sigma_{k=0}^n u_{\varphi (k)})_{n \in \mathbb{N}}$ et $(\Sigma_{k=0}^n u_{\psi (k)})_{n \in \mathbb{N}}$ tendent respectivement vers $+ \infty$ et $- \infty$.\\
\item Montrer qu'il existe une permutation $\sigma$ de $\mathbb{N}$ telle que %
la suite des sommes partielles de la série de terme général $u_{\sigma(n)}$ n'admette pas de limite %
dans $\mathbb{R} \bigcup \{ +\infty ,-\infty \}$.
%Indication : Ici encore, on utilise les suites $(u_{\varphi(n)})$ et $(u_{\psi(n)})$.
\end{enumerate}
\end{exer}

\begin{exer}
\begin{enumerate}
\item Soit $(q_n)$ une suite croissante d'entiers strictement supérieurs à $1$. %
Montrer que la série de terme général $\frac{1}{\Pi_{j=0}^n q_j}$ converge vers un élément de $]0,1]$, %
que nous noterons $\varphi ((q_n)_{n \in \mathbb{N}})$.
\item A titre d'exemple, soit $k$ un entier naturel non nul. On constate que $1/k$ est élément de $]0,1]$. %
Décomposer ce nombre en somme d'une telle série.
\item Montrer que l'application $\varphi$ de l'ensemble $S$ des suites croissantes de %
$\mathbb{N} \backslash \{0,1\}$ dans $]0,1]$ définie à la question 1. est bijective.

L'antécédent par $\varphi$ d'un élément $x$ de $]0,1]$ est appelé \textit{développement de $x$ en série de Engel}.
\item\label{rateng} A quelle condition sur $(q_n)$ le réel $\varphi ((q_n))$ est-il rationel ?

On appelle nombre Egyptien l'inverse d'un entier naturel non nul.
\item Montrer que tout rationel de $]0,1]$ s'écrit d'une unique manière comme somme d'une suite finie de nombres Egyptiens distincts.
\item Déduire de la question \ref{rateng} l'irrationalité d'un réel bien connu en analyse.
%Indication : Il s'agit de $e$.
\end{enumerate}
\end{exer}

\begin{exer}[Convergence et densit\'e pour une suite d'entiers naturels]
Soit $A$ un ensemble d'entiers naturels. On dit que $A$ admet une densité naturelle sus $\mathbb{N}$ si et seulement si :
\[\left(\frac{\sharp A \cap [\![1,n]\!]}{n}\right)_n\text{ admet une limite en $+ \infty$.}\]
Lorsque cette limite existe, nous l'appelons densité de $A$, et la notons $d(A)$.

Soit maintenant $(a_n)_n$ une suite strictement croissante d'entiers natuels non nuls. %
Montrer que si la série $\sum \frac{1}{a_k}$ converge, %
alors l'ensemble des termes de $(a_k)$ admet, dans $\mathbb{N}$, une densit\'e naturelle égale à $0$.
\end{exer}

\begin{exer}[Crit\g{e}re de condensation de Cauchy]
\begin{enumerate}
\item Soit $(u_n)_n$ une suite de réels décroisssante qui tend vers $0$ en $+ \infty$. %
Montrer que, pour tout entier $p$ strictement supérieur à $1$, %
la série $\sum u_n$ converge si et seulement si la série $\sum p^n u_{p^n}$ converge.
\item Soit $(u_n)_n$ une suite de réels positifs telle que $\sum u_n$ diverge. %
Montrer que $\sum min(u_n,1/n)$ diverge.
\end{enumerate}
\end{exer}

\begin{exer}
Déterminer la nature de la série des entiers naturels qui s'écrivent, en base $10$, sans le chiffre $9$.
\end{exer}

\begin{exer}
Soient $(a_n)$ une suite d'éléments d'un espace de Banach telle que %
$\left(\sum\limits_{k=0}^n a_k\right)_{n \in \mathbb{N}}$ soit bornée, %
et $(\epsilon_n)$ une suite réelle décroissante qui converge vers $0$.
\begin{enumerate}
\item Montrer que la série de terme général $a_n \epsilon_n$ converge. Cette règle est appelée critère de convergence d'Abel.
\item Démontrer, en utilisant la question précédente, le critère spécial des séries alternées.
\end{enumerate}
\end{exer}

% \chapter{Analyse sur des fonctions}
\chapter{Analyse sur des fonctions}
% \include{analyse_functions/planche_c}

\section{Suites et s\'eries de fonctions}
\section{Suites et séries de fonctions}

\begin{exer}
Posons : $(a,b) \in \mathbb{R}$ $a < b$.
\begin{enumerate}
\item On dit qu'une application de $[a,b]$ dans $\mathbb{R}$ est réglée si et seulement si %
elle est limite uniforme d'une suite de fonctions en escalier.
\begin{enumerate}
\item Soit : $f \in \mathbb{R}^{[a,b]}$. Montrer que $f$ est réglée si et seulement si elle satisfait la condition : %
$f$ admet en tout point de $[a,b]$ une limite à gauche et une limite à droite, lorsque ces limites sont définies.
%Indications : (i) implique (ii) d'après le cours. Pour montrer que (ii) implique (i), utiliser la définition des limites à gauche et à droite puis constater que $[a,b]$ vérifie la propriété de Borel-Lebesgue. J'indique que ce raisonnement généralise la démonstration du théorème d'approximation des applications continues définies sur un segment par des fonctions en escalier, dont le point clé est l'utilisation du théorème de Heine, qui peut se démontrer avec Borel-Lebesgue.\\
\item Montrer que l'ensemble des points de discontinuité d'une telle application est au plus dénombrable.
%Indication : La réunion d'une famille dénombrable d'ensembles finis est au plus dénombrable.\\
\end{enumerate}
\item On s'int\'eresse maintenant aux fonctions int\'egrables au sens de Riemann
\begin{enumerate}
\item Rappeler la définition de l'intégrabilité au sens de Riemann d'une application $f$ de $[a,b]$ dans $\mathbb{R}$.
%\smallskip
%\textit{J'introduis ici les notations : $E^+(f)$, respectivement $E^-(f)$, %
%est l'ensemble des applications en escalier de $[a,b]$ dans $\mathbb{R}$ majorant, respectivement minorant $f$.}
Pour toute fonction r\'eelle $f$, on note $E^+(f)$, respectivement $E^-(f)$, l'ensembe de fonctions en escalier d\'efinies sur $[a,b]$ et inf\'erieures, respectivement sup\'erieures, \g{a} $f$ sur $[a,b]$.
\item Montrer que, lorsque $E^+(f)$ et $E^-(f)$ sont tous les deux non vides, %
que $\inf_{\varphi \in E^+(f)}(\int \varphi)$ et $\sup_{\varphi \in E^-(f)}(\int \varphi)$ %
sont bien définies dans $\mathbb{R}$ et vérifient : %
$\inf_{\varphi \in E^+(f)}(\int \varphi) \geq \sup_{\varphi \in E^-(f)}(\int \varphi)$
\end{enumerate}
\item Montrer que toute application réglée $f$ de $[a,b]$ dans $\mathbb{R}$ est intégrable au sens de Riemann.
%En particulier, une telle application est bornée sauf sur un ensemble dénombreable.
%Indications : On montre simultanément que $E^+(f)$ et $E^-(f)$ sont tous les deux non vides, et que :
%\[\forall \epsilon \in \mathbb{R}_+^* , \inf_{\varphi \in E^+(f)}(\int \varphi) - \sup_{\varphi \in E^-(f)}(\int \varphi) \leq \epsilon\]
%Pour cela, poser : $\epsilon \in \mathbb{R}_+^*$ ; remarquer qu'il existe une application $\varphi$ en escalier de $[a,b]$ dans $\mathbb{R}$ située à une distance inférieure à $\frac{\epsilon}{b-a}$ de $f$, puis encadrer $f$ par $\varphi + \frac{\epsilon}{b-a}$ et $\varphi - \frac{\epsilon}{b-a}$.
\end{enumerate}
\end{exer}

\begin{exer}
Etudier la convergence de la s\'erie de fonctions :
\[\sum x\mapsto \frac{x\exp -nx}{\ln n}\]
d\'efinies sur $\mathbb{R}$, pour tout entier $n$ strictement positif.
\end{exer}

\begin{exer}
Soit $S$ la fonction, somme de la s\'erie : $\sum x\mapsto nx\exp(-nx^2)$. 
\begin{enumerate}
\item Donner le domaine de d\'efinition de $S$.
\item La convergence de la s\'erie est-elle normale, uniforme, sur ce domaine ?
\item Exprimer $S$ \`a l'aide de fonctions usuelles.
\end{enumerate}
\end{exer}

\begin{exer}
Soit $E$ l'espace vectoriel des suites r\'eelles born\'ees, %
que l'on munit de sa norme de convergence uniforme, $\| \|_{\infty}$.

D\'eterminer si les sous-ensembles qui suivent sont ferm\'es ou non :
\begin{itemize}
\item L'ensemble $F$ des suites croissantes.
\item L'ensemble $c_0$ des suites qui convergent vers $0$.
\item L'ensemble $V_0$ des suites qui admettent $0$ comme valeur d'adh\'erence.
\item L'ensemble $S$ des suites sommables.
\item L'ensemble $P$ des suites p\'eriodiques.
\item Bonus : l'ensemble $c$ des suites convergentes.
\end{itemize}
\end{exer}

%\textit{Le détail des questions dans les deux exercices qui suivent est seulement suggéré. Je laisse le soin à l'élève de réfléchir personnellement à ces deux problèmes, qui concernent les applications de classe $C^{\infty}$ de $\mathbb{R}$ dans lui-même.}

\begin{exer}
\begin{enumerate}
\item Soient $I$ un intervalle de $\mathbb{R}$ non vide, non borné supérieurement, %
et $f$ une application $C^{\infty}$ de $I$ dans $\mathbb{R}$ admettant une limite finie en $+ \infty$. %
$f'$ tend-t-elle vers $0$ en $+ \infty$ ? Est-elle bornée au voisinage de $+ \infty$ ?
%Réponse : Non. Considérer $x \mapsto \frac{\sin x^3}{x}$.\\
\item Que devient ce résultat si l'on suppose, de plus, $f$ croissante ?
%Nous allons construire un contre-exemple comme limite, en un sens que l'on précisera, d'une série de fonctions.\\
\item  Soit $g$ l'application définie sur $\mathbb{R}$ par :
\[\forall x \in \mathbb{R}_+^*, g(x) = e^{- \frac{1}{x}}\]
et $g|_{\mathbb{R}_-} = 0$
\begin{enumerate}
\item Montrer que $g$ est de classe $C^{\infty}$ sur $\mathbb{R}$.
%Indication : Raisonner par récurrence; on utilisera le théorème limite de la dérivation :\\
%soit $u$ une application définie sur un voisinage $V$ d'un point $a$ de $\mathbb{R}$, dérivable sur $V \setminus \{ a \}$, continue en $a$. Si $u'$ admet une limite $l$ en $a$, alors $u$ est dérivable, de dérivée $l$, en $a$.\\
\item Soit $\theta \in \mathbb{R}_+^*$. Définir, à l'aide de $g$, une application $h_{\theta}$ de $\mathbb{R}$ dans $\mathbb{R}$, $C^{\infty}$, nulle sur $\mathbb{R} \setminus ]0 , \theta[$, et strictement positive sur $]0 , \theta[$.
\item Soit $(\alpha , \theta) \in (\mathbb{R}_+^*)^2$. Définir une application $H_{\theta , \alpha}$ de $\mathbb{R}$ dans $\mathbb{R}$, $C^{\infty}$, nulle sur $\mathbb{R}_-$, égale à $\alpha$ sur $[ \theta , + \infty[$, et strictement croissante sur $[0, \theta]$.
%Indication : On pourra considérer une primitive de $h_{\theta}$.\\
%\textit{Je donne un temps de réflexion à l'élève afin qu'il trouve une ébauche de solution. Je poursuis l'interrogation en posant, éventuellement, la question intermédiaire qui suit.}

\medskip
Soit maintenant $(f_n)$ la suite de fonctions définies sur $\mathbb{R}$ par :
\[\forall n \in \mathbb{N}^* , \forall x \in \mathbb{R} , f_n(x) = H_{1/n^3 , 1/n^2} (x - n)\]
On constate que $\sum f_n$ est convergente -en quel sens ?- sur $\mathbb{R}$.
\item Montrer que la somme $f$ de cette série de fonctions est $C^{\infty}$, croissante sur $\mathbb{R}$, admet une limite finie en $+ \infty$, mais que $f'$ est non bornée au voisinage de $+ \infty$.\\
%Indications : $\Sigma f_n$ est normalement convergente sur $\mathbb{R}$, mais cela ne nous permet pas de connaître la dérivabilité de $f$.\\
%Toutefois, $\Sigma f_n$ est stationnaire sur tout compact de $\mathbb{R}$.\\
%On en déduit, d'après l'unicité de la limite -uniforme, ou même simple- d'une suite de fonctions, que $f$ est entièrement déterminée, sur un compact de $\mathbb{R}$, par l'une des sommes partielles de la série $\Sigma f_n$. Conclure, on appliquera le théorème des accroissements finis à $f$ sur des intervalles bien choisis.
\end{enumerate}
\end{enumerate}
\end{exer}

\begin{exer}[R\'ealisation d'un ferm\'e de $\mathbb{R}$ comme ensemble des z\'eros d'une fonction $C^{\infty}$]
\begin{enumerate}
\item Montrer que tout ouvert de $\mathbb{R}$ est une union disjointe d'intervalles ouverts, et que cette union est d\'enombrable.
\end{enumerate}
%\newcounter{stock}
\setcounter{stock}{\value{enumi}}
Soit maintenant $F$ un ferm\'e de $\mathbb{R}$, on note $O$ son compl\'ementaire. D'apr\g{e}s la question pr\'ec\'edente, on peut \'ecrire :
\[O=\bigsqcup\limits_{n\in\mathbb{N}}I_n\text{ o\g{u} $(I_n)$ est une suite d'intervalles ouverts disjoints.}\] 
On admet qu'il existe, pour tout intervalle ouvert $I$ de $\mathbb{R}$, une fonction $f_I$ d\'efinie et de classe $C^{\infty}$ sur $\mathbb{R}$, %
non nulle sur $I$ et nulle en dehors de cet intervalle, de norme infinie $1$.
\begin{enumerate}
\setcounter{enumi}{\value{stock}}
\item En appliquant une op\'eration judicieusement choisie aux termes de la suite $(f_{I_n})_n$, construire une suite $g_n$ de fonctions de $C^{\infty}(\mathbb{R},\mathbb{R})$, normalement convergente, %
dont la somme s'annulle exactement sur $F$.

\emph{On remarque que la s\'erie $\sum g_n$ est \emph{ponctuellement stationnaire}.}
\item Etude des s\'eries de d\'eriv\'ees successives de la suite $(g_n)$ :
\begin{enumerate}
\item Montrer que, pour tout entier naturel non nul $k$, la s\'erie $\sum g_n^{(k)}$ est ponctuellement stationnaire.
\item Donner une condition suffisante pour la d\'erivabilit\'e de la somme $\sum\limits_{n=0}^{\infty}g_n$.
\item Donner enfin une condition n\'ecessaire et suffisante simple à ce que la série soit %globalement stationnaire
uniform\'ement convergente sur un ouvert \'eventuellement petit, typiquement un intervalle.
\item Etudier les deux contre-exemples suivants :
\begin{description}
\item[Cas non born\'e] $O=\bigcup\limits_{n\in\mathbb{N}}\left]n,n+\frac{1}{2}\right[$ dans $\mathbb{R}$ ;
\item[Cas born\'e] $O=\bigcup\limits_{n \in \mathbb{N}^{\ast}} \left]\frac{\|g_n\|_{\infty}}{n+1},\frac{\|g_n\|_{\infty}}{n}\right[$ dans $]0,1[$.
\end{description}
%\item Donner, dans le dernier cas, un exemple où l'interversion de limites avec la suite convergente de fonctions qui convergent en $0$ ne fonctionne pas. On remarquera que les intervalles considérés sont de longueurs arbitrairement petite.\\
\end{enumerate}
\item %[R\'esolution quantitative]
%On note : $\underset{i \in D}{\bigcup}I_i$ la décomposition de $O$ en intervalles ouverts disjoints, %
%$D$ est un ensemble dénombrable.
Soit $f$ une fonction de classe $C^{\infty}$ de $\mathbb{R}$ dans $[0,1]$, nulle en dehors de $]-1,1[$, %
strictement positive sur cet intervalle, paire, croissante sur $[-1,0]$ et telle que $f(0) = 1$.
\begin{enumerate}
\item Comment construit-on une fonction $f_{a,\epsilon}$, de support $[a-\epsilon ,a+\epsilon]$, positive majorée en $a$ par $a$ ?% Par exemple changement de variable affine.
\item Construire une nouvelle suite $(h_n)_{n\in\mathbb{N}}$ de fonctions de classe $C^{\infty}$ qui converge normalement sur toute partie born\'ee de $\mathbb{R}$.
\item Majorer les normes des d\'erivées successives des fonctions de la famille $(h_n)$.
\item Corriger la construction précédente afin que les s\'eries de d\'eriv\'ees convergent, pour tout ordre.
\item Conclure.
%Indication : la norme des dérivées $k-$ièmes vaut $\frac{2}{l}^k$, à une constante multiplicative près, où $l$ est la longueur des intervalles de support. Comment compenser, uniformément en $k$, ces croissancesz polynomiales ?
\end{enumerate}
\end{enumerate}
\end{exer}

\begin{exer}
Soit $f$ une application définie et deux fois dérivable sur un intervalle $I$ de $\mathbb{R}$. Supposons : $f''$ est bornée sur $I$.

Montrer que $x \mapsto n\left(f\left(x + \frac{1}{n}\right) - f(x)\right)_n$ converge uniformément vers $f'$ sur $I$.
\end{exer}

\begin{exer}[Th\'eor\`eme de Dini]
Soit $(K,d)$ un espace m\'etrique compact. %
On consid\`ere une suite $(f_n)_n$ d'applications continues de $K$ dans lui-m\^eme, %
qui converge simplement vers une application continue $f$ de $K$ dans $K$, de sorte que :
\begin{center}Si $x$ est un \'el\'ement de $K$, la suite $(d(f_n(x),f(x))_n$ est d\'ecroissante.\end{center}
Montrer que $(f_n)$ converge uniform\'ement.
\end{exer}

\begin{exer}
Soit $(K,d)$ un espace m\'etrique compact.\\
On consid\`ere une application $f$ de $K$ dans lui-m\^eme telle que :\[\forall (x,y) \in K^2 , d(f(x),f(y)) < d(x,y)\]
\begin{enumerate}
\item Montrer que $f$ admet un unique point fixe $x_0$.
\item Montrer que, pout tout \'el\'ement $x$ de $K$, la suite $(f_n(x))$ converge vers $x_0$.
\item A l'aide du th\'eor\`eme de Dini, montrer que cettte convergence est uniforme.
\end{enumerate}
\end{exer}

\section{Analyse sur des s\'eries enti\`eres}
\section{Analyse sur des s\'eries enti\`eres}

\begin{exer}
Calculer le rayon de convergense de la s\'erie enti\`ere $\sum a_n z^n$, o\`u $a_n$ d\'esigne la $n-$i\`eme d\'ecimale de $\pi$.
\end{exer}

\begin{exer}
Soit $g$ la fonction d\'efinie par la formule :\[g(x)=\sum\limits_{n=2}^{+\infty}\frac{(-1)^n}{n(n-1)}x^n\]
\begin{enumerate}
\item D\'terminer le domaine de convergence de la s\'erie enti\`ere d\'efinie ci-dessus.
\item Calculer explicitement $g(x)$ pout tout $x$ dans ce domaine.
\end{enumerate}
\end{exer}

\begin{exer}
Calculer, pour tout r\'eel $x$, la somme : \[\sum\limits_{n=0}^{+\infty}\frac{x^{3n}}{(3n)!}\]
\end{exer}

\begin{exer}[Th\'eor\`emes Taub\'eriens]
Soit $(a_n)$ une suite complexe telle que $f:x\mapsto\sum\limits_{n=0}^{+\infty}a_nx^n$ soit d\'efinie pour tout $x$ de $]-1,1[$.\\
On suppose de plus que $f$ admet une limite $l$ \`a gauche de $1$.\\
Montrer que la s\'erie $\sum a_n$ converge, vers $l$, dans les deux cas suivants :
\begin{enumerate}
\item $\forall n \in \mathbb{N} , a_n \in \mathbb{R}_+$
\item $a_n = \underset{n\rightarrow +\infty}{o}(\frac{1}{n})$
\end{enumerate}
\end{exer}

\section{Fonctions int\'egrables}
\section{Fonctions int\'egrables}

\begin{exer}
\begin{enumerate}
\item Soient $a$ et $b$ deux réels positifs non tous les deux nuls. %
Etudier l'intégrabilité sur $]0,1]^{2}$ de $(x,y) \mapsto 1/(ax + by)$.
\item Soit $P$ une application polynomiale qui ne s'annulle, sur $[0,1]^{2}$, qu'en $(0,0)$. %
On suppose, de plus, que $P$ s'écrit: $aX + bY + Q$, où $a$ et $b$ sont deux réels strictement positifs, %
et $Q$ une somme de monômes de degré au moins $2$.

Montrer que $1/P$ est intégrable sur $]0,1]^{2}$.
\item Même question, en supposant: $P = aX + Q$, avec la même notation.
\end{enumerate}
\end{exer}

\begin{exer}[Orthogonalit\'e avec des polyn\^omes]
On se place ici sur l'intervalle $[0,1]$.
\begin{enumerate}
\item Que dire d'une application continue de $[0,1]$ dans $\mathbb{R}$ telle que :
\[\forall p \in \mathbb{R}[X] , \int\limits_0^1 f(t)P(t) dt = 0\]
\item On fixe : $n \in \mathbb{N}$. %
Soit $f$ une application continue de $[0,1]$ dans $\mathbb{R}$ telle que :
\[\forall k \in [0,n] , \int\limits_0^1 x^k f(x) dx = 0\]
Montrer que $f$ admet au moins $n+1$ z\'eros dans $[0,1]$.
\end{enumerate}
\end{exer}

\begin{exer}
Soient $a$ et $b$ deux réels positifs.\\
Montrer que $\int\limits_0^{\infty} \frac{\exp(-at)}{1 - \exp(-bt)} dt$ existe %
et vaut $\sum\limits_0^{\infty} \frac{1}{(a + nb)^2}$.
\end{exer}

\begin{exer}
\begin{enumerate}
\item Soient $p$ et $q$ deux réels strictement positifs.
\begin{enumerate}
\item Montrer que $\int\limits_0^{\infty} \frac{x^{p-1}}{1+x^q} dx$ existe et vaut $\sum\limits_0^{\infty} \frac{(-1)^k}{p+kq}$.
\item Ecrire des formules explicites pour $\frac{\pi}{4}$ et $\ln 2$.
\end{enumerate}

Soit maintenant $p$ un réel compris entre $0$ et $1$, strictement.
\item Montrer que l'intégrale $\int\limits_0^{\infty} \frac{x^{p-1}}{1+x} dx$ existe et vaut
\[\frac{1}{p} + \sum\limits_0^{\infty} \frac{2p}{p^2 - n^2}\]
\end{enumerate}
\end{exer}

\begin{exer}
Montrer que la série de fonctions de terme général $x \mapsto n \exp{-nx}$, définies sur $[1 , + \infty [$, %
converge simplement, que la somme de cette série est intégrable, calculer son intégrale.
\end{exer}

\begin{exer}
Déterminer les fonctions réelles continues $f$ telles que :
\[\forall (x,h) \in \mathbb{R}^2 , f(x) = \frac{1}{2h} \int\limits_{x-h}^{x+h} f(t) dt\]
\end{exer}

\begin{exer}
\begin{enumerate}
\item Soit $f$ une fonction réelle positive, décroissante et intégrable sur $]0,1]$. Etudier la limile en $0$ de la fonction $x \mapsto xf(x)$.
\item Même question, en $+ \infty$, pour une fonction définie sur $\mathbb{R}_+$.
\end{enumerate}
\end{exer}

\begin{exer}
\begin{enumerate}
\item Soit $f$ une fonction continue définie sur $\mathbb{R}_+$. On suppose que $f$ est intégrable. %
F tend-elle vers $0$ en $+ \infty$ ?
\item Montrer qu'une fonction réelle intégrable sur $\mathbb{R}_+$, uniformément continue, tend vers $0$ en $+ \infty$.
\end{enumerate}
\end{exer}

\begin{exer}[Intégrabilité et produit]%Pas de Cauchy dans la solution, même si la prorpiété de Cauchy sous-tend l'exercice avec la convergence absolue des intégrales.
%Cf Cassini, analyse 3, p 186 pour l'exercice complet avec sa troisième question, semi-convergence.
\begin{enumerate}
\item Soit $u$ une fonction continue et bornée sur $\mathbb{R}$. Montrer que %
pour toute fonction réelle $v$ intégrable sur $\mathbb{R}$, $uv$ est intégrable sur $\mathbb{R}$.
\item Montrer que réciproquement, si $u$ est une fonction continue de $\mathbb{R}$ dans $\mathbb{R}$ telle que, %
pour toute fonction intégrable $v$, $uv$ est intégrable, alors $u$ est bornée sur $\mathbb{R}$.
\end{enumerate}
\end{exer}

\begin{exer}
Soient $a$ et $b$ deux réels tels que $a < b$, $f$ une application continue de $[a,b]$ dans $\mathbb{R}$.\\
Montrer que :\[\int_a^b f(t) | \sin nt | dt \underset{n\rightarrow + \infty}{\longrightarrow} \frac{2}{\pi} \int_a^b f\]
%Indications :\\
%i) Constater que $\frac{2}{\pi}$ est la valeur moyenne de la fonction continue $\pi -$périodique $t \mapsto | \sin t |$, et de ses composées à droite par les fonctions $t \mapsto nt$.\\
%ii) Appliquer une égalité de la moyenne sur des intervalles bien choisis.\\
%Cette égalité est : si $g$ est une application continue d'un segment $I$ dans $\mathbb{R}$, et $h$ une fonction positive, intégrable au sens de Riemann, définie sur $I$, alors :\begin{center}$\exists c \in I / \int_I gh = g(c) \int_I h$\end{center}
%iii) En déduire une relation entre les termes de la suite d'intégrales considérée et des sommes de Riemann de $f$ sur des intervalles bien choisis, puis sur $[a,b]$.\\
%iv) Montrer que les intégrales résiduelles tendent vers $0$ quand $n$ tend vers $+ \infty$. Conclure.
\end{exer}

\begin{exer}
Soient $a$ et $b$ deux réels tels que $a < b$, et $f$ une application continue de $[a,b]$ dans $\mathbb{R}$.\\
Déterminer la limite de la suite :\[\left(\int_a^b \frac{f(x)}{3 + 2 \cos nx} dx \right)_{n \in \mathbb{N}}\]
%On peut procéder de la même manière que dans l'exercice précédent, puis calculer la valeur moyenne de $x \mapsto \frac{1}{3 + 2 \cos x}$.\\
%Indications supplémentaires :\\
%Il suffit de développer cette fonction en série de puissances de $\frac{2}{3} \cos$. On peut intervertir les signes somme -pourquoi ?-, les termes de la nouvelle somme se calculent, par exemple, à l'aide des intégrales de Wallis.
\end{exer}

\begin{exer}
Soit $(a_n)_n$ une suite complexe telle que : $\sum a_n n!$ converge.\\
Montrer que $\int\limits_0^{+\infty} e^{-x}\sum\limits_{n=0}^{+\infty}a_nx^n dx$ existe et vaut %
$\sum\limits_{n=0}^{+\infty}n! a_n$.
\end{exer}

\section{Fonctions r\'eelles, autres propri\'et\'es}
% \section{Fonctions réelles, autres propriétés}

\begin{exer}
Soit $f$ une application de classe $C^{\infty}$ de $\mathbb{R}$ dans lui-m\^eme, qui tend vers $0$ au voisinage de $+ \infty$.\\
On suppose :\[\exists x_0 \in \mathbb{R} / f(x_0)f'(x_0) \geqslant 0\]
\begin{enumerate}
\item Montrer que : $\exists x_1 \in [x_0 , \infty[ / f'(x_1) = 0$.
\item Montrer quil existe une suite $(x_k)_{k \in \mathbb{N}^{\ast}}$ strictement croissante telle que :
\[\forall k \in \mathbb{N}^{\ast} , f^{(k)} (x_k) = 0\]
\end{enumerate}
\end{exer}

\begin{exer}%Sommes de Riemann pour la solution.
Soit $f$ une application continue de $[0,1]$ dans $[0,1]$, dérivable en $0$, telle que : $f(0) = 0$.\\
On pose : $p \in \mathbb{N}^*$.\\
Déterminer la limite de $\left(\sum\limits_{k=0}^n f\left(\frac{1}{n+kp}\right)\right)_{n\in\mathbb{N}}$. 
\end{exer}


% \section{Ancien programme : s\'eries de Fourier}
% \input{analyse_fonctions/anc_series_fourier.tex}


% \section{Suites et séries de fonctions}

\begin{exer}
Posons : $(a,b) \in \mathbb{R}$ $a < b$.
\begin{enumerate}
\item On dit qu'une application de $[a,b]$ dans $\mathbb{R}$ est réglée si et seulement si %
elle est limite uniforme d'une suite de fonctions en escalier.
\begin{enumerate}
\item Soit : $f \in \mathbb{R}^{[a,b]}$. Montrer que $f$ est réglée si et seulement si elle satisfait la condition : %
$f$ admet en tout point de $[a,b]$ une limite à gauche et une limite à droite, lorsque ces limites sont définies.
%Indications : (i) implique (ii) d'après le cours. Pour montrer que (ii) implique (i), utiliser la définition des limites à gauche et à droite puis constater que $[a,b]$ vérifie la propriété de Borel-Lebesgue. J'indique que ce raisonnement généralise la démonstration du théorème d'approximation des applications continues définies sur un segment par des fonctions en escalier, dont le point clé est l'utilisation du théorème de Heine, qui peut se démontrer avec Borel-Lebesgue.\\
\item Montrer que l'ensemble des points de discontinuité d'une telle application est au plus dénombrable.
%Indication : La réunion d'une famille dénombrable d'ensembles finis est au plus dénombrable.\\
\end{enumerate}
\item On s'int\'eresse maintenant aux fonctions int\'egrables au sens de Riemann
\begin{enumerate}
\item Rappeler la définition de l'intégrabilité au sens de Riemann d'une application $f$ de $[a,b]$ dans $\mathbb{R}$.
%\smallskip
%\textit{J'introduis ici les notations : $E^+(f)$, respectivement $E^-(f)$, %
%est l'ensemble des applications en escalier de $[a,b]$ dans $\mathbb{R}$ majorant, respectivement minorant $f$.}
Pour toute fonction r\'eelle $f$, on note $E^+(f)$, respectivement $E^-(f)$, l'ensembe de fonctions en escalier d\'efinies sur $[a,b]$ et inf\'erieures, respectivement sup\'erieures, \g{a} $f$ sur $[a,b]$.
\item Montrer que, lorsque $E^+(f)$ et $E^-(f)$ sont tous les deux non vides, %
que $\inf_{\varphi \in E^+(f)}(\int \varphi)$ et $\sup_{\varphi \in E^-(f)}(\int \varphi)$ %
sont bien définies dans $\mathbb{R}$ et vérifient : %
$\inf_{\varphi \in E^+(f)}(\int \varphi) \geq \sup_{\varphi \in E^-(f)}(\int \varphi)$
\end{enumerate}
\item Montrer que toute application réglée $f$ de $[a,b]$ dans $\mathbb{R}$ est intégrable au sens de Riemann.
%En particulier, une telle application est bornée sauf sur un ensemble dénombreable.
%Indications : On montre simultanément que $E^+(f)$ et $E^-(f)$ sont tous les deux non vides, et que :
%\[\forall \epsilon \in \mathbb{R}_+^* , \inf_{\varphi \in E^+(f)}(\int \varphi) - \sup_{\varphi \in E^-(f)}(\int \varphi) \leq \epsilon\]
%Pour cela, poser : $\epsilon \in \mathbb{R}_+^*$ ; remarquer qu'il existe une application $\varphi$ en escalier de $[a,b]$ dans $\mathbb{R}$ située à une distance inférieure à $\frac{\epsilon}{b-a}$ de $f$, puis encadrer $f$ par $\varphi + \frac{\epsilon}{b-a}$ et $\varphi - \frac{\epsilon}{b-a}$.
\end{enumerate}
\end{exer}

\begin{exer}
Etudier la convergence de la s\'erie de fonctions :
\[\sum x\mapsto \frac{x\exp -nx}{\ln n}\]
d\'efinies sur $\mathbb{R}$, pour tout entier $n$ strictement positif.
\end{exer}

\begin{exer}
Soit $S$ la fonction, somme de la s\'erie : $\sum x\mapsto nx\exp(-nx^2)$. 
\begin{enumerate}
\item Donner le domaine de d\'efinition de $S$.
\item La convergence de la s\'erie est-elle normale, uniforme, sur ce domaine ?
\item Exprimer $S$ \`a l'aide de fonctions usuelles.
\end{enumerate}
\end{exer}

\begin{exer}
Soit $E$ l'espace vectoriel des suites r\'eelles born\'ees, %
que l'on munit de sa norme de convergence uniforme, $\| \|_{\infty}$.

D\'eterminer si les sous-ensembles qui suivent sont ferm\'es ou non :
\begin{itemize}
\item L'ensemble $F$ des suites croissantes.
\item L'ensemble $c_0$ des suites qui convergent vers $0$.
\item L'ensemble $V_0$ des suites qui admettent $0$ comme valeur d'adh\'erence.
\item L'ensemble $S$ des suites sommables.
\item L'ensemble $P$ des suites p\'eriodiques.
\item Bonus : l'ensemble $c$ des suites convergentes.
\end{itemize}
\end{exer}

%\textit{Le détail des questions dans les deux exercices qui suivent est seulement suggéré. Je laisse le soin à l'élève de réfléchir personnellement à ces deux problèmes, qui concernent les applications de classe $C^{\infty}$ de $\mathbb{R}$ dans lui-même.}

\begin{exer}
\begin{enumerate}
\item Soient $I$ un intervalle de $\mathbb{R}$ non vide, non borné supérieurement, %
et $f$ une application $C^{\infty}$ de $I$ dans $\mathbb{R}$ admettant une limite finie en $+ \infty$. %
$f'$ tend-t-elle vers $0$ en $+ \infty$ ? Est-elle bornée au voisinage de $+ \infty$ ?
%Réponse : Non. Considérer $x \mapsto \frac{\sin x^3}{x}$.\\
\item Que devient ce résultat si l'on suppose, de plus, $f$ croissante ?
%Nous allons construire un contre-exemple comme limite, en un sens que l'on précisera, d'une série de fonctions.\\
\item  Soit $g$ l'application définie sur $\mathbb{R}$ par :
\[\forall x \in \mathbb{R}_+^*, g(x) = e^{- \frac{1}{x}}\]
et $g|_{\mathbb{R}_-} = 0$
\begin{enumerate}
\item Montrer que $g$ est de classe $C^{\infty}$ sur $\mathbb{R}$.
%Indication : Raisonner par récurrence; on utilisera le théorème limite de la dérivation :\\
%soit $u$ une application définie sur un voisinage $V$ d'un point $a$ de $\mathbb{R}$, dérivable sur $V \setminus \{ a \}$, continue en $a$. Si $u'$ admet une limite $l$ en $a$, alors $u$ est dérivable, de dérivée $l$, en $a$.\\
\item Soit $\theta \in \mathbb{R}_+^*$. Définir, à l'aide de $g$, une application $h_{\theta}$ de $\mathbb{R}$ dans $\mathbb{R}$, $C^{\infty}$, nulle sur $\mathbb{R} \setminus ]0 , \theta[$, et strictement positive sur $]0 , \theta[$.
\item Soit $(\alpha , \theta) \in (\mathbb{R}_+^*)^2$. Définir une application $H_{\theta , \alpha}$ de $\mathbb{R}$ dans $\mathbb{R}$, $C^{\infty}$, nulle sur $\mathbb{R}_-$, égale à $\alpha$ sur $[ \theta , + \infty[$, et strictement croissante sur $[0, \theta]$.
%Indication : On pourra considérer une primitive de $h_{\theta}$.\\
%\textit{Je donne un temps de réflexion à l'élève afin qu'il trouve une ébauche de solution. Je poursuis l'interrogation en posant, éventuellement, la question intermédiaire qui suit.}

\medskip
Soit maintenant $(f_n)$ la suite de fonctions définies sur $\mathbb{R}$ par :
\[\forall n \in \mathbb{N}^* , \forall x \in \mathbb{R} , f_n(x) = H_{1/n^3 , 1/n^2} (x - n)\]
On constate que $\sum f_n$ est convergente -en quel sens ?- sur $\mathbb{R}$.
\item Montrer que la somme $f$ de cette série de fonctions est $C^{\infty}$, croissante sur $\mathbb{R}$, admet une limite finie en $+ \infty$, mais que $f'$ est non bornée au voisinage de $+ \infty$.\\
%Indications : $\Sigma f_n$ est normalement convergente sur $\mathbb{R}$, mais cela ne nous permet pas de connaître la dérivabilité de $f$.\\
%Toutefois, $\Sigma f_n$ est stationnaire sur tout compact de $\mathbb{R}$.\\
%On en déduit, d'après l'unicité de la limite -uniforme, ou même simple- d'une suite de fonctions, que $f$ est entièrement déterminée, sur un compact de $\mathbb{R}$, par l'une des sommes partielles de la série $\Sigma f_n$. Conclure, on appliquera le théorème des accroissements finis à $f$ sur des intervalles bien choisis.
\end{enumerate}
\end{enumerate}
\end{exer}

\begin{exer}[R\'ealisation d'un ferm\'e de $\mathbb{R}$ comme ensemble des z\'eros d'une fonction $C^{\infty}$]
\begin{enumerate}
\item Montrer que tout ouvert de $\mathbb{R}$ est une union disjointe d'intervalles ouverts, et que cette union est d\'enombrable.
\end{enumerate}
%\newcounter{stock}
\setcounter{stock}{\value{enumi}}
Soit maintenant $F$ un ferm\'e de $\mathbb{R}$, on note $O$ son compl\'ementaire. D'apr\g{e}s la question pr\'ec\'edente, on peut \'ecrire :
\[O=\bigsqcup\limits_{n\in\mathbb{N}}I_n\text{ o\g{u} $(I_n)$ est une suite d'intervalles ouverts disjoints.}\] 
On admet qu'il existe, pour tout intervalle ouvert $I$ de $\mathbb{R}$, une fonction $f_I$ d\'efinie et de classe $C^{\infty}$ sur $\mathbb{R}$, %
non nulle sur $I$ et nulle en dehors de cet intervalle, de norme infinie $1$.
\begin{enumerate}
\setcounter{enumi}{\value{stock}}
\item En appliquant une op\'eration judicieusement choisie aux termes de la suite $(f_{I_n})_n$, construire une suite $g_n$ de fonctions de $C^{\infty}(\mathbb{R},\mathbb{R})$, normalement convergente, %
dont la somme s'annulle exactement sur $F$.

\emph{On remarque que la s\'erie $\sum g_n$ est \emph{ponctuellement stationnaire}.}
\item Etude des s\'eries de d\'eriv\'ees successives de la suite $(g_n)$ :
\begin{enumerate}
\item Montrer que, pour tout entier naturel non nul $k$, la s\'erie $\sum g_n^{(k)}$ est ponctuellement stationnaire.
\item Donner une condition suffisante pour la d\'erivabilit\'e de la somme $\sum\limits_{n=0}^{\infty}g_n$.
\item Donner enfin une condition n\'ecessaire et suffisante simple à ce que la série soit %globalement stationnaire
uniform\'ement convergente sur un ouvert \'eventuellement petit, typiquement un intervalle.
\item Etudier les deux contre-exemples suivants :
\begin{description}
\item[Cas non born\'e] $O=\bigcup\limits_{n\in\mathbb{N}}\left]n,n+\frac{1}{2}\right[$ dans $\mathbb{R}$ ;
\item[Cas born\'e] $O=\bigcup\limits_{n \in \mathbb{N}^{\ast}} \left]\frac{\|g_n\|_{\infty}}{n+1},\frac{\|g_n\|_{\infty}}{n}\right[$ dans $]0,1[$.
\end{description}
%\item Donner, dans le dernier cas, un exemple où l'interversion de limites avec la suite convergente de fonctions qui convergent en $0$ ne fonctionne pas. On remarquera que les intervalles considérés sont de longueurs arbitrairement petite.\\
\end{enumerate}
\item %[R\'esolution quantitative]
%On note : $\underset{i \in D}{\bigcup}I_i$ la décomposition de $O$ en intervalles ouverts disjoints, %
%$D$ est un ensemble dénombrable.
Soit $f$ une fonction de classe $C^{\infty}$ de $\mathbb{R}$ dans $[0,1]$, nulle en dehors de $]-1,1[$, %
strictement positive sur cet intervalle, paire, croissante sur $[-1,0]$ et telle que $f(0) = 1$.
\begin{enumerate}
\item Comment construit-on une fonction $f_{a,\epsilon}$, de support $[a-\epsilon ,a+\epsilon]$, positive majorée en $a$ par $a$ ?% Par exemple changement de variable affine.
\item Construire une nouvelle suite $(h_n)_{n\in\mathbb{N}}$ de fonctions de classe $C^{\infty}$ qui converge normalement sur toute partie born\'ee de $\mathbb{R}$.
\item Majorer les normes des d\'erivées successives des fonctions de la famille $(h_n)$.
\item Corriger la construction précédente afin que les s\'eries de d\'eriv\'ees convergent, pour tout ordre.
\item Conclure.
%Indication : la norme des dérivées $k-$ièmes vaut $\frac{2}{l}^k$, à une constante multiplicative près, où $l$ est la longueur des intervalles de support. Comment compenser, uniformément en $k$, ces croissancesz polynomiales ?
\end{enumerate}
\end{enumerate}
\end{exer}

\begin{exer}
Soit $f$ une application définie et deux fois dérivable sur un intervalle $I$ de $\mathbb{R}$. Supposons : $f''$ est bornée sur $I$.

Montrer que $x \mapsto n\left(f\left(x + \frac{1}{n}\right) - f(x)\right)_n$ converge uniformément vers $f'$ sur $I$.
\end{exer}

\begin{exer}[Th\'eor\`eme de Dini]
Soit $(K,d)$ un espace m\'etrique compact. %
On consid\`ere une suite $(f_n)_n$ d'applications continues de $K$ dans lui-m\^eme, %
qui converge simplement vers une application continue $f$ de $K$ dans $K$, de sorte que :
\begin{center}Si $x$ est un \'el\'ement de $K$, la suite $(d(f_n(x),f(x))_n$ est d\'ecroissante.\end{center}
Montrer que $(f_n)$ converge uniform\'ement.
\end{exer}

\begin{exer}
Soit $(K,d)$ un espace m\'etrique compact.\\
On consid\`ere une application $f$ de $K$ dans lui-m\^eme telle que :\[\forall (x,y) \in K^2 , d(f(x),f(y)) < d(x,y)\]
\begin{enumerate}
\item Montrer que $f$ admet un unique point fixe $x_0$.
\item Montrer que, pout tout \'el\'ement $x$ de $K$, la suite $(f_n(x))$ converge vers $x_0$.
\item A l'aide du th\'eor\`eme de Dini, montrer que cettte convergence est uniforme.
\end{enumerate}
\end{exer}
% \section{Analyse sur des s\'eries enti\`eres}

\begin{exer}
Calculer le rayon de convergense de la s\'erie enti\`ere $\sum a_n z^n$, o\`u $a_n$ d\'esigne la $n-$i\`eme d\'ecimale de $\pi$.
\end{exer}

\begin{exer}
Soit $g$ la fonction d\'efinie par la formule :\[g(x)=\sum\limits_{n=2}^{+\infty}\frac{(-1)^n}{n(n-1)}x^n\]
\begin{enumerate}
\item D\'terminer le domaine de convergence de la s\'erie enti\`ere d\'efinie ci-dessus.
\item Calculer explicitement $g(x)$ pout tout $x$ dans ce domaine.
\end{enumerate}
\end{exer}

\begin{exer}
Calculer, pour tout r\'eel $x$, la somme : \[\sum\limits_{n=0}^{+\infty}\frac{x^{3n}}{(3n)!}\]
\end{exer}

\begin{exer}[Th\'eor\`emes Taub\'eriens]
Soit $(a_n)$ une suite complexe telle que $f:x\mapsto\sum\limits_{n=0}^{+\infty}a_nx^n$ soit d\'efinie pour tout $x$ de $]-1,1[$.\\
On suppose de plus que $f$ admet une limite $l$ \`a gauche de $1$.\\
Montrer que la s\'erie $\sum a_n$ converge, vers $l$, dans les deux cas suivants :
\begin{enumerate}
\item $\forall n \in \mathbb{N} , a_n \in \mathbb{R}_+$
\item $a_n = \underset{n\rightarrow +\infty}{o}(\frac{1}{n})$
\end{enumerate}
\end{exer}
% \section{Fonctions int\'egrables}

\begin{exer}
\begin{enumerate}
\item Soient $a$ et $b$ deux réels positifs non tous les deux nuls. %
Etudier l'intégrabilité sur $]0,1]^{2}$ de $(x,y) \mapsto 1/(ax + by)$.
\item Soit $P$ une application polynomiale qui ne s'annulle, sur $[0,1]^{2}$, qu'en $(0,0)$. %
On suppose, de plus, que $P$ s'écrit: $aX + bY + Q$, où $a$ et $b$ sont deux réels strictement positifs, %
et $Q$ une somme de monômes de degré au moins $2$.

Montrer que $1/P$ est intégrable sur $]0,1]^{2}$.
\item Même question, en supposant: $P = aX + Q$, avec la même notation.
\end{enumerate}
\end{exer}

\begin{exer}[Orthogonalit\'e avec des polyn\^omes]
On se place ici sur l'intervalle $[0,1]$.
\begin{enumerate}
\item Que dire d'une application continue de $[0,1]$ dans $\mathbb{R}$ telle que :
\[\forall p \in \mathbb{R}[X] , \int\limits_0^1 f(t)P(t) dt = 0\]
\item On fixe : $n \in \mathbb{N}$. %
Soit $f$ une application continue de $[0,1]$ dans $\mathbb{R}$ telle que :
\[\forall k \in [0,n] , \int\limits_0^1 x^k f(x) dx = 0\]
Montrer que $f$ admet au moins $n+1$ z\'eros dans $[0,1]$.
\end{enumerate}
\end{exer}

\begin{exer}
Soient $a$ et $b$ deux réels positifs.\\
Montrer que $\int\limits_0^{\infty} \frac{\exp(-at)}{1 - \exp(-bt)} dt$ existe %
et vaut $\sum\limits_0^{\infty} \frac{1}{(a + nb)^2}$.
\end{exer}

\begin{exer}
\begin{enumerate}
\item Soient $p$ et $q$ deux réels strictement positifs.
\begin{enumerate}
\item Montrer que $\int\limits_0^{\infty} \frac{x^{p-1}}{1+x^q} dx$ existe et vaut $\sum\limits_0^{\infty} \frac{(-1)^k}{p+kq}$.
\item Ecrire des formules explicites pour $\frac{\pi}{4}$ et $\ln 2$.
\end{enumerate}

Soit maintenant $p$ un réel compris entre $0$ et $1$, strictement.
\item Montrer que l'intégrale $\int\limits_0^{\infty} \frac{x^{p-1}}{1+x} dx$ existe et vaut
\[\frac{1}{p} + \sum\limits_0^{\infty} \frac{2p}{p^2 - n^2}\]
\end{enumerate}
\end{exer}

\begin{exer}
Montrer que la série de fonctions de terme général $x \mapsto n \exp{-nx}$, définies sur $[1 , + \infty [$, %
converge simplement, que la somme de cette série est intégrable, calculer son intégrale.
\end{exer}

\begin{exer}
Déterminer les fonctions réelles continues $f$ telles que :
\[\forall (x,h) \in \mathbb{R}^2 , f(x) = \frac{1}{2h} \int\limits_{x-h}^{x+h} f(t) dt\]
\end{exer}

\begin{exer}
\begin{enumerate}
\item Soit $f$ une fonction réelle positive, décroissante et intégrable sur $]0,1]$. Etudier la limile en $0$ de la fonction $x \mapsto xf(x)$.
\item Même question, en $+ \infty$, pour une fonction définie sur $\mathbb{R}_+$.
\end{enumerate}
\end{exer}

\begin{exer}
\begin{enumerate}
\item Soit $f$ une fonction continue définie sur $\mathbb{R}_+$. On suppose que $f$ est intégrable. %
F tend-elle vers $0$ en $+ \infty$ ?
\item Montrer qu'une fonction réelle intégrable sur $\mathbb{R}_+$, uniformément continue, tend vers $0$ en $+ \infty$.
\end{enumerate}
\end{exer}

\begin{exer}[Intégrabilité et produit]%Pas de Cauchy dans la solution, même si la prorpiété de Cauchy sous-tend l'exercice avec la convergence absolue des intégrales.
%Cf Cassini, analyse 3, p 186 pour l'exercice complet avec sa troisième question, semi-convergence.
\begin{enumerate}
\item Soit $u$ une fonction continue et bornée sur $\mathbb{R}$. Montrer que %
pour toute fonction réelle $v$ intégrable sur $\mathbb{R}$, $uv$ est intégrable sur $\mathbb{R}$.
\item Montrer que réciproquement, si $u$ est une fonction continue de $\mathbb{R}$ dans $\mathbb{R}$ telle que, %
pour toute fonction intégrable $v$, $uv$ est intégrable, alors $u$ est bornée sur $\mathbb{R}$.
\end{enumerate}
\end{exer}

\begin{exer}
Soient $a$ et $b$ deux réels tels que $a < b$, $f$ une application continue de $[a,b]$ dans $\mathbb{R}$.\\
Montrer que :\[\int_a^b f(t) | \sin nt | dt \underset{n\rightarrow + \infty}{\longrightarrow} \frac{2}{\pi} \int_a^b f\]
%Indications :\\
%i) Constater que $\frac{2}{\pi}$ est la valeur moyenne de la fonction continue $\pi -$périodique $t \mapsto | \sin t |$, et de ses composées à droite par les fonctions $t \mapsto nt$.\\
%ii) Appliquer une égalité de la moyenne sur des intervalles bien choisis.\\
%Cette égalité est : si $g$ est une application continue d'un segment $I$ dans $\mathbb{R}$, et $h$ une fonction positive, intégrable au sens de Riemann, définie sur $I$, alors :\begin{center}$\exists c \in I / \int_I gh = g(c) \int_I h$\end{center}
%iii) En déduire une relation entre les termes de la suite d'intégrales considérée et des sommes de Riemann de $f$ sur des intervalles bien choisis, puis sur $[a,b]$.\\
%iv) Montrer que les intégrales résiduelles tendent vers $0$ quand $n$ tend vers $+ \infty$. Conclure.
\end{exer}

\begin{exer}
Soient $a$ et $b$ deux réels tels que $a < b$, et $f$ une application continue de $[a,b]$ dans $\mathbb{R}$.\\
Déterminer la limite de la suite :\[\left(\int_a^b \frac{f(x)}{3 + 2 \cos nx} dx \right)_{n \in \mathbb{N}}\]
%On peut procéder de la même manière que dans l'exercice précédent, puis calculer la valeur moyenne de $x \mapsto \frac{1}{3 + 2 \cos x}$.\\
%Indications supplémentaires :\\
%Il suffit de développer cette fonction en série de puissances de $\frac{2}{3} \cos$. On peut intervertir les signes somme -pourquoi ?-, les termes de la nouvelle somme se calculent, par exemple, à l'aide des intégrales de Wallis.
\end{exer}

\begin{exer}
Soit $(a_n)_n$ une suite complexe telle que : $\sum a_n n!$ converge.\\
Montrer que $\int\limits_0^{+\infty} e^{-x}\sum\limits_{n=0}^{+\infty}a_nx^n dx$ existe et vaut %
$\sum\limits_{n=0}^{+\infty}n! a_n$.
\end{exer}
% % \section{Fonctions réelles, autres propriétés}

\begin{exer}
Soit $f$ une application de classe $C^{\infty}$ de $\mathbb{R}$ dans lui-m\^eme, qui tend vers $0$ au voisinage de $+ \infty$.\\
On suppose :\[\exists x_0 \in \mathbb{R} / f(x_0)f'(x_0) \geqslant 0\]
\begin{enumerate}
\item Montrer que : $\exists x_1 \in [x_0 , \infty[ / f'(x_1) = 0$.
\item Montrer quil existe une suite $(x_k)_{k \in \mathbb{N}^{\ast}}$ strictement croissante telle que :
\[\forall k \in \mathbb{N}^{\ast} , f^{(k)} (x_k) = 0\]
\end{enumerate}
\end{exer}

\begin{exer}%Sommes de Riemann pour la solution.
Soit $f$ une application continue de $[0,1]$ dans $[0,1]$, dérivable en $0$, telle que : $f(0) = 0$.\\
On pose : $p \in \mathbb{N}^*$.\\
Déterminer la limite de $\left(\sum\limits_{k=0}^n f\left(\frac{1}{n+kp}\right)\right)_{n\in\mathbb{N}}$. 
\end{exer}

% \chapter{Equations diff\'erentielles ordinaires}
\chapter{Equations diff\'erentielles ordinaires : cas lin\'eaire}

\section{Equations diff\'erentielles lin\'eaires}

\begin{exer}
Donner la solution g\'en\'erale du syst\`eme :
\begin{equation}
  \left\{
      \begin{aligned}
        x'&=x+8y+e^t \\
        y'&=2x+y+e^{-3t} \\
      \end{aligned}
    \right.
\end{equation}
\end{exer}

\begin{exer}
R\'esoudre le syst\`eme diff\'erentiel suivant :
$\left\{
\begin{array}{l}
  x'=y+z\\
  y'=x\\
  z'=x+y+z
\end{array}
\right.$
\end{exer}

\begin{exer}
R\'esoudre, sur $\mathbb{R}$, le syst\`eme suivant :
\begin{equation}
\left\{
\begin{aligned}
x'&=x+6y\\
y'&=-3x-5y\\
z'&=-3x-6y-5z\\
\end{aligned}
\right.
\end{equation}
\end{exer}

\begin{exer}
\begin{enumerate}
\item Déterminer les sous-espaces de dimension finie de $C^{\infty}(\mathbb{R},\mathbb{C})$, stables par la dérivation.
\item Cas réel ?
\end{enumerate}
\end{exer}

\begin{exer}
Soient $f$ une fonction de $\mathbb{R}$ dans $\mathbb{R}$, continue et born\'ee et $a$ un r\'eel strictement positif.\\
Montrer que l'\'equation diff\'erentielle $y''-a^2y=f$ adment une unique solution born\'ee sur $\mathbb{R}$.
\end{exer}

\begin{exer}
Soit $p$ une application de $\mathbb{R}$ dans $\mathbb{R}_+$.\\
Montrer que toutes les solutions de l'équation différentielle $y'' + py = 0$ s'annulent dans $\mathbb{R}$, sauf si $p = 0$.
\end{exer}

\begin{exer}
Soit $y$ une application de classe $C^2$ de $\mathbb{R}$ dans $\mathbb{R}$, telle que : $y(0) = 1$, $y'(0) = 0$; solution de
\[y'' = -X|y|\]
Montrer que $y$ tend vers $- \infty$ en $+ \infty$.
%Indication : démontrer et utiliser la concavité de $y$ sur $\mathbb{R}_+$.
\end{exer}

\begin{exer}
Trouver le solution g\'en\'erale de l'\'equation diff\'erentielle :
\[(1-x^2)y''-xy'+y=0\]
\end{exer}

\begin{exer}
Soit $E$ l'espace vectoriel : $\mathcal{C}^0([0,1],\mathbb{R}$.\\
On d\'efinit sur $E$ l'op\'erateur \[T:E\rightarrow [0,1]^{\mathbb{R}}:f\mapsto x\mapsto \int\limits_0^x \int\limits_t^1 f(u)dudt\]
Montrer que $T$ induit un endomorphisme de $E$, et pr\'eciser ses \'el\'ements propres.
\end{exer}

\begin{exer}[Un th\'eor\`eme de Floquet]
Soient $n$ un entier naturel non nul, $T$ un r\'eel strictement positif %
et $t\mapsto A(t)$ une application continue et $T-$p\'eriodique de $\mathbb{R}$ dans $\mathcal{M}_n(\mathbb{C})$.\\
On se propose d'\'etudier l'\'equation  diff\'erentielle :
\begin{equation}
%(\ast)
Y'(t)=A(t)Y(t)
\label{ex1}
\end{equation}
\begin{enumerate}
\item Montrer qu'il existe une solution $V$ du syst\`eme, et un complexe $\lambda$, tels que :
\[\forall t \in \mathbb{R} , V(t+T)=\lambda V(t)\]

\smallskip
Soit maintenant $(V_k)$ un syst\`eme fondamental de solutions de~\eqref{ex1}. %
On note, pour tout r\'eel $t$, $M(t)$ la matrice dont les vecteurs colonne sont les $V_k(t)$.
\item Montrer qu'il existe une matrice $C$ de $\mathcal{GL}_n(\mathbb{C}$ telle que :\[\forall t \in \mathbb{R} , M(t+T)=M(t)C\]
\end{enumerate}
\end{exer}

\begin{exer}
Soit $n$ un entier naturel non nul.\\
On pose : $(U,V)\in\mathcal{M}_{n,1}(\mathbb{R})$ et $\lambda\in\mathbb{R}$.\\
On se propose, dans cet exercice, d'\'etudier l'\'equation diff\'erentielle :
\begin{equation}
X'=AX
\label{ex2}
\end{equation}
o\`u :$A=\lambda I_n+B$ et $B=U{}^TV$.
\begin{enumerate}
\item Exprimer, pour tout r\'eel $t$, $X(t)$ en fonction de $X(0)$, $t$, $\lambda$, $B$ et $Tr B$.
\item D\'eterminer les sous-espaces vectoriels $E$ de $\mathbb{R}^n$ tels que :

\smallskip
$X(0)\in E$ implique $X(t)\in E$ pour tout $t$.
\item Donner une condition n\'ecessaire et suffisante pour que toutes les solutions de l'\'equation~\eqref{ex2} admettent une limite finie en $+\infty$.
\end{enumerate}
\end{exer}

\begin{exer}[Deux problèmes cousins : équations linéaires perturbées]
On associe, à un polynôme $\sum\limits_{k=0}^n a_k X^k$ que nous noterons également $P$, les deux équations, dites linéaires perturbées, suivantes :
\[(D) : \sum\limits_{k=0}^n a_k x^{(k)} = y\] où $x$ et $y$ sont des applications de classe $C^{\infty}$ de $\mathbb{R}$ dans $\mathbb{C}$ ;
\[(R) : \forall m \in \mathbb{N} , \sum\limits_{k=0}^n a_k x_{m+k} = y_m\] où $(x_m)$ et $(y_m)$ sont deux suites complexes.\\
On se propose, dans les deux cas, de donner une condition nécessaire et suffisante à ce que la convergence vers $0$ de $y$ à l'infini implique celle de $x$.
\begin{enumerate}
\item Cas différentiel : \`a l'aide de solutions particuli\`eres de l'équation linéaire associée, donner une condition nécessaire $(CD)$ sur les racines de $P$ pour que la propriété de convergence considérée soit vérifiée.
\item D\'eterminer, de la m\^eme mani\`ere, une condition nécessaire $(CR)$ à la propriété de convergence pour $(R)$.

\smallskip
On se propose maintenant de montrer que $(CD)$ et $(CR)$ sont suffisantes aux ropriétés de convergence des solutions des équations perturbées $(D)$ et $(R)$.\\
\item Montrer que pour chacun des deux problèmes, il suffit de résoudre le cas du degré $1$.
\item Résoudre le problème dans le cas de $(D)$.
\item Résoudre le problème dans le cas de $(R)$.
\end{enumerate}
\end{exer}

\begin{exer}[Une démonstration du comte Napoléon Daru]% : équations différentielles et lemme de Schwarz]
On considère, dans tout cet exercice, l'équation différentielle scalaire : \[\sum\limits_{k=0}^n a_k y^{(k)} = 0\]
où $\sum\limits_{k=0}^n a_k X^k$ est un polynôme réel scindé, unitaire de degré $n$, que nous noterons $P$.
\begin{enumerate}
\item Calculer la dimension de l'espace vectoriel des solutions de cette équation différentielle.

\smallskip
On cherche maintenant à déterminer une base de cet espace vectoriel, à l'aide de la méthode de Daru. On s'int\'eresse pour cela à la famille de fonctions $(x \mapsto \exp mx)_{m \in \mathbb{R}}$.
\item Calculer l'image d'une telle fonction par l'opérateur différentiel $P(\frac{d}{dx})$.

\smallskip
On consid\`ere l'application, de classe $C^{\infty}$ sur $\mathbb{R}^2$, $(m,x) \mapsto \exp mx$.
\item Exprimer les d\'eriv\'ees successives de cette application par rapport à $m$.\\
\item Que dire des opérateurs différentiels de $\mathcal{C}^{\infty}(\mathbb{R}^2,\mathbb{R})$, $\frac{\partial}{\partial m}$ et $P(\frac{\partial}{\partial x})$ ? En déduire les images par $P(\frac{d}{dx})$ des applications $x \mapsto x^k \exp(mx)$ - $m \in \mathbb{R}$, $k \in \mathbb{N}$.
%Indications : Utiliser le lemme de Schwarz, puis la formule de Leibniz.\\
\item D\'eduire de ce qui pr\'ec\`ede les solutions de l'\'equation diff\'erentielle consid\'er\'ee.
\end{enumerate}
\end{exer}% pas d'extension .tex
% \input{edo/anc_non_lineaires.tex}



% \section{Equations diff\'erentielles lin\'eaires}

\begin{exer}
Donner la solution g\'en\'erale du syst\`eme :
\begin{equation}
  \left\{
      \begin{aligned}
        x'&=x+8y+e^t \\
        y'&=2x+y+e^{-3t} \\
      \end{aligned}
    \right.
\end{equation}
\end{exer}

\begin{exer}
R\'esoudre le syst\`eme diff\'erentiel suivant :
$\left\{
\begin{array}{l}
  x'=y+z\\
  y'=x\\
  z'=x+y+z
\end{array}
\right.$
\end{exer}

\begin{exer}
R\'esoudre, sur $\mathbb{R}$, le syst\`eme suivant :
\begin{equation}
\left\{
\begin{aligned}
x'&=x+6y\\
y'&=-3x-5y\\
z'&=-3x-6y-5z\\
\end{aligned}
\right.
\end{equation}
\end{exer}

\begin{exer}
\begin{enumerate}
\item Déterminer les sous-espaces de dimension finie de $C^{\infty}(\mathbb{R},\mathbb{C})$, stables par la dérivation.
\item Cas réel ?
\end{enumerate}
\end{exer}

\begin{exer}
Soient $f$ une fonction de $\mathbb{R}$ dans $\mathbb{R}$, continue et born\'ee et $a$ un r\'eel strictement positif.\\
Montrer que l'\'equation diff\'erentielle $y''-a^2y=f$ adment une unique solution born\'ee sur $\mathbb{R}$.
\end{exer}

\begin{exer}
Soit $p$ une application de $\mathbb{R}$ dans $\mathbb{R}_+$.\\
Montrer que toutes les solutions de l'équation différentielle $y'' + py = 0$ s'annulent dans $\mathbb{R}$, sauf si $p = 0$.
\end{exer}

\begin{exer}
Soit $y$ une application de classe $C^2$ de $\mathbb{R}$ dans $\mathbb{R}$, telle que : $y(0) = 1$, $y'(0) = 0$; solution de
\[y'' = -X|y|\]
Montrer que $y$ tend vers $- \infty$ en $+ \infty$.
%Indication : démontrer et utiliser la concavité de $y$ sur $\mathbb{R}_+$.
\end{exer}

\begin{exer}
Trouver le solution g\'en\'erale de l'\'equation diff\'erentielle :
\[(1-x^2)y''-xy'+y=0\]
\end{exer}

\begin{exer}
Soit $E$ l'espace vectoriel : $\mathcal{C}^0([0,1],\mathbb{R}$.\\
On d\'efinit sur $E$ l'op\'erateur \[T:E\rightarrow [0,1]^{\mathbb{R}}:f\mapsto x\mapsto \int\limits_0^x \int\limits_t^1 f(u)dudt\]
Montrer que $T$ induit un endomorphisme de $E$, et pr\'eciser ses \'el\'ements propres.
\end{exer}

\begin{exer}[Un th\'eor\`eme de Floquet]
Soient $n$ un entier naturel non nul, $T$ un r\'eel strictement positif %
et $t\mapsto A(t)$ une application continue et $T-$p\'eriodique de $\mathbb{R}$ dans $\mathcal{M}_n(\mathbb{C})$.\\
On se propose d'\'etudier l'\'equation  diff\'erentielle :
\begin{equation}
%(\ast)
Y'(t)=A(t)Y(t)
\label{ex1}
\end{equation}
\begin{enumerate}
\item Montrer qu'il existe une solution $V$ du syst\`eme, et un complexe $\lambda$, tels que :
\[\forall t \in \mathbb{R} , V(t+T)=\lambda V(t)\]

\smallskip
Soit maintenant $(V_k)$ un syst\`eme fondamental de solutions de~\eqref{ex1}. %
On note, pour tout r\'eel $t$, $M(t)$ la matrice dont les vecteurs colonne sont les $V_k(t)$.
\item Montrer qu'il existe une matrice $C$ de $\mathcal{GL}_n(\mathbb{C}$ telle que :\[\forall t \in \mathbb{R} , M(t+T)=M(t)C\]
\end{enumerate}
\end{exer}

\begin{exer}
Soit $n$ un entier naturel non nul.\\
On pose : $(U,V)\in\mathcal{M}_{n,1}(\mathbb{R})$ et $\lambda\in\mathbb{R}$.\\
On se propose, dans cet exercice, d'\'etudier l'\'equation diff\'erentielle :
\begin{equation}
X'=AX
\label{ex2}
\end{equation}
o\`u :$A=\lambda I_n+B$ et $B=U{}^TV$.
\begin{enumerate}
\item Exprimer, pour tout r\'eel $t$, $X(t)$ en fonction de $X(0)$, $t$, $\lambda$, $B$ et $Tr B$.
\item D\'eterminer les sous-espaces vectoriels $E$ de $\mathbb{R}^n$ tels que :

\smallskip
$X(0)\in E$ implique $X(t)\in E$ pour tout $t$.
\item Donner une condition n\'ecessaire et suffisante pour que toutes les solutions de l'\'equation~\eqref{ex2} admettent une limite finie en $+\infty$.
\end{enumerate}
\end{exer}

\begin{exer}[Deux problèmes cousins : équations linéaires perturbées]
On associe, à un polynôme $\sum\limits_{k=0}^n a_k X^k$ que nous noterons également $P$, les deux équations, dites linéaires perturbées, suivantes :
\[(D) : \sum\limits_{k=0}^n a_k x^{(k)} = y\] où $x$ et $y$ sont des applications de classe $C^{\infty}$ de $\mathbb{R}$ dans $\mathbb{C}$ ;
\[(R) : \forall m \in \mathbb{N} , \sum\limits_{k=0}^n a_k x_{m+k} = y_m\] où $(x_m)$ et $(y_m)$ sont deux suites complexes.\\
On se propose, dans les deux cas, de donner une condition nécessaire et suffisante à ce que la convergence vers $0$ de $y$ à l'infini implique celle de $x$.
\begin{enumerate}
\item Cas différentiel : \`a l'aide de solutions particuli\`eres de l'équation linéaire associée, donner une condition nécessaire $(CD)$ sur les racines de $P$ pour que la propriété de convergence considérée soit vérifiée.
\item D\'eterminer, de la m\^eme mani\`ere, une condition nécessaire $(CR)$ à la propriété de convergence pour $(R)$.

\smallskip
On se propose maintenant de montrer que $(CD)$ et $(CR)$ sont suffisantes aux ropriétés de convergence des solutions des équations perturbées $(D)$ et $(R)$.\\
\item Montrer que pour chacun des deux problèmes, il suffit de résoudre le cas du degré $1$.
\item Résoudre le problème dans le cas de $(D)$.
\item Résoudre le problème dans le cas de $(R)$.
\end{enumerate}
\end{exer}

\begin{exer}[Une démonstration du comte Napoléon Daru]% : équations différentielles et lemme de Schwarz]
On considère, dans tout cet exercice, l'équation différentielle scalaire : \[\sum\limits_{k=0}^n a_k y^{(k)} = 0\]
où $\sum\limits_{k=0}^n a_k X^k$ est un polynôme réel scindé, unitaire de degré $n$, que nous noterons $P$.
\begin{enumerate}
\item Calculer la dimension de l'espace vectoriel des solutions de cette équation différentielle.

\smallskip
On cherche maintenant à déterminer une base de cet espace vectoriel, à l'aide de la méthode de Daru. On s'int\'eresse pour cela à la famille de fonctions $(x \mapsto \exp mx)_{m \in \mathbb{R}}$.
\item Calculer l'image d'une telle fonction par l'opérateur différentiel $P(\frac{d}{dx})$.

\smallskip
On consid\`ere l'application, de classe $C^{\infty}$ sur $\mathbb{R}^2$, $(m,x) \mapsto \exp mx$.
\item Exprimer les d\'eriv\'ees successives de cette application par rapport à $m$.\\
\item Que dire des opérateurs différentiels de $\mathcal{C}^{\infty}(\mathbb{R}^2,\mathbb{R})$, $\frac{\partial}{\partial m}$ et $P(\frac{\partial}{\partial x})$ ? En déduire les images par $P(\frac{d}{dx})$ des applications $x \mapsto x^k \exp(mx)$ - $m \in \mathbb{R}$, $k \in \mathbb{N}$.
%Indications : Utiliser le lemme de Schwarz, puis la formule de Leibniz.\\
\item D\'eduire de ce qui pr\'ec\`ede les solutions de l'\'equation diff\'erentielle consid\'er\'ee.
\end{enumerate}
\end{exer}

% \chapter{Calcul diff\'erentiel et quelques applications}
\chapter{Calcul diff\'erentiel et quelques applications}

\section{Calcul diff\'erentiel \'el\'ementaire}

\begin{exer}
Soit : $E = C^0([0,1],\mathbb{R})$\\
Montrer que les applications suivantes sont différentiables et calculer leurs différentielles :
\begin{enumerate}
\item $f_1 : \mathbb{R}^n \rightarrow \mathbb{R}^m : x \mapsto Ax + b$, $A \in M_{m,n}(\mathbb{R})$, $b \in \mathbb{R}^m$
\item $f_2 : E \rightarrow \mathbb{R} : u \mapsto \int_{0}^1 u(x) dx + 3u(0)$
\item $f_3 : M_n(\mathbb{R}) \times \mathbb{R}^n \rightarrow \mathbb{R}^n : (A,b) \mapsto Ab$
\item $f_4 : E^3 \rightarrow E : (u,v,w) \mapsto (x \mapsto (x^2 + 2)u(x)v(x)\int_{0}^1 a(t)w(t)dt + u(0)$, $a \in E$.
\end{enumerate}
\end{exer}

\begin{exer}
Soient $n$ un entier naturel non nul, et une application diff\'erentiable de $\mathbb{R}^n$ dans $\mathcal{L}(\mathbb{R}^n)$.\\
Montrer de deux mani\`eres que l'application de $\mathbb{R}^n$ dans $\mathbb{R}^n$ :
\[x\mapsto f(x)(x)\]
est diff\'erentiable :
\begin{itemize}
\item En évaluant directement $f(x+h)(x+h)$.
\item En l'exprimant comme composée de deux applications bien choisies.
\end{itemize}
\end{exer}

\begin{exer}
Soit :\[f : \mathbb{R}^2 \rightarrow \mathbb{R}^2 : (x_1,x_2) \mapsto (x_1^2 \sin x_2 , x_1 + \exp x_2)\]
Montrer que $f$ est de classe $C^1$ et calculer sa différentielle.
\end{exer}

\begin{exer}
Soit $U$ un ouvert de $\mathbb{R}^2$, connexe par arcs : %
on s'int\'eresse aux fonctions $f$, diff\'erentiables, de $U$ dans $\mathbb{R}$, telles que la fonction $\frac{\partial{f}}{\partial{x}}$ est nulle sur $U$.
\begin{enumerate}
\item Donner un exemple, pour $U$ et $f$, tel qu'il n'existe pas de fonction $g$ de $\mathbb{R}$ dans $\mathbb{R}$, telle que :
\[\forall (x,y)\in U , f(x,y)=g(y)\]
\item Donner une condition simple sur $U$, suffisante, pour qu'une telle fonction $g$ existe pour tout $f$ qui satisfait \`a l'hypoth\`ese de l'exercice.
\end{enumerate}
\end{exer}

\begin{exer}
R\'esoudre sur $\mathbb{R}^2$, \`a l'aide d'un changement de variable, l'\'equation aux d\'eriv\'ees partielles :
\[a\frac{\partial{f}}{\partial{x}}+b\frac{\partial{f}}{\partial{y}}=0\]
o\`u $f$ est une application diff\'erentiable de $\mathbb{R}^2$ dans $\mathbb{R}$, et o\`u l'un au moins de nombres $a$ et $b$ est non nul.
\end{exer}

\begin{exer}
Soient $n$ un entier strictement positif, et $f$ une application convexe de $\mathbb{R}^n$ dans $\mathbb{R}$.
\begin{enumerate}
\item Montrer que si $f$ admet un minimum local, alors ce minimum est global.
\item Montrer que si, de plus, $f$ est strictement convexe, alors ce minimum est unique et que : $f(x)\underset{\|x\|\rightarrow +\infty}{\longrightarrow}+\infty$.
\end{enumerate}
\end{exer}

\begin{exer}
Soit $K$ un espace topologique compact. On note $E$ l'espace vectoriel des applications continues de $K$ dans $\mathbb{R}$, que l'on munit de $\| \|_{\infty}$. %
Soient $g$ une application de classe $C^{1}$ de $\mathbb{R}$ dans $\mathbb{R}$, et $\Phi$ l'application :
\[E \rightarrow E : f \mapsto g \circ f\]
\begin{enumerate}
\item Montrer que $\Phi$ est bien définie.
\item Montrer que $\Phi$ est différentiable et calculer sa différentielle.
\item Calculer, à l'aide de ce résultat, les différentielles respectives des applications $\phi_1$ et $\phi_2$ définies par :
\[\forall u \in E , \phi_1 (u) = x \mapsto u^p(x)\]
\[\forall u \in E , \phi_2 (u) = x \mapsto \exp(u(x)) + 3 \sin u(x_0) , x_0 \in K\]
\end{enumerate}
\end{exer}

\begin{exer}
Pour chacune des fonctions :
\[f:(x,y)\mapsto\frac{x^3}{x^2+y^2}\] et \[g:(x,y)\mapsto\frac{x^3y}{x^4+y^2}\]
d\'efinies sur $\mathbb{R}^2\setminus\{(0,0)\}$, \'etudier :

\smallskip
\begin{itemize}
\item La diff\'erentiabilit\'e sur $\mathbb{R}^2\setminus\{(0,0)\}$ ;
\item La prolongeabilit\'e en $(0,0)$ ;
\item La diff\'rentiabilit\'e, en $(0,0)$, de l'\'eventuel prolongement.
\end{itemize}
\end{exer}

\begin{exer}
Soit $U$ un ouvert non vide de $\mathbb{R}^2$.
\begin{enumerate}
\item Soit $f$ une application de classe $\mathcal{C}^2$ de $U$ dans $\mathbb{R}$. %
On suppose que $\Delta f\geq f$ et que $f$ admet un maximum dans $U$. %
Montrer que $f$ est n\'egative sur $U$.
\item On suppose maintenant que $U$ est born\'e, et on note $\Gamma$ la fronti\`ere de $U$. %
Soit $g$ une application continue de $\Gamma$ dans $\mathbb{R}$. %
Montrer qu'il existe au plus une application $h$ de classe $\mathcal{C}^2$ de $U$ dans $\mathbb{R}$ telle que :
\begin{enumerate}
\item $h$ admet un prolongement continu commun avec $g$ sur $\overline{U}$
\item $\Delta h=h$
\end{enumerate}
\end{enumerate}
\end{exer}

\begin{exer}[Contre-exemple pour le lemme de Schwarz]
Soit $f$ l'application de $\mathbb{R}^2$ dans $\mathbb{R}$ d\'efinie par :
\begin{equation}\forall (x,y)\in\mathbb{R}^2
\left\{
\begin{aligned}x=0\vee y=0 &\Rightarrow f(x,y)=0\\
x\neq 0\wedge y\neq 0 &\Rightarrow f(x,y)=x^2\arctan\frac{y}{x}-y^2\arctan\frac{x}{y}
\end{aligned}
\right.
\end{equation}
Montrer que $f$ est de classe $\mathcal{C}^1$ sur $\mathbb{R}^2$, %
que $\frac{\partial^2{f}}{\partial{x}\partial{y}}$ et $\frac{\partial^2{f}}{\partial{y}\partial{x}}$ existent, mais sont diff\'erentes en $(0,0)$.
\end{exer}

\begin{exer}
Soient $n$ un entier strictement sup\'erieur \`a $1$, et soit $f$ une application diff\'erentiable de $\mathbb{R}^n$ dans $\mathbb{R}$, telle que :
\[\frac{|f(x)|}{\| x\|}\underset{\| x\|\rightarrow +\infty}{\longrightarrow}+\infty\]
Montrer que l'application $\overrightarrow{grad}(f)$ est surjective.
\end{exer}

\begin{exer}
Montrer que la s\'erie de terme g\'en\'eral :
\[U_n:(x,y)\mapsto \frac{1}{n^2}\exp(-n(x^2+y^2))\]
converge, en un sens que l'on pr\'ecisera, vers une application de classe $\mathcal{C}^1$ de $\mathbb{R}^2$ dans $\mathbb{R}$.
\end{exer}

\begin{exer}
On note respectivement $(Ox)_+$ et $(Oy)_+$ les deux demi-axes positifs des abcisses et ordonnées de $\mathbb{R}^2$.\\
Existe-t-il un arc $C^1$ dont le support soit égal à $(Ox)_+ \bigcup (Oy)_+$ ? Un tel arc peut-il être régulier ?
\end{exer}
% \input{calcul_differentiel/anc_acc_finis.tex}
% \section{Ancien programme : g\'eom\'etrie diff\'erentielle}

\begin{exer}
Soit $f$ une fonction r\'eelle \`a valeurs complexes, d\'erivable et  $2 \pi$ p\'eriodique.\\
Montrer que : $\frac{1}{2i\pi} \int_0^{2 \pi} \frac{f'(t)}{f(t)} dt$ est un entier.
\end{exer}

\begin{exer}
On se propose de démontrer l'inégalité isopérimétrique :\\
si $\gamma$ est un arc de Jordan régulier de $\mathbb{R}^2$, de longueur $l$ et dont l'intérieur admet pour aire $A$, alors :
\[4 \pi A \leq l^2\]
\begin{enumerate}
\item Démontrer l'inégalité de Wirtinger : si $y$ est une application $2 \pi$-périodique $C^1$ de $\mathbb{R}$ dans lui-même, alors :
\[\int_{0}^{2 \pi} y^2 \leq \int_{0}^{2 \pi} y'^2\]
Etudier le cas d'égalité.

\smallskip
On reprend les notations de l'énoncé, et on écrit : $\gamma := (x,y)$.
\item Montrer que l'on peut supposer $\gamma$ centré en $(0,0)$, et de longueur $2 \pi$.

\smallskip
On supposera par la suite que $\gamma$ est paramétré par longueur d'arc.
\item Démontrer, en utilisant la formule de Green-Riemann, que :\[2A \leq \int_{0}^{2 \pi} (x^2 (s) + y'^2 (s))ds\]
\item Conclure.
%On remarquera que : $\int_{0}^{2 \pi} (x'^2 (s) + y'^2 (s)) ds = \frac{l^2}{2 \pi}$
\item Etudier le cas d'égalité.
\end{enumerate}
\end{exer}


% \section{Calcul diff\'erentiel \'el\'ementaire}

\begin{exer}
Soit : $E = C^0([0,1],\mathbb{R})$\\
Montrer que les applications suivantes sont différentiables et calculer leurs différentielles :
\begin{enumerate}
\item $f_1 : \mathbb{R}^n \rightarrow \mathbb{R}^m : x \mapsto Ax + b$, $A \in M_{m,n}(\mathbb{R})$, $b \in \mathbb{R}^m$
\item $f_2 : E \rightarrow \mathbb{R} : u \mapsto \int_{0}^1 u(x) dx + 3u(0)$
\item $f_3 : M_n(\mathbb{R}) \times \mathbb{R}^n \rightarrow \mathbb{R}^n : (A,b) \mapsto Ab$
\item $f_4 : E^3 \rightarrow E : (u,v,w) \mapsto (x \mapsto (x^2 + 2)u(x)v(x)\int_{0}^1 a(t)w(t)dt + u(0)$, $a \in E$.
\end{enumerate}
\end{exer}

\begin{exer}
Soient $n$ un entier naturel non nul, et une application diff\'erentiable de $\mathbb{R}^n$ dans $\mathcal{L}(\mathbb{R}^n)$.\\
Montrer de deux mani\`eres que l'application de $\mathbb{R}^n$ dans $\mathbb{R}^n$ :
\[x\mapsto f(x)(x)\]
est diff\'erentiable :
\begin{itemize}
\item En évaluant directement $f(x+h)(x+h)$.
\item En l'exprimant comme composée de deux applications bien choisies.
\end{itemize}
\end{exer}

\begin{exer}
Soit :\[f : \mathbb{R}^2 \rightarrow \mathbb{R}^2 : (x_1,x_2) \mapsto (x_1^2 \sin x_2 , x_1 + \exp x_2)\]
Montrer que $f$ est de classe $C^1$ et calculer sa différentielle.
\end{exer}

\begin{exer}
Soit $U$ un ouvert de $\mathbb{R}^2$, connexe par arcs : %
on s'int\'eresse aux fonctions $f$, diff\'erentiables, de $U$ dans $\mathbb{R}$, telles que la fonction $\frac{\partial{f}}{\partial{x}}$ est nulle sur $U$.
\begin{enumerate}
\item Donner un exemple, pour $U$ et $f$, tel qu'il n'existe pas de fonction $g$ de $\mathbb{R}$ dans $\mathbb{R}$, telle que :
\[\forall (x,y)\in U , f(x,y)=g(y)\]
\item Donner une condition simple sur $U$, suffisante, pour qu'une telle fonction $g$ existe pour tout $f$ qui satisfait \`a l'hypoth\`ese de l'exercice.
\end{enumerate}
\end{exer}

\begin{exer}
R\'esoudre sur $\mathbb{R}^2$, \`a l'aide d'un changement de variable, l'\'equation aux d\'eriv\'ees partielles :
\[a\frac{\partial{f}}{\partial{x}}+b\frac{\partial{f}}{\partial{y}}=0\]
o\`u $f$ est une application diff\'erentiable de $\mathbb{R}^2$ dans $\mathbb{R}$, et o\`u l'un au moins de nombres $a$ et $b$ est non nul.
\end{exer}

\begin{exer}
Soient $n$ un entier strictement positif, et $f$ une application convexe de $\mathbb{R}^n$ dans $\mathbb{R}$.
\begin{enumerate}
\item Montrer que si $f$ admet un minimum local, alors ce minimum est global.
\item Montrer que si, de plus, $f$ est strictement convexe, alors ce minimum est unique et que : $f(x)\underset{\|x\|\rightarrow +\infty}{\longrightarrow}+\infty$.
\end{enumerate}
\end{exer}

\begin{exer}
Soit $K$ un espace topologique compact. On note $E$ l'espace vectoriel des applications continues de $K$ dans $\mathbb{R}$, que l'on munit de $\| \|_{\infty}$. %
Soient $g$ une application de classe $C^{1}$ de $\mathbb{R}$ dans $\mathbb{R}$, et $\Phi$ l'application :
\[E \rightarrow E : f \mapsto g \circ f\]
\begin{enumerate}
\item Montrer que $\Phi$ est bien définie.
\item Montrer que $\Phi$ est différentiable et calculer sa différentielle.
\item Calculer, à l'aide de ce résultat, les différentielles respectives des applications $\phi_1$ et $\phi_2$ définies par :
\[\forall u \in E , \phi_1 (u) = x \mapsto u^p(x)\]
\[\forall u \in E , \phi_2 (u) = x \mapsto \exp(u(x)) + 3 \sin u(x_0) , x_0 \in K\]
\end{enumerate}
\end{exer}

\begin{exer}
Pour chacune des fonctions :
\[f:(x,y)\mapsto\frac{x^3}{x^2+y^2}\] et \[g:(x,y)\mapsto\frac{x^3y}{x^4+y^2}\]
d\'efinies sur $\mathbb{R}^2\setminus\{(0,0)\}$, \'etudier :

\smallskip
\begin{itemize}
\item La diff\'erentiabilit\'e sur $\mathbb{R}^2\setminus\{(0,0)\}$ ;
\item La prolongeabilit\'e en $(0,0)$ ;
\item La diff\'rentiabilit\'e, en $(0,0)$, de l'\'eventuel prolongement.
\end{itemize}
\end{exer}

\begin{exer}
Soit $U$ un ouvert non vide de $\mathbb{R}^2$.
\begin{enumerate}
\item Soit $f$ une application de classe $\mathcal{C}^2$ de $U$ dans $\mathbb{R}$. %
On suppose que $\Delta f\geq f$ et que $f$ admet un maximum dans $U$. %
Montrer que $f$ est n\'egative sur $U$.
\item On suppose maintenant que $U$ est born\'e, et on note $\Gamma$ la fronti\`ere de $U$. %
Soit $g$ une application continue de $\Gamma$ dans $\mathbb{R}$. %
Montrer qu'il existe au plus une application $h$ de classe $\mathcal{C}^2$ de $U$ dans $\mathbb{R}$ telle que :
\begin{enumerate}
\item $h$ admet un prolongement continu commun avec $g$ sur $\overline{U}$
\item $\Delta h=h$
\end{enumerate}
\end{enumerate}
\end{exer}

\begin{exer}[Contre-exemple pour le lemme de Schwarz]
Soit $f$ l'application de $\mathbb{R}^2$ dans $\mathbb{R}$ d\'efinie par :
\begin{equation}\forall (x,y)\in\mathbb{R}^2
\left\{
\begin{aligned}x=0\vee y=0 &\Rightarrow f(x,y)=0\\
x\neq 0\wedge y\neq 0 &\Rightarrow f(x,y)=x^2\arctan\frac{y}{x}-y^2\arctan\frac{x}{y}
\end{aligned}
\right.
\end{equation}
Montrer que $f$ est de classe $\mathcal{C}^1$ sur $\mathbb{R}^2$, %
que $\frac{\partial^2{f}}{\partial{x}\partial{y}}$ et $\frac{\partial^2{f}}{\partial{y}\partial{x}}$ existent, mais sont diff\'erentes en $(0,0)$.
\end{exer}

\begin{exer}
Soient $n$ un entier strictement sup\'erieur \`a $1$, et soit $f$ une application diff\'erentiable de $\mathbb{R}^n$ dans $\mathbb{R}$, telle que :
\[\frac{|f(x)|}{\| x\|}\underset{\| x\|\rightarrow +\infty}{\longrightarrow}+\infty\]
Montrer que l'application $\overrightarrow{grad}(f)$ est surjective.
\end{exer}

\begin{exer}
Montrer que la s\'erie de terme g\'en\'eral :
\[U_n:(x,y)\mapsto \frac{1}{n^2}\exp(-n(x^2+y^2))\]
converge, en un sens que l'on pr\'ecisera, vers une application de classe $\mathcal{C}^1$ de $\mathbb{R}^2$ dans $\mathbb{R}$.
\end{exer}

\begin{exer}
On note respectivement $(Ox)_+$ et $(Oy)_+$ les deux demi-axes positifs des abcisses et ordonnées de $\mathbb{R}^2$.\\
Existe-t-il un arc $C^1$ dont le support soit égal à $(Ox)_+ \bigcup (Oy)_+$ ? Un tel arc peut-il être régulier ?
\end{exer}

% \chapter{Combinatoire}
\chapter{Combinatoire}


\section{Exercices}

\begin{exer}
Soit $(u_n)$ la suite numérique définie par :
$u_0 = 1$ ; $u_1 = 1$ ; $\forall n \in \mathbb{N} , u_{n+2} = u_{n+1} + 2u_n +(-1)^n$
On se propose de déterminer une formule du terme général de $(u_n)$, par la méthode dite des séries génératrices.
\begin{enumerate}
\item Montrer que : \[\forall n \in \mathbb{N} , u_n < 2^{n+1} - 1\]
et en déduire une minoration du rayon de convergence de la série entière $\sum z \mapsto u_n z^n$.

Cette série entière est appelée série génératrice de $(u_n)$.
\item Soit $f$ la somme de cette série sur son disque de convergence. %
Exprimer simplement $f(z)$, pour tout complexe $z$ dans ce domaine.
%Indications : (i) Utiliser la relation de récurrence qui définit $(u_n)$.\\
%(ii) Considérer, en premier lieu, le développement en série entière de la fonction $z \mapsto f(z) - z - 1$.\\
\item En déduire une formule de $u_n$.
%Indication : s'il ne l'a pas fait spontanément dans la question précédente, j'invite l'élève à décomposer la fonction rationelle $f$ en éléments simples.\\
%\textit{Je pose alors la question de la généralisation de cette démarche à la détermination des termes généraux d'autres suites, et celle de la nécessité de l'utilisation de l'analyse dans ces calculs -unicité du développement en série entière d'une fonction réelle ou complexe lorsque le rayon de convergence est non nul et manipulation d'une telle fonction sur son disque de convergence.\\
%J'indique que dans cet exercice, l'analyse, dont
%la question 1) quantifie le cadre de l'utilisation, intervient dès
%la question 2) lorsque l'on vérifie que la formule établie a bien un sens, et conjointement à cette formule par le biais de l'unicité du développement en série entière dans
%la question 3) où l'on résout le problème initial.\\
%J'ajoute que cependant, pour cette question et pour des problèmes apparentés qui se résolvent de la même manière, tout recours à l'analyse est superflu.\\
%En effet, l'utilisation de la relation de récurrence définissant $(u_n)$ pour relier $f$ à des polynômes, puis la décomposition en éléments simples de la fraction rationelle obtenue, sont possibles dans le cadre purement algébrique des séries formelles et des séries de Laurent, qui est l'objet d'un autre exercice sur cette planche.}
\end{enumerate}
\end{exer}

\begin{exer}
Soit $F_n$ la suite de Fibonacci, définie par :\[F_0 = 0 ; F_1 = 1 ; \forall n \in \mathbb{N} , F_{n+2} = F_{n+1} + F_n\]
\begin{enumerate}
\item Calculer la fonction génératrice de $(F_n)$ -je définis ce terme, si cela s'avère nécessaire.
\item En déduire une formule explicite du terme général de $(F_n)$.
\end{enumerate}
\end{exer}

\begin{exer}
On définit, pour tout entier naturel $n$, le $n-$ième nombre de Bell $B_n$ comme le nombre de partitions d'un ensemble à $n$ éléments.\\
\begin{enumerate}
\item Calculer $B_1$, $B_2$, $B_3$. Montrer que : $\forall n \in \mathbb{N} , B_{n+1} = \sum\limits_{k=0}^n C_n^k B_k$.

On appelle série génératrice exponentielle des nombres de Bell la série entière -réelle- %
$\sum x \mapsto \frac{B_n}{n!} x^n$.
\item Minorer le rayon de convergence de cette série entière, et exprimer cette fonction sur son disque de convergence.
\item En déduire une formule explicite du terme général de $B_n$.
\end{enumerate}
\end{exer}

\begin{exer}[Nombre de d\'erangements et série génératrice]
\begin{enumerate}
\item Montrer, à l'aide d'un argument combinatoire, que :\[\forall n \in \mathbb{N} n! = \sum\limits_{k=0}^n C_n^k D_{n-k}\]
On connaît ainsi une "transformée" de la suite $(D_n)$. Les calculs sont ici plus commodes avec la suite $\left(\frac{D_n}{n!}\right)$, la série formelle $\sum \frac{D_n}{n!} X^n$ est appelée série génératrice exponentielle de la suite $(D_n)$.\\
\item A l'aide d'une équation déduite de la formule établie à la question 1., calculer la série génératrice de $(D_n)$.\\
\item En déduire une formule explicite pour $D_n$.
\end{enumerate}
\end{exer}

\begin{exer}[Dérangements, autre méthode]
\begin{enumerate}
\item Montrer, à l'aide d'un argument combinatoire, que :\[\forall n \in \mathbb{N}^{\ast} D_{n+1} = n(D_n + D_{n-1})\]
%Indication : on discutera suivant la position de $n+1$ dans la décomposition en cycles d'un dérangement de $[1,n+1]$.\\
\item En déduire : \[\forall n \in \mathbb{N}^{\ast} D_n = n D_{n-1} + (-1)^n\]
\item Donner une formule explicite pour $D_n$.
%Indication -?- : on utilise encore $(\frac{D_n}{n!})$.
\end{enumerate}
\end{exer}

\begin{exer}
Un chef d'entreprise décide de distribuer des badges à ses employés, qui seront chacun identifiés par un numéro à cinq chiffres, %
afin de r\'eguler l'entr\'ee des diff\'erents locaux d'un site de production. %
Afin de limiter les risques d'erreur liés à une mauvaise lecture de la carte d'un employé, %
chaque numéro devra différer des numéros déjà affectés, de deux chiffres au moins.
\begin{enumerate}
\item Combien de badges est-il possible de réaliser compte tenu de cette contrainte ?
\item Pour les élèves de l'option informatique : écrire un algorithme, par exemple dans le langage Caml, %
donnant une distribution optimale de badges pour quatre chiffres.
\end{enumerate}
\end{exer}

%\subsection{Illustrer}

\begin{exer}
\begin{multicols}{2}
%
%\columnbreak
%
Combien trouve-t-on de parall\'elogrammes sur cette figure ?

\columnbreak

\psset{unit=0.6cm,algebraic=true}
\begin{pspicture*}(0,0)(6,6)
\pstilt{75}{
\psaxes[xAxis=true,yAxis=true,labels=none]{->}(0,0)(5,5)
%\psline[linewidth=1pt]
\multido{\na=1+1,\nb=5+-1}{5}%{\psframe(0,\n)(5-\n,1+\n)}
{\psline(0,\na)(\nb,\na)}
\multido{\na=1+1,\nb=5+-1}{5}%{\psframe(0,\n)(5-\n,1+\n)}
{\psline(\na,0)(\na,\nb)}
}
\end{pspicture*}


\psset{xunit=0.5cm,yunit=0.5cm}%,algebraic=true}
\end{multicols}
\end{exer}

%\begin{pspicture*}(0,0)(5,5)
%\pspolygon(0,0)(1,0)(1.2,1)(0.2,1)

%\psaxes[xAxis=true,yAxis=true]{->}(0,0)(-2,-4)(2,4)

%\psline[linecolor=ldgris,linewidth=0.4pt,linestyle=dashed]{*-*}(4,6)(6,6)

%\psline[linecolor=black,linewidth=0.6pt,linestyle=dotted]{*-}(5,1)(5.8,1)
%\psline[linecolor=black,linewidth=0.6pt,linestyle=dotted]{*-}(2,3)(2,3.8)
%\end{pspicture*}



\begin{exer}[Labyrinthes]
Soit $n$ un entier naturel non nul. On construit un labyrinthe, sur un carr\'e de $n\times n$ cases entour\'e de murs infranchissables, %
en ajoutant des parois aux bords des $n^2$ cases.
\begin{multicols}{2}
Cas $3\times 3$ : 5 parois, bloquage.

\vspace{3cm}

Cas $4\times 4$ : 9 parois. Chaque case est accessible depuis n'importe quelle autre.
\columnbreak

\psset{unit=1cm}
\begin{pspicture*}(0,0)(3,3)
\psframe[linewidth=1.8pt](0,0)(3,3)
\psframe[fillstyle=solid,fillcolor=gray](0,2)(2,3)
\psgrid[gridwidth=0.6pt,gridcolor=darkgray,subgriddiv=1,gridlabels=0](0,0)(3,3)
\psline[linewidth=0.9pt](0,2)(2,2)
\psline[linewidth=0.9pt](2,2)(2,3)
\psline[linewidth=0.9pt](1,0)(1,1)
\psline[linewidth=0.9pt](1,1)(2,1)
\end{pspicture*}

\bigskip

\psset{unit=0.8cm}
\begin{pspicture*}(0,0)(4,4)
\psframe[linewidth=1.8pt](0,0)(4,4)
\psgrid[gridwidth=0.6pt,gridcolor=gray,subgriddiv=1,gridlabels=0](0,0)(4,4)
\psline[linewidth=0.9pt](0,3)(2,3)
\psline[linewidth=0.9pt](0,2)(1,2)
\psline[linewidth=0.9pt](0,1)(1,1)
\psline[linewidth=0.9pt](2,0)(2,2)
\psline[linewidth=0.9pt](3,1)(4,1)
\psline[linewidth=0.9pt](3,2)(4,2)
\psline[linewidth=0.9pt](3,3)(3,4)
%\psline[linewidth=0.5pt](2,2)(2,3)
%\psline[linewidth=0.5pt](0,2)(2,2)
\end{pspicture*}
\end{multicols}
D\'eterminer, en fonction de $n$, le nombre maximal de parois que l'on peut ajouter en gardant le labyrithe connexe, %
c'est-\g{a}-dire de sorte que l'on puisse atteindre une case quelconque du labyrithe depuis n'importe quelle autre, %
compte tenu des contraintes impos\'ees par les quatre murs d'enceinte et les parois.
\end{exer}

% \section{Exercices}

\begin{exer}
Soit $(u_n)$ la suite numérique définie par :
$u_0 = 1$ ; $u_1 = 1$ ; $\forall n \in \mathbb{N} , u_{n+2} = u_{n+1} + 2u_n +(-1)^n$
On se propose de déterminer une formule du terme général de $(u_n)$, par la méthode dite des séries génératrices.
\begin{enumerate}
\item Montrer que : \[\forall n \in \mathbb{N} , u_n < 2^{n+1} - 1\]
et en déduire une minoration du rayon de convergence de la série entière $\sum z \mapsto u_n z^n$.

Cette série entière est appelée série génératrice de $(u_n)$.
\item Soit $f$ la somme de cette série sur son disque de convergence. %
Exprimer simplement $f(z)$, pour tout complexe $z$ dans ce domaine.
%Indications : (i) Utiliser la relation de récurrence qui définit $(u_n)$.\\
%(ii) Considérer, en premier lieu, le développement en série entière de la fonction $z \mapsto f(z) - z - 1$.\\
\item En déduire une formule de $u_n$.
%Indication : s'il ne l'a pas fait spontanément dans la question précédente, j'invite l'élève à décomposer la fonction rationelle $f$ en éléments simples.\\
%\textit{Je pose alors la question de la généralisation de cette démarche à la détermination des termes généraux d'autres suites, et celle de la nécessité de l'utilisation de l'analyse dans ces calculs -unicité du développement en série entière d'une fonction réelle ou complexe lorsque le rayon de convergence est non nul et manipulation d'une telle fonction sur son disque de convergence.\\
%J'indique que dans cet exercice, l'analyse, dont
%la question 1) quantifie le cadre de l'utilisation, intervient dès
%la question 2) lorsque l'on vérifie que la formule établie a bien un sens, et conjointement à cette formule par le biais de l'unicité du développement en série entière dans
%la question 3) où l'on résout le problème initial.\\
%J'ajoute que cependant, pour cette question et pour des problèmes apparentés qui se résolvent de la même manière, tout recours à l'analyse est superflu.\\
%En effet, l'utilisation de la relation de récurrence définissant $(u_n)$ pour relier $f$ à des polynômes, puis la décomposition en éléments simples de la fraction rationelle obtenue, sont possibles dans le cadre purement algébrique des séries formelles et des séries de Laurent, qui est l'objet d'un autre exercice sur cette planche.}
\end{enumerate}
\end{exer}

\begin{exer}
Soit $F_n$ la suite de Fibonacci, définie par :\[F_0 = 0 ; F_1 = 1 ; \forall n \in \mathbb{N} , F_{n+2} = F_{n+1} + F_n\]
\begin{enumerate}
\item Calculer la fonction génératrice de $(F_n)$ -je définis ce terme, si cela s'avère nécessaire.
\item En déduire une formule explicite du terme général de $(F_n)$.
\end{enumerate}
\end{exer}

\begin{exer}
On définit, pour tout entier naturel $n$, le $n-$ième nombre de Bell $B_n$ comme le nombre de partitions d'un ensemble à $n$ éléments.\\
\begin{enumerate}
\item Calculer $B_1$, $B_2$, $B_3$. Montrer que : $\forall n \in \mathbb{N} , B_{n+1} = \sum\limits_{k=0}^n C_n^k B_k$.

On appelle série génératrice exponentielle des nombres de Bell la série entière -réelle- %
$\sum x \mapsto \frac{B_n}{n!} x^n$.
\item Minorer le rayon de convergence de cette série entière, et exprimer cette fonction sur son disque de convergence.
\item En déduire une formule explicite du terme général de $B_n$.
\end{enumerate}
\end{exer}

\begin{exer}[Nombre de d\'erangements et série génératrice]
\begin{enumerate}
\item Montrer, à l'aide d'un argument combinatoire, que :\[\forall n \in \mathbb{N} n! = \sum\limits_{k=0}^n C_n^k D_{n-k}\]
On connaît ainsi une "transformée" de la suite $(D_n)$. Les calculs sont ici plus commodes avec la suite $\left(\frac{D_n}{n!}\right)$, la série formelle $\sum \frac{D_n}{n!} X^n$ est appelée série génératrice exponentielle de la suite $(D_n)$.\\
\item A l'aide d'une équation déduite de la formule établie à la question 1., calculer la série génératrice de $(D_n)$.\\
\item En déduire une formule explicite pour $D_n$.
\end{enumerate}
\end{exer}

\begin{exer}[Dérangements, autre méthode]
\begin{enumerate}
\item Montrer, à l'aide d'un argument combinatoire, que :\[\forall n \in \mathbb{N}^{\ast} D_{n+1} = n(D_n + D_{n-1})\]
%Indication : on discutera suivant la position de $n+1$ dans la décomposition en cycles d'un dérangement de $[1,n+1]$.\\
\item En déduire : \[\forall n \in \mathbb{N}^{\ast} D_n = n D_{n-1} + (-1)^n\]
\item Donner une formule explicite pour $D_n$.
%Indication -?- : on utilise encore $(\frac{D_n}{n!})$.
\end{enumerate}
\end{exer}

\begin{exer}
Un chef d'entreprise décide de distribuer des badges à ses employés, qui seront chacun identifiés par un numéro à cinq chiffres, %
afin de r\'eguler l'entr\'ee des diff\'erents locaux d'un site de production. %
Afin de limiter les risques d'erreur liés à une mauvaise lecture de la carte d'un employé, %
chaque numéro devra différer des numéros déjà affectés, de deux chiffres au moins.
\begin{enumerate}
\item Combien de badges est-il possible de réaliser compte tenu de cette contrainte ?
\item Pour les élèves de l'option informatique : écrire un algorithme, par exemple dans le langage Caml, %
donnant une distribution optimale de badges pour quatre chiffres.
\end{enumerate}
\end{exer}

%\subsection{Illustrer}

\begin{exer}
\begin{multicols}{2}
%
%\columnbreak
%
Combien trouve-t-on de parall\'elogrammes sur cette figure ?

\columnbreak

\psset{unit=0.6cm,algebraic=true}
\begin{pspicture*}(0,0)(6,6)
\pstilt{75}{
\psaxes[xAxis=true,yAxis=true,labels=none]{->}(0,0)(5,5)
%\psline[linewidth=1pt]
\multido{\na=1+1,\nb=5+-1}{5}%{\psframe(0,\n)(5-\n,1+\n)}
{\psline(0,\na)(\nb,\na)}
\multido{\na=1+1,\nb=5+-1}{5}%{\psframe(0,\n)(5-\n,1+\n)}
{\psline(\na,0)(\na,\nb)}
}
\end{pspicture*}


\psset{xunit=0.5cm,yunit=0.5cm}%,algebraic=true}
\end{multicols}
\end{exer}

%\begin{pspicture*}(0,0)(5,5)
%\pspolygon(0,0)(1,0)(1.2,1)(0.2,1)

%\psaxes[xAxis=true,yAxis=true]{->}(0,0)(-2,-4)(2,4)

%\psline[linecolor=ldgris,linewidth=0.4pt,linestyle=dashed]{*-*}(4,6)(6,6)

%\psline[linecolor=black,linewidth=0.6pt,linestyle=dotted]{*-}(5,1)(5.8,1)
%\psline[linecolor=black,linewidth=0.6pt,linestyle=dotted]{*-}(2,3)(2,3.8)
%\end{pspicture*}



\begin{exer}[Labyrinthes]
Soit $n$ un entier naturel non nul. On construit un labyrinthe, sur un carr\'e de $n\times n$ cases entour\'e de murs infranchissables, %
en ajoutant des parois aux bords des $n^2$ cases.
\begin{multicols}{2}
Cas $3\times 3$ : 5 parois, bloquage.

\vspace{3cm}

Cas $4\times 4$ : 9 parois. Chaque case est accessible depuis n'importe quelle autre.
\columnbreak

\psset{unit=1cm}
\begin{pspicture*}(0,0)(3,3)
\psframe[linewidth=1.8pt](0,0)(3,3)
\psframe[fillstyle=solid,fillcolor=gray](0,2)(2,3)
\psgrid[gridwidth=0.6pt,gridcolor=darkgray,subgriddiv=1,gridlabels=0](0,0)(3,3)
\psline[linewidth=0.9pt](0,2)(2,2)
\psline[linewidth=0.9pt](2,2)(2,3)
\psline[linewidth=0.9pt](1,0)(1,1)
\psline[linewidth=0.9pt](1,1)(2,1)
\end{pspicture*}

\bigskip

\psset{unit=0.8cm}
\begin{pspicture*}(0,0)(4,4)
\psframe[linewidth=1.8pt](0,0)(4,4)
\psgrid[gridwidth=0.6pt,gridcolor=gray,subgriddiv=1,gridlabels=0](0,0)(4,4)
\psline[linewidth=0.9pt](0,3)(2,3)
\psline[linewidth=0.9pt](0,2)(1,2)
\psline[linewidth=0.9pt](0,1)(1,1)
\psline[linewidth=0.9pt](2,0)(2,2)
\psline[linewidth=0.9pt](3,1)(4,1)
\psline[linewidth=0.9pt](3,2)(4,2)
\psline[linewidth=0.9pt](3,3)(3,4)
%\psline[linewidth=0.5pt](2,2)(2,3)
%\psline[linewidth=0.5pt](0,2)(2,2)
\end{pspicture*}
\end{multicols}
D\'eterminer, en fonction de $n$, le nombre maximal de parois que l'on peut ajouter en gardant le labyrithe connexe, %
c'est-\g{a}-dire de sorte que l'on puisse atteindre une case quelconque du labyrithe depuis n'importe quelle autre, %
compte tenu des contraintes impos\'ees par les quatre murs d'enceinte et les parois.
\end{exer}

% \chapter{Probabilit\'es}
\chapter{Probabilit\'es}

\section{Formalisme des probabilit\'es}

\begin{exer}[Tribus sur $\mathbb{N}$]
Soit $\mathcal{B}$ une tribu sur $\mathbb{N}$. Pour tout entier naturel $a$, on note $B_a$ l'ensemble des \'el\'ements $b$ de $\mathbb{N}$ tels que :
\[\forall B \in \mathcal{B} , \{a,b\}\subseteq B \vee \{a,b\}\subseteq B^c\]
Montrer, \`a l'aide de la famille $(B_a)_a$, qu'il existe une famille $(A_i)$ de parties de $\mathbb{N}$ telles que tout \'el\'ement de $\mathcal{B}$ est une union disjointe de $A_i$.
\end{exer}

\begin{exer}
Soit $X$ une variable al\'eatoire \`a valeurs dans $\mathbb{N}$.\\
Montrer que : \[\mathbb{E}(X)=\sum\limits_{n=0}^{\infty} \mathbb{P}(X>n)\]
\end{exer}

\begin{exer}
Montrer qu'une intersection d\'enombrable d'\'ev\`enements presque s\^urs est encore un \'ev\`enement presque s\^ur.\\
En est-il de même pour une intersection non d\'enombrable ?
\end{exer}
\section{calculs \'el\'ementaires}

\begin{exer}[Une distraction du savant cosinus]
%\begin{center}
%\begin{minipage}{0.8\linewidth}
\textit{N voyageurs (au moins deux) s'apprêtent à monter dans un avion contenant N places num\'erot\'ees. %
Le premier d'entre eux s'av\`ere être le savant cosinus qui, distrait comme il l'est toujours, ne regarde pas le num\'ero de sa r\'eservation. %
Les passagers suivants, quand ils montent, en ordre, dans l'avion, s'asseyent alors \`a leur place r\'eserv\'ee si elle est encore libre, %
et sinon choisissent une place au hasard parmi celles qui restent.}
%\end{minipage}
%\end{center}
\ligneinter
\begin{enumerate}
\item Calculer, par r\'ecurrence sur $N$, la probabilit\'e $p_N$ que le dernier passager soit assis \`a sa place.
\item Retrouver le r\'esultat pr\'ec\'edent par un argument direct.
\end{enumerate}
\end{exer}

\begin{exer}
On jette plusieurs fois de suite et ind\'ependamment une pi\`ece de monnaie non \'equilibr\'ee, la probabilit\'e de tomber sur "Pile" est un r\'eel $p$ compris entre $0$ et $1$ strictement.\\
Calculer :
\begin{enumerate}
\item La probabilit\'e de ne pas avoir de "Face" au cours des $n$ premiers jets pour tout $n$ sup\'erieur \`a $1$ ;
\item La prpobabilit\'e d'obtenir "Face" pour la premi\`ere fios au $n-$i\`eme jet ;
\item l'esp\'erance du nombre de jets jusqu'\`a la premi\`ere apparition de "Face".
\end{enumerate}
\end{exer}

\begin{exer}
Pour tout entier $n$ sup\'erieur ou \'egal à $1$, on consid\`ere $n$ boules num\'erot\'ees de $1$ \`a $n$, que l'on place dans $n$ urnes, num\'erot\'ees de la m\^eme mani\`ere.
\begin{enumerate}
\item Calculer, pour tout $n$, la probabilit\'e $p_n$ de l'\'ev\`enement : chaque urne contient exactement une boule \`a la fin de l'op\'eration.
\item Montrer que la suite $(p_n)$ est d\'ecroissante et tend vers $0$ en $+\infty$.
\end{enumerate}
\end{exer}
\section{Lois de probabilit\'es}

\begin{exer}[Recensement d'une population d'\'ecureuils]
%\begin{center}
%\begin{minipage}{\0.8\linewidth}
\textit{On veut estimer le nombre $N$ d'\'ecureuils dans une for\^et. %
Pour cela on en capture $k$, on leur met une petite marque sur la patte et on les rel\^ache. %
Une semaine apr\`es (on suppose qu'aucun \'ecureuil est mort ou n\'e dans cet intervalle), %
on en capture $l$ et on compte ceux d'entre eux qui portent la marque.}
%\end{minipage}
%\end{center}
\ligneinter
\begin{enumerate}
\item Calculer la probabilit\'e d'observer une valeur $m$ \'ecureuils marqu\'es  en fonction de $N$, $k$ et $l$.
\item Calculer, quand $N$ est grand, la valeur $N_{max}$ pour laquelle la probabilit\'e ci-dessus est la plus grande. On appelle cette valeur \textit{estimateur du maximum de vraisemblance} pour le nombre d'\'ecureuils.
\end{enumerate}
\end{exer}

\begin{exer}
Soit $(X_n)$ une suite de variables al\'eatoires de lois respectives $\mathcal{B}(n,p_n)$ o\`u $np_n$ est constant, de valeur unique $\lambda$.\\
On note, pour tout $n$ : $A_n = (X_n \geq 1)$.\\
Soit, de plus, $Y$ une variable de Poisson de param\`etre $\lambda$.\\
Montrer, pour $j$ fix\'e et sup\'erieur \`a $1$, que :\[\mathbb{P}(X_n=j|A_n)\underset{n\rightarrow +\infty}{\longrightarrow}\mathbb{P}(Y=j|y\geq 1)\]
\end{exer}

\begin{exer}
Soit $X$ une variable al\'eatoire de loi $\mathcal{B}(n,p)$.\\
Quel entier $j$ compris entre $0$ et $n$, maximise $\mathbb{P}(X=j)$ ?
\end{exer}

\begin{exer}[Optimisations avec la loi de Poisson]
\begin{enumerate}
\item Soit $\lambda$ un r\'eel strictement positif. Quel entier $j$ maximise $\mathbb{P}(X=j)$, pour une variable de Poisson de param\`etre $\lambda$ ?
\item Soit $j$ un entier positif. Quel est le r\'eel $\lambda$ pour lequel $\mathbb{P}(X=j)$ et maximal, o\`u $X$ est choisie avec une loi $\mathcal{P}(\lambda)$ ?
\end{enumerate}
\end{exer}

\begin{exer}
Soit $X$ une variable al\'eatoire de loi $\mathcal{G}(p)$.\\
Montrer que :\[\mathbb{E}\left(\frac{1}{1+X}\right)=\ln((1-p)^{\frac{p}{p-1}}))\]
\end{exer}

%\newpage

\begin{exer}
Soit $p$ un réel compris, strictement entre $0$ et $1$, et soient $X$ et $Y$ deux variables al\'eatoires ind\'ependantes, de m\^eme loi $\mathcal{B}(p)$.\\
Donner les lois de $max(X,Y)$ et $min(X,Y)$.
\end{exer}

\begin{exer}
Soient $X$ et $Y$ deux variables al\'eatoires d\'efinies sur un m\^eme espace $\Omega$, ind\'ependantes, et telles que :
\begin{itemize}
\item $X$ suit une loi de Poisson de param\`etre $\lambda$ ;
\item $Y$ vaut $1$ ou $2$, avec la m\^eme probabilit\'e $\frac{1}{2}$.
\end{itemize}
\begin{enumerate}
\item Donner la loi de la variable $Z$ d\'efinie par : $Z=XY$.
\item Calculer la probabilit\'e que $Z$ prenne une valeur paire.
\end{enumerate}
\end{exer}

\begin{exer}
Soit $X$ une variable al\'eatoire de Poisson, de param\`etre $\lambda$. On note $Y$ la variable $(-1)^X$.
\begin{enumerate}
\item Donner l'ensemble des valeurs prises par $Y$, puis calculer son esp\'erance.
\item D\'eterminer la loi de $Y$.
\end{enumerate}
\end{exer}
\section{$L^2$}

\begin{exer}
Soit $X$ une variable al\'eatoire $L^2$. Quel r\'eel minimise $c\mapsto\mathbb{E}((X-c)^2)$ ? Quel est le minimum atteint ?
\end{exer}

\begin{exer}
Soit $X$ une variable al\'eatoire $L^2$.\\
On suppose que : $V(X)=0$.\\
Montrer que $X$ prend la valeur $\mathbb{E}(X)$ avec une probabilit\'e $1$.
\end{exer}

\begin{exer}[L'in\'egalit\'e de Tchebycheff est-elle optimale ?]
\begin{enumerate}
\item (Oui) Montrer que, si $a$ est un r\'eel strictement positif, alors il existe une variable al\'eatoire $X$, $L^2$, telle que :\[\mathbb{P}(|X-\mathbb{E}(X)|\geq a)=\frac{V(X)}{a^2}\]
\item (Non) Soit $X$ une variable al\'eatoire $L^2$. Montrer que : \[a^2\mathbb{P}(|X-\mathbb{E}(X)|\geq a)\underset{a\rightarrow\infty}{\rightarrow}0\]
\end{enumerate}
\end{exer}
\section{Classiques, ou presque \dots mais beaux !}

\begin{exer}[Ruine du joueur]
\textit{
Un joueur dispose d'une mise initiale de $k$ euros. %
A chaque partie qu'il joue, il a la probabilit\'e $p$ de perdre un euro, dans le cas contraire sa mise augmente d'un euro. %
Ce joueur se fixe comme objectif de jouer jusqu'\`a ce qu'il soit ruin\'e, ou jusqu'\`a ce que son capital atteigne la somme de $M$ euros, o\`u $M$ est un entier fix\'e \`a l'avance.
}
\ligneinter
\begin{enumerate}
\item On veut calculer la probabilit\'e de l'\'ev\`enement : "le joueur finit ruin\'e".
\begin{enumerate}
\item Mod\'eliser, pour une mise $k$ quelconque au d\'ebut du jeu, le rpobl\`eme comme une marche al\'eatoire.

\medskip
Pour tout entier $k$ compris entre $0$ et $M$, on note $E_k$ l'\'ev\'enement correspondant \`a : %
"le joueur, ayant mis\'e $k$ euros, finit ruin\'e avant d'avoir enpoch\'e un capital de $M$". On note $q_k$ la probabinit\'e correspondant \`a un tel \'ev\`enement.
\item Calculer $q_0$ et $q_M$.
\item Montrer la relation de r\'ecurrence : $q_j=(1-p)q_{j+1}+pq_{j-1}$, pour tout indice $j$ o\`u cette formule est d\'efinie.
\item Montrer, pour tout $j$, que : $q_{j+1}-q_j=\left(\frac{p}{1-p}\right)^j(q_1-q_0)$
\item En d\'eduire que : $\forall k \in [\![0,M]\!] , q_k-q_0=(q_1-q_0)\sum\limits_{j=0}^{k-1}\left(\frac{p}{1-p}\right)^j$.
\item En appliquant la formule pr\'ec\'edente au rang $M$, donner une expression de $q_1$ en fonction de $p$. %
On distinguera bien le cas o\`u $p=\frac{1}{2}$.
\item Donner une expression de $q_k$ en fonction de $p$.
\end{enumerate}
\item Soit maintenant pour tout $k$ compris entre $0$ et $M$, $F_k$ l'\'ev\`enement : "le joueur empoche la mise de $M$ euros sans avoir jamais \'et\'e ruin\'e".
\begin{enumerate}
\item Peut-on dire que les \'ev\`enements $E_k$ et $F_k$ sont contraires ?
\item En utilisant la m\^eme m\'ethode que pour $E_k$, exprimer la probabilit\'e de $F_k$ en fonction de $p$. On distiguera bien le cas o\`u $p=\frac{1}{2}$.
\item Conclusion ?
\end{enumerate}
\end{enumerate}
\end{exer}

\begin{exer}[Nombre de cycles dans une grande permutation al\'eatoire]
Dans cet exercice, on cherche \`a estimer le nombre de cycles d'une permutation choisie au hasard dans le groupe sym\'etrique $\mathcal{S}_n$ de $[\![1,n]\!]$, quand $n$ est un entier naturel qui tend vers $+\infty$.\\
On \'etudie donc une suite $(S_n)$ de variables al\'eatoires, de lois uniformes, respectivement sur les termes de $(\mathcal{S}_n)$.\\
Pour toute permutation $\sigma$ d'un ensemble $[\![1,n]\!]$, on note $c(\sigma)$ le nombre de cycles de $\sigma$.\\
Par exemple : $c(S_n)=1$ lorsque la valeur de $S_n$ est un cycle, $c(S_n)=n$ lorsque $S_n$ prend comme valeur l'identit\'e.
\ligneinter
\vspace{-0.2pt}
\begin{enumerate}
\item Pour tout entier $n$ strictement positif, on note $\Phi^n$ l'application :
\[\mathcal{S}_n\times [\![1,n+1]\!] \longrightarrow \mathcal{S}_n : (\sigma ,i) \mapsto \tilde{\sigma}\circ (i , n+1)\]
o\`u $\tilde{\sigma}$ est la permutation de $[\![1,n+1]\!]$ qui agit comme $\sigma$ sur $[\![1,n]\!]$, et fixe $n+1$.\\
Montrer que cette application est une bijection.
\item Avec les conventions qui pr\'ec\`edent, discuter, selon $\sigma$ et $i$, de la valeur de $c(\Phi^n(\sigma))$ en fonction de $c(\sigma)$.

\medskip
On se fixe maintenant un entier naturel non nul $n$ et on cherche \`a conna\^itre la loi de $S_n$.\\
Soit $(U_k)_{k\in [\![1,n-1]\!]}$ une famille de variables al\'eatoires, ind\'ependantes, respectivement sur les termes de $([\![1,k+1]\!])_k$. Soit $(X_k)$ la suite de variables al\'eatoires, d\'efinie par r\'ecurrence finie sur $k$ par :\\
$X_1$ prend seulement la valeur $Id_{\{1\}}$ ; $\forall k \in [\![1,n-1]\!] , X_{k+1}=\Phi^k(X_k,U_k)$.\\
\item Montrer que pour tout $k$, $c(X_k)$ est une sommme de variables al\'eatoires de Bernoulli, ind\'ependantes entre elles, dont on pr\'ecisera les param\`etres.
\item Montrer que, pour tout $k$, $X_k$ et $S_k$ ont la m\^eme loi. En d\'eduire la loi de $c(S_n)$.
\item Calculer l'esp\'erance et la variance de $S_n$.\\
Pour tout entier $n$ strictement positif, on note $H_n$ le $n-$i\`eme nombre harmonique d\'efini par : $H_n=\sum\limits_{k=1}^{n}\frac{1}{k}$.
\item Montrer que, pour tout r\'eel $\epsilon$ strictement positif : $\mathbb{P}\left(\left|\frac{c(S_n)}{H_n}-1\right|\geq 1\right)\underset{n\rightarrow \infty}{\rightarrow}0$.\\
On dit que $\frac{c(S_n)}{H_n}$ converge vers $1$ en probabilit\'e.
\item En d\'eduire que $\frac{c(S_n)}{\log n}$ converge vers $1$ en probabilit\'e. Conclusion ?
\end{enumerate}
\end{exer}

%\newpage

\begin{exer}[Marche al\'eatoire cyclique]
Soit $N$ un entier naturel sup\'erieur \`a $2$. On d\'efinit une suite $(z_n)$ de variables al\'eatoires ind\'ependantes, de même loi donn\'ee par :
\[\forall n \in \mathbb{N}^{\ast}, \mathbb{P}\left(z_n =\exp\frac{2i\pi}{N}\right) = \frac{1}{2} \wedge \mathbb{P}\left(z_n =\exp -\frac{2i\pi}{N}\right) = \frac{1}{2}\]
On d\'efinit ensuite la marche $(s_n)$, sur $\mathbb{U}_N$, par :
\[s_0=1 \wedge \forall n \in \mathbb{N} , s_{n+1}=s_n z_n\]
\ligneinter
\begin{enumerate}
\item Montrer que $(s_n)$ parcourt tout $\mathbb{U}_N$ avec probabilit\'e $1$.
On d\'efinit la variable al\'eatoire $V$ par :\\
$V$ est la derni\`ere valeur dans $\mathbb{U}_N$ atteinte par la suite $(s_n)$ si la marche parcourt tout $\mathbb{U}_N$, %
et vaut $1$ si cet \'ev\`enement ne se r\'ealise pas.
\item Montrer que $V$ suit la loi uniforme sur $\mathbb{U}_N\setminus \{1\}$.
\end{enumerate}
\end{exer}

\begin{exer}[Norme euclidienne d'un sommet de l'hypercube choisi au hasard]
Pour un entier $n$ strictement positif, on note $H_n$ l'ensemble des sommets de l'hypercube de dimension $n$, $\{0,1\}^n$. %
Soit de plus $X_n$ une variable al\'eatoire uniforme sur $H_n$. Montrer que, si $n$ tend vers $+\infty$, %
alors la norme euclidienne de $X_n$ divisée par $\sqrt{n}$ converge en probabilit\'e vers $\frac{\sqrt{2}}{2}$ :
\[\forall\varepsilon\in\mathbb{R}_+^{\ast},\mathbb{P}\left(\left|\frac{\|X_n\|}{\sqrt{n}}-\frac{\sqrt{2}}{2}\right|\geq\varepsilon\right)\underset{n\rightarrow +\infty}{\longrightarrow}0\]
\end{exer}
\section{Processus de branchement}

\begin{exer}%[Processus de branchement]
\begin{center}
\textit{
Un processus de branchement est une famille de variables al\'eatoires qui permet d'\'etudier l'\'evolution de la taille d'une population.% 
Historiquement, c'est F Galton, pour estimer le probabilit\'e d'extinction des noms nobles, qui a \'etudi\'e le premier de ces processus : le processus de Galton-Watson.\\
}
\end{center}
\ligneinter
Soit $X_0$ la taille initiale de la population \'etudi\'ee.\\
Chaque individu donne naissance, ind\'ependamment des autres, \`a $m$ nouveaux individus, suivant une loi $(p_m)$. %
Les descendants directs de la population initiale forment la premi\`ere g\'en\'eration, dont on note la taille, al\'eatoire, $X_1$. Ces descendants ont des enfants, chacun suivant la m\^eme loi $(p_k)$, et engendrent la deuxi\`eme g\'en\'eration, de taille $X_2$. %
De proche en proche, on d\'efinit ainsi la suite $(X_n)$, dont le terme g\'en\'eral nous donne le nombre d'individus de la $n-$i\`eme g\'en\'eration.\\
Formellement, supposons que : $X_0=1$.\\
On d\'efinit la famille de $((\xi_k^{n})_{k\in\mathbb{N}^{\ast}})_{n\in\mathbb{N}^{\ast}}$ des variables al\'eatoires $L^2$, ind\'ependantes entre elles et toutes de loi $(p_m)_m$, qui r\'egissent la reproduction des individus des diff\'erentes g\'en\'erations, de sorte que:
\[\forall n \in \mathbb{N} , X_{n+1}=\sum\limits_{k=1}^{X_n} \xi_k^{n+1}\]
La fonction g\'en\'eratrice $\varphi$ de la loi commune des termes de $(\xi_k^n)$ est d\'efinie, pour tout $s$ de $[-1,1]$, par :\[\varphi(s)=\sum\limits_{m=0}^{\infty}p_m s^m\]
On note, pour tout entier $n$ strictement positif, $\varphi_n$ la fonction g\'en\'eratrice de $X_n$.
\ligneinter
\begin{enumerate}
\item Quelle est la loi de $X_1$ ?
On notera, respectivement, $m$ et $\sigma$ sa variance et son \'ecart-type.
\item Justifier la formule : $\forall s \in [-1,1], G_Y(s)=\mathbb{E}(s^Y)$, valable pour toute variable al\'eatoire discr\`ete $Y$ de fonction g\'en\'eratrice $G_Y$ %
(on ne s'int\'eresse pas \`a l'espace de probabilit\'e sous-jacent).
\item D\'emontrer soigneusement que, pour tout entier positif $n$ :
\[\mathbb{E}\left(s^{X_{n+1}}\right)=\sum\limits_{x\in X_n(\Omega)}\mathbb{E}\left(s^{\sum\limits_{k=1}^{x}\xi_k^{n+1}}\right)\mathbb{P}(X_n=x)\]
\item En d\'eduire que : $\forall n\in\mathbb{N},\varphi_n=\varphi^{\circ n}$ o\`u $\varphi^{\circ n}$ est le $n-$i\`eme it\'er\'e de $\varphi$.
\item D\'emontrer que, pour tout entier $n$ strictement positif $n$, $X_n$ est $L^2$ et :
\begin{center}$\mathbb{E}(X_n)=m^n$ ; $V(X_n)=\sigma^2 m^{n-1}\frac{m^n-1}{m-1}$ si $m\neq 1$ ; $V(X_n)=n\sigma^2$ si $m=1$.\end{center}
\hspace*{-2em}\textit{On s'int\'eresse \`a la probabilit\'e $\pi$ d'extinction de la population, d\'efinie par : %
$\pi=\mathbb{P}(\exists n\in \mathbb{N} | x_n=0)$.}
\item Pourquoi peut-on \'ecrire $\pi =\underset{n\rightarrow +\infty}{lim}\mathbb{P}(X_n=0)$ ?

\medskip
On note, pour tout entier naturel $n$ : $q_n=\mathbb{P}(X_n=0)$.
\item Que dire de $\pi$ si $p_0=0$, $p_0=1$ ?
On suppose maintenant : $p_0\in ]0,1[$.
\item Montrer que : $\pi=\varphi (\pi)$. On \'etablira pour cela une relation de r\'ecurrence sur les termes de $(q_n)$.
\item Discuter, suivant $m$, de la valeur de $\pi$ \`a l'aide des propri\'et\'es analytiques de $\varphi$.
\end{enumerate}
\end{exer}

%Ajouter, après une étude du cas m>0 et de l'éventuelle croissance exponentielle de la population, un exercice concernant les multiplicateurs d'électrons.

\begin{exer}
A l'instant $0$, une culture biologique d\'emarre avec une cellule rouge. %
Au bout d'une minute, cete cellule meurt et est remplacée par :
\begin{itemize}
\item deux cellules rouges avec probabilit\'e $\frac{1}{4}$,
\item Une cellule rouge et une cellule blanche avec probabilit\'e $\frac{2}{3}$,
\item deux cellules blanches probabilit\'e $\frac{1}{12}$.
\end{itemize}
Chaque cellule rouge vit une minute et se reproduit \`a son tour suivant la m\^eme r\`egle, %
chaque cellule blanche meurt dans le m\^eme temps sans se reproduire.
\begin{enumerate}
\item A l'instant $n+\frac{1}{2}$, quelle est la probabilit\'e qu'aucune cellule blanche %
n'ait encore fait son apparition ?
\item Quelle est la probabilit\'e que la population tout enti\`ere disparaise ?
\end{enumerate}
\end{exer}

\begin{exer}[Division cellulaire]
On garde la notation de l'exercice qui précède.\\
Soit $(Z_n)$ un proc\'ed\'e de division dans une population de cellules, v\'erifiant : %
$p_0>0$ ; $p_2>0$ ; $p_1\in[0,1[$ et $p_n=0$ pour tout $n$ sup\'erieur ou \'egal \`a $3$.
\begin{itemize}
\item Etudier la probabilit\'e d'extinction de $(Z_n)$.
\item Quelle est la signification, biologiquement, du cas : $p_1=0$ ?
\end{itemize}
\end{exer}


% \section{Formalisme des probabilit\'es}

\begin{exer}[Tribus sur $\mathbb{N}$]
Soit $\mathcal{B}$ une tribu sur $\mathbb{N}$. Pour tout entier naturel $a$, on note $B_a$ l'ensemble des \'el\'ements $b$ de $\mathbb{N}$ tels que :
\[\forall B \in \mathcal{B} , \{a,b\}\subseteq B \vee \{a,b\}\subseteq B^c\]
Montrer, \`a l'aide de la famille $(B_a)_a$, qu'il existe une famille $(A_i)$ de parties de $\mathbb{N}$ telles que tout \'el\'ement de $\mathcal{B}$ est une union disjointe de $A_i$.
\end{exer}

\begin{exer}
Soit $X$ une variable al\'eatoire \`a valeurs dans $\mathbb{N}$.\\
Montrer que : \[\mathbb{E}(X)=\sum\limits_{n=0}^{\infty} \mathbb{P}(X>n)\]
\end{exer}

\begin{exer}
Montrer qu'une intersection d\'enombrable d'\'ev\`enements presque s\^urs est encore un \'ev\`enement presque s\^ur.\\
En est-il de même pour une intersection non d\'enombrable ?
\end{exer}
% \section{calculs \'el\'ementaires}

\begin{exer}[Une distraction du savant cosinus]
%\begin{center}
%\begin{minipage}{0.8\linewidth}
\textit{N voyageurs (au moins deux) s'apprêtent à monter dans un avion contenant N places num\'erot\'ees. %
Le premier d'entre eux s'av\`ere être le savant cosinus qui, distrait comme il l'est toujours, ne regarde pas le num\'ero de sa r\'eservation. %
Les passagers suivants, quand ils montent, en ordre, dans l'avion, s'asseyent alors \`a leur place r\'eserv\'ee si elle est encore libre, %
et sinon choisissent une place au hasard parmi celles qui restent.}
%\end{minipage}
%\end{center}
\ligneinter
\begin{enumerate}
\item Calculer, par r\'ecurrence sur $N$, la probabilit\'e $p_N$ que le dernier passager soit assis \`a sa place.
\item Retrouver le r\'esultat pr\'ec\'edent par un argument direct.
\end{enumerate}
\end{exer}

\begin{exer}
On jette plusieurs fois de suite et ind\'ependamment une pi\`ece de monnaie non \'equilibr\'ee, la probabilit\'e de tomber sur "Pile" est un r\'eel $p$ compris entre $0$ et $1$ strictement.\\
Calculer :
\begin{enumerate}
\item La probabilit\'e de ne pas avoir de "Face" au cours des $n$ premiers jets pour tout $n$ sup\'erieur \`a $1$ ;
\item La prpobabilit\'e d'obtenir "Face" pour la premi\`ere fios au $n-$i\`eme jet ;
\item l'esp\'erance du nombre de jets jusqu'\`a la premi\`ere apparition de "Face".
\end{enumerate}
\end{exer}

\begin{exer}
Pour tout entier $n$ sup\'erieur ou \'egal à $1$, on consid\`ere $n$ boules num\'erot\'ees de $1$ \`a $n$, que l'on place dans $n$ urnes, num\'erot\'ees de la m\^eme mani\`ere.
\begin{enumerate}
\item Calculer, pour tout $n$, la probabilit\'e $p_n$ de l'\'ev\`enement : chaque urne contient exactement une boule \`a la fin de l'op\'eration.
\item Montrer que la suite $(p_n)$ est d\'ecroissante et tend vers $0$ en $+\infty$.
\end{enumerate}
\end{exer}
% \section{Lois de probabilit\'es}

\begin{exer}[Recensement d'une population d'\'ecureuils]
%\begin{center}
%\begin{minipage}{\0.8\linewidth}
\textit{On veut estimer le nombre $N$ d'\'ecureuils dans une for\^et. %
Pour cela on en capture $k$, on leur met une petite marque sur la patte et on les rel\^ache. %
Une semaine apr\`es (on suppose qu'aucun \'ecureuil est mort ou n\'e dans cet intervalle), %
on en capture $l$ et on compte ceux d'entre eux qui portent la marque.}
%\end{minipage}
%\end{center}
\ligneinter
\begin{enumerate}
\item Calculer la probabilit\'e d'observer une valeur $m$ \'ecureuils marqu\'es  en fonction de $N$, $k$ et $l$.
\item Calculer, quand $N$ est grand, la valeur $N_{max}$ pour laquelle la probabilit\'e ci-dessus est la plus grande. On appelle cette valeur \textit{estimateur du maximum de vraisemblance} pour le nombre d'\'ecureuils.
\end{enumerate}
\end{exer}

\begin{exer}
Soit $(X_n)$ une suite de variables al\'eatoires de lois respectives $\mathcal{B}(n,p_n)$ o\`u $np_n$ est constant, de valeur unique $\lambda$.\\
On note, pour tout $n$ : $A_n = (X_n \geq 1)$.\\
Soit, de plus, $Y$ une variable de Poisson de param\`etre $\lambda$.\\
Montrer, pour $j$ fix\'e et sup\'erieur \`a $1$, que :\[\mathbb{P}(X_n=j|A_n)\underset{n\rightarrow +\infty}{\longrightarrow}\mathbb{P}(Y=j|y\geq 1)\]
\end{exer}

\begin{exer}
Soit $X$ une variable al\'eatoire de loi $\mathcal{B}(n,p)$.\\
Quel entier $j$ compris entre $0$ et $n$, maximise $\mathbb{P}(X=j)$ ?
\end{exer}

\begin{exer}[Optimisations avec la loi de Poisson]
\begin{enumerate}
\item Soit $\lambda$ un r\'eel strictement positif. Quel entier $j$ maximise $\mathbb{P}(X=j)$, pour une variable de Poisson de param\`etre $\lambda$ ?
\item Soit $j$ un entier positif. Quel est le r\'eel $\lambda$ pour lequel $\mathbb{P}(X=j)$ et maximal, o\`u $X$ est choisie avec une loi $\mathcal{P}(\lambda)$ ?
\end{enumerate}
\end{exer}

\begin{exer}
Soit $X$ une variable al\'eatoire de loi $\mathcal{G}(p)$.\\
Montrer que :\[\mathbb{E}\left(\frac{1}{1+X}\right)=\ln((1-p)^{\frac{p}{p-1}}))\]
\end{exer}

%\newpage

\begin{exer}
Soit $p$ un réel compris, strictement entre $0$ et $1$, et soient $X$ et $Y$ deux variables al\'eatoires ind\'ependantes, de m\^eme loi $\mathcal{B}(p)$.\\
Donner les lois de $max(X,Y)$ et $min(X,Y)$.
\end{exer}

\begin{exer}
Soient $X$ et $Y$ deux variables al\'eatoires d\'efinies sur un m\^eme espace $\Omega$, ind\'ependantes, et telles que :
\begin{itemize}
\item $X$ suit une loi de Poisson de param\`etre $\lambda$ ;
\item $Y$ vaut $1$ ou $2$, avec la m\^eme probabilit\'e $\frac{1}{2}$.
\end{itemize}
\begin{enumerate}
\item Donner la loi de la variable $Z$ d\'efinie par : $Z=XY$.
\item Calculer la probabilit\'e que $Z$ prenne une valeur paire.
\end{enumerate}
\end{exer}

\begin{exer}
Soit $X$ une variable al\'eatoire de Poisson, de param\`etre $\lambda$. On note $Y$ la variable $(-1)^X$.
\begin{enumerate}
\item Donner l'ensemble des valeurs prises par $Y$, puis calculer son esp\'erance.
\item D\'eterminer la loi de $Y$.
\end{enumerate}
\end{exer}
% \section{$L^2$}

\begin{exer}
Soit $X$ une variable al\'eatoire $L^2$. Quel r\'eel minimise $c\mapsto\mathbb{E}((X-c)^2)$ ? Quel est le minimum atteint ?
\end{exer}

\begin{exer}
Soit $X$ une variable al\'eatoire $L^2$.\\
On suppose que : $V(X)=0$.\\
Montrer que $X$ prend la valeur $\mathbb{E}(X)$ avec une probabilit\'e $1$.
\end{exer}

\begin{exer}[L'in\'egalit\'e de Tchebycheff est-elle optimale ?]
\begin{enumerate}
\item (Oui) Montrer que, si $a$ est un r\'eel strictement positif, alors il existe une variable al\'eatoire $X$, $L^2$, telle que :\[\mathbb{P}(|X-\mathbb{E}(X)|\geq a)=\frac{V(X)}{a^2}\]
\item (Non) Soit $X$ une variable al\'eatoire $L^2$. Montrer que : \[a^2\mathbb{P}(|X-\mathbb{E}(X)|\geq a)\underset{a\rightarrow\infty}{\rightarrow}0\]
\end{enumerate}
\end{exer}
% \section{Classiques, ou presque \dots mais beaux !}

\begin{exer}[Ruine du joueur]
\textit{
Un joueur dispose d'une mise initiale de $k$ euros. %
A chaque partie qu'il joue, il a la probabilit\'e $p$ de perdre un euro, dans le cas contraire sa mise augmente d'un euro. %
Ce joueur se fixe comme objectif de jouer jusqu'\`a ce qu'il soit ruin\'e, ou jusqu'\`a ce que son capital atteigne la somme de $M$ euros, o\`u $M$ est un entier fix\'e \`a l'avance.
}
\ligneinter
\begin{enumerate}
\item On veut calculer la probabilit\'e de l'\'ev\`enement : "le joueur finit ruin\'e".
\begin{enumerate}
\item Mod\'eliser, pour une mise $k$ quelconque au d\'ebut du jeu, le rpobl\`eme comme une marche al\'eatoire.

\medskip
Pour tout entier $k$ compris entre $0$ et $M$, on note $E_k$ l'\'ev\'enement correspondant \`a : %
"le joueur, ayant mis\'e $k$ euros, finit ruin\'e avant d'avoir enpoch\'e un capital de $M$". On note $q_k$ la probabinit\'e correspondant \`a un tel \'ev\`enement.
\item Calculer $q_0$ et $q_M$.
\item Montrer la relation de r\'ecurrence : $q_j=(1-p)q_{j+1}+pq_{j-1}$, pour tout indice $j$ o\`u cette formule est d\'efinie.
\item Montrer, pour tout $j$, que : $q_{j+1}-q_j=\left(\frac{p}{1-p}\right)^j(q_1-q_0)$
\item En d\'eduire que : $\forall k \in [\![0,M]\!] , q_k-q_0=(q_1-q_0)\sum\limits_{j=0}^{k-1}\left(\frac{p}{1-p}\right)^j$.
\item En appliquant la formule pr\'ec\'edente au rang $M$, donner une expression de $q_1$ en fonction de $p$. %
On distinguera bien le cas o\`u $p=\frac{1}{2}$.
\item Donner une expression de $q_k$ en fonction de $p$.
\end{enumerate}
\item Soit maintenant pour tout $k$ compris entre $0$ et $M$, $F_k$ l'\'ev\`enement : "le joueur empoche la mise de $M$ euros sans avoir jamais \'et\'e ruin\'e".
\begin{enumerate}
\item Peut-on dire que les \'ev\`enements $E_k$ et $F_k$ sont contraires ?
\item En utilisant la m\^eme m\'ethode que pour $E_k$, exprimer la probabilit\'e de $F_k$ en fonction de $p$. On distiguera bien le cas o\`u $p=\frac{1}{2}$.
\item Conclusion ?
\end{enumerate}
\end{enumerate}
\end{exer}

\begin{exer}[Nombre de cycles dans une grande permutation al\'eatoire]
Dans cet exercice, on cherche \`a estimer le nombre de cycles d'une permutation choisie au hasard dans le groupe sym\'etrique $\mathcal{S}_n$ de $[\![1,n]\!]$, quand $n$ est un entier naturel qui tend vers $+\infty$.\\
On \'etudie donc une suite $(S_n)$ de variables al\'eatoires, de lois uniformes, respectivement sur les termes de $(\mathcal{S}_n)$.\\
Pour toute permutation $\sigma$ d'un ensemble $[\![1,n]\!]$, on note $c(\sigma)$ le nombre de cycles de $\sigma$.\\
Par exemple : $c(S_n)=1$ lorsque la valeur de $S_n$ est un cycle, $c(S_n)=n$ lorsque $S_n$ prend comme valeur l'identit\'e.
\ligneinter
\vspace{-0.2pt}
\begin{enumerate}
\item Pour tout entier $n$ strictement positif, on note $\Phi^n$ l'application :
\[\mathcal{S}_n\times [\![1,n+1]\!] \longrightarrow \mathcal{S}_n : (\sigma ,i) \mapsto \tilde{\sigma}\circ (i , n+1)\]
o\`u $\tilde{\sigma}$ est la permutation de $[\![1,n+1]\!]$ qui agit comme $\sigma$ sur $[\![1,n]\!]$, et fixe $n+1$.\\
Montrer que cette application est une bijection.
\item Avec les conventions qui pr\'ec\`edent, discuter, selon $\sigma$ et $i$, de la valeur de $c(\Phi^n(\sigma))$ en fonction de $c(\sigma)$.

\medskip
On se fixe maintenant un entier naturel non nul $n$ et on cherche \`a conna\^itre la loi de $S_n$.\\
Soit $(U_k)_{k\in [\![1,n-1]\!]}$ une famille de variables al\'eatoires, ind\'ependantes, respectivement sur les termes de $([\![1,k+1]\!])_k$. Soit $(X_k)$ la suite de variables al\'eatoires, d\'efinie par r\'ecurrence finie sur $k$ par :\\
$X_1$ prend seulement la valeur $Id_{\{1\}}$ ; $\forall k \in [\![1,n-1]\!] , X_{k+1}=\Phi^k(X_k,U_k)$.\\
\item Montrer que pour tout $k$, $c(X_k)$ est une sommme de variables al\'eatoires de Bernoulli, ind\'ependantes entre elles, dont on pr\'ecisera les param\`etres.
\item Montrer que, pour tout $k$, $X_k$ et $S_k$ ont la m\^eme loi. En d\'eduire la loi de $c(S_n)$.
\item Calculer l'esp\'erance et la variance de $S_n$.\\
Pour tout entier $n$ strictement positif, on note $H_n$ le $n-$i\`eme nombre harmonique d\'efini par : $H_n=\sum\limits_{k=1}^{n}\frac{1}{k}$.
\item Montrer que, pour tout r\'eel $\epsilon$ strictement positif : $\mathbb{P}\left(\left|\frac{c(S_n)}{H_n}-1\right|\geq 1\right)\underset{n\rightarrow \infty}{\rightarrow}0$.\\
On dit que $\frac{c(S_n)}{H_n}$ converge vers $1$ en probabilit\'e.
\item En d\'eduire que $\frac{c(S_n)}{\log n}$ converge vers $1$ en probabilit\'e. Conclusion ?
\end{enumerate}
\end{exer}

%\newpage

\begin{exer}[Marche al\'eatoire cyclique]
Soit $N$ un entier naturel sup\'erieur \`a $2$. On d\'efinit une suite $(z_n)$ de variables al\'eatoires ind\'ependantes, de même loi donn\'ee par :
\[\forall n \in \mathbb{N}^{\ast}, \mathbb{P}\left(z_n =\exp\frac{2i\pi}{N}\right) = \frac{1}{2} \wedge \mathbb{P}\left(z_n =\exp -\frac{2i\pi}{N}\right) = \frac{1}{2}\]
On d\'efinit ensuite la marche $(s_n)$, sur $\mathbb{U}_N$, par :
\[s_0=1 \wedge \forall n \in \mathbb{N} , s_{n+1}=s_n z_n\]
\ligneinter
\begin{enumerate}
\item Montrer que $(s_n)$ parcourt tout $\mathbb{U}_N$ avec probabilit\'e $1$.
On d\'efinit la variable al\'eatoire $V$ par :\\
$V$ est la derni\`ere valeur dans $\mathbb{U}_N$ atteinte par la suite $(s_n)$ si la marche parcourt tout $\mathbb{U}_N$, %
et vaut $1$ si cet \'ev\`enement ne se r\'ealise pas.
\item Montrer que $V$ suit la loi uniforme sur $\mathbb{U}_N\setminus \{1\}$.
\end{enumerate}
\end{exer}

\begin{exer}[Norme euclidienne d'un sommet de l'hypercube choisi au hasard]
Pour un entier $n$ strictement positif, on note $H_n$ l'ensemble des sommets de l'hypercube de dimension $n$, $\{0,1\}^n$. %
Soit de plus $X_n$ une variable al\'eatoire uniforme sur $H_n$. Montrer que, si $n$ tend vers $+\infty$, %
alors la norme euclidienne de $X_n$ divisée par $\sqrt{n}$ converge en probabilit\'e vers $\frac{\sqrt{2}}{2}$ :
\[\forall\varepsilon\in\mathbb{R}_+^{\ast},\mathbb{P}\left(\left|\frac{\|X_n\|}{\sqrt{n}}-\frac{\sqrt{2}}{2}\right|\geq\varepsilon\right)\underset{n\rightarrow +\infty}{\longrightarrow}0\]
\end{exer}
% \section{Processus de branchement}

\begin{exer}%[Processus de branchement]
\begin{center}
\textit{
Un processus de branchement est une famille de variables al\'eatoires qui permet d'\'etudier l'\'evolution de la taille d'une population.% 
Historiquement, c'est F Galton, pour estimer le probabilit\'e d'extinction des noms nobles, qui a \'etudi\'e le premier de ces processus : le processus de Galton-Watson.\\
}
\end{center}
\ligneinter
Soit $X_0$ la taille initiale de la population \'etudi\'ee.\\
Chaque individu donne naissance, ind\'ependamment des autres, \`a $m$ nouveaux individus, suivant une loi $(p_m)$. %
Les descendants directs de la population initiale forment la premi\`ere g\'en\'eration, dont on note la taille, al\'eatoire, $X_1$. Ces descendants ont des enfants, chacun suivant la m\^eme loi $(p_k)$, et engendrent la deuxi\`eme g\'en\'eration, de taille $X_2$. %
De proche en proche, on d\'efinit ainsi la suite $(X_n)$, dont le terme g\'en\'eral nous donne le nombre d'individus de la $n-$i\`eme g\'en\'eration.\\
Formellement, supposons que : $X_0=1$.\\
On d\'efinit la famille de $((\xi_k^{n})_{k\in\mathbb{N}^{\ast}})_{n\in\mathbb{N}^{\ast}}$ des variables al\'eatoires $L^2$, ind\'ependantes entre elles et toutes de loi $(p_m)_m$, qui r\'egissent la reproduction des individus des diff\'erentes g\'en\'erations, de sorte que:
\[\forall n \in \mathbb{N} , X_{n+1}=\sum\limits_{k=1}^{X_n} \xi_k^{n+1}\]
La fonction g\'en\'eratrice $\varphi$ de la loi commune des termes de $(\xi_k^n)$ est d\'efinie, pour tout $s$ de $[-1,1]$, par :\[\varphi(s)=\sum\limits_{m=0}^{\infty}p_m s^m\]
On note, pour tout entier $n$ strictement positif, $\varphi_n$ la fonction g\'en\'eratrice de $X_n$.
\ligneinter
\begin{enumerate}
\item Quelle est la loi de $X_1$ ?
On notera, respectivement, $m$ et $\sigma$ sa variance et son \'ecart-type.
\item Justifier la formule : $\forall s \in [-1,1], G_Y(s)=\mathbb{E}(s^Y)$, valable pour toute variable al\'eatoire discr\`ete $Y$ de fonction g\'en\'eratrice $G_Y$ %
(on ne s'int\'eresse pas \`a l'espace de probabilit\'e sous-jacent).
\item D\'emontrer soigneusement que, pour tout entier positif $n$ :
\[\mathbb{E}\left(s^{X_{n+1}}\right)=\sum\limits_{x\in X_n(\Omega)}\mathbb{E}\left(s^{\sum\limits_{k=1}^{x}\xi_k^{n+1}}\right)\mathbb{P}(X_n=x)\]
\item En d\'eduire que : $\forall n\in\mathbb{N},\varphi_n=\varphi^{\circ n}$ o\`u $\varphi^{\circ n}$ est le $n-$i\`eme it\'er\'e de $\varphi$.
\item D\'emontrer que, pour tout entier $n$ strictement positif $n$, $X_n$ est $L^2$ et :
\begin{center}$\mathbb{E}(X_n)=m^n$ ; $V(X_n)=\sigma^2 m^{n-1}\frac{m^n-1}{m-1}$ si $m\neq 1$ ; $V(X_n)=n\sigma^2$ si $m=1$.\end{center}
\hspace*{-2em}\textit{On s'int\'eresse \`a la probabilit\'e $\pi$ d'extinction de la population, d\'efinie par : %
$\pi=\mathbb{P}(\exists n\in \mathbb{N} | x_n=0)$.}
\item Pourquoi peut-on \'ecrire $\pi =\underset{n\rightarrow +\infty}{lim}\mathbb{P}(X_n=0)$ ?

\medskip
On note, pour tout entier naturel $n$ : $q_n=\mathbb{P}(X_n=0)$.
\item Que dire de $\pi$ si $p_0=0$, $p_0=1$ ?
On suppose maintenant : $p_0\in ]0,1[$.
\item Montrer que : $\pi=\varphi (\pi)$. On \'etablira pour cela une relation de r\'ecurrence sur les termes de $(q_n)$.
\item Discuter, suivant $m$, de la valeur de $\pi$ \`a l'aide des propri\'et\'es analytiques de $\varphi$.
\end{enumerate}
\end{exer}

%Ajouter, après une étude du cas m>0 et de l'éventuelle croissance exponentielle de la population, un exercice concernant les multiplicateurs d'électrons.

\begin{exer}
A l'instant $0$, une culture biologique d\'emarre avec une cellule rouge. %
Au bout d'une minute, cete cellule meurt et est remplacée par :
\begin{itemize}
\item deux cellules rouges avec probabilit\'e $\frac{1}{4}$,
\item Une cellule rouge et une cellule blanche avec probabilit\'e $\frac{2}{3}$,
\item deux cellules blanches probabilit\'e $\frac{1}{12}$.
\end{itemize}
Chaque cellule rouge vit une minute et se reproduit \`a son tour suivant la m\^eme r\`egle, %
chaque cellule blanche meurt dans le m\^eme temps sans se reproduire.
\begin{enumerate}
\item A l'instant $n+\frac{1}{2}$, quelle est la probabilit\'e qu'aucune cellule blanche %
n'ait encore fait son apparition ?
\item Quelle est la probabilit\'e que la population tout enti\`ere disparaise ?
\end{enumerate}
\end{exer}

\begin{exer}[Division cellulaire]
On garde la notation de l'exercice qui précède.\\
Soit $(Z_n)$ un proc\'ed\'e de division dans une population de cellules, v\'erifiant : %
$p_0>0$ ; $p_2>0$ ; $p_1\in[0,1[$ et $p_n=0$ pour tout $n$ sup\'erieur ou \'egal \`a $3$.
\begin{itemize}
\item Etudier la probabilit\'e d'extinction de $(Z_n)$.
\item Quelle est la signification, biologiquement, du cas : $p_1=0$ ?
\end{itemize}
\end{exer}


%%% ------------------------------------------ %%%
\appendix
%%% ------------------------------------------ %%%

% \chapter{Exercices \g{a} retravailler}
% \section{Enonc\'es \g{a} retravailler}

\begin{exer}
Soient $n$ un entier naturel non nul, $A$ une matrice telle que :
\[\forall M \in M_n (\mathbb{C}) , \det (A+M) = (\det A) + (\det M)\]
Montrer que $A$ est nulle.
\end{exer}

\begin{exer}
\begin{enumerate}
\item Enoncer le théorème de convergence d'un produit de Cauchy qui a été démontré en cours.
\item Démontrer le théorème de Mertens : si $\sum u_n$ et $\sum v_m$ sont deux séries d'une algèbre normée, %
respectivement absolument convergente et convergente, alors le produit de Cauchy de ces deux séries est convergent, de somme :
\[(\sum\limits_{n=0}^{+ \infty} u_n)(\sum\limits_{m=0}^{+ \infty} v_m)\]
\end{enumerate}
\end{exer}

\begin{exer}
Théorème de Stone-Weierstrass général.
\end{exer}


\begin{exer}
Soit $E$ un espace préhilbertien réel, $F$ un espace vectoriel normé, et $f$ une application continue de $E$ dans $F$ vérifiant :
\[\forall (x,y) \in E^2 , x \perp y \Rightarrow f(x+y) = f(x) + f(y)\]
On se propose de déterminer la nature de $f$.
\begin{enumerate}
\item Examiner le cas où $E$ est de dimension inférieure ou égale à $1$.
\item Que peut-on dire de $x+y$ et $x-y$ lorsque $x$ et $y$ sont deux vecteurs de $E$ de même norme ?
\item Examiner le cas où $f$ est paire.
\item Examiner le cas où $f$ est impaire.
\item Examiner le cas général.
\end{enumerate}
\end{exer}


\begin{exer}
Soient $E$ un espace Euclidien de dimension $3$, $x$, $y$, et $z$ des vecteurs de $E$.\\
Montrer :
\[[y \wedge z, z \wedge x, x \wedge y] = ([x,y,z])^2\]
et \[(x \wedge y)\wedge (x \wedge z) = [x,y,z]x\]
\end{exer}

\begin{exer}
Soit $(z_n)$ une suite complexe vérifiant :
\[\forall (n,m) \in \mathbb{N}^2 , \lvert z_n - z_m \rvert \geq 1\]
\begin{enumerate}
\item Soit $\alpha$ un réel strictement positif. Montrer que $\sum \frac{1}{\lvert z_n \rvert^{2 + \alpha}}$ converge.
\item $\sum \frac{1}{\lvert z_n \rvert^2}$ converge-t-elle ?
\end{enumerate}
\end{exer}

\begin{exer}
Soit $(K,d)$ un espace métrique compact, par exemple un fermé borné d'un espace vectoriel normé de dimension finie.\\
On considère une isométrie $f$ de $K$ dans lui-même, c'est-à-dire une application de $K$ dans $K$ qui conserve la distance.
\begin{enumerate}
\item $f$ admet-elle nécessairement un point fixe ?
\item Montrer que $f$ est surjective.
%Indications : supposer que ce ne soit pas le cas, et considérer un élément $x_0$ de $K$ qui n'est pas dans l'image de $f$. Définir la suite $(x_n)$ des images de $x_0$ par les itérées de $f$ et montrer que $f$ induit une permutation de l'ensemble $L$ des valeurs d'adhérence de $(x_n)$. Utiliser la compacité de $K$ pour montrer que $L$ est non vide. En déduire une contradiction en considérant la suite des distances entre les termes de $(x_n)$ et $L$.\\
%Indications : Supposer que ce ne soit pas le cas, et considérer un point de $K$ dont la distance à l'image de $f$ soit maximale -pourquoi ce point existe-t-il ?-, étudier la suite des itérées de ce point par $f$ pour déduire une contradiction.
\end{enumerate}
\end{exer}

\begin{exer}
Considérons la solution $f$ l'équation différentielle $y'=1+Idy^{2}$, maximale, nulle en $0$, %
dont on notera $I_{f}$ l'intervalle de définition.\\
On suppose que $f$ admet un développement en série entière au voisinage de $0$.
\begin{enumerate}
\item Montrer que ce développement est de la forme $\sum_{n=0}^{\propto} a_{n}X^{3n+1}$ et que $(a_{n})$ vérifie:
$a_{0}=1$ ; \[\forall n \in \mathbb{N}, (3n+4)a_{n+1}=\sum_{q=0}^{n} a_{n-q}a_{q}\]
\item Montrer que la suite $(a_{n})$ définie par les relations ci-dessus est décroissante.
\item Montrer que $f$ est développable en série entière au voisinage de $0$, et minorer son rayon de convergence $R$.
\item Montrer que $R < \pi/2 - Arctan(5/4) + 1$
\item Montrer que $I_{f}$ n'est pas borné inférieurement.
\end{enumerate}
\end{exer}

Théorème de réalisation de Borel.

Série entière semi-convergente en tout point de son cercle de convergence ?

Condition de Frédéric pour la développabilité en sérien entière :\\
Existence d'un réel strictement positif $M$ tel que :
\[forall k \in \mathbb{N} , \frac{f^{(k)}(0)}{k!} \leq M^k\]
??

\begin{exer}
Sous-groupes compacts de $GL_n(\mathbb{R})$, $n \in \mathbb{N}^{\ast}$.\\
Cet exercice utilise le résultat suivant :
\textit{
Soit un espace vectoriel normé $E$, une partie compacte et convexe $K$ de $E$ et un groupe $G$ affine, %
équicontinu -les normes des parties linéaires sont majorées par une constante $M$ fixée-, sur $E$ qui laisse stable $K$. %
Il suffit par exemple que $G$ soit compact. %
Alors les éléments de $G$ admettent un point fixe commun dans $K$.
}
Soit $G$ un sous-groupe compact de $GL_n(\mathbb{R})$. %
On se propose de montrer que $G$ est conjugué à un sous-groupe de $O_n(\mathbb{R})$, %
soit que les éléments de $G$ préservent un produit scalaire.
%On étudie à cette fin l'action de $G$ sur l'ensemble des produits scalaires de $\mathbb{R}^n$, il s'agit de prouver que cette action admet un point fixe.\\
\begin{itemize}
\item Montrer que l'ensemble $S_n^{++}(\mathbb{R})$ est convexe.
\item Que dire de l'ensemble des termes de la famille $(^tMM)_{M \in G}$ ?
\item Etudier l'adhérence de son enveloppe convexe.
\item Conclure à l'aide de la propriété des groupes affines compacts énoncée au début de cet exercice.
\end{itemize}
\end{exer}

\begin{exer}
On considère une famille $(\Gamma_{\lambda})_{\lambda \in \Lambda}$ de courbes de $\mathbb{R}^2$, %
indexée sur une partie $\Lambda$ de $\mathbb{R}$, %
définie par une famille d'équations $(G(x,y,\lambda)=0)_{\lambda \in \Lambda}$, %
où $G$ est une fonction de classe $C^{1}$. %
Dans les cas suivants, trouver une équation différentielle en $x \mapsto y$ -dont aura disparu $\lambda$- %
dont les solutions ont pour graphes des arcs des termes de $\Gamma_{\lambda}$.\\
Les termes de $(\Gamma_{\lambda})$ sont:
\begin{enumerate}
\item Les hyperboles d'équations respectives: $xy = \lambda$.
\item Les cercles d'équations respectives: $x^{2} + y^{2} = \lambda$.
\item Les ellipses de foyers $(-1,0)$ et $(1,0)$ -trouver $G$.
\item Les courbes d'équations repectives :
\[(c(x)\lambda + d(x))y = a(x)\lambda + b(x)\]
où $a$, $b$, $c$, et $d$ sont des applications de classe $C^{1}$ de $\mathbb{R}$ dans lui-même.
\end{enumerate}
\end{exer}

\begin{exer}
Déterminer le rayon de convergence de la série entière $\sum a_n X^n$, où $a_n$ désigne la $n$-ième décimale de $\pi$.
\end{exer}

\begin{exer}
Soit $\sum a_n$ une série divergente à termes strictement positifs. %
On note $(S_n)$ la suite des sommmes partielles de $\sum a_n$, et on suppose que de plus, %
$\frac{a_n}{S_n}$ tend vers $0$ en $+ \infty$. %
Calculer les rayons de convergence respectifs de $\sum a_n z^n$ et $\sum S_n z^n$, %
et étudier la relation qu'entretiennent leurs sommes respectives.
\end{exer}

\begin{exer}
Soient $n$ un entier naturel supérieur à $2$, $(A_i)_i$ une famille de $n$ points du plan Euclidien.
\begin{enumerate}
\item Donner une condition nécessaire et suffisante pour qu'il existe un $n-$gone %
dont les milieux des côtés soient les termes de $(A_i)$.
\item Construire, le cas échéant, ces polygones.
\end{enumerate}
%Indication : on discutera selon la parité de $n$.
\end{exer}

\begin{exer}[Décompostion de Dunford d'un endomorphisme d'espace vectoriel complexe]
Soit $E$ un espace vectoriel complexe de dimension finie.
\begin{itemize}
\item Montrer qu'un endomorphisme $u$ de $E$ se décompose de manière unique d'une forme $d+n$, %
où $d$ et $n$ sont deuxendomorphismes, respectivement diagonolisable et nilpotent, qui commutent.
\item Montrer de plus que $d$ et $n$ sont des polynômes en $u$.
\end{itemize}
%Indications : on trigonalise $u$. On montre facilement l'existence et la commutativité de la décomposition sur les sous-espaces caractéristiques. Ensuite, on montre que les projections sur les sous-espaces caractéristiques sont des polynômes en $u$ en appliquant la propriété de Bézout à un polynôme bien choisi -il s'agit du polynôme caractéristique de la somme des autres espacees caractéristiques.\\
%On montre l'unicité en se rappelant qu'un endomorphisme diagonalisable et nilpotent est nul.
\end{exer}

\begin{exer}
Soient $n$ un entier naturel non nul, et $\langle | \rangle$ le produit scalaire canonique de $R^n$.
\begin{enumerate}
\item Soient $p$ un entier naturel non nul, $(u_i)$ une famille de $p$ vecteurs de $R^n$ vérifiant :
\[\forall (i,j) \in [[1,p]]^2 , \langle u_i | u_j \rangle < 0\]
Montrer que toute sous-famille de $p-1$ vecteurs de $(u_i)$ est libre.
\item Montrer que l'on ne peut trouver plus de $n+1$ vecteurs vérifiant ces conditions.
\item Montrer que l'on  peut trouver $n+1$ vecteurs vérifiant ces conditions.
\end{enumerate}
\end{exer}

\begin{exer}
Soient $I$ un intervalle compact de $\mathbb{R}$, et $(f_n)_{n \in \mathbb{N}})$ %
une suite d'applications de $I$ dans $\mathbb{R}$ telles que :\\
Si $(x_n)$ est une suite convergente de $I$, alors $(f_n(x_n))_{n \in \mathbb{N}}$ converge.
\begin{enumerate}
\item Montrer que $(f_n)$ converge simplement vers une limite $f$.
\item Montrer que $(f_n)$ converge uniform\'ement.
\end{enumerate}
\end{exer}

\begin{exer}
Donner un \'equivalent en $+ \infty$ de $x \mapsto \exp(-x^2) \int\limits_0^{x^2} \exp(t^2) dt$.
\end{exer}

\begin{exer}[$\mathbb{R}^3$ se partitionne en cercles Euclidiens]
\begin{enumerate}
\item Donner une partition simple d'un tore plein en cercles Euclidiens.
\item Comment construit-on simplement un tore deuxième partitionné, %
dont le cercle radial contient le centre du premier tore, et qui englobe la première figure ? %
On s'inspirera, dans une large mesure, de la construction précédente.
\item Construire une partition de $\mathbb{R}^3$ en cercles Euclidiens.
%Indications : procéder séquentiellement.\\
%Au cours de l'étape de la question 2), on consruit un objet intermédiaire -nature ?- qui contient les points de l'espace situés à une distance inférieure à $1$ du tore initial, avant de poursuivre.
\end{enumerate}
\end{exer}
\chapter{Exercices \g{a} retravailler}

% \section{}
\section{Enonc\'es \g{a} retravailler}

\begin{exer}
Soient $n$ un entier naturel non nul, $A$ une matrice telle que :
\[\forall M \in M_n (\mathbb{C}) , \det (A+M) = (\det A) + (\det M)\]
Montrer que $A$ est nulle.
\end{exer}

\begin{exer}
\begin{enumerate}
\item Enoncer le théorème de convergence d'un produit de Cauchy qui a été démontré en cours.
\item Démontrer le théorème de Mertens : si $\sum u_n$ et $\sum v_m$ sont deux séries d'une algèbre normée, %
respectivement absolument convergente et convergente, alors le produit de Cauchy de ces deux séries est convergent, de somme :
\[(\sum\limits_{n=0}^{+ \infty} u_n)(\sum\limits_{m=0}^{+ \infty} v_m)\]
\end{enumerate}
\end{exer}

\begin{exer}
Théorème de Stone-Weierstrass général.
\end{exer}


\begin{exer}
Soit $E$ un espace préhilbertien réel, $F$ un espace vectoriel normé, et $f$ une application continue de $E$ dans $F$ vérifiant :
\[\forall (x,y) \in E^2 , x \perp y \Rightarrow f(x+y) = f(x) + f(y)\]
On se propose de déterminer la nature de $f$.
\begin{enumerate}
\item Examiner le cas où $E$ est de dimension inférieure ou égale à $1$.
\item Que peut-on dire de $x+y$ et $x-y$ lorsque $x$ et $y$ sont deux vecteurs de $E$ de même norme ?
\item Examiner le cas où $f$ est paire.
\item Examiner le cas où $f$ est impaire.
\item Examiner le cas général.
\end{enumerate}
\end{exer}


\begin{exer}
Soient $E$ un espace Euclidien de dimension $3$, $x$, $y$, et $z$ des vecteurs de $E$.\\
Montrer :
\[[y \wedge z, z \wedge x, x \wedge y] = ([x,y,z])^2\]
et \[(x \wedge y)\wedge (x \wedge z) = [x,y,z]x\]
\end{exer}

\begin{exer}
Soit $(z_n)$ une suite complexe vérifiant :
\[\forall (n,m) \in \mathbb{N}^2 , \lvert z_n - z_m \rvert \geq 1\]
\begin{enumerate}
\item Soit $\alpha$ un réel strictement positif. Montrer que $\sum \frac{1}{\lvert z_n \rvert^{2 + \alpha}}$ converge.
\item $\sum \frac{1}{\lvert z_n \rvert^2}$ converge-t-elle ?
\end{enumerate}
\end{exer}

\begin{exer}
Soit $(K,d)$ un espace métrique compact, par exemple un fermé borné d'un espace vectoriel normé de dimension finie.\\
On considère une isométrie $f$ de $K$ dans lui-même, c'est-à-dire une application de $K$ dans $K$ qui conserve la distance.
\begin{enumerate}
\item $f$ admet-elle nécessairement un point fixe ?
\item Montrer que $f$ est surjective.
%Indications : supposer que ce ne soit pas le cas, et considérer un élément $x_0$ de $K$ qui n'est pas dans l'image de $f$. Définir la suite $(x_n)$ des images de $x_0$ par les itérées de $f$ et montrer que $f$ induit une permutation de l'ensemble $L$ des valeurs d'adhérence de $(x_n)$. Utiliser la compacité de $K$ pour montrer que $L$ est non vide. En déduire une contradiction en considérant la suite des distances entre les termes de $(x_n)$ et $L$.\\
%Indications : Supposer que ce ne soit pas le cas, et considérer un point de $K$ dont la distance à l'image de $f$ soit maximale -pourquoi ce point existe-t-il ?-, étudier la suite des itérées de ce point par $f$ pour déduire une contradiction.
\end{enumerate}
\end{exer}

\begin{exer}
Considérons la solution $f$ l'équation différentielle $y'=1+Idy^{2}$, maximale, nulle en $0$, %
dont on notera $I_{f}$ l'intervalle de définition.\\
On suppose que $f$ admet un développement en série entière au voisinage de $0$.
\begin{enumerate}
\item Montrer que ce développement est de la forme $\sum_{n=0}^{\propto} a_{n}X^{3n+1}$ et que $(a_{n})$ vérifie:
$a_{0}=1$ ; \[\forall n \in \mathbb{N}, (3n+4)a_{n+1}=\sum_{q=0}^{n} a_{n-q}a_{q}\]
\item Montrer que la suite $(a_{n})$ définie par les relations ci-dessus est décroissante.
\item Montrer que $f$ est développable en série entière au voisinage de $0$, et minorer son rayon de convergence $R$.
\item Montrer que $R < \pi/2 - Arctan(5/4) + 1$
\item Montrer que $I_{f}$ n'est pas borné inférieurement.
\end{enumerate}
\end{exer}

Théorème de réalisation de Borel.

Série entière semi-convergente en tout point de son cercle de convergence ?

Condition de Frédéric pour la développabilité en sérien entière :\\
Existence d'un réel strictement positif $M$ tel que :
\[forall k \in \mathbb{N} , \frac{f^{(k)}(0)}{k!} \leq M^k\]
??

\begin{exer}
Sous-groupes compacts de $GL_n(\mathbb{R})$, $n \in \mathbb{N}^{\ast}$.\\
Cet exercice utilise le résultat suivant :
\textit{
Soit un espace vectoriel normé $E$, une partie compacte et convexe $K$ de $E$ et un groupe $G$ affine, %
équicontinu -les normes des parties linéaires sont majorées par une constante $M$ fixée-, sur $E$ qui laisse stable $K$. %
Il suffit par exemple que $G$ soit compact. %
Alors les éléments de $G$ admettent un point fixe commun dans $K$.
}
Soit $G$ un sous-groupe compact de $GL_n(\mathbb{R})$. %
On se propose de montrer que $G$ est conjugué à un sous-groupe de $O_n(\mathbb{R})$, %
soit que les éléments de $G$ préservent un produit scalaire.
%On étudie à cette fin l'action de $G$ sur l'ensemble des produits scalaires de $\mathbb{R}^n$, il s'agit de prouver que cette action admet un point fixe.\\
\begin{itemize}
\item Montrer que l'ensemble $S_n^{++}(\mathbb{R})$ est convexe.
\item Que dire de l'ensemble des termes de la famille $(^tMM)_{M \in G}$ ?
\item Etudier l'adhérence de son enveloppe convexe.
\item Conclure à l'aide de la propriété des groupes affines compacts énoncée au début de cet exercice.
\end{itemize}
\end{exer}

\begin{exer}
On considère une famille $(\Gamma_{\lambda})_{\lambda \in \Lambda}$ de courbes de $\mathbb{R}^2$, %
indexée sur une partie $\Lambda$ de $\mathbb{R}$, %
définie par une famille d'équations $(G(x,y,\lambda)=0)_{\lambda \in \Lambda}$, %
où $G$ est une fonction de classe $C^{1}$. %
Dans les cas suivants, trouver une équation différentielle en $x \mapsto y$ -dont aura disparu $\lambda$- %
dont les solutions ont pour graphes des arcs des termes de $\Gamma_{\lambda}$.\\
Les termes de $(\Gamma_{\lambda})$ sont:
\begin{enumerate}
\item Les hyperboles d'équations respectives: $xy = \lambda$.
\item Les cercles d'équations respectives: $x^{2} + y^{2} = \lambda$.
\item Les ellipses de foyers $(-1,0)$ et $(1,0)$ -trouver $G$.
\item Les courbes d'équations repectives :
\[(c(x)\lambda + d(x))y = a(x)\lambda + b(x)\]
où $a$, $b$, $c$, et $d$ sont des applications de classe $C^{1}$ de $\mathbb{R}$ dans lui-même.
\end{enumerate}
\end{exer}

\begin{exer}
Déterminer le rayon de convergence de la série entière $\sum a_n X^n$, où $a_n$ désigne la $n$-ième décimale de $\pi$.
\end{exer}

\begin{exer}
Soit $\sum a_n$ une série divergente à termes strictement positifs. %
On note $(S_n)$ la suite des sommmes partielles de $\sum a_n$, et on suppose que de plus, %
$\frac{a_n}{S_n}$ tend vers $0$ en $+ \infty$. %
Calculer les rayons de convergence respectifs de $\sum a_n z^n$ et $\sum S_n z^n$, %
et étudier la relation qu'entretiennent leurs sommes respectives.
\end{exer}

\begin{exer}
Soient $n$ un entier naturel supérieur à $2$, $(A_i)_i$ une famille de $n$ points du plan Euclidien.
\begin{enumerate}
\item Donner une condition nécessaire et suffisante pour qu'il existe un $n-$gone %
dont les milieux des côtés soient les termes de $(A_i)$.
\item Construire, le cas échéant, ces polygones.
\end{enumerate}
%Indication : on discutera selon la parité de $n$.
\end{exer}

\begin{exer}[Décompostion de Dunford d'un endomorphisme d'espace vectoriel complexe]
Soit $E$ un espace vectoriel complexe de dimension finie.
\begin{itemize}
\item Montrer qu'un endomorphisme $u$ de $E$ se décompose de manière unique d'une forme $d+n$, %
où $d$ et $n$ sont deuxendomorphismes, respectivement diagonolisable et nilpotent, qui commutent.
\item Montrer de plus que $d$ et $n$ sont des polynômes en $u$.
\end{itemize}
%Indications : on trigonalise $u$. On montre facilement l'existence et la commutativité de la décomposition sur les sous-espaces caractéristiques. Ensuite, on montre que les projections sur les sous-espaces caractéristiques sont des polynômes en $u$ en appliquant la propriété de Bézout à un polynôme bien choisi -il s'agit du polynôme caractéristique de la somme des autres espacees caractéristiques.\\
%On montre l'unicité en se rappelant qu'un endomorphisme diagonalisable et nilpotent est nul.
\end{exer}

\begin{exer}
Soient $n$ un entier naturel non nul, et $\langle | \rangle$ le produit scalaire canonique de $R^n$.
\begin{enumerate}
\item Soient $p$ un entier naturel non nul, $(u_i)$ une famille de $p$ vecteurs de $R^n$ vérifiant :
\[\forall (i,j) \in [[1,p]]^2 , \langle u_i | u_j \rangle < 0\]
Montrer que toute sous-famille de $p-1$ vecteurs de $(u_i)$ est libre.
\item Montrer que l'on ne peut trouver plus de $n+1$ vecteurs vérifiant ces conditions.
\item Montrer que l'on  peut trouver $n+1$ vecteurs vérifiant ces conditions.
\end{enumerate}
\end{exer}

\begin{exer}
Soient $I$ un intervalle compact de $\mathbb{R}$, et $(f_n)_{n \in \mathbb{N}})$ %
une suite d'applications de $I$ dans $\mathbb{R}$ telles que :\\
Si $(x_n)$ est une suite convergente de $I$, alors $(f_n(x_n))_{n \in \mathbb{N}}$ converge.
\begin{enumerate}
\item Montrer que $(f_n)$ converge simplement vers une limite $f$.
\item Montrer que $(f_n)$ converge uniform\'ement.
\end{enumerate}
\end{exer}

\begin{exer}
Donner un \'equivalent en $+ \infty$ de $x \mapsto \exp(-x^2) \int\limits_0^{x^2} \exp(t^2) dt$.
\end{exer}

\begin{exer}[$\mathbb{R}^3$ se partitionne en cercles Euclidiens]
\begin{enumerate}
\item Donner une partition simple d'un tore plein en cercles Euclidiens.
\item Comment construit-on simplement un tore deuxième partitionné, %
dont le cercle radial contient le centre du premier tore, et qui englobe la première figure ? %
On s'inspirera, dans une large mesure, de la construction précédente.
\item Construire une partition de $\mathbb{R}^3$ en cercles Euclidiens.
%Indications : procéder séquentiellement.\\
%Au cours de l'étape de la question 2), on consruit un objet intermédiaire -nature ?- qui contient les points de l'espace situés à une distance inférieure à $1$ du tore initial, avant de poursuivre.
\end{enumerate}
\end{exer}



% \chapter{Ancien programme}
\chapter{Ancien programme}

% \section{Ancien programme}

\subsection{Alg\g{e}bre bilin\'eaire}

\begin{exer}
Soit $A = \mathbb{R} \rightarrow M_n(\mathbb{R}) : t \mapsto A(t)$ une application continue qui prend des valeurs antisymétriques.\\
Montrer que toute solution de l'équation différentielle en $X$, fonction de $\mathbb{R}$ dans $M_n(\mathbb{R})$ :
\[\forall t \in \mathbb{R} , X'(t) = A(t)X(t)\]
telle que $X(0)$ soit une matrice orthogonale, prend ses valeurs dans l'ensemble des matrices orthogonales.
\end{exer}

\begin{exer}
%Cet exercice utilise les résultats de l'exercice ??\\
Soit $n$ un entier naturel non nul. On notera $Asym(n,\mathbb{R})$ l'espace des matrices antisymétriques d'ordre $n$ %
sur $\mathbb{R}$.\\
Montrer que l'exponentielle de matrices induit une surjection de $Asym(n,\mathbb{R})$ sur $SO(n,\mathbb{R})$.
\end{exer}

\subsection{Formes bilin\'eaires et quadratiques}

\begin{exer}
Sioent $q$ et $q'$ deux formes quadratiques de même cône isotrope.

Donner une relation simple entre $q$ et $q'$.
\end{exer}

\begin{exer}
Montrer que les formes bilinéaires $\phi$, non dégénérées, d'un espace vectoriel vérifiant :
\[\forall (x,y) \in E^2, \phi (x,y) = 0 \Rightarrow \phi (y,x) = 0\]
sont les formes bilinéaires symétriques et antisymétriques.

\medskip
On pourra \'etudier les familles de formes lin\'eaires, indexées en $x$ 
$d_x : E \rightarrow \mathbb{K} : y \mapsto \phi(y,x)$ et $g_x : E \rightarrow \mathbb{K} : y \mapsto \phi(x,y)$
\end{exer}

\newpage

\begin{center}
\fbox{
\begin{minipage}{15cm}
\textit{
Pour toute forme quadratique $q$, on appelle groupe orthogonal de $q$ et note $O(q)$ %
l'ensemble des automorphismes linéaires $u$ de $E$, encore appelés isométries, tels que :
\[\forall x \in E , q(u(x)) = q(x)\]
En particulier, une isom\'etrie pour $q$ pr\'eserve sa forme polaire $\varphi$
}
\end{minipage}
}
\end{center}

\begin{exer}[Commutant du groupe $O(q)$ dans $\mathcal{L}(E)$]
Soit $E$ un espace vectoriel de dimension finie. On considère, dans cet exercice, une forme quadratique $q$ non dégénérée sur $E$, on note $\varphi$ sa forme polaire.

Soit de plus $a$ un vecteur de $E$, non isotrope pour $q$. %
On note $A$ la droite vectorielle de $E$ engendrée par $a$, et $B$ l'espace $\{ x \in E | \varphi (a,x) = 0 \}$.
\begin{enumerate}
\item Montrer que : $E = A \oplus B$.
\item Avec cette notation, montrer que la symétrie linéaire par rapport à $A$, et parallèlement à $B$, est une isométrie pour $q$.

\medskip
On note :\[C = \{ v \in L(E) | \forall u \in O(q) , uv = vu\}\]
cet ensemble est appelé le commutant de $O(q)$ dans $L(E)$.\\
Soit $v$ un élément de $C$.
\item Soit encore $a$ un vecteur de $E$ non isotrope pour $q$. Montrer que : %
$\exists \lambda_a \in \mathbb{R} | v(a) = \lambda_a a$.
\item Que dire du scalaire $\lambda_b$, défini pour un autre quelconque vecteur non isotrope de $q$ ?
\item Etudier le cas d'un vecteur isotrope de $q$.
\item Déterminer $C$.
\end{enumerate}
\end{exer}

\begin{exer}[Condition de minimalité de $O(q)$]
%\textit{Cet exercice reprend les notations adoptées au début de l'exercice précédent, %
%et utilise le résultat concernant le vecteur isotrope de $q$.}
On reprend la notation de l'exercice pr\'ec\'edent pour le groupe orthogonal.

\medskip
Soient $q$ et $q'$ deux formes quadratiques sur $E$, on suppose que $q$ est non dégénérée. %
On note encore $b$ et $b'$ les formes polaires respectives pour $q$ et $q'$.
\begin{enumerate}
\item Montrer l'existence d'un endomorphisme $u$ de $E$ tel que :
\[\forall (x,y) \in E^2 , b'(x,y) = b(u(x),y)\]
autrement dit :
\[\forall (x,y) \in E^2 , b'(x,y) = b(x,u(y))\]
On suppose maintenant que : $O(q') \subseteq O(q)$.
\item Montrer que cette inclusion est une égalité.
\end{enumerate}
\end{exer}

\subsection{Formes sesquilin\'eaires complexes}

\begin{exer}
Montrer que la norme de $M_n (\mathbb{C})$ qui dérive du produit scalaire $(A,B) \mapsto Tr(A^{\ast} B)$ %
est une norme d'algèbre.\\
Calculer la norme d'une matrice Hermitienne.
\end{exer}

\begin{exer}
Soient $A$ et $B$ deux matrices Hermitiennes. Montrer que les valeurs propres de $AB - BA$ sont imaginaires pures.
\end{exer}

\begin{exer}
Montrer qu'une matrice réelle -respectivement complexe- inversible est le produit d'une matrice orthogonale %
par une matrice symétrique définie positive -respectivement une matrice unitaire par une matrice Hermitienne définie positive.
\end{exer}

\begin{exer}
On définit un ordre $\preceq$ sur l'ensemble des matrices hermitiennes positives par : $H \preceq K$ si et seulement si $K - H$ est positive.

Montrer que $M_n(\mathbb{C}) \rightarrow M_n(\mathbb{C}) : A \mapsto A^{\ast} A$ est convexe pour cette relation d'ordre.
\end{exer}
% \section{Ancien programme}

\subsection{Alg\g{e}bre bilin\'eaire}

\begin{exer}
Soit $A = \mathbb{R} \rightarrow M_n(\mathbb{R}) : t \mapsto A(t)$ une application continue qui prend des valeurs antisymétriques.\\
Montrer que toute solution de l'équation différentielle en $X$, fonction de $\mathbb{R}$ dans $M_n(\mathbb{R})$ :
\[\forall t \in \mathbb{R} , X'(t) = A(t)X(t)\]
telle que $X(0)$ soit une matrice orthogonale, prend ses valeurs dans l'ensemble des matrices orthogonales.
\end{exer}

\begin{exer}
%Cet exercice utilise les résultats de l'exercice ??\\
Soit $n$ un entier naturel non nul. On notera $Asym(n,\mathbb{R})$ l'espace des matrices antisymétriques d'ordre $n$ %
sur $\mathbb{R}$.\\
Montrer que l'exponentielle de matrices induit une surjection de $Asym(n,\mathbb{R})$ sur $SO(n,\mathbb{R})$.
\end{exer}

\subsection{Formes bilin\'eaires et quadratiques}

\begin{exer}
Sioent $q$ et $q'$ deux formes quadratiques de même cône isotrope.

Donner une relation simple entre $q$ et $q'$.
\end{exer}

\begin{exer}
Montrer que les formes bilinéaires $\phi$, non dégénérées, d'un espace vectoriel vérifiant :
\[\forall (x,y) \in E^2, \phi (x,y) = 0 \Rightarrow \phi (y,x) = 0\]
sont les formes bilinéaires symétriques et antisymétriques.

\medskip
On pourra \'etudier les familles de formes lin\'eaires, indexées en $x$ 
$d_x : E \rightarrow \mathbb{K} : y \mapsto \phi(y,x)$ et $g_x : E \rightarrow \mathbb{K} : y \mapsto \phi(x,y)$
\end{exer}

\newpage

\begin{center}
\fbox{
\begin{minipage}{15cm}
\textit{
Pour toute forme quadratique $q$, on appelle groupe orthogonal de $q$ et note $O(q)$ %
l'ensemble des automorphismes linéaires $u$ de $E$, encore appelés isométries, tels que :
\[\forall x \in E , q(u(x)) = q(x)\]
En particulier, une isom\'etrie pour $q$ pr\'eserve sa forme polaire $\varphi$
}
\end{minipage}
}
\end{center}

\begin{exer}[Commutant du groupe $O(q)$ dans $\mathcal{L}(E)$]
Soit $E$ un espace vectoriel de dimension finie. On considère, dans cet exercice, une forme quadratique $q$ non dégénérée sur $E$, on note $\varphi$ sa forme polaire.

Soit de plus $a$ un vecteur de $E$, non isotrope pour $q$. %
On note $A$ la droite vectorielle de $E$ engendrée par $a$, et $B$ l'espace $\{ x \in E | \varphi (a,x) = 0 \}$.
\begin{enumerate}
\item Montrer que : $E = A \oplus B$.
\item Avec cette notation, montrer que la symétrie linéaire par rapport à $A$, et parallèlement à $B$, est une isométrie pour $q$.

\medskip
On note :\[C = \{ v \in L(E) | \forall u \in O(q) , uv = vu\}\]
cet ensemble est appelé le commutant de $O(q)$ dans $L(E)$.\\
Soit $v$ un élément de $C$.
\item Soit encore $a$ un vecteur de $E$ non isotrope pour $q$. Montrer que : %
$\exists \lambda_a \in \mathbb{R} | v(a) = \lambda_a a$.
\item Que dire du scalaire $\lambda_b$, défini pour un autre quelconque vecteur non isotrope de $q$ ?
\item Etudier le cas d'un vecteur isotrope de $q$.
\item Déterminer $C$.
\end{enumerate}
\end{exer}

\begin{exer}[Condition de minimalité de $O(q)$]
%\textit{Cet exercice reprend les notations adoptées au début de l'exercice précédent, %
%et utilise le résultat concernant le vecteur isotrope de $q$.}
On reprend la notation de l'exercice pr\'ec\'edent pour le groupe orthogonal.

\medskip
Soient $q$ et $q'$ deux formes quadratiques sur $E$, on suppose que $q$ est non dégénérée. %
On note encore $b$ et $b'$ les formes polaires respectives pour $q$ et $q'$.
\begin{enumerate}
\item Montrer l'existence d'un endomorphisme $u$ de $E$ tel que :
\[\forall (x,y) \in E^2 , b'(x,y) = b(u(x),y)\]
autrement dit :
\[\forall (x,y) \in E^2 , b'(x,y) = b(x,u(y))\]
On suppose maintenant que : $O(q') \subseteq O(q)$.
\item Montrer que cette inclusion est une égalité.
\end{enumerate}
\end{exer}

\subsection{Formes sesquilin\'eaires complexes}

\begin{exer}
Montrer que la norme de $M_n (\mathbb{C})$ qui dérive du produit scalaire $(A,B) \mapsto Tr(A^{\ast} B)$ %
est une norme d'algèbre.\\
Calculer la norme d'une matrice Hermitienne.
\end{exer}

\begin{exer}
Soient $A$ et $B$ deux matrices Hermitiennes. Montrer que les valeurs propres de $AB - BA$ sont imaginaires pures.
\end{exer}

\begin{exer}
Montrer qu'une matrice réelle -respectivement complexe- inversible est le produit d'une matrice orthogonale %
par une matrice symétrique définie positive -respectivement une matrice unitaire par une matrice Hermitienne définie positive.
\end{exer}

\begin{exer}
On définit un ordre $\preceq$ sur l'ensemble des matrices hermitiennes positives par : $H \preceq K$ si et seulement si $K - H$ est positive.

Montrer que $M_n(\mathbb{C}) \rightarrow M_n(\mathbb{C}) : A \mapsto A^{\ast} A$ est convexe pour cette relation d'ordre.
\end{exer}
% \section{Ancien programme : espaces complets}

\begin{exer}
Soit $E$ un espace vectoriel normé.

Montrer que $E$ est un espace de Banach si et seulement si toute série absolument convergente de $E$ est convergente.
\end{exer}

\begin{exer}
%Je donne, si cela s'avère nécessaire, la définition d'un ensemble dénombrable, ainsi que quelques exemples de tels ensembles, sur suggestion de l'élève. Plus particulièrement, j'établis que $\mathbb{Q}$ est dénombrable.\\
Soit $(E,d)$ un espace métrique complet, par exemple un fermé d'un espace de Banach, muni de la distance associée à la norme.
\begin{enumerate}
\item Théorème de Baire : soit $(U_n)_{n \in \mathbb{N}}$ une suite d'ouverts denses de $E$.
Montrer que $\bigcap_{n \in \mathbb{N}}U_n$ est dense dans $E$.
\item Montrer que la réunion d'une suite de fermés d'intérieurs vides de $E$ est d'intérieur vide.
\item Montrer que $\mathbb{R}$ n'est pas dénombrable.
\item Montrer que $\mathbb{R} \backslash \mathbb{Q}$ est dense dans $\mathbb{R}$.

\medskip
On note maintenant $\mathbb{K}$ l'un des deux corps $\mathbb{R}$ ou $\mathbb{C}$.
\item Soit $E$ un espace de Banach sur $\mathbb{K}$. Montrer que $E$ n'admet pas de base dénombrable.
%Indication : Soit $(x_n)$ une éventuelle base dénombrable de $E$. Considérer la suite $(Vect(x_k)_{k \in 0,n})_n$.\\
\item Montrer qu'il n'existe pas de norme pour laquelle $\mathbb{K}[X]$ soit un espace de Banach.
\end{enumerate}
\end{exer}

\begin{exer}
$\mathbb{K}$ est ici le corps $\mathbb{R}$ ou $\mathbb{C}$.\\
On note $l^{\infty}(\mathbb{K})$ l'espace vectoriel des suites bornées de $\mathbb{K}$ muni de la norme $\|.\|_{\infty}$ définie par :
\[\forall (u_n) \in l^{\infty}(\mathbb{K}) , \|(u_n)\|_{\infty} = \sup_{n \in \mathbb{N}} u_n\]
Montrer que $l^{\infty}(\mathbb{K})$ est un espace de Banach.
\end{exer}

\begin{exer}
$\mathbb{K}$ est ici le corps $\mathbb{R}$ ou $\mathbb{C}$. $p \in [1,+\infty[$\\
On note, pour tout réel $p$ supérieur ou égal à $1$, $l^{p}(\mathbb{K})$ l'espace vectoriel des suites $(u_n)$ de $\mathbb{K}$ %
telles que $\sum \lvert u_n \rvert^{p}$ converge, muni de la norme $\|.\|_{p}$ définie par :
\[\forall (u_n) \in l^{p}(\mathbb{K}) , \|(u_n)\|_{p} = (\sum_{n \in \mathbb{N}} \lvert u_n \rvert^{p})^{1/p}\]
\begin{enumerate}
\item Quelles relations d'inclusion existe-t-il entre les espaces $l^{p}(\mathbb{K})$ ?
\item Montrer que ces espaces vectoriels normés sont de Banach.
\end{enumerate}
%Eventuellement, afin de faciliter les calculs, je demande de se restreindre au cas de $l^{1}(\mathbb{K})$.
\end{exer}

\begin{exer}
Soit $E$ un espace vectoriel normé.\\
Montrer que $E$ est un espace de Banach si et seulement si toute série absolument convergente de $E$ est convergente.
\end{exer}

\begin{exer}
$\mathbb{K}$ est ici le corps $\mathbb{R}$ ou $\mathbb{C}$.\\
On note $l^{\infty}(\mathbb{K})$ l'espace vectoriel des suites bornées de $\mathbb{K}$ muni de la norme %
$\|.\|_{\infty}$ définie par :
\[\forall (u_n) \in l^{\infty}(\mathbb{K}) , \|(u_n)\|_{\infty} = \sup_{n \in \mathbb{N}} u_n\]
Montrer que $l^{\infty}(\mathbb{K})$ est un espace de Banach.
\end{exer}

\begin{exer}[Dual de $l^1$]
On se place dans l'espace des suites réelles $(u_n)_n$ telles que la série $\sum_n u_n$ est absolument convergente, nous notons cet espace $l^1(\mathbb{R})$ et le munissons de la norme $\| \|_1$ définie par : $\forall (u_n)_n \in l^1 , \| (u_n) \|_1 = \sum\limits_{n=0}^{+\infty} | u_n |$. On appelle dual topologique de cet espace, et note $(l^1)'$, l'espace vectoriel des formes linéaires continues de $l^1$, que nous munirons de la norme subordonnée à $\| \|_1$à la source, $| |$ à l'arrivée.\\
Montrer qu'il existe une isométrie linaire $\phi$ de l'espace $l^{\infty}$ des suites réelles bornées muni de la norme infinie, sur $(l^1)'$, telle que :
\[\forall (a_n)_n \in l^{\infty} , \forall (u_n) \in l^1 , \phi ((a_n)_n) ((u_n)_n) = \sum\limits_{n=0}^{+\infty} a_n u_n\]
\end{exer}

\begin{exer}
\begin{enumerate}
\item Enoncer le th\'eor\`eme de convergence domin\'ee de Lebesgue.
\item Rappeler la d\'efinition de l'int\'egrabilit\'e au sens de Riemann sur un segment.
\item Que dire de l'int\'egrale d'une limite uniforme de fonctions Riemann-int\'egrables ?

\medskip
Soit maintenant $I$ un intervalle r\'eel. On appelle fonction localement int\'egrable sur $I$ une fonction int\'egrable sur tout segment de $I$. %
On peut se ramener au cas au cas d'un intervalle $I$ de la forme $[a,b[$, o\`u $a$ est un nombre r\'eel, et $b$ un e\'el\'ement de $\mathbb{R}\cup\{+\infty\}$, strictement sup\'erieur \`a $a$.
\item Montrer le th\'eor\`eme de convergence dominée pour l'intégrale de Riemann impropre :

\medskip
\fbox{
\begin{minipage}{15cm}
Soit $(f_n)$ une suite de fonctions r\'eelles d\'efinies sur $I$, localement int\'egrables et domin\'ees par une fonction $\phi$, positive et int\'egrable sur $I$. On note que les int\'egrales des termes de $(f_n)$ convergent. %
Si $(f_n)$ converge vers une limite $f$ d\'efinie sur $I$, alors $f$ est localement int\'egrable sur $I$, son int\'egrale sur $I$ converge et :
\[\lim_n \int\limits_a^b f_n(t) dt = \int\limits_a^b f(t) dt\]
\end{minipage}
}
\end{enumerate}
\end{exer}

\begin{exer}
Soit $f$ une fonction réelle définie sur $\mathbb{R}$, telle que $f$ et $f'^2$ soient intégrables.\\
Montrer que $f$ tend vers $0$ en $+ \infty$ et $- \infty$.
%Indication : penser au critère de Cauchy. On pourra utiliser le résultat de l'exercice précédent.
\end{exer}
% \input{analyse_fonctions/anc_series_fourier.tex}
% \input{edo/anc_non_lineaires.tex}
% \input{calcul_differentiel/anc_acc_finis.tex}
% \section{Ancien programme : g\'eom\'etrie diff\'erentielle}

\begin{exer}
Soit $f$ une fonction r\'eelle \`a valeurs complexes, d\'erivable et  $2 \pi$ p\'eriodique.\\
Montrer que : $\frac{1}{2i\pi} \int_0^{2 \pi} \frac{f'(t)}{f(t)} dt$ est un entier.
\end{exer}

\begin{exer}
On se propose de démontrer l'inégalité isopérimétrique :\\
si $\gamma$ est un arc de Jordan régulier de $\mathbb{R}^2$, de longueur $l$ et dont l'intérieur admet pour aire $A$, alors :
\[4 \pi A \leq l^2\]
\begin{enumerate}
\item Démontrer l'inégalité de Wirtinger : si $y$ est une application $2 \pi$-périodique $C^1$ de $\mathbb{R}$ dans lui-même, alors :
\[\int_{0}^{2 \pi} y^2 \leq \int_{0}^{2 \pi} y'^2\]
Etudier le cas d'égalité.

\smallskip
On reprend les notations de l'énoncé, et on écrit : $\gamma := (x,y)$.
\item Montrer que l'on peut supposer $\gamma$ centré en $(0,0)$, et de longueur $2 \pi$.

\smallskip
On supposera par la suite que $\gamma$ est paramétré par longueur d'arc.
\item Démontrer, en utilisant la formule de Green-Riemann, que :\[2A \leq \int_{0}^{2 \pi} (x^2 (s) + y'^2 (s))ds\]
\item Conclure.
%On remarquera que : $\int_{0}^{2 \pi} (x'^2 (s) + y'^2 (s)) ds = \frac{l^2}{2 \pi}$
\item Etudier le cas d'égalité.
\end{enumerate}
\end{exer}



\section{Alg\g{e}bre bilin\'eaire et formes quadratiques}
\chapter{Alg\g{e}bre bilin\'eaire, formes quadratiques, espaces pr\'ehilbertiens}

\section{Alg\`ebre bilin\'eaire, espaces pr\'ehilbertiens}

\begin{exer}
Soit $E$ l'espace vectoriel des suites r\'eeles de carr\'e sommable, c'est-\`a-dire les suites $(u_n)_{n\in\mathbb{N}}$ telles que $\sum u_n^2$ converge. %
On pose : \[\langle(u_n)|(v_n)\rangle =\sum\limits_{n=0}^{\infty} u_n v_n\]
pour toutes les suites $(u_n)$ et $(v_n)$ de $E$.
\begin{enumerate}
\item Montrer que l'application d\'efinie pr\'ec\'edemment est un produit scalaire sur $E$. Quelle norme d\'erive de ce roduit scalaire ?

\smallskip
On note maintenant, pour tout entier naturel $k$, $\delta^{(k)}$ la suite dont le $ki$i\`eme terme vaut $1$, et dont les autres termes sont nuls.
\item D\'ecrire l'espace vectoriel engendr\'e par la suite $(\delta^{(k)})_k$.
\item Montrer que cette suite est totale dans $E$.
\item Montrer qu'il existe une partie d\'enombrable de $E$, qui est dense dans $E$. On dit que $E$ est \textit{s\'eparable}.
\end{enumerate}
On note souvent $l^2(\mathbb{R})$ l'espace $E$, muni de la structure pr\'ehilbertienne \'etud\i\'ee ici.
\end{exer}

%élève2
\begin{exer}
Calculer \[\underset{a,b\in\mathbb{R}^2}{\inf}\int\limits_0^1(x\ln x -ax^2-bx)^2 dx\]
\end{exer}

%élève3
\begin{exer}
Montrer qu'une matrice réelle inversible est le produit d'une matrice orthogonale et d'une matrice triangulaire supérieure %
-décomposition d'Iwasawa.
\end{exer}

\begin{exer}[Polyn\^omes de Tchebycheff et produit scalaire]
On d\'efinit, pour deux polyn\^omes $P$ et $Q$ de $\mathbb{R}[X]$ :
\[\langle P|Q \rangle =\int\limits_{-1}^1 \frac{\tilde{P}(t)\tilde{Q}(t)}{\sqrt{1-t^2}}dt\]
\begin{enumerate}
\item Montrer que l'application $\langle |\rangle$ d\'efinie ci-dessus est un produit scalaire sur $\mathbb{R}[X]$.
\item Soit $n$ un entier naturel. Montrer qu'il existe un unique polyn\^ome $T_n$ tel que :
\[\forall\theta\in\mathbb{R} , T_n(\cos x)=\cos nx\]
\item Etablir la relation de r\'ecurrence d'ordre $2$ entre les termes de la suite $(T_n)$.
\item Montrer que la suite $(T_n)$ est orthogonale pour $\langle |\rangle$ et calculer la norme des termes de $(T_n)$.
\end{enumerate}
\end{exer}

\begin{exer}
Soit $E$ un espace pr\'ehilbertien r\'eel. Soient de plus $F$ et $G$ deux sous-espaces vectoriels de $E$.
\begin{enumerate}
\item Comparer $(F+G)^{\perp}$ et $F^{\perp}\cap G^{\perp}$.
\item Comparer $(F\cap G)^{\perp}$ et $F^{\perp}+G^{\perp}$.
\item Que dire si $E$ est de dimension finie ?
\end{enumerate}
\end{exer}

\begin{exer}
Soit $n$ un entier naturel non nul.\\
D\'eterminer, pour toute matrice $A$ de $\mathcal{M}_n(\mathbb{R})$, la valeur :
\[\underset{M\in\mathcal{S}_n(\mathbb{R})}{\min}\sum\limits_{(i,j)\in [\![1,n]\!]^2} (A_{i,j}-M_{i,j})^2\]
\end{exer}

\begin{exer}
Soit $(E,\langle |\rangle )$ un espace euclidien de dimension sup\'erieure ou \'egale \`a $2$, $a$ et $b$ deux vecteurs unitaires et ind\'ependants de $E$.

On d\'efinit l'endomorphisme lin\'eaire $f$ de $E$ par :\[\forall x \in E, f(x)=\langle a|x\rangle a+\langle b|x \rangle b\]
\begin{enumerate}
\item Caract\'eriser $f$ lorsque $a$ et $b$ sont deux vecteurs orthogonaux.
\item Cas g\'en\'eral : d\'eterminer l'image et les \'el\'ements propres de $f$. $f$ est-il diagonalisable ?
\end{enumerate}
\end{exer}

\begin{exer}
Soit $E$ un espace Euclidien. %
On appelle centre d'un groupe $G$ l'ensemble des éléments de $G$ qui commutent avec tous les éléments de $G$.

D\'eterminer le centre de $O(E)$ et celui de $SO(E)$.
\end{exer}

\begin{exer}
Soit $E$ un espace vectoriel réel de dimension finie.
\begin{enumerate}
\item Montrer que tout endomorphisme de $E$ admet un sous-espace stable de dimension $1$ ou $2$.

\smallskip
On suppose maintenant $E$ Euclidien.
\item Montrer que si $f$ est un endomorphisme de $E$ qui stabilise l'orthogonal de tout sous-espace stable, alors $E$ se décompose en somme directe orthogonale de sous-espaces stables de $f$, de dimension inférieure ou égale à $2$.
\item En déduire, dans des bases orthogonales bien choisies, les matrices des endomorphismes symétriques ou orthogonaux.% Anien programme : endomorphjismes antisymétriques.
\end{enumerate}
\end{exer}

\begin{exer}
Soit $E$ un espace vectoriel réel de dimension finie.

Montrer que si $G$ est un sous-groupe fini de $GL(E)$, %
alors tout sous-espace vectoriel de $E$ stable par tous les éléments de $G$ admet un suppl\'ementaire stable par $G$.
\end{exer}

\begin{exer}
Soit $E$ un espace préhilbertien réel.
\begin{enumerate}
\item Soit $(e_k)$ une suite libre ordonnée, finie ou infinie, de vecteurs de $E$. Définir l'orthonormalisée de Gram-Schmidt de $(e_k)$.
On suppose maintenant $E$ Euclidien de dimension $n$.
\item Que dire de l'ensemble $B$ des bases ordonnées de $E$ par rapport à $E^n$ ?
\item Montrer que l'application de $B$ dans $B$ qui associe, à une base de $E$, son orthonormalisée, est continue.
\end{enumerate}
\end{exer}

\begin{exer}
Trouver tous les couples de sym\'etries orthogonales qui commutent.
\end{exer}% pas d'extension .tex
% \section{Ancien programme}

\subsection{Alg\g{e}bre bilin\'eaire}

\begin{exer}
Soit $A = \mathbb{R} \rightarrow M_n(\mathbb{R}) : t \mapsto A(t)$ une application continue qui prend des valeurs antisymétriques.\\
Montrer que toute solution de l'équation différentielle en $X$, fonction de $\mathbb{R}$ dans $M_n(\mathbb{R})$ :
\[\forall t \in \mathbb{R} , X'(t) = A(t)X(t)\]
telle que $X(0)$ soit une matrice orthogonale, prend ses valeurs dans l'ensemble des matrices orthogonales.
\end{exer}

\begin{exer}
%Cet exercice utilise les résultats de l'exercice ??\\
Soit $n$ un entier naturel non nul. On notera $Asym(n,\mathbb{R})$ l'espace des matrices antisymétriques d'ordre $n$ %
sur $\mathbb{R}$.\\
Montrer que l'exponentielle de matrices induit une surjection de $Asym(n,\mathbb{R})$ sur $SO(n,\mathbb{R})$.
\end{exer}

\subsection{Formes bilin\'eaires et quadratiques}

\begin{exer}
Sioent $q$ et $q'$ deux formes quadratiques de même cône isotrope.

Donner une relation simple entre $q$ et $q'$.
\end{exer}

\begin{exer}
Montrer que les formes bilinéaires $\phi$, non dégénérées, d'un espace vectoriel vérifiant :
\[\forall (x,y) \in E^2, \phi (x,y) = 0 \Rightarrow \phi (y,x) = 0\]
sont les formes bilinéaires symétriques et antisymétriques.

\medskip
On pourra \'etudier les familles de formes lin\'eaires, indexées en $x$ 
$d_x : E \rightarrow \mathbb{K} : y \mapsto \phi(y,x)$ et $g_x : E \rightarrow \mathbb{K} : y \mapsto \phi(x,y)$
\end{exer}

\newpage

\begin{center}
\fbox{
\begin{minipage}{15cm}
\textit{
Pour toute forme quadratique $q$, on appelle groupe orthogonal de $q$ et note $O(q)$ %
l'ensemble des automorphismes linéaires $u$ de $E$, encore appelés isométries, tels que :
\[\forall x \in E , q(u(x)) = q(x)\]
En particulier, une isom\'etrie pour $q$ pr\'eserve sa forme polaire $\varphi$
}
\end{minipage}
}
\end{center}

\begin{exer}[Commutant du groupe $O(q)$ dans $\mathcal{L}(E)$]
Soit $E$ un espace vectoriel de dimension finie. On considère, dans cet exercice, une forme quadratique $q$ non dégénérée sur $E$, on note $\varphi$ sa forme polaire.

Soit de plus $a$ un vecteur de $E$, non isotrope pour $q$. %
On note $A$ la droite vectorielle de $E$ engendrée par $a$, et $B$ l'espace $\{ x \in E | \varphi (a,x) = 0 \}$.
\begin{enumerate}
\item Montrer que : $E = A \oplus B$.
\item Avec cette notation, montrer que la symétrie linéaire par rapport à $A$, et parallèlement à $B$, est une isométrie pour $q$.

\medskip
On note :\[C = \{ v \in L(E) | \forall u \in O(q) , uv = vu\}\]
cet ensemble est appelé le commutant de $O(q)$ dans $L(E)$.\\
Soit $v$ un élément de $C$.
\item Soit encore $a$ un vecteur de $E$ non isotrope pour $q$. Montrer que : %
$\exists \lambda_a \in \mathbb{R} | v(a) = \lambda_a a$.
\item Que dire du scalaire $\lambda_b$, défini pour un autre quelconque vecteur non isotrope de $q$ ?
\item Etudier le cas d'un vecteur isotrope de $q$.
\item Déterminer $C$.
\end{enumerate}
\end{exer}

\begin{exer}[Condition de minimalité de $O(q)$]
%\textit{Cet exercice reprend les notations adoptées au début de l'exercice précédent, %
%et utilise le résultat concernant le vecteur isotrope de $q$.}
On reprend la notation de l'exercice pr\'ec\'edent pour le groupe orthogonal.

\medskip
Soient $q$ et $q'$ deux formes quadratiques sur $E$, on suppose que $q$ est non dégénérée. %
On note encore $b$ et $b'$ les formes polaires respectives pour $q$ et $q'$.
\begin{enumerate}
\item Montrer l'existence d'un endomorphisme $u$ de $E$ tel que :
\[\forall (x,y) \in E^2 , b'(x,y) = b(u(x),y)\]
autrement dit :
\[\forall (x,y) \in E^2 , b'(x,y) = b(x,u(y))\]
On suppose maintenant que : $O(q') \subseteq O(q)$.
\item Montrer que cette inclusion est une égalité.
\end{enumerate}
\end{exer}

\subsection{Formes sesquilin\'eaires complexes}

\begin{exer}
Montrer que la norme de $M_n (\mathbb{C})$ qui dérive du produit scalaire $(A,B) \mapsto Tr(A^{\ast} B)$ %
est une norme d'algèbre.\\
Calculer la norme d'une matrice Hermitienne.
\end{exer}

\begin{exer}
Soient $A$ et $B$ deux matrices Hermitiennes. Montrer que les valeurs propres de $AB - BA$ sont imaginaires pures.
\end{exer}

\begin{exer}
Montrer qu'une matrice réelle -respectivement complexe- inversible est le produit d'une matrice orthogonale %
par une matrice symétrique définie positive -respectivement une matrice unitaire par une matrice Hermitienne définie positive.
\end{exer}

\begin{exer}
On définit un ordre $\preceq$ sur l'ensemble des matrices hermitiennes positives par : $H \preceq K$ si et seulement si $K - H$ est positive.

Montrer que $M_n(\mathbb{C}) \rightarrow M_n(\mathbb{C}) : A \mapsto A^{\ast} A$ est convexe pour cette relation d'ordre.
\end{exer}


\section{G\'eom\'etrie euclidienne}
% \section{G\'eom\'etrie euclidienne}

\begin{exer}
Enoncer et d\'emontrer une r\g{e}gle permettant de calculer explicitement les coordonnées du centre \'eventuel d'une conique, %
dont on conna\^it une équation cart\'esienne.\\
On pourra écrire l'équation d'une conique du plan affine Euclidien sous forme matricielle.
\end{exer}

\begin{exer}
Soit $q$ une forme quadratique non d\'eg\'en\'er\'ee de $\mathbb{R}^2$, d\'efinissant donc une conique $C$ centrée en $0$. %
D\'eterminer une équation cartésienne de la tangente à $\mathbb{C}$ en un point $(x_0,y_0)$ de deux manières :
\begin{itemize}
\item Considérer le vecteur dérivé d'un arc différentiable injectif traçé sur la conique, %
dans un voisinage de $(x_0,y_0)$, au paramètre associé à $(x_0,y_0)$;
\item Calculer (pourquoi ?) le noyau de la différentielle de $q$ en $(x_0,y_0)$.
\end{itemize}
%\textit{Je poursuivrai éventuellement l'exercice avec une généralisation de cette méthode aux submersions -tangence à une fibre et noyau de la différentielle...}
\end{exer}

\section{Espaces complets}
\input{anc_programme/esp_complets.tex}

\section{S\'eries de Fourier}
% \section{S\'eries de Fourier}

\begin{exer}[Inégalité de Wirtinger]
Montrer que, si $y$ est une application $2 \pi$-périodique $C^1$ de $\mathbb{R}$ dans lui-même, alors :
\[\int_{0}^{2 \pi} y^2 \leq \int_{0}^{2 \pi} y'^2\]
Etudier le cas d'égalité.
\end{exer}

\begin{exer}
Soit $f$ une fonction continue, $2\pi-$périodique dont les coefficients de Fourier %
de degré inférieur ou égal à un entier positif $n$ sont nuls.\\
Montrer que $f$ admet au moins $2n$ zéros sur une période.
\end{exer}

\begin{exer}
Soit $P$ un polynôme trigonométrique de degré inférieur ou égal à $n$.\\
Démontrer qu'il existe un réel positif $c$, que l'on calculera explicitement, ne dépendant que de $n$, telle que :
\[\Arrowvert P' \Arrowvert \leq c \Arrowvert P \Arrowvert\]
On pourra, pour tout élément la fonction $I_n$ définie sur $[0,2\pi]$ par :
\[\forall x \in [0,2\pi], I_n(x) = \int_0^{2\pi} P(x-y)F_n(y) dy\]
\end{exer}

\begin{exer}
Soit $f$ une application de classe $C^{\infty}$ de $\mathbb{R}$ dans lui-même, à décroissance rapide ainsi que ses dérivées, c'est-à-dire 
\[\forall (m,n) \in \mathbb{N}^2 , \lim\limits_{\substack{+ \infty \\ - \infty}} x^m f^{(n)}(x) = 0\]
On considère la fonction $\phi$ définie sur $\mathbb{R}$ par :\[\forall x \in \mathbb{R} , \phi (x) = \sum\limits_{n \in \mathbb{Z}} f(x+2k\pi)\]
\begin{enumerate}
\item Montrer que cette somme est bien définie -en quels sens ? Que dire de la régularité de $\phi$ ?
%Indication : On pourra étudier ces questions de convergence sur les compacts de $\mathbb{R}$.\\
\item Montrer que $\phi$ se développe en série de Fourier de la manière suivante
\[\forall x \in \mathbb{R} , \phi (x) = \sum\limits_{n \in \mathbb{Z}} c_n(f) e^{inx}\]
où $\forall n \in \mathbb{Z} , c_n(f) = \int\limits_{\mathbb{R}} f(t) e^{-int} dt$
\end{enumerate}
\end{exer}

\begin{exer}
Développer en série de Fourier la fonction : $x \mapsto \ln (2 + \cos x)$.\\
On pourra, en premier lieu, développer la dérivée en série de Fourier.
\end{exer}

\begin{exer}
Soit $E$ l'espace des fonctions complexes, continues et $2\pi-$périodiques sur $\mathbb{R}$, qui sont limites uniformes de leur série de Fourier.\\
On définit sur $E$ la norme $\| \|_E$ par :
\[\forall f \in E \|f\| _E = \sup\limits_{n \in \mathbb{N}} \| S_n(f) \|_{\infty}\]
Montrer que $(E , \| \|_E)$ est un espace de Banach.
\end{exer}

\begin{exer}[Equation à retard]
Soit $\lambda$ un réel. On considère l'équation en $f$ suivante, dite équation à retard :\[\forall t \in \mathbb{R} , f'(t) = f(t + \lambda)\]
\begin{enumerate}
\item Montrer que si $f$ est solution, alors $f$ et $f'$ sont développables en série de Fourier.
\item Avec la même notation, montrer que :\[\forall n \in \mathbb{Z} , (in - e^{in \lambda})c_n(f) = 0\]
En déduire toutes les solutions possibles.
\end{enumerate}
\end{exer}

\begin{exer}[Lemme de Riemann-Lebesgue généralisé]
Soit : $(a,b) \in \mathbb{R}$ , $a<b$.\\
Montrer que si $f$ est une fonction réelle continue définie sur $[a,b]$, %
$g$ une fonction intégrable au sens de Riemann, $2\pi -$périodique et positive, alors :
\[\lim\limits_{n \rightarrow \infty} \int\limits_a^b f(t)g(nt) dt = \frac{1}{2\pi} \int\limits_a^b f(t) dt \int\limits_0^{2\pi} g(t) dt\]
\end{exer}

\section{EDO non lin\'eaires}
% \section{EDO r\'esolues : cas g\'en\'eral}

\begin{exer}[Flot de vecteur associé à une équation différentielle résolue]% et théorème de Cauchy-Lipschitz]
Soit $E$ un espace de Banach, et $U$ un ouvert de $E \times \mathbb{R}$. %
On considère une fonction $f$ de $U$ dans $E$, autrement dit un champ de vecteurs sur $U$, localement Lipschitzienne en $x$, par exemple $C^1$. %
On s'intéresse à l'équation différentielle en $x$ : \[x' = f(x,t)\]
On note, pour tout couple $(x,s)$ de $U$ et tout temps $t$ suffisamment proche de $s$, $\varphi(t;s,x)$ la valeur en $t$ de la solution globale de $(Eq)$ associée à la condition initiale $(x,s)$ -pourquoi existe-t-elle ?\\
\ligneinter
\begin{enumerate}
\item Montrer que, pour tous $(x,s_1)$, $s_2$ et $t$ tels que cette exession est définie : \[\varphi(t;s_2,\varphi(s_2;s_1,x)) = \varphi(t;s_1,x)\]
\begin{center}
\begin{minipage}{16cm}
\textit{
On s'intéresse au cas autonome. %
$U$ s'écrit alors $V \times$ $I$, où $V$ est un ouvert de $E$ et $I$ un intervalle ouvert de $\mathbb{R}$. %
Soit $y_0$ une solution du problème : $y_0' = f(y_0,u)$ , $y_0(s_0) = Y$ avec $s_0$ un temps dans $I$ et $Y$ un vecteur de $V$.
}
\end{minipage}
\end{center}
\item Montrer que l'on peut définir un nouveau problème de Cauchy : $y' = f(y,t)$, $y(0) = Y$\\
eu égard la constance de $f$ en $t$, on précisera l'intervalle de définition de la nouvelle solution $y_{0 Y}$.

\smallskip
On suppose maintenant : $s_0 = 0$, $I$ en particulier contient $0$.
\item A l'aide d'un changement de variable inspiré de la question précédente %
et en utilisant une solution bien choisie à l'équation différentielle $y' = f(y)$, montrer :
\[\varphi(t+s;0,Y) = \varphi(t;0,\varphi(s;0,Y))\]
\end{enumerate}
%On note : $\phi = \mathbb{R} \times U \rightarrow V : (t,x) \mapsto \varphi(t;0,x)$, on remarque que $\phi$ n'est pas toujours défini pour tout temps $t$.\\
%$\phi$ définit, lorsque $I$ est $\mathbb{R}$ -flot complet-, un morphisme du groupe additif $(\mathbb{R} , +)$ dans le groupe des bijections de $V$ sur lui-même. l'image de ce morphisme peut être incluse dans le groupe des homéomorphismes, des difféomorphismes... de $V$ sur lui-même, suivant la régularité de $f$.
%\textit{On appelle parfois groupe à un paramètre d'automorphismes continus, différentiels... un morphisme continu de $\mathbb{R}$ dans le groupe d'automlorphismes continus, différentiels, de l'espace sous-jacent, ici $V$.}
\end{exer}

\begin{exer}[Explosion en temps fini, espace des phases de dimension finie]
Soient $n$ un entier strictement positif, $J$ un intervalle ouvert réel %
et $f(.,.)$ un champ de vecteurs défini sur $J \times \mathbb{R}^n$. %
On choisit un couple $(t_0,y_0)$ de conditions initiales sur $J \times \mathbb{R}^n$. %
On note $y$ la solution du problème de Cauchy associé, $] T_{\ast} , T^{\ast} [$ son intervalle de définition.
\begin{enumerate}
\item Montrer que l'on se trouve dans l'alternative -non exclusive- suivante :
\begin{enumerate}
\item $T^{\ast} = \sup J$ ;
\item $\lim\limits_{t \rightarrow T^{\ast}} \| y(t) \| = + \infty$.
\end{enumerate}

\smallskip
\textit{Application.} Soit $t_0$ un nombre réel, $g$ et $h$ des fonctions, respectivement $C^1$ %
définie sur $\mathbb{R}_+^{\ast}$ et continue sur $[t_0 , + \infty [$, dans $\mathbb{R}_+^{\ast}$, telles que :
\[\forall a \in \mathbb{R}_+^{\ast} , \int\limits_a^{+ \infty} \frac{1}{g(x)} dx = + \infty\]
\item Montrer que les solutions de l'équation différentielle : $x^{'} = h \cdot g(x)$ sont définies globalement.
\end{enumerate}
\end{exer}

\begin{exer}
Déterminer la forme générale du portrait de phase autonome en dimension 1.
\end{exer}

\begin{exer}
Montrer que le problème général : $y' = f(y,t)$, posé sur un ouvert $U$ d'un produit $E \times I$ %
où $E$ est un espace de Banach et $I$ un intervalle réel, %
est équivalent à une équation différentielle autonome sur un espace que l'on précisera.\\
Critiquer la pertinence d'une telle simplification.
\end{exer}

\section{Calcul diff\'erentiel}
% \section{Calcul diff\'erentiel}

\subsection{Avec les accroissements finis}

\begin{exer}
Soient $E$ et $F$ deux espaces vectoriels normés, $U$ un ouvert de $E$, %
et $f$ une application continue de $U$ dans $F$ différentiable sur $U \setminus \{ a \}$, où $a$ est un élément de $U$. %
On suppose, de plus, que $df$ admet une limite en $a$.\\
Montrer que $f$ est différentiable en $a$.
\end{exer}

\begin{exer}
\begin{enumerate}
\item Donner un exemple d'application de $\mathbb{R}^2$ dans lui-même qui soit un difféomorphisme local mais non global.
\item Montrer que \[\mathbb{R}^2 \rightarrow \mathbb{R}^2 : (x,y) \mapsto (\sin x + \sinh y, \sinh x - \sin y)\]
est un difféomorphisme de $\mathbb{R}^2$ sur lui-même.
\end{enumerate}
\end{exer}

\begin{exer}
Soit $n$ un entier naturel non nul. On munit $\mathbb{R}^n$ de son produit scalaire canonique. %
Soit de plus $f$ une application de $\mathbb{R}^n$ dans lui-même, différentiable, telle que :
\[\exists \alpha \in \mathbb{R}_+ | \forall a \in \mathbb{R}^n , \forall h \in \mathbb{R}^n , \langle df(x)h | h \rangle \geq \alpha \| h \|^2\]
\begin{enumerate}
\item Soient $a$ et $b$ deux vecteurs de $\mathbb{R}^n$. Montrer que :
\[\langle f(b)-f(a) | b-a \rangle \geq \| b-a \|^2\]
%Indication : Appliquer le théorème des accroissements finis à $[0,1] \rightarrow \mathbb{R}^n : t \mapsto f(tb + (1-t)a)$.\\
\item Monterer que $f$ est une application fermée.
%On montrera que : \begin{center}$\forall (a,b) \in (\mathbb{R}^n)^2 , \| f(b)-f(a) \| \geq \| b-a\|$\end{center}
\item Montrer que $f$ est un difféomorphisme local.
\item Montrer que $f$ est un difféomorphisme global.
%Indications : $f$ est injective d'après l'inégalité montrée à la question 2). D'après la question précédente, $f$ est ouverte.
\end{enumerate}
\end{exer}

\begin{exer}
Montrer que le système 
\[xu^2 + yzv + x^2 z = 3 , xyv^3 + 2zu - u^2 v^2 = 2\]
permet de définir, au voisinage du point $(1,1,1,1,1)$, $(u,v)$ comme une fonction de $(x,y,z)$.\\
Quelle est la différentielle de cette fonction en $(1,1,1)$ ?
\end{exer}

\begin{exer}
Soient $H_0$ l'espace de Banach des applications continues de $[0,1]$ dans $\mathbb{R}$ %
muni de la norme de la convergence uniforme $\| \|_{\infty}$, %
$H_1$ l'espace vectoriel de applications $C^1$ de $[0,1]$ dans $\mathbb{R}$ %
qui s'annulent en $0$, normé par $\| \|_{\infty} + f \mapsto \| f' \|_{\infty}$ que nous noterons $\| \|_1$.
\begin{enumerate}
\item Montrer que $(H_1 , \| \|_1)$ est un espace de Banach.
\item Soit $\phi H_1 \rightarrow H_0 : f \mapsto f' + ff'$. Montrer que $\phi$ est différentiable et calculer sa différentielle.
\item Que dire de la différentielle de $\phi$ en $0$ ?
\item Montrer que si la norme de l'application continue $g$ est suffisament petite, %
alors l'\'equation différentielle $f' + ff' = g$ admet une solution dans $H_1$.
\end{enumerate}
\end{exer}

\begin{exer}
Soient $U$ un ouvert de $\mathbb{R}^2$ et $f$ une application de $U$ dans $\mathbb{R}$  %
qui admet des dérivées partielles bornées.
\begin{enumerate}
\item Montrer que si $U$ est convexe, alors $f$ est uniformément continue.
\item Que devient le résultat si on suppose seulement $U$ connexe ?
\end{enumerate}
\end{exer}

\begin{exer}
Montrer qu'une application de classe $C^1$ de $\mathbb{R}^2$ dans $\mathbb{R}$ ne peut pas être injective.
\end{exer}

\subsection{G\'eom\'etrie diff\'erentielle}

\begin{exer}
Soit $f$ une fonction r\'eelle \`a valeurs complexes, d\'erivable et  $2 \pi$ p\'eriodique.\\
Montrer que : $\frac{1}{2i\pi} \int_0^{2 \pi} \frac{f'(t)}{f(t)} dt$ est un entier.
\end{exer}

\begin{exer}
On se propose de démontrer l'inégalité isopérimétrique :\\
si $\gamma$ est un arc de Jordan régulier de $\mathbb{R}^2$, de longueur $l$ et dont l'intérieur admet pour aire $A$, alors :
\[4 \pi A \leq l^2\]
\begin{enumerate}
\item Démontrer l'inégalité de Wirtinger : si $y$ est une application $2 \pi$-périodique $C^1$ de $\mathbb{R}$ dans lui-même, alors :
\[\int_{0}^{2 \pi} y^2 \leq \int_{0}^{2 \pi} y'^2\]
Etudier le cas d'égalité.

\smallskip
On reprend les notations de l'énoncé, et on écrit : $\gamma := (x,y)$.
\item Montrer que l'on peut supposer $\gamma$ centré en $(0,0)$, et de longueur $2 \pi$.

\smallskip
On supposera par la suite que $\gamma$ est paramétré par longueur d'arc.
\item Démontrer, en utilisant la formule de Green-Riemann, que :\[2A \leq \int_{0}^{2 \pi} (x^2 (s) + y'^2 (s))ds\]
\item Conclure.
%On remarquera que : $\int_{0}^{2 \pi} (x'^2 (s) + y'^2 (s)) ds = \frac{l^2}{2 \pi}$
\item Etudier le cas d'égalité.
\end{enumerate}
\end{exer}



% \section{Ancien programme}

\subsection{Alg\g{e}bre bilin\'eaire}

\begin{exer}
Soit $A = \mathbb{R} \rightarrow M_n(\mathbb{R}) : t \mapsto A(t)$ une application continue qui prend des valeurs antisymétriques.\\
Montrer que toute solution de l'équation différentielle en $X$, fonction de $\mathbb{R}$ dans $M_n(\mathbb{R})$ :
\[\forall t \in \mathbb{R} , X'(t) = A(t)X(t)\]
telle que $X(0)$ soit une matrice orthogonale, prend ses valeurs dans l'ensemble des matrices orthogonales.
\end{exer}

\begin{exer}
%Cet exercice utilise les résultats de l'exercice ??\\
Soit $n$ un entier naturel non nul. On notera $Asym(n,\mathbb{R})$ l'espace des matrices antisymétriques d'ordre $n$ %
sur $\mathbb{R}$.\\
Montrer que l'exponentielle de matrices induit une surjection de $Asym(n,\mathbb{R})$ sur $SO(n,\mathbb{R})$.
\end{exer}

\subsection{Formes bilin\'eaires et quadratiques}

\begin{exer}
Sioent $q$ et $q'$ deux formes quadratiques de même cône isotrope.

Donner une relation simple entre $q$ et $q'$.
\end{exer}

\begin{exer}
Montrer que les formes bilinéaires $\phi$, non dégénérées, d'un espace vectoriel vérifiant :
\[\forall (x,y) \in E^2, \phi (x,y) = 0 \Rightarrow \phi (y,x) = 0\]
sont les formes bilinéaires symétriques et antisymétriques.

\medskip
On pourra \'etudier les familles de formes lin\'eaires, indexées en $x$ 
$d_x : E \rightarrow \mathbb{K} : y \mapsto \phi(y,x)$ et $g_x : E \rightarrow \mathbb{K} : y \mapsto \phi(x,y)$
\end{exer}

\newpage

\begin{center}
\fbox{
\begin{minipage}{15cm}
\textit{
Pour toute forme quadratique $q$, on appelle groupe orthogonal de $q$ et note $O(q)$ %
l'ensemble des automorphismes linéaires $u$ de $E$, encore appelés isométries, tels que :
\[\forall x \in E , q(u(x)) = q(x)\]
En particulier, une isom\'etrie pour $q$ pr\'eserve sa forme polaire $\varphi$
}
\end{minipage}
}
\end{center}

\begin{exer}[Commutant du groupe $O(q)$ dans $\mathcal{L}(E)$]
Soit $E$ un espace vectoriel de dimension finie. On considère, dans cet exercice, une forme quadratique $q$ non dégénérée sur $E$, on note $\varphi$ sa forme polaire.

Soit de plus $a$ un vecteur de $E$, non isotrope pour $q$. %
On note $A$ la droite vectorielle de $E$ engendrée par $a$, et $B$ l'espace $\{ x \in E | \varphi (a,x) = 0 \}$.
\begin{enumerate}
\item Montrer que : $E = A \oplus B$.
\item Avec cette notation, montrer que la symétrie linéaire par rapport à $A$, et parallèlement à $B$, est une isométrie pour $q$.

\medskip
On note :\[C = \{ v \in L(E) | \forall u \in O(q) , uv = vu\}\]
cet ensemble est appelé le commutant de $O(q)$ dans $L(E)$.\\
Soit $v$ un élément de $C$.
\item Soit encore $a$ un vecteur de $E$ non isotrope pour $q$. Montrer que : %
$\exists \lambda_a \in \mathbb{R} | v(a) = \lambda_a a$.
\item Que dire du scalaire $\lambda_b$, défini pour un autre quelconque vecteur non isotrope de $q$ ?
\item Etudier le cas d'un vecteur isotrope de $q$.
\item Déterminer $C$.
\end{enumerate}
\end{exer}

\begin{exer}[Condition de minimalité de $O(q)$]
%\textit{Cet exercice reprend les notations adoptées au début de l'exercice précédent, %
%et utilise le résultat concernant le vecteur isotrope de $q$.}
On reprend la notation de l'exercice pr\'ec\'edent pour le groupe orthogonal.

\medskip
Soient $q$ et $q'$ deux formes quadratiques sur $E$, on suppose que $q$ est non dégénérée. %
On note encore $b$ et $b'$ les formes polaires respectives pour $q$ et $q'$.
\begin{enumerate}
\item Montrer l'existence d'un endomorphisme $u$ de $E$ tel que :
\[\forall (x,y) \in E^2 , b'(x,y) = b(u(x),y)\]
autrement dit :
\[\forall (x,y) \in E^2 , b'(x,y) = b(x,u(y))\]
On suppose maintenant que : $O(q') \subseteq O(q)$.
\item Montrer que cette inclusion est une égalité.
\end{enumerate}
\end{exer}

\subsection{Formes sesquilin\'eaires complexes}

\begin{exer}
Montrer que la norme de $M_n (\mathbb{C})$ qui dérive du produit scalaire $(A,B) \mapsto Tr(A^{\ast} B)$ %
est une norme d'algèbre.\\
Calculer la norme d'une matrice Hermitienne.
\end{exer}

\begin{exer}
Soient $A$ et $B$ deux matrices Hermitiennes. Montrer que les valeurs propres de $AB - BA$ sont imaginaires pures.
\end{exer}

\begin{exer}
Montrer qu'une matrice réelle -respectivement complexe- inversible est le produit d'une matrice orthogonale %
par une matrice symétrique définie positive -respectivement une matrice unitaire par une matrice Hermitienne définie positive.
\end{exer}

\begin{exer}
On définit un ordre $\preceq$ sur l'ensemble des matrices hermitiennes positives par : $H \preceq K$ si et seulement si $K - H$ est positive.

Montrer que $M_n(\mathbb{C}) \rightarrow M_n(\mathbb{C}) : A \mapsto A^{\ast} A$ est convexe pour cette relation d'ordre.
\end{exer}
% \section{Ancien programme}

\subsection{Alg\g{e}bre bilin\'eaire}

\begin{exer}
Soit $A = \mathbb{R} \rightarrow M_n(\mathbb{R}) : t \mapsto A(t)$ une application continue qui prend des valeurs antisymétriques.\\
Montrer que toute solution de l'équation différentielle en $X$, fonction de $\mathbb{R}$ dans $M_n(\mathbb{R})$ :
\[\forall t \in \mathbb{R} , X'(t) = A(t)X(t)\]
telle que $X(0)$ soit une matrice orthogonale, prend ses valeurs dans l'ensemble des matrices orthogonales.
\end{exer}

\begin{exer}
%Cet exercice utilise les résultats de l'exercice ??\\
Soit $n$ un entier naturel non nul. On notera $Asym(n,\mathbb{R})$ l'espace des matrices antisymétriques d'ordre $n$ %
sur $\mathbb{R}$.\\
Montrer que l'exponentielle de matrices induit une surjection de $Asym(n,\mathbb{R})$ sur $SO(n,\mathbb{R})$.
\end{exer}

\subsection{Formes bilin\'eaires et quadratiques}

\begin{exer}
Sioent $q$ et $q'$ deux formes quadratiques de même cône isotrope.

Donner une relation simple entre $q$ et $q'$.
\end{exer}

\begin{exer}
Montrer que les formes bilinéaires $\phi$, non dégénérées, d'un espace vectoriel vérifiant :
\[\forall (x,y) \in E^2, \phi (x,y) = 0 \Rightarrow \phi (y,x) = 0\]
sont les formes bilinéaires symétriques et antisymétriques.

\medskip
On pourra \'etudier les familles de formes lin\'eaires, indexées en $x$ 
$d_x : E \rightarrow \mathbb{K} : y \mapsto \phi(y,x)$ et $g_x : E \rightarrow \mathbb{K} : y \mapsto \phi(x,y)$
\end{exer}

\newpage

\begin{center}
\fbox{
\begin{minipage}{15cm}
\textit{
Pour toute forme quadratique $q$, on appelle groupe orthogonal de $q$ et note $O(q)$ %
l'ensemble des automorphismes linéaires $u$ de $E$, encore appelés isométries, tels que :
\[\forall x \in E , q(u(x)) = q(x)\]
En particulier, une isom\'etrie pour $q$ pr\'eserve sa forme polaire $\varphi$
}
\end{minipage}
}
\end{center}

\begin{exer}[Commutant du groupe $O(q)$ dans $\mathcal{L}(E)$]
Soit $E$ un espace vectoriel de dimension finie. On considère, dans cet exercice, une forme quadratique $q$ non dégénérée sur $E$, on note $\varphi$ sa forme polaire.

Soit de plus $a$ un vecteur de $E$, non isotrope pour $q$. %
On note $A$ la droite vectorielle de $E$ engendrée par $a$, et $B$ l'espace $\{ x \in E | \varphi (a,x) = 0 \}$.
\begin{enumerate}
\item Montrer que : $E = A \oplus B$.
\item Avec cette notation, montrer que la symétrie linéaire par rapport à $A$, et parallèlement à $B$, est une isométrie pour $q$.

\medskip
On note :\[C = \{ v \in L(E) | \forall u \in O(q) , uv = vu\}\]
cet ensemble est appelé le commutant de $O(q)$ dans $L(E)$.\\
Soit $v$ un élément de $C$.
\item Soit encore $a$ un vecteur de $E$ non isotrope pour $q$. Montrer que : %
$\exists \lambda_a \in \mathbb{R} | v(a) = \lambda_a a$.
\item Que dire du scalaire $\lambda_b$, défini pour un autre quelconque vecteur non isotrope de $q$ ?
\item Etudier le cas d'un vecteur isotrope de $q$.
\item Déterminer $C$.
\end{enumerate}
\end{exer}

\begin{exer}[Condition de minimalité de $O(q)$]
%\textit{Cet exercice reprend les notations adoptées au début de l'exercice précédent, %
%et utilise le résultat concernant le vecteur isotrope de $q$.}
On reprend la notation de l'exercice pr\'ec\'edent pour le groupe orthogonal.

\medskip
Soient $q$ et $q'$ deux formes quadratiques sur $E$, on suppose que $q$ est non dégénérée. %
On note encore $b$ et $b'$ les formes polaires respectives pour $q$ et $q'$.
\begin{enumerate}
\item Montrer l'existence d'un endomorphisme $u$ de $E$ tel que :
\[\forall (x,y) \in E^2 , b'(x,y) = b(u(x),y)\]
autrement dit :
\[\forall (x,y) \in E^2 , b'(x,y) = b(x,u(y))\]
On suppose maintenant que : $O(q') \subseteq O(q)$.
\item Montrer que cette inclusion est une égalité.
\end{enumerate}
\end{exer}

\subsection{Formes sesquilin\'eaires complexes}

\begin{exer}
Montrer que la norme de $M_n (\mathbb{C})$ qui dérive du produit scalaire $(A,B) \mapsto Tr(A^{\ast} B)$ %
est une norme d'algèbre.\\
Calculer la norme d'une matrice Hermitienne.
\end{exer}

\begin{exer}
Soient $A$ et $B$ deux matrices Hermitiennes. Montrer que les valeurs propres de $AB - BA$ sont imaginaires pures.
\end{exer}

\begin{exer}
Montrer qu'une matrice réelle -respectivement complexe- inversible est le produit d'une matrice orthogonale %
par une matrice symétrique définie positive -respectivement une matrice unitaire par une matrice Hermitienne définie positive.
\end{exer}

\begin{exer}
On définit un ordre $\preceq$ sur l'ensemble des matrices hermitiennes positives par : $H \preceq K$ si et seulement si $K - H$ est positive.

Montrer que $M_n(\mathbb{C}) \rightarrow M_n(\mathbb{C}) : A \mapsto A^{\ast} A$ est convexe pour cette relation d'ordre.
\end{exer}
% \section{Ancien programme : espaces complets}

\begin{exer}
Soit $E$ un espace vectoriel normé.

Montrer que $E$ est un espace de Banach si et seulement si toute série absolument convergente de $E$ est convergente.
\end{exer}

\begin{exer}
%Je donne, si cela s'avère nécessaire, la définition d'un ensemble dénombrable, ainsi que quelques exemples de tels ensembles, sur suggestion de l'élève. Plus particulièrement, j'établis que $\mathbb{Q}$ est dénombrable.\\
Soit $(E,d)$ un espace métrique complet, par exemple un fermé d'un espace de Banach, muni de la distance associée à la norme.
\begin{enumerate}
\item Théorème de Baire : soit $(U_n)_{n \in \mathbb{N}}$ une suite d'ouverts denses de $E$.
Montrer que $\bigcap_{n \in \mathbb{N}}U_n$ est dense dans $E$.
\item Montrer que la réunion d'une suite de fermés d'intérieurs vides de $E$ est d'intérieur vide.
\item Montrer que $\mathbb{R}$ n'est pas dénombrable.
\item Montrer que $\mathbb{R} \backslash \mathbb{Q}$ est dense dans $\mathbb{R}$.

\medskip
On note maintenant $\mathbb{K}$ l'un des deux corps $\mathbb{R}$ ou $\mathbb{C}$.
\item Soit $E$ un espace de Banach sur $\mathbb{K}$. Montrer que $E$ n'admet pas de base dénombrable.
%Indication : Soit $(x_n)$ une éventuelle base dénombrable de $E$. Considérer la suite $(Vect(x_k)_{k \in 0,n})_n$.\\
\item Montrer qu'il n'existe pas de norme pour laquelle $\mathbb{K}[X]$ soit un espace de Banach.
\end{enumerate}
\end{exer}

\begin{exer}
$\mathbb{K}$ est ici le corps $\mathbb{R}$ ou $\mathbb{C}$.\\
On note $l^{\infty}(\mathbb{K})$ l'espace vectoriel des suites bornées de $\mathbb{K}$ muni de la norme $\|.\|_{\infty}$ définie par :
\[\forall (u_n) \in l^{\infty}(\mathbb{K}) , \|(u_n)\|_{\infty} = \sup_{n \in \mathbb{N}} u_n\]
Montrer que $l^{\infty}(\mathbb{K})$ est un espace de Banach.
\end{exer}

\begin{exer}
$\mathbb{K}$ est ici le corps $\mathbb{R}$ ou $\mathbb{C}$. $p \in [1,+\infty[$\\
On note, pour tout réel $p$ supérieur ou égal à $1$, $l^{p}(\mathbb{K})$ l'espace vectoriel des suites $(u_n)$ de $\mathbb{K}$ %
telles que $\sum \lvert u_n \rvert^{p}$ converge, muni de la norme $\|.\|_{p}$ définie par :
\[\forall (u_n) \in l^{p}(\mathbb{K}) , \|(u_n)\|_{p} = (\sum_{n \in \mathbb{N}} \lvert u_n \rvert^{p})^{1/p}\]
\begin{enumerate}
\item Quelles relations d'inclusion existe-t-il entre les espaces $l^{p}(\mathbb{K})$ ?
\item Montrer que ces espaces vectoriels normés sont de Banach.
\end{enumerate}
%Eventuellement, afin de faciliter les calculs, je demande de se restreindre au cas de $l^{1}(\mathbb{K})$.
\end{exer}

\begin{exer}
Soit $E$ un espace vectoriel normé.\\
Montrer que $E$ est un espace de Banach si et seulement si toute série absolument convergente de $E$ est convergente.
\end{exer}

\begin{exer}
$\mathbb{K}$ est ici le corps $\mathbb{R}$ ou $\mathbb{C}$.\\
On note $l^{\infty}(\mathbb{K})$ l'espace vectoriel des suites bornées de $\mathbb{K}$ muni de la norme %
$\|.\|_{\infty}$ définie par :
\[\forall (u_n) \in l^{\infty}(\mathbb{K}) , \|(u_n)\|_{\infty} = \sup_{n \in \mathbb{N}} u_n\]
Montrer que $l^{\infty}(\mathbb{K})$ est un espace de Banach.
\end{exer}

\begin{exer}[Dual de $l^1$]
On se place dans l'espace des suites réelles $(u_n)_n$ telles que la série $\sum_n u_n$ est absolument convergente, nous notons cet espace $l^1(\mathbb{R})$ et le munissons de la norme $\| \|_1$ définie par : $\forall (u_n)_n \in l^1 , \| (u_n) \|_1 = \sum\limits_{n=0}^{+\infty} | u_n |$. On appelle dual topologique de cet espace, et note $(l^1)'$, l'espace vectoriel des formes linéaires continues de $l^1$, que nous munirons de la norme subordonnée à $\| \|_1$à la source, $| |$ à l'arrivée.\\
Montrer qu'il existe une isométrie linaire $\phi$ de l'espace $l^{\infty}$ des suites réelles bornées muni de la norme infinie, sur $(l^1)'$, telle que :
\[\forall (a_n)_n \in l^{\infty} , \forall (u_n) \in l^1 , \phi ((a_n)_n) ((u_n)_n) = \sum\limits_{n=0}^{+\infty} a_n u_n\]
\end{exer}

\begin{exer}
\begin{enumerate}
\item Enoncer le th\'eor\`eme de convergence domin\'ee de Lebesgue.
\item Rappeler la d\'efinition de l'int\'egrabilit\'e au sens de Riemann sur un segment.
\item Que dire de l'int\'egrale d'une limite uniforme de fonctions Riemann-int\'egrables ?

\medskip
Soit maintenant $I$ un intervalle r\'eel. On appelle fonction localement int\'egrable sur $I$ une fonction int\'egrable sur tout segment de $I$. %
On peut se ramener au cas au cas d'un intervalle $I$ de la forme $[a,b[$, o\`u $a$ est un nombre r\'eel, et $b$ un e\'el\'ement de $\mathbb{R}\cup\{+\infty\}$, strictement sup\'erieur \`a $a$.
\item Montrer le th\'eor\`eme de convergence dominée pour l'intégrale de Riemann impropre :

\medskip
\fbox{
\begin{minipage}{15cm}
Soit $(f_n)$ une suite de fonctions r\'eelles d\'efinies sur $I$, localement int\'egrables et domin\'ees par une fonction $\phi$, positive et int\'egrable sur $I$. On note que les int\'egrales des termes de $(f_n)$ convergent. %
Si $(f_n)$ converge vers une limite $f$ d\'efinie sur $I$, alors $f$ est localement int\'egrable sur $I$, son int\'egrale sur $I$ converge et :
\[\lim_n \int\limits_a^b f_n(t) dt = \int\limits_a^b f(t) dt\]
\end{minipage}
}
\end{enumerate}
\end{exer}

\begin{exer}
Soit $f$ une fonction réelle définie sur $\mathbb{R}$, telle que $f$ et $f'^2$ soient intégrables.\\
Montrer que $f$ tend vers $0$ en $+ \infty$ et $- \infty$.
%Indication : penser au critère de Cauchy. On pourra utiliser le résultat de l'exercice précédent.
\end{exer}
% \input{analyse_fonctions/anc_series_fourier.tex}
% \input{edo/anc_non_lineaires.tex}
% \input{calcul_differentiel/anc_acc_finis.tex}
% \section{Ancien programme : g\'eom\'etrie diff\'erentielle}

\begin{exer}
Soit $f$ une fonction r\'eelle \`a valeurs complexes, d\'erivable et  $2 \pi$ p\'eriodique.\\
Montrer que : $\frac{1}{2i\pi} \int_0^{2 \pi} \frac{f'(t)}{f(t)} dt$ est un entier.
\end{exer}

\begin{exer}
On se propose de démontrer l'inégalité isopérimétrique :\\
si $\gamma$ est un arc de Jordan régulier de $\mathbb{R}^2$, de longueur $l$ et dont l'intérieur admet pour aire $A$, alors :
\[4 \pi A \leq l^2\]
\begin{enumerate}
\item Démontrer l'inégalité de Wirtinger : si $y$ est une application $2 \pi$-périodique $C^1$ de $\mathbb{R}$ dans lui-même, alors :
\[\int_{0}^{2 \pi} y^2 \leq \int_{0}^{2 \pi} y'^2\]
Etudier le cas d'égalité.

\smallskip
On reprend les notations de l'énoncé, et on écrit : $\gamma := (x,y)$.
\item Montrer que l'on peut supposer $\gamma$ centré en $(0,0)$, et de longueur $2 \pi$.

\smallskip
On supposera par la suite que $\gamma$ est paramétré par longueur d'arc.
\item Démontrer, en utilisant la formule de Green-Riemann, que :\[2A \leq \int_{0}^{2 \pi} (x^2 (s) + y'^2 (s))ds\]
\item Conclure.
%On remarquera que : $\int_{0}^{2 \pi} (x'^2 (s) + y'^2 (s)) ds = \frac{l^2}{2 \pi}$
\item Etudier le cas d'égalité.
\end{enumerate}
\end{exer}

%%% ------------------------------------------ %%%

\end{document}          