\section{Propositions \`a d\'emontrer}

\begin{ques}
Enoncer et d\'emontrer le th\'eor\g{e}me de d\'rivabilit\'e, et la formule de d\'erivation, d'une fonction r\'eciproque.
\end{ques}

\begin{ques}[Ancien programme]
Exprimer $\arg\cosh$ et $\arg\sinh$ \`a l'aide de la fonction logarithme et de fonctions alg\'ebriques.
\end{ques}

\begin{ques}
Enoncer et d\'emontrer le crit\`ere de convergence des int\'egrales de Bertrand.
\end{ques}

\begin{ques}
Enoncer et d\'emontrer le th\'eor\`eme de convergence des s\'eries de Bertrand.
\end{ques}

\begin{ques}
D\'ecomposer sur $\mathbb{C}$, $\frac{P'}{P}$ en \'el\'ements simples, o\`u $P$ est un polyn\^ome complexe non constant.
\end{ques}

\begin{ques}
Définir les polynômes de Tchebycheff, et établir une relation de récurrence entre les termes de la suite de ces polynômes. Calculer le polynôme de Tchebycheff de degré $5$.
\end{ques}

\begin{ques}
Montrer que, pour tout espace vectoriel de dimension finie $E$, %
toute famille $(u_i)$ d'endomorphismes de $E$ qui commutent deux \`a deux admet une base de diagonalisation commune dans $E$.
\end{ques}

\begin{ques}
Démontrer qu'une matrice r\'eelle sym\'etrique est diagonalisable \textit{id est} \'enoncer et d\'emontrer le th\'eor\g{e}me spectral.
\end{ques}

\begin{ques}
Calculer $\int\limits_{\mathbb{R}} e^{-\frac{x^2}{2}} dx$.
\end{ques}

\begin{ques}
Soient $r_1$ et $r_2$ deux rotations vectorielles de $\mathbb{R}^3$ qui ne sont pas des retournements. Montrer que ces rotations commutent si et seulement si elles ont le même axe.
\end{ques}

\begin{ques}[Ancien programme]
Donner une classification complète des quadriques vectorielles en dimension $3$.
\end{ques}

\begin{ques}
Classifier totalement les automorphismes de l'espace Euclidien $\mathbb{R}^3$.
\end{ques}