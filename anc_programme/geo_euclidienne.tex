% \section{G\'eom\'etrie euclidienne}

\begin{exer}
Enoncer et d\'emontrer une r\g{e}gle permettant de calculer explicitement les coordonnées du centre \'eventuel d'une conique, %
dont on conna\^it une équation cart\'esienne.\\
On pourra écrire l'équation d'une conique du plan affine Euclidien sous forme matricielle.
\end{exer}

\begin{exer}
Soit $q$ une forme quadratique non d\'eg\'en\'er\'ee de $\mathbb{R}^2$, d\'efinissant donc une conique $C$ centrée en $0$. %
D\'eterminer une équation cartésienne de la tangente à $\mathbb{C}$ en un point $(x_0,y_0)$ de deux manières :
\begin{itemize}
\item Considérer le vecteur dérivé d'un arc différentiable injectif traçé sur la conique, %
dans un voisinage de $(x_0,y_0)$, au paramètre associé à $(x_0,y_0)$;
\item Calculer (pourquoi ?) le noyau de la différentielle de $q$ en $(x_0,y_0)$.
\end{itemize}
%\textit{Je poursuivrai éventuellement l'exercice avec une généralisation de cette méthode aux submersions -tangence à une fibre et noyau de la différentielle...}
\end{exer}