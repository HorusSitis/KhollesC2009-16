\section{Formalisme des probabilit\'es}

\begin{exer}[Tribus sur $\mathbb{N}$]
Soit $\mathcal{B}$ une tribu sur $\mathbb{N}$. Pour tout entier naturel $a$, on note $B_a$ l'ensemble des \'el\'ements $b$ de $\mathbb{N}$ tels que :
\[\forall B \in \mathcal{B} , \{a,b\}\subseteq B \vee \{a,b\}\subseteq B^c\]
Montrer, \`a l'aide de la famille $(B_a)_a$, qu'il existe une famille $(A_i)$ de parties de $\mathbb{N}$ telles que tout \'el\'ement de $\mathcal{B}$ est une union disjointe de $A_i$.
\end{exer}

\begin{exer}
Soit $X$ une variable al\'eatoire \`a valeurs dans $\mathbb{N}$.\\
Montrer que : \[\mathbb{E}(X)=\sum\limits_{n=0}^{\infty} \mathbb{P}(X>n)\]
\end{exer}

\begin{exer}
Montrer qu'une intersection d\'enombrable d'\'ev\`enements presque s\^urs est encore un \'ev\`enement presque s\^ur.\\
En est-il de même pour une intersection non d\'enombrable ?
\end{exer}