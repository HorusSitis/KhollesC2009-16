\section{Espaces vectoriels norm\'es}

\begin{exer}
Dans tout l'exercice, $K$ est un compact, $E$ est l'espace vectoriel des applications continues de $K$ dans $\mathbb{R}$, muni de la norme infinie.\\
\begin{enumerate}
\item Soit $\varphi$ une application uniformément continue de $\mathbb{R}$ dans $\mathbb{R}$. %
Montrer que $E \rightarrow E : f \mapsto \varphi \circ f$ est uniformément continue. Est-elle linéaire ?
\item Soit $\psi$ une application continue de $K$ dans $K$. %
Montrer que $E \rightarrow E : f \mapsto f \circ \psi$ est linéaire continue, et calculer sa norme.
\end{enumerate}
\end{exer}

\begin{exer}
Soit $\varphi$ une application lin\'eaire entre deux espaces vectoriels norm\'es $E$ et $F$.

Montrer que $\varphi$ est continue si et seulement si elle transforme toute suite qui tend vers $0$ en une suite bornée.
\end{exer}

\begin{exer}
Soit $n$ un entier naturel non nul. %
Soit de plus $G$ l'ensemble des matrices de $\mathcal{M}_n(\mathbb{R})$ triangulaires supérieures et de d\'eterminant $1$.

$G$ est-il fermé, borné, connexe par arcs ?
\end{exer}

\begin{exer}
Soit $E$ un espace vectoriel normé de dimension finie. %
On \'etudie ici, pour tout compact $A$ dans $E$, l'ensemble $L_A$ des endomorphismes de $E$ qui stabilisent $A$.
\begin{enumerate}
\item Pourquoi les \'el\'ements de $L_A$ sont-ils continus ?
\item Montrer que $L_A$ est fermé, quel que soit le compact $A$ de $E$.
\begin{center}
On veut maintenant caract\'eriser les compacts $A$ de $E$ tels que $L_A$ est lui-m\^eme compact.
\end{center}
\item D'apr\`es ce qui pr\'ec\^ede, quelle propri\'et\'e doit-on rechercher sur $A$ pour conclure ? Montrer que cette propri\'et\'e ne d\'epend pas de la norme choisie sur $\mathcal{L}(E)$.
\begin{center}
On suppose que $A$ est un compact de $E$ qui contient les vecteurs d'une base $(e_i)_i$ de $E$. On peut \'ecrire : $Vect A = E$.
\end{center}
\item Montrer que l'application qui \`a une application lin\'eaire $f$ de $E$ dans lui-m\^eme asocie le r\'eel positif : $\max\limits_i \|f(e_i)\|$ est une norme de $\mathcal{L}(E)$.
\item En d\'eduire que $L_A$ est born\'e, conclure.
\item Montrer r\'eciproquement que tout compact $A$ de $E$ tel que $L_A$ est compact v\'erifie : $Vect A = E$. On pourra raisonner par l'absurde.
\end{enumerate}
\end{exer}