\section{Analyse sur des s\'eries enti\`eres}

\begin{exer}
Calculer le rayon de convergense de la s\'erie enti\`ere $\sum a_n z^n$, o\`u $a_n$ d\'esigne la $n-$i\`eme d\'ecimale de $\pi$.
\end{exer}

\begin{exer}
Soit $g$ la fonction d\'efinie par la formule :\[g(x)=\sum\limits_{n=2}^{+\infty}\frac{(-1)^n}{n(n-1)}x^n\]
\begin{enumerate}
\item D\'terminer le domaine de convergence de la s\'erie enti\`ere d\'efinie ci-dessus.
\item Calculer explicitement $g(x)$ pout tout $x$ dans ce domaine.
\end{enumerate}
\end{exer}

\begin{exer}
Calculer, pour tout r\'eel $x$, la somme : \[\sum\limits_{n=0}^{+\infty}\frac{x^{3n}}{(3n)!}\]
\end{exer}

\begin{exer}[Th\'eor\`emes Taub\'eriens]
Soit $(a_n)$ une suite complexe telle que $f:x\mapsto\sum\limits_{n=0}^{+\infty}a_nx^n$ soit d\'efinie pour tout $x$ de $]-1,1[$.\\
On suppose de plus que $f$ admet une limite $l$ \`a gauche de $1$.\\
Montrer que la s\'erie $\sum a_n$ converge, vers $l$, dans les deux cas suivants :
\begin{enumerate}
\item $\forall n \in \mathbb{N} , a_n \in \mathbb{R}_+$
\item $a_n = \underset{n\rightarrow +\infty}{o}(\frac{1}{n})$
\end{enumerate}
\end{exer}